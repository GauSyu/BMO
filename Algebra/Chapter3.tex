\chapter{Modules}
\section{Some Definitions}
\subsection{Modules}
  \begin{defn}
    Let $R$ be a ring. A \termin[left module]{module}\index{left!module} over $R$, or a left $R-$module $M$ is an abelian group together with an left operation of $R$ on $M$ satisfies both left and right distributive laws.
  \end{defn}
%  \begin{rem}
%    We usually deal only with left modules and hence call these simply modules.
%  \end{rem}

  \begin{defn}
    Let $R$ be an \emph{entire} ring and let $M$ be a $R-$module. The \termin[torsion submodule]{torsion!submodule} $M_{\tor}$ is the set of \emph{torsion} elements in $M$.
  \end{defn}


  \begin{prop}
    Let $_R\mathfrak{I}$ be the ring of left ideals of $R$, then a $R-$module is also a $_R\mathfrak{I}-$module.
  \end{prop}
  \begin{rem}
    We denote $\mathfrak{I}$ to be the ring of ideals of $R$, then the same proposition holds for $\mathfrak{I}$.
  \end{rem}

\subsection{Algebras}
  \begin{defn}
    By an \termin{$R-$algebra}, we mean a $R-$module together with a bilinear map.
  \end{defn}

  \begin{defn}
    By an \termin{algebra over $A$}, we mean a ring-homomorphism $f\colon A\To B$ such that $f(A)$ is contained in the center of $B$. We say that the algebra is \termin[finitely generated]{finitely generated!algebra} if $B$ is \emph{finitely generated as a ring} over $f(A)$.
  \end{defn}
  \begin{rem}
    In this case, $B$ can be view as an $A-$module by the operation
    \begin{equation*}
      (a,b)\longmapsto f(a)b
    \end{equation*}

    More over, any $B-$module can be view as an $A-$module by the above operation.
  \end{rem}

\newpage\section{Homomorphisms}
In this section, we should assume $R$ is commutative.
\subsection{Exactness}
  \begin{prop}
    The functor $\Hom_R(-,N)$ is \termin[right exact]{right!exact}. Which means each sequence
    \begin{equation*}
      M'\To M\To M''\To 0
    \end{equation*}
    is exact, then the sequence
    \begin{equation*}
      \Hom_R(M',N)\Ot \Hom_R(M,N)\Ot \Hom_R(M'',N)\Ot 0
    \end{equation*}
    is exact.
  \end{prop}
  \begin{prop}
    The functor $\Hom_R(M,-)$ is \termin[left exact]{left!exact}. Which means each sequence
    \begin{equation*}
      0\To N'\To N\To N''
    \end{equation*}
    is exact, then the sequence
    \begin{equation*}
      0\To \Hom_R(M,N')\To \Hom_R(M,N)\To \Hom_R(M,N'')
    \end{equation*}
    is exact.
  \end{prop}

\subsection{Representation}

  Let $M$ be an $A-$module, then $\End_A(M)$ is a ring and $M$ is a module over $\End_A(M)$. Moreover, if $A$ is commutative and $\rho\colon R\To \End_A(M)$ is a ring homomorphism, then $M$ is also a module over $R$.
  \begin{defn}
    Let $R$ be a ring and let $\rho\colon R\To \End_A(M)$ be a ring homomorphism. Then $\rho$ is called a \termin{representation} of $R$ on $M$.
  \end{defn}

  \begin{defn}
    A \emph{morphism} of representation $\rho\colon R\To \End_A(M)$ into another $\rho'\colon R\To \End_A(M')$ is an $A-$module homomorphism $h\colon M\To M'$ such that the following diagram is commutative.
    \begin{displaymath}
      \xymatrix@R=0.5cm{
                &         \End_A(M) \ar[dd]^{[h]}     \\
              R \ar[ur]^{\rho} \ar[dr]_{\rho'}                 \\
                &         \End_A(M')                 }
    \end{displaymath}
    where $[h]$ is defined by $[h]f=h\circ g\circ h^{-1}$.
  \end{defn}

  \begin{defn}
    Let $G$ be a \emph{monoid}. By a \termin[representation]{representation!of monoid} of $G$ on an $A-$module $M$, we mean a homomorphism $\rho\colon G\To \End_A(M)^{\ast}$.
  \end{defn}
  \begin{rem}
    We may extend $\rho$ to be a representation of $A[G]$ on $\End_A(M)$.
  \end{rem}

\newpage\section{Category of Modules}
  \begin{defn}
    A module $M$ is said to be \termin[finitely generated]{finitely generated!module} or of \termin[finite type]{module!of finite type}, or \emph{\red finite} over $R$, if it has a finite number of generators.
  \end{defn}
  \begin{defn}
    An exact sequence of modules $0\To\longexseq{M'}{f}{M}{g}{M''}\To0$ is said to \termin[split]{split!exact sequence}, if it is \termin[equivalent]{equivalent!exact sequences} to the canonical one: $0\To\longexseq{M'}{\imath}{M'\oplus M''}{p}{M''}\To0$. That is there exist an isomorphism $M\To M'\oplus M''$ make the following diagram commutative:
  \begin{displaymath}
      \xymatrix{
         0\ar[r] & M'\ar[r]^{f}\ar@{=}[d] & M\ar[r]^{g}\ar[d]_{\cong} & M''\ar[r]\ar@{=}[d] & 0 \\
         0\ar[r] & M'\ar[r]_-{\imath} & M'\oplus M''\ar[r]_-{p} & M''\ar[r] & 0
      }
  \end{displaymath}
  \end{defn}

  \begin{prop}
    Let $0\To\longexseq{M'}{f}{M}{g}{M''}\To0$ be an exact sequence of modules. Then the following statements are equivalent:
    \begin{enumerate}
      \setlength{\itemindent}{2ex}
      \item this sequence is split;
      \item $f$ is split monomorphism;
      \item $g$ is split epimorphism.
    \end{enumerate}
  \end{prop}
  \begin{rem}
    Such a proposition is true in general abelian category.
  \end{rem}

  \begin{defn}
    In abelian category, a \termin{kernel} of a morphism $f\colon A\To B$, is a morphism $k\colon K\To A$ such that for any object $X$, the following sequence is exact:
    \begin{equation*}
      0 \To \Hom(X,K) \To \Hom(X,A) \To \Hom(X,B) \To 0
    \end{equation*}
  \end{defn}

  Similarly, we have
  \begin{defn}
    In abelian category, a \termin{cokernel} of a morphism $f\colon A\To B$, is a morphism $c\colon B\To C$ such that for any object $X$, the following sequence is exact:
    \begin{equation*}
      0 \To \Hom(A,X) \To \Hom(B,X) \To \Hom(C,X) \To 0
    \end{equation*}
  \end{defn}
  \begin{rem}
    It is clear that these definitions are consist with the definitions in Chapter1.
  \end{rem}

  \begin{thm}
    The category of modules over a ring is an abelian category.
  \end{thm}

\newpage\section{Free Module}
  \begin{defn}
    A non-empty family of elements of $M$ is called a \termin[basis]{basis!of module} of $M$ if it is linearly independent and generates $M$. By a \termin[free]{free!module} module we mean a module admits a basis, or zero module.
  \end{defn}

  \begin{defn}
    Let $S$ be a non-empty set, we define the \emph{free} module over a ring $R$ \emph{generated by} $S$, denoted by $R\<S\>$, to be the unique object in $\mathbf{Mod}_R$ satisfying the following universal property
    \begin{quote}
      For any $R-$module $M$ and any set-map $f\colon S\To M$, there exist exact one $R-$homomorphism $\tilde{f}\colon F\To M$ making the following diagram commutative
      \begin{displaymath}
        \xymatrix@R=0.5cm{
                &         F \ar[dd]^{\tilde{f}}     \\
              S \ar[ur]^{t} \ar[dr]_{f}                 \\
                &         M                 }
      \end{displaymath}
    \end{quote}
  \end{defn}

  These two definitions are consist. If $S$ is a non-empty set then it is consist with a diagram of type $S$ (view as a discrete category) say $\{Rx_i\}_{i\in S}$. For any $R-$module $M$ and any set-map $f\colon S\To M$, $(M,f)$ induces a nature cocone $(M,\phi_i)$ where $\phi_i$ is the composite of canonical maps
  \begin{alignat*}{5}
       Rx_i & \longrightarrow & R & \longrightarrow & M \\
       rx_i & \longmapsto & r & \longmapsto & ry_i
  \end{alignat*}
  where $y_i\in M$ is arbitrary. By such a translation, the solution of the above universal property is exact the colimit of $\{Rx_i\}_{i\in S}$ and hence is $\oplus Rx_i$.

  \begin{prop}
    If sets $S$ and $S'$ have same cardinality, then $R\<S\>\cong R\<S'\>$.
  \end{prop}

  \begin{warn}
     The converse may not be true! If it is true, we say $R$ has \termin[invariant dimension property]{invariant!dimension property}. In this case, we define the \termin{dimension} of a free module $F$ to be the cardinality of a basis of $F$.
  \end{warn}

  \begin{exam}
    Let $F$ be a free $A-$module with countable infinite basis, then $R=\End_A(F)$ is a ring. And as $R-$modules $R\cong R\oplus R\oplus \cdots \oplus R$ for any finite number of summands.
  \end{exam}

  The following propositions about invariant dimension property can be found in \cite{hungerford1974algebra}. We just list them here.

  \begin{prop}
    Let $F$ be a free $R-$module with infinite basis, then every basis of $F$ has same cardinality.
  \end{prop}

  \begin{prop}\label{3.5.2}
    Fields have invariant dimension property.
  \end{prop}

  \begin{prop}
    Let $I$ be an ideal of $R$ and $F$ be a free $R-$module with basis $X$. $\pi\colon F\To F/IF$ is the canonical map. Then $F/IF$ is a free $R/I-$module with basis $\pi(X)$ and $|\pi(X)|=|X|$.
  \end{prop}

  \begin{prop}
    If $f\colon R\To A$ is an epimorphism where $A$ has invariant dimension property, then so does $R$.
  \end{prop}
  \begin{proof}
    Consider $I=\ker f$, then $A\cong R/I$, and the following comes from the above proposition.
  \end{proof}

  \begin{prop}
    Commutative rings have invariant dimension property.
  \end{prop}


  \begin{defn}
    A module $M$ is called \termin[principal]{principal!module} if there exists an element $x\in M$ such that $M=Ax$.
  \end{defn}

\subsection{Vector Spaces}
  \begin{prop}
    If any $A-$module is free, then $A$ is a division ring.
  \end{prop}
  \begin{rem}
    A ring over which every module is projective is called \termin[semisimple]{semisimple!ring}.
  \end{rem}

  \begin{thm}[Kernel and Image]
    If $f\colon V\To W$ is a homomorphism of vector spaces over a field $k$, then
    \begin{equation*}
      \dim V=\dim\ker f+\dim\im f
    \end{equation*}
  \end{thm}

  \begin{ex}
    Let $V$ be a vector spaces over a field $k$, and let $U,W$ be subspaces. Show that
    \begin{equation*}
      \dim U+\dim W=\dim(U+W)+\dim(U\cap W)
    \end{equation*}
  \end{ex}
  \begin{proof}
    Let $\{v_1,v_2,\cdots,v_k\}$ be a basis of $U\cap W$. One can extend it to a basis of $U$, say $\{v_1,v_2,\cdots,v_k,u_{k+1},u_{k+2},\cdots,u_n\}$. Similarly, we get a basis of $W$, say $\{v_1,v_2\cdots,v_k,w_{k+1},w_{k+2},\cdots,w_m\}$.
    Then it is clear that $\{v_1,v_2,\cdots,v_k,u_{k+1},u_{k+2},\cdots,u_n,w_{k+1},w_{k+2},\cdots,w_m\}$ is a basis of $U+W$. Hence we get the formula.
  \end{proof}

  \begin{ex}
    Generalize the dimension statement of \ref{3.5.2} to free modules over a commutative ring.
  \end{ex}
  \begin{proof}
    Notice that $F/\mm F$ is a vector space over $A/\mm$.
  \end{proof}

  \begin{ex}
    Let $R$ be an entire ring containing a field $k$ as subring. Suppose that $R$ is a finite dimensional vector space over $k$ under the ring multiplication. Show that $R$ is a field.
  \end{ex}
  \begin{proof}
    Let $e^1,e^2,\cdots,e^n$ be a basis of $R$, and $e^ie^j=a^{ij}_ke^k$, then for any $\alpha=\alpha_ie^i,\beta=\beta_je^j\neq0$, the equations $\alpha X=\beta$ has a solution $X=\alpha^{-1}\beta$. Otherwise, $\det(\alpha_ia^{ij}_k)=0$, hence $\alpha X=0$ has a non-zero solution, which contradict to the entireness of $R$.
  \end{proof}

\newpage\section{Duality}
  \begin{defn}
    Let $E$ be a free module over a commutative ring $A$. By the \termin[dual module]{dual!module} $\codual{E}$ of $E$ we mean the module $\Hom(E,A)$. Its elements will be called \termin[functionals]{functional}.
  \end{defn}
  \begin{rem}
    If $x\in E,f\in \codual{E}$, we sometimes denote $f(x)$ by $\<x,f\>$. Keeping $x$ fixed, we get a linear map on $\codual{E}$, which is $0$ if and only if $x=0$. Hence we get an injection $E\To E^{\vee\vee}$ which is not always a surjective. If it is, we call $E$ is \termin{reflexive}.
  \end{rem}

  \begin{prop}
    Let $E$ ba a finite free module over the commutative ring $A$, of dimension $n$. Then $\codual{E}$ is also free, and $\dim\codual{E}=n$. If $\{x_1,\cdots,x_n\}$ is a basis for $E$, and $f_i$ is the functional such that $f_i(x_j)=\delta_{ij}$, then $\{f_1,\cdots,f_n\}$ is a basis for $\codual{E}$.
  \end{prop}

  We call such a basis $\{f_i\}$ the \termin[dual basis]{dual!basis} of $\{x_i\}$.

  \begin{cor}
    When $E$ is of finite type, then the natural map $E\To E^{\vee\vee}$ is an isomorphism.
  \end{cor}

  \begin{defn}
    Let $V,V'$ be two vector spaces over a field $K$. Given a bilinear map
    \mapdes{V\times V'}{K}{(x,x')}{\<x,x'\>}
    An element $x\in V$ is said to be \termin{orthogonal} to a subset $s'$ of $V'$ if $\<x,x'\>=0$ for all $x'\in S'$. We define the \termin[kernel]{kernel!of bilinear map} of the bilinear map on the left to be the subspace of $V$ which is orthogonal to $V'$, and similarly for the kernel on the right.
  \end{defn}
  \begin{thm}
    Let $V\times V'\To K$ be a bilinear map, let $W,W'$ be its kernels on the left and right respectively, and assume that $V'/W'$ is finite dimensional. Then the induced homomorphism $V'/W'\To\codual{(V/W)}$ is an iosmorphism.
  \end{thm}

  \begin{rem}
    Let $E$ be a module over a commutative ring $A$, then we may form two types of dual:

    $\dual{E}=\Hom(E,\QQ/\ZZ)$, viewing $E$ as an abelian group;

    $\codual{E}=\Hom_A(E,A)$, viewing $E$ as an $A-$module.

    Both are called dual. If we need to distinguish them, we call $\dual{E}$ the \termin{Pontrjagin dual}.
  \end{rem}

\newpage\section{Modules over Principal Rings}
  \emph{Throughout this section, we assume that $R$ is a principal entire ring. All modules are over $R$ unless otherwise specified.}

  \begin{prop}
    Principal rings have invariant dimension property.
  \end{prop}

  \begin{thm}
    Let $F$ be a free module, and $M$ a submodule. Then $M$ is free, and its dimension is less than or equal to the dimension of $F$.
  \end{thm}
  \begin{proof}
    Let $\{v_i\}_{i\in I}$ be a basis of $F$. For any subset $J$ of $I$, we let $F_J$ to be the free submodule generated by $\{v_j\}_{j\in J}$, and we let $M_J=F_J\cap M$. Let $S$ be the set of all pairs $(M_J,w)$ where $J$ is a subset of $I$, and $w$ is a basis of $M_J$ indexed by a subset $J'$ of $J$. For two such pairs $(M_J,w), (M_K,u)$, we define $(M_J,w)\prec(M_K,u)$ if $J\subset K$ and if the basis $u$ of $M_K$ is a extension of the basis $w$ of $M_J$. Then we will use Zorn's lemma and the rest are easy to prove.
  \end{proof}
  \begin{warn}
    This is not true in general case. For example, $\ZZ/6$ is a free $\ZZ/6-$module, but  its submodule $\{0,2,4\}$ is not free.
  \end{warn}

  \begin{cor}
    Let $E$ be a finitely generated module and $E'$ a submodule. Then $E'$ is finitely generated.
  \end{cor}

  \begin{defn}
    A free $1-$dimensional module over $R$ is called \termin{infinite cyclic}.
  \end{defn}
  \begin{defn}
    Let $E$ be a module. An element $x$ of $E$ is called a \termin[torsion element]{torsion!element} if there exists $a\in R,a\neq0$ such that $ax=0$. The \termin[torsion submodule]{torsion!submodule} $E_{\tor}$ is the set of torsion elements in $E$.
    We say that $E$ is a \termin[torsion module]{torsion!module} if it consists of torsion elements.
    If $E_{\tor}=0$, we say that $E$ is \termin[torsion free]{torsion!free}.
  \end{defn}

  \begin{thm}[Structure Theorem of f.g.Modules over PID]\label{f.g.module over PID}
    Let $E$ be finitely generated. Then $E/E_{\tor}$ is free. There exists a free submodule $F$ of $E$ such that
    \begin{equation*}
      E=E_{\tor}\oplus F
    \end{equation*}
    The dimension of such a submodule $F$ is uniquely determined.
  \end{thm}
  \begin{proof}
    First prove that $E/E_{\tor}$ is torsion free, then use lemma \ref{torsion free is free} and lemma \ref{free summand}.
  \end{proof}
  \begin{lem}\label{torsion free is free}
    A finitely generated torsion free module is free.
  \end{lem}
  \begin{lem}\label{free summand}
    Let $E,E'$ be modules, and assume that $E'$ is free. Let $f\colon E\To E'$ be surjective. Then there exist a free submodule $F$ of $E$ such that $\local{f}{F}$ is an isomorphism, and such that
    \begin{equation*}
      E=F\oplus\ker f
    \end{equation*}
  \end{lem}

  \begin{defn}
    The dimension of the free module $F$ in Theorem \ref{f.g.module over PID} is called the \termin{rank} of $E$.
  \end{defn}

  \begin{defn}
    Let $E$ be a module over $R$. For $x\in E$, the map $a\mapsto ax$ is a homomorphism of $R$ onto the submodule generated by $x$, and the kernel $\ann(x)$ is an ideal called the \termin{annihilator} of $x$. We say that $m\in R$ is a \termin{period} of $x$ if $(m)=\ann(x)$. An element $c\in R,c\neq0$ is said to be an \termin{exponent} for $E$ (resp. for $x$) if $cE=0$ (resp. $cx=0$).

    Let $p$ be a rime element. We denoted by $E(p)$ the submodule of $E$ consisting of all elements having an exponent which is a power $p^r(r\geqslant1)$. A \termin{$p-$submodule} of $E$ is a submodule contained in $E(p)$.

    We select once and for all a system of representatives for the prime elements of $R$ (modulo units).

    Let $m\in R,m\neq0$. We denoted by $E_m$ the kernel of the map $x\mapsto mx$. It consists of all elements of $E$ having exponent $m$.

    A module $E$ is said to be \termin{cyclic} if  it is isomorphic to $R/(a)$ for some element $a\in R$. Without loss of generality if $a\neq0$, one may assume that $a$ is a product of primes in our system of representatives, and then we could say that $a$ is the \termin{order} of the module.
  \end{defn}
  \begin{defn}
    Let $r_1,\cdots,r_s$ be integers $\geqslant1$. A $p-$module $E$ is said to be of \termin{type}
    \begin{equation*}
      (p^{r_1},\cdots,p^{r_s})
    \end{equation*}
    if it is isomorphic to the product of cyclic modules $R/(p^{r_i})$ ($i=1,\cdots,s$). If $p$ is fixed, then one could say that the module is of type $(r_1,\cdots,r_s)$ (relative to $p$).
  \end{defn}

  \begin{defn}
    Let $y_1,\cdots,y_m$ be elements of a module, we say that they are \termin{independent} if whenever we have a relation
    \begin{equation*}
      a_1y_1+\cdots+a_my_m=0
    \end{equation*}
    with $a_i\in R$, then we must have $a_iy_i=0$ for all $i$.
  \end{defn}
  \begin{warn}
    Observe that \emph{independent} does not mean \emph{linearly independent}.
  \end{warn}
  \begin{lem}
    Let $E$ be a torsion module of exponent $p^r(r\geqslant1)$ for some prime element $p$. Let $x_1\in E$ be an element of period $p^r$. Let $\overline{E}=E/(x_1)$. Let $\overline{y}_1,\cdots,\overline{y}_m$ be independent elements of $\overline{E}$. Then for each $i$ there exists a representative $y_i\in E$ of $\overline{y}_i$, such that the period of $y_i$ is the same as the period of $\overline{y}_i$. The elements $x_1,y_1,\cdots,y_m$ are independent.
  \end{lem}

  \begin{thm}
    Let $E$ be a finitely generated torsion module $\neq0$. Then $E$ is the direct sum
    \begin{equation*}
      E=\bigoplus_p E(p)
    \end{equation*}
    taken over all primes $p$ such that $E(p)\neq0$.
    Each $E(p)$ can be written as a direct sum
    \begin{equation*}
      E(p)=R/(p^{r_1})\oplus\cdots\oplus R/(p^{r_s})
    \end{equation*}
    with $1\leqslant r_1\leqslant\cdots\leqslant r_s$. The sequence $r_1,\cdots,r_s$ is uniquely determined.
  \end{thm}

  \begin{thm}
    Let $E$ be a finitely generated torsion module $\neq0$. Then $E$ is isomorphic to a direct sum of non-zero factors
    \begin{equation*}
      R/(q_1)\oplus\cdots\oplus R/(q_r)
    \end{equation*}
    where $q_1,\cdots,q_r$ are non-zero elements of $R$, and $q_1|q_2|\cdots|q_r$. The sequence of ideals $(q_1),\cdots,(q_r)$ is uniquely determined by the above conditions.
  \end{thm}
  \begin{rem}
    The ideals $(q_1),\cdots,(q_r)$ are called the \termin[invariants]{invariant!of module} of $E$.
  \end{rem}

  \begin{thm}[Elementary Divisors]
    Let $F$ be a free module, and let $M$ be a finitely generated submodule $\neq0$. Then there exists a basis $\BBb$ of $F$, elements $e_1,\cdots,e_m$ in this basis, and non-zero elements $a_1,\cdots,a_m\in R$ such that
    \begin{enumerate}[(i)]
      \item The elements $a_1e_1,\cdots,a_me_m$ form a basis of $M$ over $R$.
      \item We have $a_i|a_{i+1}$ for $i=1,\cdots,m-1$.
    \end{enumerate}
    The sequence of ideals $(a_1),\cdots,(a_m)$ is uniquely determined by the preceding conditions.
  \end{thm}
  \begin{rem}
    We call the ideals $(a_1),\cdots,(a_m)$ the \termin[invariants]{invariant!of module} of $M$ in $F$.
  \end{rem}

  \begin{thm}
    Assume that the elementary matrices in $R^{n\times n}$ generate $\GL_n(R)$. Let $(c_{ij})$ be a non-zero matrix with components in $R$. Then with a finite number of row and column operations, it is possible to bring the matrix to the form
    \begin{equation*}
      \diag\{a_1,a_2,\cdots,a_m,0,\cdots,0\}
    \end{equation*}
    with $a_1\cdots a_m\neq0$ and $a_1|a_2|\cdots|a_m$.
  \end{thm}



\subsection{Lattice}
  \begin{ex}
    Let $A$ be an additive subgroup of Euclidean space $\RR^n$, and assume that in every bounded region of space, there is only a finite number of elements of $A$. Show that $A$ is a free abelian group on $\leqslant n$ generators.
  \end{ex}
  \begin{proof}
    Induction on the maximal number of linearly independent elements of $A$ over $\RR$.

    When this number is $1$, since in every bounded region of space, there is only a finite number of elements of $A$, there must one $a\in A$, such that $d(a,0)=\min\limits_{x\in A\setminus\{0\}}d(x,0)$. Then it is clear that $A=\ZZ a$.

    Let $v_1,\cdots,v_m$ be a maximal set of linearly independent elements of $A$ over $\RR$, and let $A_0$ be the subgroup of $A$ contained in $\RR-$space generated by $v_1,\cdots,v_{m-1}$. By induction, one may assume that any element of $A_0$ is a linear integral combination of $v_1,\cdots,v_{m-1}$.

    Let $S$ be the subset of elements $v\in A$ of the form $v=a_1v_1+\cdots+a_mv_m$ with real coefficients $a_i$ satisfying
    \begin{align*}
      &0\leqslant a_i<1  \quad\text{if } i=1,\cdots,m-1 \\
      &0\leqslant a_m\leqslant1
    \end{align*}

    Since $S$ is in a bounded region of space, there must be an element $v'_m$ of $S$ with the smallest $a_m\neq0$. It is clear that $\{v_1,\cdots,v_{m-1},v'_m\}$ is a basis of $A$ over $\ZZ$.
  \end{proof}
  \begin{rem}
    Such $A$ is called a \termin{lattice} in a Euclidean space. The above exercise is applied in algebraic number theory to show that the group of units in the ring of integers of a number field modulo torsion is isomorphic to a lattice in a Euclidean space. See.
  \end{rem}

  \begin{ex}[Artin-Tate]
    Let $G$ be a finite group operating on a finite set $S$. For $w\in S$, denote $1\cdot w$ by $[w]$, so that we have the direct sum
    \begin{equation*}
      \ZZ\<S\>=\sum_{w\in S}\ZZ[w]
    \end{equation*}
    Define an action of $G$ on $\ZZ\<S\>$ by defining $\sigma[w]=[\sigma w]$ (for $w\in S$), and extending $\sigma$ to $\ZZ\<S\>$ by linearity. Let $M$ be a subgroup of $\ZZ\<S\>$ of rank $\#[S]$. Show that $M$ has a $\ZZ-$basis $\{y_w\}_{w\in S}$ such that $\sigma y_w=y_{\sigma w}$ for all $w\in S$.
  \end{ex}
    \begin{proof}
      First, we consider a Euclidean space $E$ generated by $S$. Then $M$ is a lattice of this space, hence there exists a real number $r>0$ such that for any $X\in E$, there exist a $x\in M$ such that
      \begin{equation*}
        \|X-x\|<r
      \end{equation*}

      For any integer $n$, we can find some $\{x_v\}_{v\in S}\subset M$ such that
      \begin{equation*}
        \left\|n[v]-x_v\right\|<r
      \end{equation*}

      For any $w\in S$, define $y_w$ as follow
      \begin{equation*}
        y_w=\sum_{\substack{
                                          \sigma\in G\\
                                          \sigma v=w}}\sigma x_v
      \end{equation*}

      Then it is clear that $\sigma y_w=y_{\sigma w}$. We will show that $\{y_w\}_{w\in S}$ is a $\RR-$basis of $M$.

      If not, then there exist some real numbers $c_w$, such that
      \begin{equation*}
        \sum_{w\in S} c_wy_w=0
      \end{equation*}
      Moreover, we can assume that, some $c_w=1$.

      Let $x_v=n[v]+b_v$, then
      \begin{align*}
        0&=\sum_{w\in S} c_w\sum_{\substack{
                                          \sigma\in G\\
                                          \sigma v=w}}\sigma n[v]+b_v\\
          &=\sum_{w\in S} \sum_{\substack{
                                          \sigma\in G\\
                                          \sigma v=w}} c_wn[w]+b_v
      \end{align*}

      Hence
      \begin{equation*}
        n\leqslant\left\|\sum_{w\in S} \sum_{\substack{
                                          \sigma\in G\\
                                          \sigma v=w}} c_wn[w]\right\|
        =\left\|\sum_{w\in S} \sum_{\substack{
                                          \sigma\in G\\
                                          \sigma v=w}} b_v\right\|<|S||G|r
      \end{equation*}

      However, $n$ can be large enough to make this inequality fail to hold. Thence $\{y_w\}_{w\in S}$ is a $\RR-$basis of $M$, and by multiply some rational numbers, we can get a $\ZZ-$basis of $M$ satisfying the condition.
    \end{proof}

\subsection{Seminorm}
  \begin{ex}\label{seminorm}
    Let $M$ be a finitely generated abelian group. By a \termin{seminorm} on $M$ we mean a real-valued function $v\mapsto|v|$ satisfying the following properties:
    \begin{enumerate}
      \item $|v|\geqslant0$ for all $v\in M$;
      \item $|nv|=|n||v|$ for $n\in\ZZ$;
      \item $|v+w|\leqslant|v|+|w|$ for all $v,w\in M$.
    \end{enumerate}
    By the \emph{kernel} of seminorm we mean the subset of elements $v$ such that $|v|=0$.
    \begin{enumerate}[a)]
      \item Let $M_0$ be the kernel. Show that $M_0$ is a subgroup. If $M_0=\{0\}$, then the seminorm is called a \termin{norm}.
      \item Assume that $M$ has rank $r$. Let $v_1,\cdots,v_r\in M$ be linearly independent over $\ZZ$ mod $M_0$. Prove that there exists a basis $\{w_1,\cdots,w_r\}$ of $M/M_0$ such that
          \begin{equation*}
            |w_i|\leqslant\sum_{j=1}^i|v_j|
          \end{equation*}
    \end{enumerate}
  \end{ex}
  \begin{proof}
    \begin{enumerate}[a)]
      \item For $v\in M_0$, by \emph{2}, $0,-v\in M_0$. For $u,v\in M_0$, by \emph{3}, $u+v\in M_0$.
      \item Without loss of generality, we can assume $M_0=\{0\}$.

               Let $M_1=\<v_1,\cdots,v_r\>$. Let $d$ be the exponent of $M/M_1$.
               Then $dM$ has a finite index in $M_1$ ($d\geqslant[M:dM]=[M:M_1][M_1:dM]$).

               Let $n_{j,j}$ be the smallest positive integer such that there exist integers $n_{j,1},\cdots,n_{j,j-1}$ satisfying
               \begin{equation*}
                 n_{j,1}v_1+\cdots+n_{j,j}v_j=dw_j
               \end{equation*}
               for some $w_j\in M$.

               Without loss of generality, we may assume $0\leqslant n_{j,k}\leqslant d-1$. Then $w_1,\cdots,w_r$ form the desired basis:

               First, $w_1,\cdots,w_r$ obversely form a $\RR-$basis. We need to prove that it is also a $\ZZ-$basis. For any $a\in M$, we have $a=a_1w_1+\cdots+a_rw_r$. Without loss of generality, we can assume $0\leqslant a_j<1$. Hence
               \begin{align*}
                 da & = a_1dw_1+\cdots+a_rdw_r \\
                  & = \sum_{j=1}^r a_j \sum_{i=1}^j n_{j,i}v_i \\
                  & = \sum_{i=1}^r\sum_{j=1}^r a_jn_{j,i}v_i
               \end{align*}

               Notice that $0\leqslant a_rn_{r,r}<n_{r,r}$, by the minimum of $n_{r,r}$, $a_r$ must be $0$, and the equality become
               \begin{equation*}
                 da=\sum_{i=1}^{r-1}\sum_{j=1}^{r-1} a_jn_{j,i}v_i
               \end{equation*}

               Since $0\leqslant a_jn_{j,j}<n_{j,j}$, we can get all $a_j=0$ by induction. Hence $a=0$, which means that $w_1,\cdots,w_r$ a $\ZZ-$basis.

               The desired inequality comes from
               \begin{align*}
                 |w_j| & = \frac{1}{d}|n_{j,1}v_1+\cdots+n_{j,j}v_j| \\
                  & \leqslant \sum_{i=1}^j\frac{n_{j,i}}{d}|v_i|\leqslant\sum_{i=1}^j|v_i|
               \end{align*}
    \end{enumerate}
  \end{proof}

  \begin{ex}
    Consider the multiplicative group $\QQ^{\ast}$ of non-zero rational numbers. For a non-zero rational number $x=a/b$ with $a,b\in\ZZ$ and $(a,b)=1$, define the \emph{height}
    \begin{equation*}
      h(x)=\log\max(|a|,|b|)
    \end{equation*}
    \begin{enumerate}[a)]
      \item Show that $h$ defines a seminorm on $\QQ^{\ast}$, whose kernel consists of $\pm1$ (the torsion group).
      \item Let $M_1$ be a subgroup of $\QQ^{\ast}$, generated by $x_1,\cdots,x_m$. Let $M$ be the subgroup of $\QQ^{\ast}$ consisting of those elements $x$ such that $x^s\in M_1$ for some positive integer $s$. Show that $M$ is finitely generated, and using Exercise \ref{seminorm}, find a bound for the seminorm of a set of generators of $M$ in terms of the seminorms of $x_1,\cdots,x_m$.
    \end{enumerate}
  \end{ex}
  \begin{proof}
    \begin{enumerate}[a)]
      \item \emph{1} is clear. For $x=a/b,(a,b)=1$, if $n>0$, then $(a^n,b^n)=1$, hence
               \begin{equation*}
                  h(x^n)=\log\max(|a^n|,|b^n|)=\log(\max(|a|,|b|))^n=nh(x)
               \end{equation*}
               if $n<0$, then
               \begin{equation*}
                  h(x^n)=\log\max(|b^{|n|}|,|a^{|n|}|)=\log(\max(|b|,|a|))^{|n|}=|n|h(x)
               \end{equation*}
               if $n=0$, then
               \begin{equation*}
                 h(x^0)=\log\max(1,1)=1=h(1)
               \end{equation*}
               Which proves \emph{2}.

               Let $v=a/b,w=c/d,(a,b)=(c,d)=1$, assume $(ac,bd)=e$, then
               \begin{align*}
                 h(vw) &=\log\max(|ac/e|,|bd/e|)\\
                          &\leqslant\log\max(|ac|,|bd|)\\
                          &\leqslant\log\max(|a|,|b|)+\log\max(|c|,|d|)=h(v)+h(w)
               \end{align*}
               Which is \emph{3}.

               Hence $h(x)$ is a seminorm on $\QQ^{\ast}$.

               If $h(a/b)=0$, then $\max(|a|,|b|)=1$, hence $|a|=|b|=1$. So the kernel consists of $\pm1$.
     \item For $x_i=a_i/b_i$, let $a_i$ and $b_i$ be factorized into power of primes $p_{i,j}$ and $q_{i,j}$ respectively.
              Then For any $x=a/b\in M_1$, $a$ (resp. $b$) must be a product of some power of $p_{i,j}$ (resp. $q_{i,j}$).

              For $x=a/b\in M$, assume $x^s\in M_1$, then $a^s$ (resp. $b^s$) must be a product of some power of $p_{i,j}$ (resp. $q_{i,j}$),
              and hence so do $a$ and $b$. Whence $M$ must be a subgroup of the subgroup of $\QQ^{\ast}$ which is generated by those primes.

              Moreover, since for any $x\in M$, there exist a $s$ such that $x^s\in M_1$ and hence can be written by a product of $x_1,\cdots,x_m$. This fact shows that $x_1,\cdots,x_m$ is a maximal linear independent term in $M$.
              By Exercise \ref{seminorm}, the generators of $y_1,\cdots,y_m$ of $M$ must satisfying
              \begin{equation*}
                h(y_i)\leqslant\sum_{j=1}^ih(x_j)
              \end{equation*}
              for $i=1,\cdots,m$. Whence we get a bound of the set of generators of $M$, say $\sum\limits_{i=1}^m h(x_i)$.
    \end{enumerate}
  \end{proof}

\newpage\section{Localization}
  \emph{Throughout this section, we assume that $A$ is a commutative ring. All modules are over $A$ unless otherwise specified.}

  \begin{ex}
    Let $A$ be a commutative ring and let $M$ be an $A-$module. Let $S$ be a multiplicative subset of $A$. Define $S^{-1}M$ in a manner analogous to the one we used to define $S^{-1}A$. Show that
    \begin{enumerate}[a)]
      \item $S^{-1}M$ is an $S^{-1}A-$module.
      \item the functor $M\mapsto S^{-1}M$ is exact.
    \end{enumerate}
  \end{ex}
  \begin{proof}
    \begin{enumerate}[a)]
      \item We consider pairs $(x,s)$ with $x\in M$ and $s\in S$. we define a relation
      \begin{equation*}
        (x,s)\sim(x',s')
      \end{equation*}
      between such pairs, by the condition that there exists an element $s_1\in S$ such that
      \begin{equation*}
        s_1(s'x-sx')=0
      \end{equation*}
      It is then trivially verified that this is an equivalence relation, and the equivalence class containing a pair $(x,s)$ is denoted by $x/s$. The set of equivalence classes is denoted by $S^{-1}M$.

      We define a $S^{-1}A-$module structure on $S^{-1}M$ by
      \begin{equation*}
        (a/s)(x/s')=ax/ss'
      \end{equation*}
      It is trivially verified that this is well defined. For any $a/s_1,a'/s_1'\in S^{-1}A$ and $x/s_2,x'/s_2'\in S^{-1}M$, we have the left and right distributive laws:
      \begin{align*}
        (\frac{a}{s_1}+\frac{a'}{s_1'})\frac{x}{s_2} & = \frac{s_1'a+s_1a'}{s_1s_1'}\frac{x}{s_2} = \frac{s_1'ax+s_1a'x}{s_1s_1's_2} \\
         & = \frac{ax}{s_1s_2}+\frac{a'x}{s_1's_2} = \frac{a}{s_1}\frac{x}{s_2}+\frac{a'}{s_1'}\frac{x}{s_2} \\
        \frac{a}{s_1}(\frac{x}{s_2}+\frac{x'}{s_2'}) & = \frac{a}{s_1}\frac{s_2'x+s_2x'}{s_2s_2'} = \frac{s_2'ax+s_2ax'}{s_1s_2s_2'} \\
         & = \frac{ax}{s_1s_2}+\frac{ax'}{s_1s_2'} = \frac{a}{s_1}\frac{x}{s_2}+\frac{a}{s_1}\frac{x'}{s_2'}
      \end{align*}
      \item For any $A-$homomorphism $f\colon M\To M'$, we define $S^{-1}f\colon S^{-1}M\To S^{-1}M'$ to be
               \begin{equation*}
                 S^{-1}f(x/s)=f(x)/s
               \end{equation*}
               It is trivially verified that this is a $S^{-1}A-$homomophism and the following diagram is commutative:
               \begin{displaymath}
                 \xymatrix{
                    M \ar[r]^-{\varphi_S}\ar[d]_{f} & S^{-1}M\ar[d]^{S^{-1}f}       \\
                    M' \ar[r]_-{\varphi_S}  & S^{-1}M'             }
               \end{displaymath}
               where $\varphi_S$ is the natural map $x\mapsto x/1$. Whence $S^{-1}$ is a functor.

               Given an arbitrary short exact sequence
               \begin{equation*}
                 0\To \longexseq{M'}{f}{M}{g}{M''} \To 0
               \end{equation*}
               We need to show the sequence
               \begin{equation*}
                 0\To \longexseq{S^{-1}M'}{S^{-1}f}{S^{-1}M}{S^{-1}g}{S^{-1}M''} \To 0
               \end{equation*}
               is exact.
               \begin{enumerate}[1)]
                 \item \emph{$S^{-1}f$ is injective:}
                          For any $S^{-1}f(x'/s)=0/1$, there exists a $s'\in S$ such that $s'f(x')=0$. Hence $s'x'\in\ker f$. Since $f$ is injective, hence $s'x'=0$, which means $x'/s=0/1$.
                 \item \emph{$S^{-1}g$ is surjective:}
                          For any $x''/s\in S^{-1}M''$, there exists a $x\in M$ such that $g(x)=x''$ since $g$ is surjective. Hence $S^{-1}g(x/s)=x''/s$.
                 \item \emph{$\ker S^{-1}g=\im S^{-1}f$:}
                          Since $g\circ f=0$ and $S^{-1}$ is a functor, we have $S^{-1}g\circ S^{-1}f=0$.
                          The rest is to show that $\ker S^{-1}g\subset\im S^{-1}f$:
                          For any $x/s\in\ker S^{-1}g$, there exists a $s'\in S$ such that $s'g(x)=0$. Hence $s'x\in\ker g$. Since $\ker g=\im f$, there exist a $x'\in M'$ such that $f(x')=s'x$. Whence $S^{-1}f(x'/s's)=x/s$.
               \end{enumerate}
    \end{enumerate}
  \end{proof}

  \begin{prop}
    If $N,P$ are submodules of an $A-$module $M$, then
    \begin{enumerate}[(i)]
      \item $S^{-1}(N+P)=S^{-1}N+S^{-1}P$;
      \item $S^{-1}(N\cap P)=S^{-1}N\cap S^{-1}P$;
      \item $S^{-1}(M/N)\cong S^{-1}M/S^{-1}N$.
    \end{enumerate}
  \end{prop}
  \begin{proof}
    \emph{(i)} follows from definition. \emph{(ii)} is easy to verify: if $x/s=y/t$ ($x\in N,y\in P, s,t\in S$), then there exist a $u\in S$ such that $u(tx-sy)=0$. Hence $w=utx=usy\in N\cap P$ and $x/s=w/ust\in S^{-1}(N\cap P)$. Consequently, $S^{-1}N\cap S^{-1}P\subset S^{-1}(N\cap P)$, and the reverse inclusion is clear.

    \emph{(iii)}. Apply the functor $S^{-1}$ to the following exact sequence:
    \begin{equation*}
      0 \To N \To M \To M/N \To 0
    \end{equation*}
  \end{proof}
  \begin{cor}
    $S^{-1}(M\oplus N)=S^{-1}M\oplus S^{-1}N$.
  \end{cor}

  \begin{ex}\label{3.ex10}
    Let $\pp$ be a prime ideal, define $M_{\pp}$ in a manner analogous to the one we used to define $A_{\pp}$.
    \begin{enumerate}[a)]
      \item Show that the natural map
               \begin{equation*}
                 M\To\prod M_{\pp}
               \end{equation*}
               of a module $M$ into the direct product of all localizations $M_{\pp}$ where $\pp$ ranges over all \emph{maximal} ideals, is injective.
      \item Show that a sequence
               \begin{equation*}
                 0\To \longexseq{M'}{}{M}{}{M''} \To 0
               \end{equation*}
               is exact if and only if the sequence
               \begin{equation*}
                 0\To \longexseq{M'_{\pp}}{}{M_{\pp}}{}{M''_{\pp}} \To 0
               \end{equation*}
               is exact for all primes $\pp$.
      \item Let $A$ be entire and let $M$ be torsion-free. For each prime $\pp$ of $A$ show that the natural map $M\To M_{\pp}$ is injective.
    \end{enumerate}
  \end{ex}
  \begin{proof}
    \begin{enumerate}[a)]
      \item Suppose the natural map is not injective, and $x$ is a non-zero element in its kernel, let $\aa=\ann(x)$. It is an ideal $\neq(1)$, hence must be contained in some maximal ideal $\mm$. But by the natural map, there exists a $s\notin\mm$ such that $sx=0$ which contradicts with $\aa\subset\mm$. Whence the natural map must be injective.
      \item We leave the proof to proposition \ref{L.P.3}.
      \item The condition implies that $\ann(x)=0$ for any $x\neq0$. Let $x$ be in the kernel, then there exists a $s\notin \pp$ such that $sx=0$, which can be true only when $x=0$.
    \end{enumerate}
  \end{proof}

\subsection{Local Properties}
  See \cite{atiyah1994introduction} and \cite{eisenbud1995commutative}.

  \begin{defn}
    A property $\Pp$ of $A$ (or of an $A-$module $M$) is said to be a \termin[local property]{local!property} if the following is true:
    \begin{quote}
      $A$ (or $M$) has $\Pp$ if and only if $A_{\pp}$ (or $M_{\pp}$) has $\Pp$ for each prime ideal of $A$.
    \end{quote}
  \end{defn}

  \begin{prop}\label{L.P.1}
    Let $M$ be an $A-$module, and $x\in M$. Then the following are equivalent:
    \begin{enumerate}
      \item $x=0$;
      \item $x/1=0/1$ in $M_{\pp}$ for all prime ideals $\pp$ of $A$;
      \item $x/1=0/1$ in $M_{\mm}$ for all maximal ideals $\mm$ of $A$.
    \end{enumerate}
  \end{prop}
  \begin{proof}
    It is clear \emph{1)$\Rightarrow$2)$\Rightarrow$3)}. Suppose \emph{3)}, and $x\neq0$. Let $\aa=\ann(x)$, it is an ideal $\neq(1)$, hence must be contained in some maximal ideal $\mm$. But $x/1=0/1$ in $M_{\mm}$, there exists a $s\notin\mm$ such that $sx=0$ which contradicts with $\aa\subset\mm$.
  \end{proof}
  \begin{cor}\label{L.P.1.1}
    Let $M$ be an $A-$module. Then the following are equivalent:
    \begin{enumerate}
      \item $M=0$;
      \item $M_{\pp}=0$ for all prime ideals $\pp$ of $A$;
      \item $M_{\mm}=0$ for all maximal ideals $\mm$ of $A$.
    \end{enumerate}
  \end{cor}

  \begin{prop}\label{L.P.2}
    Let $f\colon M\To N$ be an $A-$homomorphism. Then the following are equivalent:
    \begin{enumerate}
      \item $f$ is a monomorphism (resp. epimorphism, isomorphism);
      \item $f_{\pp}\colon M_{\pp}\To N_{\pp}$ is a monomorphism (resp. epimorphism, isomorphism) for each prime ideal $\pp$ of $A$;
      \item $f_{\mm}\colon M_{\mm}\To N_{\mm}$ is a monomorphism (resp. epimorphism, isomorphism) for each maximal ideal $\mm$ of $A$.
    \end{enumerate}
  \end{prop}
  \begin{proof}
    We prove only the statements for monomorphisms:

    \emph{1)$\Rightarrow$2)}. $0\to M\to N$ is exact, hence $0\to M_{\pp}\to N_{\pp}$ is exact, i.e. $f_{\pp}$ is a monomorphism.

    \emph{2)$\Rightarrow$3)} because a maximal ideal is prime.

    \emph{3)$\Rightarrow$1)}. Let $M'$ be the kernel of $f$, then $0\to M'\to M\to N$ is exact, hence so is $0\to M'_{\mm}\to M_{\mm}\to N_{\mm}$. Then $M'_{\mm}=0$ because $f_{\mm}$ is injective. Hence $M=0$ by \ref{L.P.1.1}. Which proved $f$ is injective.
  \end{proof}

  \begin{prop}\label{L.P.3}
    Let $0\to M'\to M\to M''\to 0$ be a sequence of $A-$modules. Then the following are equivalent:
    \begin{enumerate}
      \item $0\to M'\to M\to M''\to 0$ is exact;
      \item $0\to M'_{\pp}\to M_{\pp}\to M''_{\pp}\to 0$ is exact for each prime ideal $\pp$ of $A$;
      \item $0\to M'_{\mm}\to M_{\mm}\to M''_{\mm}\to 0$ is exact for each maximal ideal $\mm$ of $A$.
    \end{enumerate}
  \end{prop}
  \begin{proof}
    By the exactness of the functor $S^{-1}$, it is clear \emph{1)$\Rightarrow$2)$\Rightarrow$3)}.

    Denote $M'\To M$ by $f$, and $M\To M''$ by $g$.

    \emph{3)$\Rightarrow$1)}. By proposition \ref{L.P.2}, we only need to prove that $\ker g=\im f$.
    It is clear that $\ker g_{\mm}=(\ker g)_{\mm}$ and $\im f_{\mm}=(\im f)_{\mm}$, hence $(\ker g)_{\mm}=(\im f)_{\mm}$. Hence $\ker g=\im f$ because the natural map $M\To\prod M_{\mm}$ is injective.
  \end{proof}

\newpage\section{Projective Modules}
  \begin{prop}
    Let $P$ be a $R-$module, then the following are equivalent.
    \begin{enumerate}[P1]
      \item Given a homomorphism $f\colon P\To M''$ and surjective homomorphism $g\colon M\To M''$, there exist a homomorphism $h\colon P\To M$ make the following diagram commutative.
        \begin{displaymath}
          \xymatrix{
             & P \ar[dl]_{h}\ar[d]^{f} &        \\
             M \ar[r]_{g} & M'' \ar[r] & 0             }
        \end{displaymath}
      \item Every exact sequence $0\To M'\To M''\To P\To 0$ splits.
      \item There exists a module $M$ such that $P\oplus M$ is free, or in words, $P$ is a direct summand of a free module.
      \item The functor $M\mapsto\Hom_R(P,M)$ is exact.
    \end{enumerate}
  \end{prop}

  \begin{defn}
    A \termin[projective module]{projective!module} is a module satisfying \emph{P1}.
  \end{defn}

  \begin{prop}
    Direct sums and direct summands of projective modules are projective.
  \end{prop}
  \begin{warn}
    Submodules of projective modules need not be projective; a ring R for which every submodule of a projective left module is projective is called left \termin{hereditary}.
  \end{warn}

  \begin{prop}
    In principal domains or local rings, projective modules and free modules coincide.
  \end{prop}
  \begin{proof}
    It is clear that projective is torsion-free, hence free over a principal domain by Theorem \ref{f.g.module over PID}. The local ring case is a deep theorem by Irving Kaplansky, see \cite{kaplansky1958projective}, but the finite generated case is just a corollary to Nakayama's lemma.
  \end{proof}
  \begin{rem}
    There may be other rings over which the similar statement is true. For example, for polynomial rings over a field, any finite projective module is free. This is Quillen-Suslin theorem, see Chapter 21, Theorem 3.7.
  \end{rem}

  \begin{defn}
    A module is called locally free if its every localization is free.
  \end{defn}
  \begin{cor}
    Projective module is locally free.
  \end{cor}
  \begin{proof}
    It is clear that a localization of a projective module is a projective module over a local ring, hence is free.
  \end{proof}
  \begin{warn}
    The converse is not always true.
  \end{warn}



\subsection{Grothendieck Group}
  Let $A$ be a ring. Isomorphism classes of finite\footnote{finitely generated} projective modules form an abelian monoid. The addition is defined by
  \begin{equation*}
    [P]+[Q]\defeq[P\oplus Q]
  \end{equation*}

  The corresponding \emph{Grothendieck group} is denoted by $K(A)$.

  We define an equivalence relation (which is called \termin{stably isomorphic}) $\sim$ as follow: $P\sim P'$ if there exist finite free modules $F,F'$ such that $P\oplus F\cong P'\oplus F'$. Under this equivalence relation we obtain another group denoted by $K_0(A)$.

  \begin{prop}
    $K(A)=K_0(A)$
  \end{prop}
  \begin{proof}
    It suffices to show that $K_0(A)$ satisfying the universal property of $K(A)$.

   Let $M(A)$ be the monoid of isomorphism classes of finite projective modules. Let $i\colon M(A)\To K_0(A)$ be $[M]\mapsto[[M]]$ where the $[[M]]$ denotes the equivalent class contains $M$. This map is well-defined since any isomorphic modules must be stably isomorphic.

    Let $G$ be an arbitrary abelian group, and $f$ be a homomorphism from $M(A)$ to $G$. We need to show that there exists a unique homomorphism $h\colon K_0(A)\To G$ satisfying the commutative diagram:
    \begin{displaymath}
      \xymatrix@R=0.5cm{
                &         K_0(A) \ar[dd]^{h}     \\
              M(A) \ar[ur]^{i} \ar[dr]_{f}                 \\
                &         G                 }
    \end{displaymath}

    \emph{Uniqueness}. Let $h,h'$ be two homomorphisms satisfying the commutative diagram, then for any $[M]\in M(A)$, $h([[M]])=h'([[M]])$.
  \end{proof}
  \begin{warn}
    Since the monoid-homomorphism from an abelian monoid to its Grothendieck group may not be injective in general, two finite projective modules which are isomorphic may not be stably isomorphic.
  \end{warn}

  A family of modules $\FFf$ is called \termin[exact]{exact!family} if for any short exact sequence $0\To M'\To M\To M''\To 0$, $M\in\FFf$ if and only if $M',M''\in\FFf$.

  Let $\Ff$ be the free abelian group generated by isomorphism classes of modules in $\FFf$. Let $\Gamma$ be the subgroup generated by all elements
  \begin{equation*}
    [M']-[M]+[M'']
  \end{equation*}
  for which there exist an exact sequence $0\To M'\To M\To M''\To 0$.

  The factor group $\Ff/\Gamma$ is called the \termin{Grothendieck group} $K(R)$.

  More general, one can consider Grothendieck group of an exact category.

  By an \termin[exact category]{exact!category} $\Aa$, we mean an additive category together with a class of distinguished short sequences $A\To B\To C$ which are called ``exact sequences''.

  It is defined in the same way as before as the abelian group with one generator $[M]$ for each isomorphism class of objects in the category and one relation
  \begin{equation*}
    [A]-[B]+[C]=0
  \end{equation*}
  for each exact sequence $A\To B\To C$.

\subsection{Euler-Poincar\'{e} Maps}

  \begin{defn}
    Let $A$ be a ring and $\Gamma$ is an abelian group. An \termin{Euler-Poincar\'{e} mapping} is a corresponding $\varphi$ from an exact family of $A-$modules, or a Serre subcategory of $\Mod_A$, to $\Gamma$ such that: for any short exact sequence
    \begin{equation*}
      0\To M'\To M\To M''\To 0
    \end{equation*}
    we have
    \begin{equation*}
      \varphi(M)=\varphi(M')+\varphi(M'')
    \end{equation*}
  \end{defn}

  \begin{defn}
    If $M$ is a module, then a sequence of submodules
    \begin{equation*}
      M=M_1\supset M_2\supset\cdots\supset M_r=0
    \end{equation*}
    is called a \termin[finite filtration]{finite!filtration}, and $r$ is called the \termin[length]{length!of filtration} of the filtration.
  \end{defn}

  \begin{defn}
    A module $M$ is said to be \termin[simple]{simple!module} if $M\neq0$ and if it does not contain any submodule other than $0$ and $M$.
    A filtration is called \termin[simple]{simple!filtration} if each $M_i/M_{i+1}$ is simple.
  \end{defn}

  \begin{thm}[Jordan-H\"{o}lder Theorem]
    Two simple filtrations of a module are equivalent.
  \end{thm}

  \begin{defn}
    A module $M$ is said to be \termin[of finite length]{module!of finite length} if it is $0$ or if it admits a simple finite filtration. The length of such a simple filtration is called the \termin[length]{length!of module} of the module.
  \end{defn}


  \begin{thm}
    Let $\varphi$ be an rule which to each simple module associates an element of an abelian group $\Gamma$, and such that if $M\cong M'$ then
    \begin{equation*}
      \varphi(M)=\varphi(M')
    \end{equation*}
    Then $\varphi$ has a unique extension to an Euler-Poincar\'{e} mapping defined on all modules of finite length.
  \end{thm}
  \begin{proof}
    Given a simple filtration
    \begin{equation*}
      M=M_1\supset M_2\supset\cdots\supset M_r=0
    \end{equation*}
    we define
    \begin{equation*}
      \varphi(M)=\sum_{i=1}^{r-1} \varphi(M_i/M_{i+1})
    \end{equation*}
  \end{proof}

\subsection{Projective Modules over Dedekind Rings}
  \emph{Let $\oo$ be a Dedekind ring and $K$ its quotient field.}

  \begin{ex}
    Let $M$ be a finitely generated torsion-free module over $\oo$. Prove that $M$ is projective.
  \end{ex}
  \begin{proof}
    Given a prime ideal $\pp$, the localized module $M_{\pp}$ is finitely generated torsion-free over $\oo_{\pp}$, which is principal. Then $M_{\pp}$ is projective, so if $F$ is finite free over $\oo$, and $f\colon F\to M$ is a surjective homomorphism, then $f_{\pp}\colon F_{\pp}\to M_{\pp}$ has a splitting $g_{\pp}\colon M_{\pp}\to F_{\pp}$. There exists $c_{\pp}\in\oo$ such that $c_{\pp}\notin\pp$ and $c_{\pp}g_{\pp}(M)\subset F$ because $g_{\pp}(M)$ is finitely generated. The family $\{c_{\pp}\}$ generates the unit ideal $\oo$: if not then $\{c_{\pp}\}$ generates an proper ideal hence belongs to some maximal ideal $\mm$, but $c_{\mm}\notin\mm$ which is a contradiction. So there is a finite number of elements $c_{\pp_i}$ and elements $x_i\in\oo$ such that $\sum x_ic_{\pp_i}=1$. Let
    \begin{equation*}
      g=\sum x_ic_{\pp_i}g_{\pp_i}
    \end{equation*}
    Then $g\colon M\to F$ is a homomorphism, and $f\circ g=f\circ(\sum x_ic_{\pp_i}g_{\pp_i})=\sum x_i f\circ c_{\pp_i} g_{\pp_i}=\sum x_ic_{\pp_i} f_{\pp_i}\circ g_{\pp_i}=\sum x_ic_{\pp_i}\id_{\pp_i}=\id$.
  \end{proof}
  \begin{ex}\label{3.ex12}
    \begin{enumerate}[a)]
      \item Let $\aa,\bb$ be ideals. Show that there is an isomorphism:
               \begin{equation*}
                 \aa\oplus\bb\markar{\cong}\oo\oplus\aa\bb
               \end{equation*}
      \item Let $\aa,\bb$ be fractional ideals, and let $f\colon\aa\to\bb$ be an isomorphism (as $\oo-$modules, of course).
               Then $f$ has an extension to a $K-$linear map $f_K\colon K\to K$. Let $c=f_K(1)$. Show that $\bb=c\aa$ and that $f$ is given by the mapping $m_c\colon x\mapsto cx$.
      \item Let $\aa$ be a fractional ideal. For each $b\in\aa^{-1}$ the map $m_b\colon\aa\to\oo$ is an element of the dual $\codual{\aa}$.
               Show that $\aa^{-1}=\codual{\aa}$ under this map, and so $\aa^{\vee\vee}=\aa$.
    \end{enumerate}
  \end{ex}
  \begin{proof}
    \begin{enumerate}[a)]
      \item Assume $\aa,\bb$ are relatively prime. Then $\aa+\bb=\oo$ and $\aa\bb=\aa\cap\bb$.
               Consider the canonical map from $\aa\oplus\bb$ to $\aa+\bb$, we get the following exact sequence:
               \begin{equation*}
                 0 \To \aa\bb \To \aa\oplus\bb \To \oo \To 0
               \end{equation*}
               Since $\oo$ is a free $\oo-$module, $\aa\oplus\bb\cong\oo\oplus\aa\bb$.

               As for the general case, thanks to Exercise 2.19, there exists a $c\in K$ such that $c\aa$ is relatively prime to $\bb$. Hence we have
               \begin{equation*}
                 \aa\oplus\bb\cong c\aa\oplus\bb\cong\oo\oplus c\aa\bb\cong\oo\oplus\aa\bb
               \end{equation*}
      \item For any $x\in\aa$, we have $c=f_K(1)=f_K(x^{-1}x)=x^{-1}f(x)$ since $f_K$ is a $K-$linear map and an extension of $f$.
               Hence $f(x)=cx$, which proved the statements since $f$ is an isomorphism.
      \item The map $b\mapsto m_b$ is clearly injective and a $\oo-$homomorphism. It suffices to show that it is surjective.
               For any $f\in\codual{\aa}$, by \emph{b)}, there exists a $c\in K$ such that $f=m_c$ and $c\aa\subset\oo$. Since $\aa^{-1}=\{c\in K|c\aa\subset\oo\}$ (if $c\aa\subset\oo$, then $c\aa\aa^{-1}\subset\aa^{-1}$, hence $c\in\aa^{-1}$), we have $c\in\aa^{-1}$.
    \end{enumerate}
  \end{proof}

  \begin{ex}
    \begin{enumerate}[a)]
      \item Let $M$ be a projective finite module over the Dedekind ring $\oo$.
               Show that there exist free modules $F$ and $F'$ such that $F\supset M\supset F'$, and $F,F'$ have the same rank, which is called the \termin{rank} of $M$.
      \item Prove that there exists a basis $\{e_1,\cdots,e_n\}$ of $F$ and ideals $\aa_1,\cdots,\aa_n$ such that $M=\aa_1e_1+\cdots+\aa_ne_n$, or in other words, $M=\bigoplus\aa_i$.
      \item Prove that $M\cong\oo^{n-1}\oplus\aa$ for some ideal $\aa$, and that the association $M\mapsto\aa$ induces isomorphism of $K_0(\oo)$ with the group of ideal classes $\Pic(\oo)$.
    \end{enumerate}
  \end{ex}
  \begin{proof}
    \begin{enumerate}[a)]
      \item Let $x_1,\cdots,x_n$ generated $M$, then there exists a maximal linear independent subterm, say $x_1,\cdots,x_k$, then $F'=\<x_1,\cdots,x_k\>\subset M$. For any $x_i, k<i\leqslant n$, assume that $a_ix_i\in F'$, then there exists a $c\in K$ such that $x_i\in F=\<cx_i,\cdots,cx_k\>$ for any $1\leqslant i\leqslant n$. Hence $F'\subset M\subset F$ and $\rank(F)=\rank(F')$.
      \item Let $p_i\colon F\to\oo$ be the projection from $F$ to the $i-$th coefficient, then $p_i(M)=\aa_i$ is an ideal of $\oo$.
               It is then clear that $M=\aa_1e_1\cdots+\aa_ne_n$.
      \item By Exercise \ref{3.ex12}, there exists an ideal $\aa$ such that $M\cong\oo^{n-1}\oplus\aa$, and the map $M\mapsto\aa$ is a homomorphism.
               Since every ideal is a projective module, the map is clearly surjective. It suffices to show that if $M\cong\oo^n\oplus\aa$, $N\cong\oo^m\oplus\bb$ and $[\aa]=[\bb]$ in $\Pic(\oo)$, then $[M]=[N]$ in $K_0(\oo)$.

               $[\aa]=[\bb]$ means there exists a principal fractional ideal $\cc$ such that $\bb=\cc\aa$.
               Hence $N\cong\oo^m\oplus\cc\aa\cong\oo^{m-1}\oplus\cc\oplus\aa$. But $\cc$ is clearly a free module.
               Hence there exist free modules $F$ and $F'$ such that $M\oplus F\cong\oo^r\oplus\aa\cong N\oplus F'$, which means $[M]=[N]$ in $K_0(\oo)$.
    \end{enumerate}
  \end{proof}

\newpage\section{Inverse Limits}

  \begin{thm}
    Inverse limits exist in the category of groups., in the category of modules over a ring, and also in the category of rings.
  \end{thm}

  \begin{ex}
    Prove that the inverse limit of a system of simple groups in which the homomorphisms are surjective is either the trivial group, or a simple group.
  \end{ex}
  \begin{proof}
    For any homomorphism $\phi$ in this system, since $\phi$ is surjective and its domisn is simple, $\phi$ must be isomorphism or trivial. If all homomorphisms are isomorphism, then it is clear that the limit is isomorphic to the groups in system, which are simple, or trivial.
    If not, there must be one trivial homomorphism whose codomain is trivial since it is surjective. Consider the commutative diagram below
        \begin{displaymath}
          \xymatrix{
             \invlim G\ar[r]\ar[dr]^-{0} & G_i\ar[d]^{0}               \\
             \invlim G\ar[r]_-{0}\ar@{-->}[u]^-{1}_-{0} & G_j }
        \end{displaymath}
    where $0$ denote the trivial homomorphisms and $1$ denote the isomorphism. By the universality of $\invlim G$, $1=0$, hence $\invlim G$ is trivial.
  \end{proof}

  \begin{defn}
    Let $A$ be a commutative ring and $I$ a proper ideal. Define a \termin{$I-$Cauchy sequence} $\{x_n\}$ to be a sequence of elements of $A$ satisfying the following condition:

    Given a positive integer $k$, there exists $N$ such that for all $n,m\geqslant N$ we have $x_n-x_m\in I^k$.

    Define a \termin{null sequence} to be a sequence for which given $k$ there exists $N$ such that for all $n\geqslant N$ we have $x_n\in I^k$.
  \end{defn}
  \begin{prop}
    Define addition and multiplication of sequences termwise. Then the $I-$Cauchy sequences form a ring $\Cc$, the null sequences form an ideal $\Nn$.
  \end{prop}
  \begin{defn}
    The factor ring $\Cc/\Nn$ is called the \termin[$I-$adic completion]{$I-$adic!completion} of $A$.
  \end{defn}
  \begin{prop}
    There is a natural isomorphism
    \begin{equation*}
      \Cc/\Nn \cong \invlim A/I^n
    \end{equation*}
  \end{prop}

  \begin{defn}
    Let $p$ be a prime number. For $n\geqslant m$ we have a canonical surjective ring homomorphism
    \begin{equation*}
      \phi^n_m\colon \ZZ/p^n\ZZ \To \ZZ/p^m\ZZ
    \end{equation*}

    The projective limit is called the ring of $p-$adic integers, and denoted by $\ZZ_p$.
  \end{defn}
  \begin{warn}
     $\ZZ_p$ is not isomorphic to $\ZZ_{(p)}$, the localization of $\ZZ$ at $(p)$. In fact, $\ZZ_p$ has the cardinality of the continuum! By the way, $\ZZ_p$ is the completion of $\ZZ_{(p)}$ at the unique maximal ideal $p\ZZ_{(p)}$.
  \end{warn}

  \begin{ex}
    \begin{enumerate}[a)]
      \item Let $n$ range over the positive integers and let $p$ be a prime number. Show that the abelian groups $A_n=\ZZ/p^n\ZZ$ form a projective system under the canonical homomorphisms. Let $\ZZ_p$ be its inverse limit. Show that $\ZZ_p$ maps surjectively on each $\ZZ/p^n\ZZ$; that $\ZZ_p$ has no divisors of $0$, and has a unique maximal ideal generated by $p$. Show that $\ZZ_p$ is factorial, with only one prime, namely $p$ itself.
      \item Next consider all ideals of $\ZZ$ as forming a directed system, by divisibility. Prove that
               \begin{equation*}
                 \invlim_{(a)} \ZZ/(a) = \prod_p \ZZ_p
               \end{equation*}
               where the limit is taken over all ideals $(a)$, and the product is taken over all primes $p$.
    \end{enumerate}
  \end{ex}
  \begin{proof}
    \begin{enumerate}[a)]
      \item We prove the statements one by one.

        \emph{1). The abelian groups $A_n$ form a projective system under the canonical homomorphisms.}

        For $n\geqslant m$, $\phi^n_m$ is the quotient of $\pi_m$ induced by $\pi_n$, where $\pi_n$ is the canonical map from $\ZZ$ to $\ZZ/p^n\ZZ$.
        \begin{displaymath}
          \xymatrix{
             \ZZ\ar[r]^-{\pi_m}\ar[d]_-{\pi_n} & \ZZ/p^m\ZZ            \\
             \ZZ/p^n\ZZ\ar[ur]_{\phi^n_m} & }
        \end{displaymath}
        By the uniqueness of quotient, $\phi^j_k\circ \phi^i_j = \phi^i_k$ for any $i \leqslant j \leqslant k$. Hence the abelian groups $A_n$ form a projective system under the canonical homomorphisms.

        \emph{2). $\ZZ_p$ maps surjectively on each $A_n$.}

        By the discuss above, $(\ZZ,\pi_n)$ is also a cone of $(A_n,\phi^n_m)$, hence there exists a unique morphism from $(\ZZ,\pi_n)$ to the inverse limit $(\ZZ_p,\phi_n)$. Consider an arbitrary commutative diagram in this morphism:
        \begin{displaymath}
          \xymatrix{
             \ZZ\ar[r]^-{\pi_n}\ar[d] & \ZZ/p^n\ZZ            \\
             \ZZ_p\ar[ur]_{\phi_n} & }
        \end{displaymath}
        Since $\pi_n$ is surjective, hence so is $\phi_n$.

        \emph{3). A $p-$adic integer $\alpha$ equals to $0$ if and only if $\phi_n(\alpha)=0$ for all $n$. }

        Let $\<\alpha\>$ be the subring of $\ZZ_p$ generated by $\alpha$. Then $(\<\alpha\>,0)$ is a cone of $(A_n,\phi^n_m)$, hence there exists a unique morohism from $(\<\alpha\>,0)$ to $(\ZZ_p,\phi_n)$. However, both $0$ and the inclusion $i$ satisfying the commutative diagrams, hence $i=0$, i.e. $\alpha=0$.
      \begin{displaymath}
        \xymatrix@R=0.5cm{
               & &         A_n\ar[dd]^{\phi^n_m}     \\
            \<\alpha\>\ar@/^/[urr]^{0}\ar@/_/[drr]_{0}\ar@<0.5ex>[r]^{i}\ar@<-0.5ex>[r]_{0}
               & \ZZ_p \ar[ur]_{\phi_n} \ar[dr]^{\phi_m}                 \\
               & &         A_m                 }
      \end{displaymath}

        \emph{4). $\ZZ_p$ has no divisors of $0$.}

        Let $\alpha,\beta$ be two non-zero $p-$adic integers, by the discuss above, when $n$ is large enough, we have $\phi_n(\alpha)\neq0$ and $\phi_n(\beta)\neq0$.
        Taking $l,m$ to be the largest integers such that $p^l,p^m$ divide $\phi_n(\alpha)$ and $\phi_n(\beta)$ respectively. then for any $N\geqslant n$, the same statements hold for $\phi_N(\alpha)$ and $\phi_N(\beta)$. Therefore, we can assuming $n$ is larger than $l+m$, and hence $\phi_n(\alpha)\phi_n(\beta)\neq0$, which implies $\alpha\beta\neq0$. That shows $\ZZ_p$ has no divisors of $0$.

        \emph{5). Any $p-$adic integer can be uniquely determined by a sequence $x_n\in A_n$ such that}
             \begin{equation*}
                x_{n+1} \equiv x_n \mod p^n
             \end{equation*}

        For any $x_n\in A_n$, since $\ZZ_p$ maps surjectively on each $A_n$, there exist a $p-$adic integer $\alpha$, such that $\phi_n(\alpha)=x_n$. To show that we can choose the same $\alpha$ satisfying $\phi_m(\alpha)=x_m$ for any $m$, it suffices to check that $\phi^n_m(x_n)=(x_m)$ for all $m,n$, which are nothing but the congruences above. The uniqueness comes from \emph{3)}.

        \emph{6). A $p-$adic integer $\alpha$, which is determined by a sequence $x_n\in A_n$, is unit if and only if $x_1\not\equiv 0 \mod p$.}

        Let $\alpha$ be a unit, then there exists a $p-$adic integer $\beta$ such that $\alpha\beta=1$. Let $\beta$ be determined by a sequence $y_n\in A_n$, then
        \begin{equation*}
          x_ny_n \equiv 1\mod p^n
        \end{equation*}
        In particular, $x_1y_1\equiv 1\mod p$, hence $x_1\not\equiv 0 \mod p$.

        Conversely, let $x_1\not\equiv 0\mod p$, then it is easy to show that
        \begin{equation*}
          x_n\equiv x_{n-1}\equiv \cdots \equiv x_1 \not\equiv 0 \mod p
        \end{equation*}

        Therefore, for any $n$ we can find a $y_n$ satisfying $x_ny_n \equiv 1\mod p^n$.
        Since $x_{n+1}\equiv x_n\mod p^n$ and $x_{n+1}y_{n+1}\equiv x_ny_n \mod p^n$, then also $y_{n+1}\equiv y_n\mod p^n$. This means the sequence $y_n\in A_n$ determined a $p-$adic integer $\beta$, and by the congruences above, $\alpha\beta=1$.

        \emph{7). $\ZZ_p$ has a unique maximal ideal generated by $p$.}

        By \emph{6)}, any $\alpha\not\in p\ZZ_p$ is a unit, hence $p\ZZ_p$ is the unique maximal ideal.

        \emph{8). $\ZZ_p$ is factorial, with only one prime, namely $p$ itself.}

        Let $\alpha$ be an arbitrary non-zero $p-$adic integer, we need to show that $\alpha$ can be uniquely written in the form $p^m\varepsilon$, where $\varepsilon$ is a unit.

        If $\alpha$ is a unit, then the statement is true. Assume not, let $\alpha$ be determined by $x_n\in A_n$.
        Then, by \emph{6)}, $x_1\equiv0\mod p$. Since $\alpha\neq0$, by \emph{3)}, the congruences $x_n\equiv0\mod p^n$ can not hold for all $n$.
        Let $m+1$ be the smallest index for which
        \begin{equation*}
          x_{m+1}\not\equiv0\mod p^{m+1}
        \end{equation*}
        Then, for $s>0$
        \begin{equation*}
          x_{m+s}\equiv x_m \mod p^m
        \end{equation*}
        and therefore $y_s=x_{m+s}/p^m$ is an integer.
        From the congruences
        \begin{equation*}
          p^my_{s+1}-p^my_s=x_{m+s+1}-x_{m+s}\equiv0 \mod p^{m+s}
        \end{equation*}
        it follows that
        \begin{equation*}
          y_{s+1}\equiv y_s \mod p^s
        \end{equation*}
        for all $s>0$. Thus the sequence $y_s$ determine a $p-$adic integer $\varepsilon$. Since $y_1=x_{m+1}/p^m\not\equiv0\mod p$, by \emph{6)}, $\varepsilon$ is a unit.
        Finally, from
        \begin{equation*}
          p^my_s= x_{m+s}\equiv x_s \mod p^s
        \end{equation*}
        it follows that $p^m\varepsilon=\alpha$ as desired.

        We assume now $\alpha$ can be also written as $\alpha=p^k\eta$, where $\eta$ is a unit. Let $\eta$ be determined by $z_n\in A_n$, then
        \begin{equation*}
          p^my_s\equiv p^kz_s \mod p^s
        \end{equation*}
        for all $s>0$, and by \emph{6)}, $p$ never divides $y_s$ or $z_s$ since $\varepsilon,\eta$ are unit.
        From the congruence
        \begin{equation*}
          p^kz_{m+1}\equiv p^my_{m+1}\not\equiv0 \mod p^{m+1}
        \end{equation*}
        it follows that $k\leqslant m$. By symmetry, we have $m\leqslant k$, hence $k=m$.
        Therefore
        \begin{equation*}
          p^m\varepsilon=\alpha=p^m\eta
        \end{equation*}
        which implies $\varepsilon=\eta$ by \emph{4)}.

      \item First, we consider the structure of the inverse system of $\ZZ/(a)$.
               For any ideals $(a),(b)$ of $\ZZ$, there exist a homomorphism $\phi^a_b\colon\ZZ/(a)\To\ZZ/(b)$ in this system if and only if it is induced from $\pi_b$ by $\pi_a$:
        \begin{displaymath}
          \xymatrix{
             \ZZ\ar[r]^-{\pi_b}\ar[d]_-{\pi_a} & \ZZ/(b)            \\
             \ZZ/(a)\ar[ur]_{\phi^a_b} & }
        \end{displaymath}
               if and only if $(a)\subset\ker\pi_b$, if and only if $b\mid a$.

               To prove our statement, it suffices to check that $\prod\limits_p \ZZ_p$ satisfying the universal property of $\invlim \ZZ/(a)$. Therefore, we need to show the existence and uniqueness of $t$ for any $T$ satisfying the following commutative diagram:
        \begin{displaymath}
          \xymatrix@R=0.5cm{
               & &         \ZZ/(a)\ar[dd]^{\phi^a_b}     \\
            T\ar@/^/[urr]\ar@/_/[drr]\ar@{-->}[r]^-{t}
               & \prod_p\ZZ_p \ar[ur]_{\phi_a} \ar[dr]^{\phi_b}                 \\
               & &         \ZZ/(b)                 }
        \end{displaymath}

               Let $a=p_1^{n_1}\cdots p_s^{n_s}, b=p_1^{m_1}\cdots p_s^{m_s}$ where $p_1,p_2,\cdots,p_s$ are distinct primes and $n_i\geqslant m_i$ for all $1\leqslant i\leqslant s$.
               Then $(a)=\bigcap(p_i^{n_i})$ and $(b)=\bigcap(p_i^{m_i})$.

               Hence, by the Chinese Remainder Theorem,
               \begin{equation*}
                 \ZZ/(a)=\prod_{i}\ZZ/(p_i^{n_i}) \qquad \ZZ/(b)=\prod_{i}\ZZ/(p_i^{m_i})
               \end{equation*}

               Hence the diagram above is just the product of the following commutative diagrams:
        \begin{displaymath}
          \xymatrix@R=0.5cm{
               & &         \ZZ/(p_i^{n_i})\ar[dd]^{\phi^{n_i}_{m_i}}     \\
            T\ar@/^/[urr]\ar@/_/[drr]\ar@{-->}[r]^-{t_{p_i}}
               & \ZZ_{p_i} \ar[ur]_{\phi_{n_i}} \ar[dr]^{\phi_{m_i}}                 \\
               & &         \ZZ/(p_i^{m_i})                 }
        \end{displaymath}
               Hence $t$ is the product of $t_p$, where each $t_p$ uniquely exists and the product is taken over all primes $p$. Thus $t$ uniquely exists.
    \end{enumerate}
  \end{proof}

\subsection{Mittag-Leffler Condition}
  \begin{defn}
    A \emph{\red morphism} between inverse system is just a natural transformation between them.
  \end{defn}

  \begin{defn}
    In an abelian category $\Cc$, a sequence of inverse systems indexed by the same index $I$
    \begin{equation*}
      0 \To (A_i) \To (B_i) \To (C_i) \To 0
    \end{equation*}
    is said to be \termin[exact]{exact!sequence} if the corresponding sequence of abelian groups is exact for each $i$.
  \end{defn}

  \begin{rem}
     $\Cc^I$ is also abelian, the definition for exact sequence of $I-$system coincide with concept of exact sequence in the abelian category $\Cc^I$.
  \end{rem}

  \begin{prop}
    Taking inverse limit is a left exact functor $\Cc^I\To\Cc$.
  \end{prop}
  \begin{proof}
    Draw a commutative diagram, then the it is trivial to verify.
  \end{proof}
  \begin{warn}
    $\invlim$ is, in general case, not right exact, since you can't descent the limit cokernel to the termwise cokernels.
  \end{warn}

  If $I$ is ordered (not simply partially ordered) and countable, and $\Cc$ is the category $\Ab$ of abelian groups, the \emph{Mittag-Leffler condition} is a condition on the transition morphisms $\phi^i_j$ that ensures the exactness of $\invlim$.

  Specifically, Eilenberg constructed a functor
  \begin{equation*}
    \invlim\nolimits^1\colon\Ab^I\To\Ab
  \end{equation*}
  (pronounced ``lim one'') such that if $(A_i)$, $(B_i)$, and $(C_i)$ are three projective systems of abelian groups, and
    \begin{equation*}
      0 \To (A_i) \To (B_i) \To (C_i) \To 0
    \end{equation*}
  is a short exact sequence of inverse systems, then
    \begin{equation*}
      0 \To \invlim A_i \To \invlim B_i \To \invlim C_i \To \invlim\nolimits^1 A_i
    \end{equation*}
  is an exact sequence in $\Ab$.

  \begin{defn}
    If the ranges of the morphisms of the inverse system of sets $(A_n, \phi^m_n)$ are \emph{stationary},
    that is, for every $n$ there exists $m \geqslant n$ such that for all $l \geqslant m$, $\im \phi^l_n=\im \phi^m_n$ one says that the system satisfies the \termin{Mittag-Leffler condition}
    \footnote{\emph{Why the Mittag-Leffler condition is so named?} This condition comes from Bourbaki's Algebra \cite{bourbaki1998algebra}, but the name first appear in Bourbaki's General Topology, Chapter II, section 3.5 \cite{bourbaki1998general}. The main theorem is attributed to Mittag-Leffler, and is concerned with inverse systems of ``complete Hausdorff uniform spaces''.
    The Mittag Leffler condition mentioned there says the functions in the system have dense image. The usual theorem about inverse limits is a corollary, for sets with the ``discrete uniformity''.
    Classical Mittag-Leffler is given as an example of the main theorem. The spaces there are essentially holomorphic functions on balls centred at $0$, continuous on the boundary, with the uniform metric.
    --- \url{http://mathoverflow.net/questions/14717/mittag-leffler-condition-whats-the-origin-of-its-name}}.
     Denote $\ML$ to simplify the expressed.
  \end{defn}

  In fact, by the commutative diagram below, ($l\geqslant m\geqslant n$)
        \begin{displaymath}
          \xymatrix{
               A_l\ar[r]^{\phi^l_m}\ar[dr]_{\phi^l_n} & A_m\ar[d]^{\phi^m_n} \\
               & A_n                 }
        \end{displaymath}
  It is clear that $\im \phi^l_n \subset \im \phi^m_n$ is always true, hence form a \emph{descending chain}:
  \begin{equation*}
    A_n=\im \phi^n_n \supset \im \phi^{n+1}_n \supset \cdots \supset \im \phi^m_n \supset \cdots
  \end{equation*}
  Then $\ML$ says nothing but the chains are finite for all $n$.

  Suppose $(A_n)$ satisfying $\ML$ and each $A_n$ is non-trivial. Let $A'_n=\bigcap\limits_{m\geqslant n}\im \phi^m_n$.
  Then $(A'_n)$ is also an inverse system with non-trivial terms and $\invlim A'_n \cong \invlim A_n$. Moreover, each $\phi^m_n\colon A'_m\To A'_n$ is surjective.
  Hence $\invlim A'_n$ must be non-trivial.

  \begin{exam}
    The following situations are examples satisfying $\ML$:
    \begin{itemize}
      \item a system in which the morphisms $\phi^m_n$ are surjective.
      \item a system of finite-dimensional vector spaces.
      \item a system of finite-length $A-$modules.
    \end{itemize}
  \end{exam}

  Let's return to consider the inverse system of \textbf{abelian groups}, here is an important theorem:
  \begin{thm}
    If $(A_i)$ satisfying $\ML$, then $\invlim^1 A_i=0$.
  \end{thm}
  \begin{proof}
    Consider an arbitrary exact sequence:
    \begin{equation*}
      0 \To (A_i) \markar{f_i} (B_i) \markar{g_i} (C_i) \To 0
    \end{equation*}
    It suffices to show that $\invlim g_i$ is surjective.

    For any $c\in\invlim C_i$, let $(c_i)$ be its corresponding sequence.
    By the short exact sequence of $i-$terms, each $D_i=g_i^{-1}(c_i)$ is non-trivial, moreover, a coset of $A_i$ in $B_i$.
    Therefore $(D_i)$ also form an inverse system (of sets) satisfying $\ML$. Hence $\invlim D_i$ is non-trivial.
    But any element of $\invlim D_i$ lies in $\invlim B_i$ and is mapped to $c$, this proves $\invlim g_i$ is surjective as desired.
  \end{proof}

  \begin{exam}
    Taking $I$ to be the non-negative integers, letting $A_i = p^i\ZZ, B_i = \ZZ$, and $C_i = \ZZ/p^i\ZZ$. Then
    \begin{equation*}
      \invlim\nolimits^1 A_i = \ZZ_p/\ZZ
    \end{equation*}
    This shows that $(p^i\ZZ)$ dissatisfies $\ML$, in fact, $\invlim p^i\ZZ=0$.
  \end{exam}

  \begin{prop}
    If $(A_i)$ satisfying $\ML$, then we have an exact sequence
    \begin{equation*}
      0 \To \invlim A_i \To \prod A_i \markar{1-\phi} \prod A_i \To 0
    \end{equation*}
    where the map $\phi$ is the product of $\phi^i_{i-1}$.
  \end{prop}
  \begin{proof}
    For any $N$ large enough, the sequence below is exact:
    \begin{equation*}
      0 \To \invlim_{1\leqslant i\leqslant N} A_i \To \prod_{i=1}^N A_i \markar{1-\phi} \prod_{i=1}^N A_i \To 0
    \end{equation*}
    Moreover, the left terms satisfies $\ML$ since $(A_i)$ satisfies. Hence the sequence in proposition is exact.
  \end{proof}

  \begin{ex}
    \begin{enumerate}[a)]
      \item Let $(A_n)$ be an inverse system of commutative rings, and let $(M_n)$ be an inverse system of modules, each $M_n$ is a module over $A_n$ such that the following diagram is commutative:
          \begin{displaymath}
            \xymatrix@1{
               A_{n+1}\ar@{}[r]|{\times}\ar[d] & M_{n+1}\ar[r]\ar[d] & M_{n+1}\ar[d] \\
               A_n\ar@{}[r]|{\times} & M_n\ar[r] & M_n                 }
          \end{displaymath}
          Show that $\invlim M_n$ is a module over $\invlim A_n$.
      \item Let $M$ be a $p-$divisible group. Show that $T_p(M)$ is a module over $\ZZ_p$.
      \item Let $M,N$ be $p-$divisible groups. Show that $T_p(M\oplus N) = T_p(M) \oplus T_p(N)$, as modules over $\ZZ_p$.
    \end{enumerate}
  \end{ex}
  \begin{proof}
    \begin{enumerate}[a)]
      \item It is clear.
      \item $M[p^n]$ is a module over $\ZZ/p^n\ZZ$. Then by \emph{a)}, $T_p(M)=\invlim M[p^n]$ is a module over $\ZZ_p$.
      \item It is easy to check that $M \oplus N [p^n] = M[p^n] \oplus N[p^n]$. Then $T_p(M\oplus N) = T_p(M) \oplus T_p(N)$.
    \end{enumerate}
  \end{proof}

\newpage\section{Direct Limit}

  The direct limit of the direct system $(A_i,\phi^i_j)$ is can be constructed by the coproduct of the $A_i$ modulo a certain subobject:
  \begin{equation*}
    \dirlim A_i = \left.\coprod A_i \right/N
  \end{equation*}
  Here, $N$ is generated by all $x^i_j$, $i\leqslant j$, defined below:

  For any $x_i\in A_i$, $x^i_j=\imath(x_i)-\imath(\phi^i_j(x_i))$, where $\imath$ is the canonical injection $A_i\injection \coprod A_i$.

  Hence, if $x_i\in A_i$ and $x_j\in A_j$, then $\imath(x_i) = \imath(x_j)$ if there is some $k$ such that $\phi^i_k(x_i)=\phi^j_k(x_j)$.
  Heuristically, two elements in the coproduct are equivalent if and only if they ``eventually become equal'' in the direct system.
  \begin{ex}\label{3.ex19}
    Let $(A_i, \phi^i_j)$ be a directed system of modules. Let $a_k\in A_k$ for some $k$, and suppose that the image of $a_k$ in the direct limit $A$ is $0$. Show that there exists some index $j\geqslant k$ such that $\phi^k_j(a_k) = 0$.
  \end{ex}

  \begin{ex}
    Let $I,J$ be two directed sets, and give the product $I\times J$ as the product of categories. Let $A_{ij}$ be a $I\times J-$system of abelian groups.
    Show that the direct limits
    \begin{equation*}
      \dirlim_i\dirlim_j A_{ij} \qquad \text{and} \qquad \dirlim_j\dirlim_i A_{ij}
    \end{equation*}
    are naturally isomorphic. State and prove the same result for inverse limits.
  \end{ex}
  \begin{proof}
    This follows immediately from Proposition \ref{product and power}, i.e.
    \begin{equation*}
      \Ab^{I\times J}\simeq(\Ab^J)^I\simeq(\Ab^I)^J
    \end{equation*}
  \end{proof}

  \begin{ex}
    The functor $\dirlim\colon\Mod_R^I\To\Mod_R$ is exact.
  \end{ex}
  \begin{proof}
    Consider an arbitrary exact sequence:
    \begin{equation*}
      0 \To (M'_i) \markar{f_i} (M_i) \markar{g_i} (M''_i) \To 0
    \end{equation*}
    Only the injectivity of $f=\dirlim f_i$ is non-trivial.

    Assume $f(x)=0$, we need to show $x=0$. Let $(x_i)$ be the corresponding sequence of $x$,
    Then, the image of $f_i(x_i)$ in $\dirlim M_i$ is $0$, by Exercise \ref{3.ex19}, there exists some index $j\geqslant i$ such that $\phi^i_j(f_i(x_i)) = 0$.
    Therefore $f_j(\phi^i_j(x_i)) = 0$. But $f_j$ is injective, hence $\phi^i_j(x_i)=0$. Therefore $\imath(x_i)=\imath(x_i)-\imath(\phi^i_j(x_i))\in N$, which shows $x=0$ as desired.
  \end{proof}

  \begin{ex}
    \begin{enumerate}[a)]
      \item Let $\{M_i\}$ be a family of modules. For any module $N$ show that
                \begin{equation*}
                  \Hom(\bigoplus M_i,N) = \prod\Hom(M_i,N)
                \end{equation*}
      \item Show that
               \begin{equation*}
                 \Hom(N,\prod M_i) = \prod\Hom(N,M_i)
               \end{equation*}
    \end{enumerate}
  \end{ex}
  \begin{proof}
    \emph{a)} is a special case of Proposition \ref{Hom-dir.lim}, while \emph{b)} is a special case of Proposition \ref{Hom-inv.lim}, here $\Jj$ is the discrete index set of $\{M_i\}$.
  \end{proof}

  \begin{ex}
    Let $(M_i)$ be an inverse system of modules. For any module $N$ show that
    \begin{equation*}
      \Hom(N,\invlim M_i) = \invlim\Hom(N,M_i)
    \end{equation*}
  \end{ex}
  \begin{proof}
    This is again a special case of Proposition \ref{Hom-inv.lim}, here $\Jj$ is the index set (which is a FPOS) of $(M_i)$.
  \end{proof}

  \begin{ex}\label{3.ex24}
    Show that any module is a direct limit of finitely generated submodules.
  \end{ex}
  \begin{proof}
    The finitely generated submodules of a module $M$ form a directed system under the order of containing.
    Then $M$ induces a cocone of this system by the inclusions, hence there exist a unique homomorphism from $M$ to the direct limit, say $f$.

    $f$ is injective: for $f(x)=0$, there must be a finitely generated submodule $N$ such that $x\in N$ and hence its image in the direct limit is $0$. By Exercise \ref{3.ex19}, there exist a $N'\supset N$, such that $\phi^N_{N'}(x)=0$, which means $x=0$.

    $f$ is surjective: for any $a$ in the direct limit, let $x$ be it's preimage, then by the definitions, $f(x)=a$.
  \end{proof}

  \begin{defn}
    A module $M$ is called \termin{finitely presented} if there is an exact sequence
    \begin{equation*}
      F_1\To F_0\To M\To 0
    \end{equation*}
    where $F_0,F_1$ are free with finite basis. The image of $F_1$ in $F_0$ is said to be the submodule of \termin{relations}, among the free basis elements of $F_0$.
  \end{defn}
  \begin{rem}
    Any module is \emph{presented}, in the sense that there is an exact sequence
    \begin{equation*}
      F_1\To F_0\To M\To 0
    \end{equation*}
    where $F_0,F_1$ are free.

    First, for any module $M$ is a quotient of a free module $F_0$, hence we have the an exact sequence
    \begin{equation*}
      0\To R\To F_0\To M\To 0
    \end{equation*}
    Then $R$ is also a quotient of a free module $F_1$, hence we get an epimorphism $F_1\To R$.
    By Lemma \ref{factor through}, we obtain the require exact sequence.
  \end{rem}

  \begin{prop}
    A module $M$ is finitely presented if and only if there is an exact sequence
    \begin{equation*}
      0\To R\To F\To M\To 0
    \end{equation*}
    where $F$ is free with finite basis, $R$ is finitely generated.
  \end{prop}

  \begin{ex}
    Show that any module is a direct limit of finitely presented modules (not necessarily submodules). In other words, given $M$, there exists a directed system $(M_i)$ with $M_i$ finitely presented for all $i$ such that
    \begin{equation*}
      M\cong\dirlim M_i
    \end{equation*}
  \end{ex}
  \begin{proof}
    First, we have an exact sequence
    \begin{equation*}
      0\To R\To F\To M\To 0
    \end{equation*}
    By Exercise \ref{3.ex24}, $R$ is the direct limit of its finitely generated submodules. Denote this directed system by $(R_i)$, then it is clear that $R\To F$ is injective. Let $M_i=F/R_i$, then we have a family of exact sequence
    \begin{equation*}
      0\To R_i\To F\To M_i\To 0
    \end{equation*}
    where each $M_i$ is finitely presented. The connection homomorphisms in $(R_i)$ naturally induce directed systems for $F$ and $\{M_i\}$ of the same type with $(R_i)$. Therefore we have an exact sequence
    \begin{equation*}
      0\To (R_i)\To (F)\To (M_i)\To 0
    \end{equation*}
    Since $\dirlim$ is exact, we obtain an exact sequence of the limits:
    \begin{equation*}
      0\To \dirlim(R_i)\To \dirlim(F)\To \dirlim(M_i)\To 0
    \end{equation*}
    where $\dirlim(R_i)=R, \dirlim(F)=F$. Compare it with the first exact sequence, we have $M=\dirlim(M_i)$ as desired.
  \end{proof}

  \begin{ex}
    Let $E$ be a module. Let $(M_i)$ be a directed system of modules. If $E$ is finitely generated, show that the natural homomorphism
    \begin{equation*}
      \dirlim \Hom(E,M_i) \To \Hom(E,\dirlim M_i)
    \end{equation*}
    is injective. If $E$ is finitely presented, show that this homomorphism is an isomorphism.
  \end{ex}
  \begin{proof}
    We finish the proof step by step.
    \begin{enumerate}
      \item Any connection homomorphism $\phi^i_j$ in $(M_i)$ induces a homomorphism
                 \longmapdes{\Phi^i_j}{\Hom(E,M_i)}{\Hom(E,M_j)}{f}{\phi^i_j\circ f}
                 Thus $(\Hom(E,M_i),\Phi^i_j)$ form a directed system of the same type with $(M_i)$.

                 Any homomorphism $\phi^i\colon M_i\To\dirlim M_i$ induces a homomorphism
                 \longmapdes{\Phi^i}{\Hom(E,M_i)}{\Hom(E,\dirlim M_j)}{f}{\phi^i\circ f}

                 Then for any $\phi^i=\phi^j\circ\phi^i_j$, $\Phi^i=\Phi^j\circ\Phi^i_j$. Therefore $(\Hom(E,\dirlim M_i),\Phi^i)$ is a cocone of $(\Hom(E,M_i),\Phi^i_j)$. Hence there exist a unique morphism
                 \begin{equation*}
                    \dirlim \Hom(E,M_i) \To \Hom(E,\dirlim M_i)
                 \end{equation*}

                 We denoted the natural homomorphism by $\Psi$.

                 Notice that for any $i,j\in I$, there exist $k\in I$ such that $i\leqslant k,j\leqslant k$,
      \item \emph{If $E$ is free with finite basis, then $\Psi$ is an isomorphism.}

                 Let $b_1,\cdots,b_n$ be a basis of $E$. We check that $\Hom(E,\dirlim M_i)$ satisfies the universal property of $\dirlim \Hom(E,M_i)$.

                 For any cocone $(S,g_i)$, we need to show there exist a unique morphism $\Hom(E,\dirlim M_i)\To S$.

                 For any $f\in\Hom(E,\dirlim M_i)$, consider the preimage $x_1,\cdots,x_n$ of $b_1,\cdots,b_n$ in $M_i$, then the map $b_j\mapsto x_j$ induces a unique homomorphism $E\To M_i$.
                 In this way, we get a map $\Hom(E,\dirlim M_i)\To \Hom(E,M_i)$ hence a map $\Hom(E,\dirlim M_i)\To S$. The commutativity is easy to check.

                 Assume there exist two morphisms
                 \begin{equation*}
                   h_1,h_2\colon\Hom(E,\dirlim M_i)\To S
                 \end{equation*}

                 For any $f\in\Hom(E,\dirlim M_i)$, we need to show $h_1(f)=h_2(f)$.

                 Consider $f_i\in g_i^{-1}(h_1(f)), f_i'\in g_i^{-1}(h_2(f))$. For $1\leqslant j\leqslant n$, there exists $k_j\geqslant i$ such that $\phi^i_{k_j}(f_i(b_j))=\phi^i_{k_j}(f_i'(b_j))$. Therefore there exists $k\geqslant i$ such that $\phi^i_{k}(f_i(b_j))=\phi^i_{k}(f_i'(b_j))$ for any $1\leqslant j\leqslant n$.
                 But the map $b_j\mapsto\phi^i_{k}(f_i(b_j))$ induces a unique homomorphism $f_k\in\Hom(E,M_k)$, hence
                 $h_1(f)=g_i(f_i)=g_k(f_k)=g_i(f_i')=h_2(f)$.
      \item \emph{If $E$ is finitely presented by an exact sequence
                 \begin{equation*}
                   F_1\To F_0\To M\To 0
                 \end{equation*}
                 then $\Psi$ is an isomorphism.}

                 Consider the following commutative diagram with exact rows:
                 \begin{displaymath}
                   \xymatrix@1{
                      0\ar[r] & \dirlim \Hom(E,M_i)\ar[r]\ar[d]_{\Psi} & \dirlim \Hom(F_0,M_i)\ar[r]\ar[d]_{\Psi_0} & \dirlim \Hom(F_1,M_i)\ar[d]_{\Psi_1} \\
                      0\ar[r] & \Hom(E,\dirlim M_i)\ar[r] & \Hom(F_0,\dirlim M_i)\ar[r] & \Hom(F_1,\dirlim M_i)              }
                 \end{displaymath}
                 Here $\Psi_0,\Psi_1$ are isomorphisms by 2.
                 Hence, by Five Lemma, $\Psi$ is an isomorphism.
      \item \emph{If $E$ is finitely generated by an epimorphism
                 \begin{equation*}
                   F\To M\To 0
                 \end{equation*}
                 then $\Psi$ is injective.}

                 Consider the following commutative diagram with exact rows:
                 \begin{displaymath}
                   \xymatrix@1{
                      0\ar[r] & \dirlim \Hom(E,M_i)\ar[r]\ar[d]_{\Psi} & \dirlim \Hom(F,M_i)\ar[d]_{\Psi_0}     \\
                      0\ar[r] & \Hom(E,\dirlim M_i)\ar[r] & \Hom(F,\dirlim M_i)              }
                 \end{displaymath}
                 Here $\Psi_0$ is an isomorphisms by 2.
                 By $\triangle$ Lemma, $\Psi$ is injective.
    \end{enumerate}
  \end{proof}

\newpage\section{Graded Algebras}

  \begin{defn}
    Let $A$ be an algebra over a field $k$. By a \termin{filtration} of $A$ we mean a sequence of $k-$vector spaces $A_i (i=0,1,\cdots)$ such that
    \begin{equation*}
      A_0\subset A_1\subset A_2\subset \cdots \qquad\quad \text{and} \qquad\qquad \bigcup A_i=A
    \end{equation*}
    and $A_iA_j\subset A_{i+j}$ for all $i,j\geqslant 0$, In particular, $A$ is an $A_0-$algebra. We then call $A$ a \termin{filtered algebra}. Let $R$ be an algebra. We say that $R$ is \termin[graded]{graded!algebra} if $R$ is a direct sum $R=\bigoplus R_i$ of subspaces such that $R_iR_j\subset R_{i+j}$ for all $i,j\geqslant0$.
  \end{defn}

  \begin{ex}
    Let $A$ be filtered algebra. Define $R_i$ for $i\geqslant0$ by $R_i=A_i/A_{i-1}$. By definition, $A_{-1}=\{0\}$. Let $R=\bigoplus R_i$, and $\gr_i(A)=R_i$. Define a natural product on $R$ making $R$ into a graded algebra, denoted by $\gr(A)$, and called the \termin{associated graded algebra}.
  \end{ex}
  \begin{proof}
    The multiplication $\gr(A)\times\gr(A)\To\gr(A)$ is combined from the natural maps
    \mapdes{A_i/A_{i-1}\times A_j/A_{j-1}}{A_{i+j}/A_{i+j-1}}{(x+A_{i-1},y+A_{j-1})}{xy+A_{i+j-1}}
    The multiplication is well defined and endows $\gr(A)$ with the structure of a graded algebra, with gradation $\gr_i(A)$.
  \end{proof}

  \begin{ex}
    Let $A,B$ be filtered algebras, $A=\bigcup A_i$ and $B=\bigcup B_i$. Let $L\colon A\To B$ be an $(A_0,B_0)-$linear map preserving the filtration, that is $L(A_i)\subset B_i$ for all $i$, and $L(ca)=L(c)L(a)$ for $c\in A_0$ and $a\in A_i$ for all $i$.
    \begin{enumerate}[a)]
      \item Show that $L$ induces an $(A_0,B_0)-$linear map
                 \begin{equation*}
                   \gr_i(L)\colon \gr_i(A)\To\gr_i(B) \qquad \text{for all }i.
                 \end{equation*}
      \item Suppose that $\gr_i(L)$ is an isomorphism for all $i$. Show that $L$ is an $(A_0,B_0)-$isomorphism.
    \end{enumerate}
  \end{ex}
  \begin{proof}
    Consider the following $(A_0,B_0)-$linear map
    \begin{equation*}
      A_i\markar{L}B_i\markar{\pi}B_i/B_{i-1}
    \end{equation*}
    its kernel contain $A_{i-1}$ since $L(A_{i-1})\subset B_{i-1}$, therefore it induces an quotient map $A_i/A_{i-1}\To B_i/B_{i-1}$.

    Consider the following commutative diagrams and using $5-$lemma, by induction on $i$, it is clear that if $\gr_i(L)$ are isomorphisms for all $i$, then $L$ is an $(A_0,B_0)-$isomorphism.
                 \begin{displaymath}
                   \xymatrix@1{
                      0\ar[r] & A_{i-1}\ar[r]\ar[d]_{L} & A_i\ar[r]\ar[d]_{L} & \gr_i(A)\ar[r]\ar[d]_{\gr_i(L)} & 0     \\
                      0\ar[r] & B_{i-1}\ar[r] & B_i\ar[r] & \gr_i(B)\ar[r] & 0            }
                 \end{displaymath}
  \end{proof}

  \begin{ex}
    Suppose $k$ has characteristic $0$. Let $\nn$ be the set of all strictly upper triangular matrices of a given size $n\times n$ over $k$.
    \begin{enumerate}[a)]
      \item For a given matrix $X\in\nn$, let $D_1(X),\cdots,D_n(X)$ be its diagonals, so $D_1=D_1(X)$ is the main diagonal, and is $0$ by the definition of $\nn$. Let $\nn_i$ be the subset of $\nn$ consisting of those matrices whose diagonals $D_1,\cdots,D_{n-i}$ are $0$. Thus $\nn_0=\{0\}$, $\nn_1$ consists of all matrices whose components are $0$ except possibly for $x_{1n}$; $\nn_2$ consists of all matrices whose components are $0$ except possibly those in the last two diagonals; and so forth. Show that each $\nn_i$ is an algebra, and its elements are nilpotent (in fact the $(i+1)-$th power of its elements is $0$).
      \item Let $U$ be the set of elements $I+X$ with $X\in\nn$. Show that $U$ is a multiplicative group.
      \item Let $\exp$ be the exponential series defined as usual. Show that $\exp$ defines a polynomial function on $\nn$ (all but a finite number of terms are $0$ when evaluated on a nilpotent matrix), and establishes a bijection
          \begin{equation*}
            \exp\colon\nn\To U
          \end{equation*}
          Show that the inverse is given by the standard $\log$ series.
    \end{enumerate}
  \end{ex}
  \begin{proof}
    It is clear that $xy\in\nn_i$ for any $x,y\in\nn_i$, therefore $\nn_i$ is a subalgebra of $\nn$. Moreover, $xy\in\nn_{i-1}$, therefore the $(i+1)-$th power of $\nn_i$'s elements is $0$.

    The identity of $U$ is $I$, for any $I+X\in U$, its inverse is $\sum(-X)^i$.

    \begin{align*}
      \exp(X) & = \sum_{k=0}^{\infty} \frac{X^k}{k!} \\
      \log(X) & = - \sum_{k=1}^{\infty} \frac{(I-X)^k}{k}
    \end{align*}
  \end{proof}

\newpage\section{Some Obvious Module Structure}
\begin{prop}
  Let $R,S$ be rings, $ _RA_S, _RB_S$ are double modules. Then
  \begin{enumerate}
    \item $\Hom_R(A,-)$ is a covatiant functor from $ _R\Mod$ to $ _S\Mod$. The left $S-$module structure is given by $(sf)a=f(as)$.
    \item $\Hom_R(-,B)$ is a contravariant functor from $ _R\Mod$ to $\Mod_S$. The right $S-$module structure is given by $(fs)a=(f(a))s$.
  \end{enumerate}
\end{prop}

  Let $A$ be a $R-$module, then $A\cong\Hom_R(R,A)$. $A^{\ast}\defeq\Hom_R(A,R)$ is called the
  \termin[dual module]{dual!module} of $A$. If $A\cong A^{\ast\ast}$, then we say it is \termin{reflexive}.

\begin{prop}
  Let $R,S$ be rings, $ _SA_R, _RB_S$ are double modules. Then
  \begin{enumerate}
    \item $A\otimes_R-$ is a covatiant functor from $ _R\Mod$ to $ _S\Mod$.
    \item $-\otimes_RB$ is a covariant functor from $\Mod_R$ to $\Mod_S$.
  \end{enumerate}
\end{prop}
