\chapter{Galois Theory}
\cite{morandi1996field}
\section{Galois Extensions}
  \begin{defn}
    Let $K$ be a field and let $G$ be a group of automorphisms of $K$. The subset of $K$ consisting of all elements which is fixed under all $\sigma\in G$ is a field. It is called the \termin{fixed field} of $G$, and denoted by $K^G$.
  \end{defn}

  \begin{defn}
    An algebraic extension $K$ of a field $k$ is called \termin[Galois]{Galois Extension} if it is both normal and separable.
  \end{defn}

  \begin{defn}
    For an extension $K/k$, the group of automorphisms of $K$ over $k$ is denoted by $\Aut(K/k)$. When the extension is Galois, $\Aut(K/k)$ is called the \termin{Galois group} of $K$ over $k$, and is denoted by $\Gal(K/k)$ or $G(K/k)$.
  \end{defn}

  The main theorem in this section is
  \begin{thm}[Galois Connection]
    Let $K$ be a Galois extension of $k$, with Galois group $G$. Denote the set of intermediate field of $K/k$ by $\Int(K/k)$, and the set of subgroups of $G$ by $\Sub(G)$. Then there exists a bijection between them:
    \mapdes{\Sub(G)}{\Int(K/k)}{H}{E=K^H}

    The field $E$ is Galois if and only if $H$ is normal in $G$. If that is the case, then the map $\sigma\mapsto\local{\sigma}{E}$ induces an isomorphism of $G/H$ onto the Galois group of $E$ over $k$.
  \end{thm}

  Lang gives the proofs step by step.

  \begin{thm}
    Let $K$ be a Galois extension of $k$, with Galois group $G$, then $k=K^G$. Denote the set of intermediate field of $K/k$ by $\Int(K/k)$, and the set of subgroups of $G$ by $\Sub(G)$. If $F\in\Int(K/k)$, then $K$ is Galois over $F$. The map
    \mapdes{\Int(K/k)}{\Sub(G)}{F}{G(K/F)}
    is injective.
  \end{thm}

  \begin{defn}
    We say a subgroup $H$ of $G$ \emph{\red belongs} to an intermediate field $F$ if $H=G(K/F)$.
  \end{defn}

  \begin{cor}
    Let $K/k$ be Galois with group $G$. Let $F,F'$ be two intermediate fields, and let $H,H'$ be the subgroups of $G$ belonging to $F,F'$ respectively. Then
    \begin{enumerate}[a)]
      \item  $H\cap H'$ belongs to $FF'$;
      \item  The fixed field of the smallest subgroup of $G$ containing $H,H'$ is $F\cap F'$;
      \item  $F\subset F'$ if and only if $H'\subset H$.
    \end{enumerate}
  \end{cor}
  Such results can be represented by the corresponding between the following two \emph{Hasse diagrams}
                 \begin{displaymath}
                   \xymatrix@!0{
                      & <H\cup H'> &\ar[ddrr]&&& FF' & \\
                      H\ar@{-}[ur]\ar@{-}[dr] && H'\ar@{-}[ul]\ar@{-}[dl] && F\ar@{-}[ur]\ar@{-}[dr] && F'\ar@{-}[ul]\ar@{-}[dl] \\
                      & H\cap H' &\ar[uurr]&&& F\cap F' &            }
                 \end{displaymath}

  \begin{cor}
    Let $E$ be a finite separable extension of a field $k$. Let $K$ be the smallest normal extension of $k$ containing $E$. Then $K$ is finite Galois over $k$, and $\Int(E/k)$ is finite.
  \end{cor}

  \begin{lem}
    Let $E$ be an algebraic separable extension of $k$. Assume that there is an integer $n\geqslant1$ such that every element of $E$ is of degree $\leqslant n$ over $k$. Then $[E:k]\leqslant n$.
  \end{lem}

  Now, we have $K^{G(K/F)}=F$, but how about $G(K/K^H)$.

  \begin{thm}[Artin]
    Let $K$ be a field and let $G$ be a finite group of automorphisms of $K$, of order $n$. Let $k=K^G$ be the fixed field. Then $K$ is a finite Galois extension of $k$, and its Galois group is $G$. We have $[K:k]=n$.
  \end{thm}

  \begin{cor}
    Let $K$ be a finite Galois extension of $k$ and let $G$ be its Galois group. Then every subgroup of $G$ belongs to some $F\in\Int(K/k)$.
  \end{cor}
  \begin{warn}
    This statement is not true when $K$ is an infinite Galois extension of $k$.
  \end{warn}

  \begin{lem}
    Let $K$ be a Galois extension of $k$. Let
    \begin{equation*}
       \lambda\colon K\To \lambda K
    \end{equation*}
    be an isomorphism, then
    \begin{equation*}
      G(\lambda K/\lambda k)^{\lambda}=G(K/k)
    \end{equation*}
    i.e.
    \begin{equation*}
      G(\lambda K/\lambda k)=\lambda G(K/k) \lambda^{-1}
    \end{equation*}
  \end{lem}

  \begin{thm}
    Let $K$ be a Galois extension of $k$ with group $G$. Let $F\in\Int(K/k)$, and let $H=G(K/F)$. Then $F$ is normal over $k$ if and only if $H$ is normal in $G$. If that is the case, then the map $\sigma\mapsto\local{\sigma}{E}$ induces a homomorphism of $G$ onto the Galois group of $F$ over $k$, whose kernel is $H$.
  \end{thm}

  \begin{defn}
    A Galois extension is said to be \termin[abelian]{abelian extension} (resp. \termin[cyclic]{cyclic extension}) if its Galois group is \emph{abelian} (resp. \emph{cyclic}).
  \end{defn}

  \begin{cor}
    Let $K/k$ be abelian (resp. cyclic). If $F\in\Int(K/k)$, then $F/k$ is also abelian (resp. cyclic).
  \end{cor}

  \begin{thm}
    Let $K/k$ be Galois, $F/k$ be an arbitrary extension, then $KF/F, K/(K\cap F)$ are Galois. Moreover, we have an isomorphism
    \isodes{\Gal(KF/F)}{\Gal(K/(K\cap F))}{\sigma}{\local{\sigma}{K}}
                 \begin{displaymath}
                   \xymatrix@1{
                      && KF\ar@{-}[dr] &\\
                      K\ar@{-}[urr] &&& F\ar@{-}[dll] \\
                      & K\cap F\ar@{-}[ul] &&\\
                      & k\ar@{-}[u] &&            }
                 \end{displaymath}
  \end{thm}

  \begin{cor}
    Let $K/k$ be finite Galois, $F/k$ be an arbitrary extension. Then $[KF:F]$ divides $[K:k]$.
  \end{cor}
  \begin{warn}
    The assertion of above corollary is not usually valid if $K/k$ is not Galois.

    Indeed, let
    \begin{equation*}
      \alpha=2^{\frac{1}{3}}\quad\zeta=\frac{-1+\sqrt{-3}}{2}\quad\beta=\alpha\zeta
    \end{equation*}
    and $k=\QQ,K=\QQ(\beta),F=\QQ(\alpha)$. Then
    \begin{equation*}
      [KF:F]=2\quad [K:k]=3
    \end{equation*}
  \end{warn}

  \begin{thm}
    Let $\Gal(K_1/k)=G_1, \Gal(K_2/k)=G_2$, then $K_1K_2/k$ is Galois. Let $G$ be its Galois group, then
    \mapdes{G}{G_1\times G_2}{\sigma}{(\local{\sigma}{K_1},\local{\sigma}{K_2})}
    is injective. If $K_1\cap K_2=k$, then it is an isomorphism.
                 \begin{displaymath}
                   \xymatrix@1{
                      & K_1K_2\ar@{-}[dr] &\\
                      K_1\ar@{-}[ur] && K_2\ar@{-}[dl] \\
                      & K_1\cap K_2\ar@{-}[ul] &\\
                      & k\ar@{-}[u] &           }
                 \end{displaymath}
  \end{thm}

  \begin{cor}
    Let $K_1,\cdots,K_n$ be Galois extensions of $k$ with Galois groups $G_1,\cdots, G_n$. Assume $K_{i+1}\cap(K_1\cdots K_i)=k$, then
    \begin{equation*}
      \Gal(K_1\cdots K_n/k)\cong G_1\times\cdots\times G_n
    \end{equation*}
  \end{cor}

  \begin{cor}
    Let $K$ be finite Galois over $k$ with group $G$, and assume $G=G_1\times\cdots\times G_n$, let $K_i$ be the fixed field of
    \begin{equation*}
      G_1\times\cdots\times 1 \times\cdots\times G_n
    \end{equation*}
    then $K_i$ is Galois over $k$ and $K_{i+1}\cap(K_1\cdots K_i)=k, K=K_1\cdots K_n$.
  \end{cor}

  \begin{thm}
    Assume all fields contained in some common field.
    \begin{enumerate}[(i)]
      \item If $K,L$ are abelian over $k$, then so is $KL$;
      \item If $K/k$ is abelian and $F/k$ is arbitrary, then $KF/F$ is abelian;
      \item IF $K/k$ is abelian and $F\in\Int(K/k)$, then $K/F,F/k$ are abelian.
    \end{enumerate}
  \end{thm}

  \begin{warn}
    The converse of last statement may not be true.
  \end{warn}

  \begin{defn}
    The composite of all abelian extensions of $k$ in $k^{\ac}$ is called the \termin{abelian closure} of $k$ and denoted by $k^{\ab}$.
  \end{defn}

\newpage\section{Examples and Applications}

\newpage\section{Norm and Trace}
  Let $E/k$ be finite, $[E:k]_s=r$, $\alpha\in E$. We define the \termin{norm} and \termin{trace} of $\alpha$ to be
  \begin{align*}
    N_{E/k} &= N^E_k(\alpha) = \prod_{v=1}^{r}\sigma_v\alpha^{[E:k]_i} = \left(\prod_{v=1}^r\sigma_v\alpha\right)^{[E:k]_i} \\
    \Tr_{E/k} &= \Tr^E_k(\alpha) = [E:k]_i\sum_{v=1}^r\sigma_v\alpha
  \end{align*}

  If $E/k$ is separable, then
  \begin{align*}
    N^E_k(\alpha) & =\prod_{\sigma}\sigma\alpha \\
    \Tr^E_k(\alpha) & = \sum_{\sigma}\sigma\alpha
  \end{align*}

  \begin{thm}
    Let $E/k$ be finite, then
    \begin{enumerate}[(i)]
      \item $N^E_k(\alpha)$ is a multiplicative homomorphism of $E^{\ast}$ into $k^{\ast}$ and  $\Tr^E_k(\alpha)$ is an additive homomorphism of $E$ into $k$.
      \item If $E\supset F\supset k$ is a tower, then
                 \begin{equation*}
                   N^E_k=N^F_k\circ N^E_F\quad\text{and}\quad \Tr^E_k=\Tr^F_k\circ\Tr^E_F
                 \end{equation*}
      \item If $E=k(\alpha)$, and $f(X)=\Irr(\alpha,k,X)=X^n+a_{n-1}X^{n-1}+\cdots+a_0$, then
                 \begin{equation*}
                   N^{k(\alpha)}_k(\alpha)=(-1)^na_0\quad\text\quad\Tr^{k(\alpha)}_k(\alpha)=-a_{n-1}
                 \end{equation*}
    \end{enumerate}
  \end{thm}

  \begin{thm}
    Let $E/k$ be finite separable. Then $\Tr^E_k$ is a nonzero functional. The map
    \mapdes{E\times E}{k}{(x,y)}{\Tr(xy)}
    is bilinear, and identifies $E$ with its dual space $\codual{E}$.
  \end{thm}

  \begin{cor}
    Let $\omega_1,\cdots,\omega_n$ be a basis of $E/k$. Then there exists a basis $\omega'_1,\cdots,\omega'_n$ of $E/k$ such that $\Tr(\omega_i\omega'_j)=\delta_{ij}$.
  \end{cor}

  \begin{cor}
    Let $E/k$ be finite separable, and let $\sigma_1,\cdots,\sigma_n$ be the distinct embeddings. Let $w_1,\cdots,w_n$ be elements of $E$. Then the vectors
    \begin{gather*}
      \xi_1=(\sigma_1w_1,\cdots,\sigma_1w_n) \\
      \cdots \\
      \xi_n=(\sigma_nw_1,\cdots,\sigma_nw_n)
    \end{gather*}
    are linearly independent over $E$ if $w_1,\cdots,w_n$ form a basis of $E/k$.
  \end{cor}

  \begin{rem}
    In characteristic $0$, one sees much more trivially that the trace is not identically $0$. Indeed, if $c\in k$ and $c\neq0$, then $\Tr(c)=nc$ where $n=[E;k]$, and $n\neq0$. This argument also holds in characteristic $p$ where $n$ is prime to $p$.
  \end{rem}

  \begin{prop}
    Let $E=k(\alpha)$ be separable. Let
    \begin{equation*}
      f(X)=\Irr(\alpha,k,X)
    \end{equation*}
    and let $f'(X)$ be its derivative. Let
    \begin{equation*}
      \frac{f(X)}{(X-\alpha)}=\beta_0+\beta_1X+\cdots+\beta_{n-1}X^{n-1}
    \end{equation*}
    with $\beta_i\in E$. Then the dual basis of $1,\alpha,\cdots,\alpha^{n-1}$ is
    \begin{equation*}
      \frac{\beta_0}{f'(\alpha)},\cdots,\frac{\beta_{n-1}}{f'(\alpha)}
    \end{equation*}
  \end{prop}

  Define
  \longmapdes{m_{\alpha}}{E}{E}{x}{\alpha x}

  \begin{prop}
    Let $E/k$ be finite and let $\alpha\in E$. Then
    \begin{equation*}
      \det(m_{\alpha})=N_{E/k}(\alpha)\quad\text{and}\quad\Tr(m_{\alpha})=\Tr_{E/k}(\alpha)
    \end{equation*}
  \end{prop}

\newpage\section{Cyclic Extensions}

  \begin{thm}[Hilbert's Theorem 90]
    Let $K/k$ be cyclic of degree $n$ with Galois group $G$. Let $\sigma$ be a generator of $G$. Let $\beta\in K$. The norm $N(\beta)=1$ if and only if there exists an element $\alpha\neq0$ in $K$ such that $\beta=\alpha/\sigma\alpha$.
  \end{thm}

  \begin{thm}
    Let $k$ be a field, $n$ an integer $>0$ prime to the characteristic of $k$, and assume that there is a primitive $n-$th root of unity in $k$.
    \begin{enumerate}[(i)]
      \item Let $K$ be a cyclic extension of degree $n$. Then there exists $\alpha\in K$ such that $K=k(\alpha)$, and $\alpha$ satisfies an equation $X^n-a=0$ for some $a\in k$.
      \item Conversely, let $a\in k$. Let $\alpha$ be a root of $X^n-a$. Then $k(\alpha)$ is cyclic over $k$, of degree $d$, $d\mid n$, and $\alpha^d$ is an element of $k$.
    \end{enumerate}
  \end{thm}

  \begin{thm}[Hilbert's Theorem 90, Additive Form]
    Let $K/k$ be cyclic of degree $n$ with Galois group $G$. Let $\sigma$ be a generator of $G$. Let $\beta\in K$. The trace $\Tr(\beta)=0$ if and only if there exists an element $\alpha\neq0$ in $K$ such that $\beta=\alpha-\sigma\alpha$.
  \end{thm}

  \begin{thm}[Artin-Schreier]
    Let $k$ be a field of characteristic $p$.
    \begin{enumerate}[(i)]
      \item Let $K$ ba a cyclic extension of $k$ of degree $p$. Then there exists $\alpha\in K$ such that $K=k(\alpha)$ and $\alpha$ satisfies an equation $X^p-X-a=0$ with some $a\in k$.
      \item Conversely, given $a\in k$, the polynomial $f(X)=X^p-X-a$ either has one root in $k$, in which case all its roots are in $k$, or it is irreducible. In this latter case, if $\alpha$ is a root then $k(\alpha)$ is cyclic of degree $p$ over $k$.
    \end{enumerate}
  \end{thm}

\newpage\section{Solvable and Radical Extensions}

  \begin{defn}
    A Galois extension is called \termin[solvable]{solvable extension} if its Galois group is solvable. A finite extension $E/k$ is said to be \termin[solvable]{solvable extension} if the smallest Galois extension $K$ of $k$ containing $E$ is solvable.
  \end{defn}
  \begin{rem}
     This is equivalent to saying that there exist a solvable Galois extension $L$ of $k$ containing $E$.
  \end{rem}

  \begin{prop}
    Solvable extension form a distinguished class.
  \end{prop}

  \begin{defn}
    A simple extension $k(\alpha)/k$ is called \termin{solvable by radicals} if it is one of the following type:
    \begin{enumerate}
      \item It is obtained by adjoining a root of unity.
      \item It is obtained by adjoining a root of a polynomial $X^n-a$ with $a\in k$ and $n$ prime to the characteristic.
      \item It is obtained by adjoining a root of an equation $X^p-X-a$ with $a\in k$, if $p$ is the character $>0$.
    \end{enumerate}

    A finite extension $F/k$ is said to be \termin{solvable by radicals} if it is separable and if there exists a finite extension $E/k$ containing $F$, and admitting a tower
    \begin{equation*}
      k=E_0\subset E_1\subset E_2\subset\cdots\subset E_m=E
    \end{equation*}
    such that each step is a simple extension solvable by radicals.
  \end{defn}

  \begin{prop}
    The class of extensions which are solvable by radical is distinguished.
  \end{prop}
  \begin{proof}
    Using lifting.
  \end{proof}

  \begin{thm}
    Let $E/k$ be separable, then $E$ is solvable by radical if and only if $E/k$ is solvable.
  \end{thm}

  \begin{defn}
    A polynomial $f\in k[X]$ is said to be \termin{solvable by radicals} if its splitting field is solvable by radicals.
  \end{defn}

  \begin{defn}
    Let $k$ be a field. The general polynomial of degree $n$ over $k$ is defined to be
    \begin{equation*}
      f(X)=X^n-t_1X^{n-1}+t_2X^{n-2}+\cdots+(-1)^nt_n\in k(t_1,t_2,\cdots,t_n)[X]
    \end{equation*}
    where $t_1,t_2,\cdots,t_n$ are algebraic independent over $k$.
  \end{defn}

  \begin{cor}
    The general polynomial of degree $n$ is solvable by radicals only if $n\leqslant4$.
  \end{cor}

\newpage\section{Abelian Kummer Theory}

  \begin{defn}
    A Galois extension $K/k$ with group $G$ is said to be of \termin{exponent} $m$ if $\sigma^m=1$ for all $\sigma\in G$.
  \end{defn}

    Let $m$ be prime to the characteristic of $k$. We assume $k$ contains a primitive $m-$th root of unity.

    Let $a\in k$, and $\alpha$ be an $m-$th root of $a$. The field $k(\alpha)$ is independent of the choice of $\alpha$ and hence denoted by $k(a^{1/m})$.

    We denote by $k^{\ast m}$ the image of $k^{\ast}$ under the homomorphism $x\mapsto x^m$.

    Let $k^{\ast m}\subset B< k^{\ast}$. We denoted by $k(B^{1/m})$ or $K_B$ the composite of all fields $k(a^{1/m})$ with $a\in B$.

    $K_B/k$ is Galois, let $G$ be its Galois group. Let $\sigma\in G$. Then $\sigma\alpha=\omega_{\sigma}\alpha$ for some $m-$th root of unity $\omega_{\sigma}\in\root_m\subset k^{\ast}$.

    There is a homomorphism:
    \mapdes{G}{\root_m}{\sigma}{\omega_{\sigma}}

    We may write $\omega_{\sigma}=\sigma\alpha/\alpha$, but it is independent of the choice of $\alpha$.

    We denote $\omega_{\sigma}$ by $\<\sigma,\alpha\>$. The map
    \mapdes{G\times B}{\root_m}{(\sigma,a)}{\<\sigma,a\>}
    is bilinear.

  \begin{thm}
    Let $k$ be a field, $m$ an integer $>0$ prime to the characteristic of $k$, and assume that a primitive $m-$root of unity lies in $k$.
    \begin{enumerate}[(i)]
      \item Let $k^{\ast m}\subset B< k^{\ast}$, $K_B=k(B^{1/m})$, then $K_B/k$ is Galois and abelian of exponent $m$.
      \item Consider the bilinear map
                 \mapdes{G\times B}{\root_m}{(\sigma,a)}{\<\sigma,a\>}
                 The kernel of left is $1$, and the kernel of right is $k^{\ast m}$.
      \item $K_B/k$ is finite if and only if $(B:k^{\ast m})$ is finite. If that is the case, then
                 \begin{equation*}
                   B/k^{\ast m}\cong \dual{G}
                 \end{equation*}
                 and
                 \begin{equation*}
                   [K_B:k]=(B:k^{\ast m})
                 \end{equation*}
    \end{enumerate}
  \end{thm}

  \begin{thm}
    There is a $1-1$ corresponding:
    \isodes{\Sub(k^{\ast};k^{\ast m})}{\{\text{abelian extension of $k$ of exponent $m$}\}}{B}{K_B}
  \end{thm}

  \begin{thm}
      Let $k$ be a field with characteristic $p$, we define the operator $\wp$ by
      \begin{equation*}
        \wp(x)=x^p-x
      \end{equation*}
      Then $\wp$ is an additive automorphism of $k$.

      For $\wp(k)\subset B\subset k$, Let $K_B=k(\wp^{-1}B)$.
    \begin{enumerate}[(i)]
      \item There is a $1-1$ corresponding
                 \isodes{\Sub(k:\wp(k))}{\{\text{abelian extension of $k$ of exponent $p$}\}}{B}{K_B}
      \item For $\sigma\in G, a\in B$ and $\alpha\in K_B$ such that $\wp\alpha=a$. Let $\<\sigma,a\>=\sigma\alpha-\alpha$.
                 There is a bilinear map
                 \mapdes{G\times B}{\ZZ/p}{(\sigma,a)}{\<\sigma,a\>}
                 The kernel of left is $1$, and the kernel of right is $\wp(k)$.
      \item $K_B/k$ is finite if and only if $(B:\wp(k))$ is finite. If that is the case, then
                 \begin{equation*}
                   [K_B:k]=(B:\wp(k))
                 \end{equation*}
    \end{enumerate}
  \end{thm}

\newpage\section{The Equation $X^n-a=0$}
  In this section, the roots of unity are not in the ground field.

  \begin{thm}
    Let $k$ be a field and $n$ an integer $\geqslant2$. Let $a\in k,a\neq0$. Assume that for all $p\mid n$, we have $a\in k^{p}$, and if $4\mid n$ then $a\notin-4k^4$. Then $X^n-a$ is irreducible in $k[X]$.
  \end{thm}

  \begin{cor}
    $a\in k^{\ast}$, and $a\notin k^p$ for some $p$. If $p$ is equal to the characteristic of $k$ or $p$ is odd, then for all $r\geqslant1$, the polynomial $X^{p^r}-a$ is irreducible over $k$.
  \end{cor}

  \begin{cor}
    If $k^{\ac}/k$ is finite of degree $>1$, then $k^{\ac}=k(i)$ and its characteristic is $0$.
  \end{cor}
  
  \begin{exam}
    Let $k$ be a field with characteristic not dividing $n$. Let $a\in k^{\ast}$ and $K$ be the splitting field of $X^n-a$. Let $\alpha$ be a root of $X^n-a$ and let $\zeta$ be a primitive $n-$th root of unity. Then
    \begin{equation*}
      K=k(\alpha,\zeta)=k(\alpha,\root_n)
    \end{equation*}
    
    Let $\sigma\in G_{K/k}$. Then there exists some integer $b(\sigma)$ uniquely determined mod $n$, such that
    \begin{equation*}
      \sigma(\alpha)=\alpha\zeta^{b(\sigma)}
    \end{equation*}
    
    There exists an integer $d(\sigma)$ uniquely determined mod $n$, such that
    \begin{equation*}
      \sigma(\zeta)=\zeta^{d(\sigma)}
    \end{equation*}
    
    Let $G(n)$ be the subgroup of $\GL_2(\ZZ/n)$ consisting of all matrices 
    \begin{equation*}
      M=\begin{pmatrix}
          1 & 0 \\
          b & d \\
        \end{pmatrix}
    \end{equation*}
    where $b\in\ZZ/n,d\in(\ZZ/n)^{\ast}$.
    
    Observe that $|G(n)|=n\varphi(n)$, and we obtain an injective map:
    \mapdes{G_{K/k}}{G(n)}{\sigma}{M(\sigma)}
  \end{exam}
  
  \begin{thm}
    Let $k$ be a field. Let $n$ be an odd integer prime to the characteristic, and assume that $[k(\root_n):k]=\varphi(n)$. Let $a\in k$, and $a\notin k^p$ for all prime $p\mid n$. Let $K$ be the splitting field of $X^n-a$ over $k$.
    
    Then the above homomorphism $\sigma\mapsto M(\sigma)$ is an isomorphism. The commutator group is $\Gal(K/k(\root_n))$, so $k(\root_n)$ is the maximal abelian subextension of $K$. 
  \end{thm}
  \begin{warn}
    When $n$ is even, there are some complications.
  \end{warn}