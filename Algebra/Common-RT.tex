%%%%%%%%%%%%%%%%%%%%%%%%%%%%%%%%%%
%
%               A Common Setting File Made by Gau Syu
%                              GauSyu@Gmail.com
%
%###########################################
% @ Basic Document Packages
%     @@ Latex Graphics
%     @@ Index and Nomenclatures
%     @@ Define Colors
% @ Theorems and References
%     @@ Change the Indentation
%     @@ Define Theorem Environments
%     @@ Define Cross Reference Names
% @ Letters and Notations
%     @@ Letters
%     @@ Math Characters
%     @@ Arrows
%     @@ Functions
%     @@ Terminology
%     @@ Map Descriptions
%     @@ Typical Commutative Diagrams
%     @@ Arrows in Diagram
%     @@ Words
% @ Chapters and Sections
%%%%%%%%%%%%%%%%%%%%%%%%%%%%%%%%%%
%
%                          Basic Document Packages
%
%%%%%%%%%%%%%%%%%%%%%%%%%%%%%%%%%%
\documentclass[11pt,a4paper,oneside]{book}
\usepackage{hyperref} % Able to Hyperref
\usepackage{amsmath} % the Standard AMS Package
\usepackage{mathtools}
\usepackage{appendix} % Allowed Special Appendix Chapters
\usepackage{enumerate} % This package gives the enumerate environment an optional argument which determines the style in which the counter is printed.
%
%                                     Latex Graphics
%
\usepackage{graphicx}% Able to Insert Pictures
\usepackage{epic} % Extending Latex Graphics
\usepackage[all]{xy} % Able to Draw Diagrams
\usepackage{tikz} % Able to Draw Pictures
\usetikzlibrary{arrows,decorations.pathmorphing,decorations.pathreplacing,decorations.markings,shapes.geometric,calc,positioning,chains,matrix,scopes}
\tikzset{%
singly/.style={postaction={decorate},decoration={markings,mark=at position .6 with {\arrow{>}}}},%
doubly/.style={double, double distance=1.5pt,postaction={decorate},decoration={markings,mark=at position .6 with {\arrow{>}}}},%
}
%
%                           Index and Nomenclatures
%
\usepackage{makeidx}
\usepackage{nomencl}
\def\nomname{Notations}
\setlength{\nomlabelwidth}{.20\hsize}
\setlength{\nomitemsep}{-\parsep}
\makeindex
\makenomenclature
%
%                                    Define Colors
%
\usepackage{color}
\newcommand{\red}{\color{red}}

%%%%%%%%%%%%%%%%%%%%%%%%%%%%%%%%%%
%
%                          Theorems and References
%
%%%%%%%%%%%%%%%%%%%%%%%%%%%%%%%%%%
\usepackage[amsmath,thmmarks,hyperref]{ntheorem} % DIY Theorem Envirnoments
%
%                          Change the Indentation
%
\makeatletter
\def\numberline#1{%
  \settowidth\@tempdimb{#1\hspace{0.5em}}%
  \ifdim\@tempdima<\@tempdimb%
    \@tempdima=\@tempdimb%
  \fi%
  \hb@xt@\@tempdima{#1\hfil}}
\makeatother
%
%                        Define Theorem Environments
%
\setlength{\theorempreskipamount}{8pt}
\theoremstyle{plain}
\theoremseparator{.}
\newtheorem{ex}{}[chapter] % Exercises
\newtheorem{thm}{Theorem}[section] % Theorems
\newtheorem{lem}[thm]{Lemma} % Lemmas
\newtheorem{prop}[thm]{Proposition} % Properties
\newtheorem{cor}[thm]{Corollary} % Corollaries

\theorembodyfont{\upshape}
\newtheorem{defn}[thm]{Definition} % Definitions
\newtheorem*{rem}{Remark} % Remarks
\newtheorem*{warn}{Warning} % Warning
\newtheorem*{exam}[thm]{Example} % Examples

\theoremstyle{nonumberplain}
\theoremheaderfont{\itshape}
\theorembodyfont{\upshape}
\theoremsymbol{\ensuremath{\Box}}
\newtheorem{proof}{Proof} % Proofs

\qedsymbol{\ensuremath{\Box}} % Define Q.E.D. Symbol
%
%                          Define Cross Reference Names
%
\usepackage{cleveref}
\crefname{equation}{equation}{equations}
\crefname{cor}{Corollary}{Corollaries}
\crefname{exam}{Example}{Examples}
\crefname{ex}{}{Exercises}
\crefname{prop}{Proposition}{Propositions}
\crefname{defn}{Definition}{Definitions}
\crefname{lemma}{Lemma}{Lemmas}
\crefname{thm}{Theorem}{Theorems}
\crefname{chapter}{Chapter}{Chapters}

%%%%%%%%%%%%%%%%%%%%%%%%%%%%%%%%%%
%
%                                Letters and Notations
%
%%%%%%%%%%%%%%%%%%%%%%%%%%%%%%%%%%
\usepackage{amssymb,amsfonts,bm,mathrsfs,upgreek}
%
%                                           Letters
%
\def\Aa{{\cal A}}
\def\Bb{{\cal B}}
\def\Cc{{\cal C}}
\def\Dd{{\cal D}}
\def\Ee{{\cal E}}
\def\Ff{{\cal F}}
\def\Gg{{\cal G}}
\def\Hh{{\cal H}}
\def\Ii{{\cal I}}
\def\Jj{{\cal J}}
\def\Kk{{\cal K}}
\def\Ll{{\cal L}}
\def\Mm{{\cal M}}
\def\Nn{{\cal N}}
\def\Oo{{\cal O}}
\def\Pp{{\cal P}}
\def\Qq{{\cal Q}}
\def\Rr{{\cal R}}
\def\Ss{{\cal S}}
\def\Tt{{\cal T}}
\def\Uu{{\cal U}}
\def\Vv{{\cal V}}
\def\Ww{{\cal W}}
\def\Xx{{\cal X}}
\def\Yy{{\cal Y}}
\def\Zz{{\cal Z}}

\def\AA{{\mathbb A}}
\def\BB{{\mathbb B}}
\def\CC{{\mathbb C}}
\def\DD{{\mathbb D}}
\def\EE{{\mathbb E}}
\def\FF{{\mathbb F}}
\def\GG{{\mathbb G}}
\def\HH{{\mathbb H}}
\def\II{{\mathbb I}}
\def\JJ{{\mathbb J}}
\def\KK{{\mathbb K}}
\def\LL{{\mathbb L}}
\def\MM{{\mathbb M}}
\def\NN{{\mathbb N}}
\def\OO{{\mathbb O}}
\def\PP{{\mathbb P}}
\def\QQ{{\mathbb Q}}
\def\RR{{\mathbb R}}
\def\SS{{\mathbb S}}
\def\TT{{\mathbb T}}
\def\UU{{\mathbb U}}
\def\VV{{\mathbb V}}
\def\WW{{\mathbb W}}
\def\XX{{\mathbb X}}
\def\YY{{\mathbb Y}}
\def\ZZ{{\mathbb Z}}

\def\Aaa{\mathscr{A}}
\def\Bbb{\mathscr{B}}
\def\Ccc{\mathscr{C}}
\def\Ddd{\mathscr{D}}
\def\Eee{\mathscr{E}}
\def\Fff{\mathscr{F}}
\def\Ggg{\mathscr{G}}
\def\Hhh{\mathscr{H}}
\def\Iii{\mathscr{I}}
\def\Jjj{\mathscr{J}}
\def\Kkk{\mathscr{K}}
\def\Lll{\mathscr{L}}
\def\Mmm{\mathscr{M}}
\def\Nnn{\mathscr{N}}
\def\Ooo{\mathscr{O}}
\def\Ppp{\mathscr{P}}
\def\Qqq{\mathscr{Q}}
\def\Rrr{\mathscr{R}}
\def\Sss{\mathscr{S}}
\def\Ttt{\mathscr{T}}
\def\Uuu{\mathscr{U}}
\def\Vvv{\mathscr{V}}
\def\Www{\mathscr{W}}
\def\Xxx{\mathscr{X}}
\def\Yyy{\mathscr{Y}}
\def\Zzz{\mathscr{Z}}

\def\aa{\mathfrak{a}}
\def\bb{\mathfrak{b}}
\def\cc{\mathfrak{c}}
\def\dd{\mathfrak{d}}
\def\ee{\mathfrak{e}}
\def\ff{\mathfrak{f}}
\def\gg{\mathfrak{g}}  % Knuth uses $\gg$ for ``>>''.
\def\hh{\mathfrak{h}}
\def\ii{\mathfrak{i}}
\def\jj{\mathfrak{j}}
\def\kk{\mathfrak{k}}
\def\ll{\mathfrak{l}}  % Knuth uses $\ll$ for ``<<''.
\def\mm{\mathfrak{m}}
\def\nn{\mathfrak{n}}
\def\oo{\mathfrak{o}}
\def\pp{\mathfrak{p}}
\def\qq{\mathfrak{q}}
\def\rr{\mathfrak{r}}
\def\ss{\mathfrak{s}}
\def\tt{\mathfrak{t}}
\def\uu{\mathfrak{u}}
\def\vv{\mathfrak{v}}
\def\ww{\mathfrak{w}}
\def\xx{\mathfrak{x}}
\def\yy{\mathfrak{y}}
\def\zz{\mathfrak{z}}

\def\BBb{\mathfrak{B}}
\def\EEe{\mathfrak{E}}
\def\FFf{\mathfrak{F}}
\def\GGg{\mathfrak{G}}
\def\SSs{\mathfrak{S}}

\def\aaa{{a}}
\def\bbb{{b}}
\def\ccc{{c}}
\def\ddd{{d}}

%
%                                  Math Characters
%
\def\<{\langle}
\def\>{\rangle}
\def\anti{\mathpzc{S}}
\def\ctimes{\textrm{\c{$\otimes$}}}
\def\sminus{\smallsetminus}
\def\Wedge{\mbox{$\bigwedge$}}
\def\lrtimes{\Join}

\def\Hodge{\widetilde{\Delta}}

%
%                                         Arrows
%
\def\acts{\curvearrowright}
\def\epi{\twoheadrightarrow}
\def\from{\leftarrow}
\def\isom{\overset{\sim}{\to}}
\def\longto{\longrightarrow}
\def\mono{\rightarrowtail}
\def\onto{\twoheadrightarrow}
\def\injection{\hookrightarrow}
\def\then{\Rightarrow}
\def\To{\longto}
\def\Ot{\longleftarrow}
\def\tofrom{\leftrightarrow}
\def\tto{\rightrightarrows}
\def\down{\downarrow}

%
%                                          Functions
%
\newcommand{\norm}[1]{\lVert #1\rVert} % Norm
\newcommand{\dual}[1]{{#1}^{\wedge}}% Dual
\newcommand{\codual}[1]{{#1}^{\vee}}% Codual

%
%          Make index and red emphasis for terminology
%
\usepackage{xifthen}% provides \isempty test
\newcommand{\termin}[2][]{%
  \ifthenelse{\isempty{#1}}%
    {\emph{\red {#2}}\index{#2}}% if #1 is empty
    {\emph{\red {#1}}\index{#2}}% if #1 is not empty
}

\newcommand{\markar}[1]{\stackrel{{#1}}{\longrightarrow}}
\newcommand{\markal}[1]{\stackrel{{#1}}{\longleftarrow}}
\newcommand{\defen}{\stackrel{\text{def}}{\iff}}
\newcommand{\local}[2]{\left.{#1}\right|_{#2}}%Local #1 at #2

%
%                                        Combinations
%
\newcommand{\exseq}[5]{{#1} \xrightarrow{{#2}} {#3} \xrightarrow{{#4}} {#5}}
\newcommand{\longexseq}[5]{{#1} \markar{{#2}} {#3} \markar{{#4}} {#5}}
\newcommand{\shortexseq}[5]{1\longrightarrow{#1}\xrightarrow{#2}{#3}\xrightarrow{#4}{#5}\longrightarrow{1}}
\newcommand{\myldto}[1]{$\mathop{\rightsquigarrow}\limits_{\mathclap{\text{\scriptsize #1}}}$}
\newcommand{\mysim}[1]{\mathop{\sim}\limits_{\mathclap{#1}}}
\newcommand{\defeq}{\stackrel{\text{def}}{=}}
%
%                                    Map Descriptions
%
\newcommand{\mapdes}[4]
  {
    \begin{align*}
       #1 & \longrightarrow #2 \\
       #3 & \longmapsto #4
    \end{align*}
  }
\newcommand{\longmapdes}[5]
  {
    \begin{align*}
      #1\colon  #2 & \longrightarrow  #3 \\
            #4 & \longmapsto  #5
    \end{align*}
  }
\newcommand{\isodes}[4]
  {
    \begin{align*}
      #1 & \cong  #2 \\
      #3 & \leftrightarrow  #4
    \end{align*}
  }
%
%                        Typical Commutative Diagrams
%
\newcommand{\initial}[6]
{
\begin{displaymath}
   \xymatrix{
     {#1} \ar[r]^{#2} \ar[dr]_{#4} & {#3} \ar@{-->}[d]^{#6} \\
     & {#5}
   }
\end{displaymath}
}
\newcommand{\terminal}[6]
{
\begin{displaymath}
   \xymatrix{
     {#5} \ar[dr]^{#4} \ar@{-->}[d]_{#6} & \\
     {#3} \ar[r]_{#2} & {#1}
   }
\end{displaymath}
}
\newcommand{\functor}[8]
{
\begin{displaymath}
   \xymatrix{
     {#1}\ar[d]_{#2}\ar[r]^{#4} & {#5}\ar[d]^{#8} \\
     {#3}\ar[r]_{#6} & {#7}
   }
\end{displaymath}
}

\newcommand{\Functor}[9]
{
\begin{displaymath}
   \xymatrix{
     {#1}\ar[rr]^{#2} & & {#3}\\
     {#4}\ar[dd]^{#5}="a" \ar@{|->}[rr]  & & {#7}\ar[dd]_{#8}="b" \ar@{|->} "a";"b"\\
     & & \\
     {#6}\ar@{|->}[rr] & & {#9}
   }
\end{displaymath}
}
%
%                                    Arrows in Diagram
%
\newdir{ (}{{}*!/-5pt/@^{(}}

%
%                                              Words
%
\DeclareMathOperator{\ac}{a}%algebraic closure
\DeclareMathOperator{\ab}{ab}%abelian group
\DeclareMathOperator{\Ab}{\mathbf{Ab}}%category of abelian groups
\DeclareMathOperator{\Abtf}{\mathbf{Ab}_{\mathrm{tf}}}%Abelian group with torsion-free
\DeclareMathOperator{\Ad}{Ad}
\DeclareMathOperator{\Add}{\mathbf{Add}}
\DeclareMathOperator{\ad}{ad}
\DeclareMathOperator{\adj}{adj}
\DeclareMathOperator{\Alt}{Alt}
\DeclareMathOperator{\ann}{ann}
\DeclareMathOperator{\Aut}{Aut}
\DeclareMathOperator{\Bil}{Bil}%Bilinear
\DeclareMathOperator{\ch}{ch}
\DeclareMathOperator{\cls}{cls}
\DeclareMathOperator{\coim}{coim}
\DeclareMathOperator{\coker}{coker}
\DeclareMathOperator{\Cat}{\mathbf{Cat}}%category of all (small) categories
\DeclareMathOperator{\di}{d}
\DeclareMathOperator{\diag}{diag}
\DeclareMathOperator*{\dirlim}{\underrightarrow{\lim}}%direct limit
\DeclareMathOperator{\D}{D}
\DeclareMathOperator{\Der}{Der}%Derivations
\DeclareMathOperator{\End}{End}
\DeclareMathOperator{\ev}{ev}
\DeclareMathOperator{\Ext}{Ext}
\DeclareMathOperator{\filt}{\mathfrak{f}}%filtration  %\ff
\DeclareMathOperator{\Fct}{\mathbf{Fct}}%Functor
\DeclareMathOperator{\Fix}{Fix}%Fixed points
\DeclareMathOperator{\Forget}{Forget}
\DeclareMathOperator{\Free}{Free}
\DeclareMathOperator{\Fun}{\mathbf{Fun}}
\DeclareMathOperator{\gr}{gr}
\DeclareMathOperator{\Gal}{Gal}
\DeclareMathOperator{\Grp}{\mathbf{Grp}}%Category of groups
\DeclareMathOperator{\Hess}{Hess}%Hessian
\DeclareMathOperator{\Hom}{Hom}%Hom bifunctor
\DeclareMathOperator{\im}{im}
\DeclareMathOperator*{\invlim}{\underleftarrow{\lim}}%inverse limit
\DeclareMathOperator{\id}{id}%Identity
\DeclareMathOperator{\Id}{Id}%Identity
\DeclareMathOperator{\Inn}{Inn}
\DeclareMathOperator{\Int}{Int} %intermediate field
\DeclareMathOperator{\Irr}{Irr} %Irreducible polynomial
\DeclareMathOperator{\Is}{\mathfrak{Is}}%Isolator
\DeclareMathOperator{\Lie}{\mathfrak{L}}%lie algebra
\DeclareMathOperator{\Mod}{\mathbf{Mod}}%Module
\DeclareMathOperator{\Mor}{Mor}
\DeclareMathOperator{\ML}{\mathbf{M.L.}}
\DeclareMathOperator{\nat}{nat}
\DeclareMathOperator{\N}{\mathbb{N}}
\DeclareMathOperator{\ob}{ob}%Object
\DeclareMathOperator{\obj}{\mathcal{T}}
\DeclareMathOperator{\one}{\mathbf{1}}
\DeclareMathOperator{\op}{op}%opposite ring
\DeclareMathOperator{\Op}{\mathcal{O}\mathfrak{p}\mathrm{ext}}%the sets of congruence classes of extensions
\DeclareMathOperator{\Pic}{Pic}%Picard group
\DeclareMathOperator{\prim}{prim}
\DeclareMathOperator{\Quot}{Qout}%Quotient functor
\DeclareMathOperator{\rad}{rad}
\DeclareMathOperator{\rank}{rank}
\DeclareMathOperator{\sgn}{sgn}
\DeclareMathOperator{\s}{s} %separable closure
\DeclareMathOperator{\Set}{\mathbf{Set}}%Category of sets
\DeclareMathOperator{\Span}{span} % Annoyingly, \span is already a command in TeX, and redefining it leads to other problems.
\DeclareMathOperator{\Split}{\mathbf{Split}}%Karoubi envelope
\DeclareMathOperator{\Sub}{Sub}%Sub functor
\DeclareMathOperator{\supp}{supp}
\DeclareMathOperator{\Top}{Top}
\DeclareMathOperator{\Tor}{Tor}
\DeclareMathOperator{\tor}{tor}
\DeclareMathOperator{\Tr}{Tr}
\DeclareMathOperator{\Vect}{Vect}

\DeclareMathOperator{\GL}{GL}
\DeclareMathOperator{\PGL}{PGL}
\DeclareMathOperator{\PSL}{PSL}
\DeclareMathOperator{\SL}{SL}
\DeclareMathOperator{\SO}{SO}
\DeclareMathOperator{\GO}{O}
\DeclareMathOperator{\SP}{Sp}
\DeclareMathOperator{\Spin}{Spin}
\DeclareMathOperator{\SU}{SU}
\DeclareMathOperator{\GU}{U}
\DeclareMathOperator{\Pt}{Pt}

\def\gl{\gg\ll}
\def\sl{\ss\ll}
\def\so{\ss\oo}
\def\sp{\ss\pp}
\def\su{\ss\uu}
\def\root{\bm{\upmu}}

\def\st{\textrm{ s.t. }}
\def\RP{\mathbb{R}\mathbf{P}}
\def\Real{\mathbb{R}}
%%%% Others %%%%

%
%                              Chapters and Sections
%
\usepackage[Lenny]{fncychap}% Chapter Style
\setcounter{secnumdepth}{2}% Depth of Sections
