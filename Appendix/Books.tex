\chapter{Famous Books}
Here is a list of some famous books in present. The articles discuss them are copied from $n$Lab.
\minitoc
\newpage
\section{Handbook of Analysis and its Foundations} 
Erich Schechter's \emph{\href{http://www.math.vanderbilt.edu/~schectex/ccc/}{Handbook of Analysis and its Foundations}} is a large book, intended for self study by beginning graduate students or senior-level undergraduates, on all of the basic topics of abstract analysis and then some. Its logical flow is very much like that of Bourbaki, but focussed on that which applies to analysis and written in a modern style. Except for the slightly broken index (check the \href{http://www.math.vanderbilt.edu/~schectex/ccc/addenda/}{\textbf{errata}}!), it is very user-friendly, with sketched proofs phrased as exercises with hints, many examples (eventually), and abundant cross references. It is also extremely self contained; the only prerequisite is mathematical maturity, and the first section even helps with that!

It begins, as the name implies, with foundations: not only the usual na\"{i}ve set theory, but also a discussion of ZFC, constructive mathematics, and enough model theory to do nonstandard analysis. There is special emphasis on the axiom of choice; throughout the book, it is explicitly pointed out whenever anything beyond dependent choice and excluded middle is required. (The axioms of replacement and foundation, on the few occasions when they appear, are also pointed out, so logically the book takes place in ${Z^-} + {DC}$.)

The book then moves on to algebra, moving from monoids to fields, on the grounds that such algebra also serves as a foundation for analysis. This culminates in a treatment of category theory; this is somewhat unsatisfactory (although very good for an analysis book!) and is not much more than Bourbaki's theory of structures reinterpreted as a theory of concrete categories. After algebra comes topology, and then analysis proper.

Those aspects of analysis that do not depend on the theory of the real numbers are also covered when appropriate in the set-theory and algebra sections of the book; thus topological spaces and convex sets (for example) are defined early, although they are not thoroughly studied until later (after the foundational parts are finished). The analysis in the book is soft and abstract, on the grounds that this material serves as the proper foundation for hard results in concrete cases. However, many concrete examples are given to illustrate the abstract ideas. The breadth of topics covered, even within analysis itself, is quite wide; from convergence spaces to ultrabarrels, from the Henstock integral to the Brouwer fixed-point theorem, it has it all.

But everything must stop somewhere; it does not cover \textbf{complex analysis}.

\subsection*{Contents}

\begin{itemize}%
\item Part A: Sets and Orderings (Chapters 1--7)
\item Part B: Algebra (Chapters 8--14)
\item Part C: Topology and Uniformity (Chapters 15--21)
\item Part D: Topological Vector Spaces (Chapters 22--30)
\end{itemize}
\begin{enumerate}%
\item Sets
\item Functions
\item Relations and Orderings
\item More About Sups and Infs
\item Filters, Topologies, and Other Sets of Sets
\item Constructivism and Choice
\item Nets and Convergences
\item Elementary Algebraic Systems
\item Concrete Categories
\item The Real Numbers
\item Linearity
\item Convexity
\item Boolean Algebras
\item Logic and Intangibles
\item Topological Spaces
\item Separation and Regularity Axioms
\item Compactness
\item Uniform Spaces
\item Metric and Uniform Completeness
\item Baire Theory
\item Positive Measure and Integration
\item Norms
\item Normed Operators
\item Generalized Riemann Integrals
\item Fr\'{e}chet Derivatives
\item Metrization of Groups and Vector Spaces
\item Barrels and Other Features of TVS��s
\item Duality and Weak Compactness
\item Vector Measures
\item Initial Value Problems
\end{enumerate}





\newpage
\section{Elephant}
The Elephant is a book on topos theory by Peter Johnstone.

The full title is \emph{Sketches of an Elephant: A Topos Theory Compendium. Like Gravitation}, the title can be taken to refer not only to the subject matter but also to the immense size and scope of the book itself. Like The \emph{Lord of the Rings}, it consists of 6 parts arranged evenly into 3 volumes (but without appendices). Actually, Volume 3 has not yet been published (so who knows? it may have appendices after all!).

The Elephant is a good reference for anything related to topos theory, and we may often cite it here. However, it introduced many terminological changes, some of which may not be widely accepted or even known. (Fortunately, it will tell you about these in the text.) 

\subsection*{Contents}
\begin{enumerate}[A]
  \item Toposes as Categories
  \begin{enumerate}[{A}1]
    \item Regular and cartesian closed categories
    \begin{enumerate}[{A1}.1]
      \item Preliminary assumptions
      \item Cartesian categories
      \item Regular categories
      \item Coherent categories
      \item Cartesian closed categories
      \item Subobject classifiers
    \end{enumerate}
    \item Toposes - basic theory
    \begin{enumerate}[{A2}.1]
      \item Definition and examples
      \item The monadicity theorem
      \item The Fundamental Theorem
      \item Effectiveness, positivity and partial maps
      \item Natural number objects
      \item Quasitoposes
    \end{enumerate}
    \item Allegories
    \begin{enumerate}[{A3}.1]
      \item Relations in regular categories
      \item Allegories and tabulations
      \item Splitting symmetric idempotents
      \item Division allegories and power allegories
    \end{enumerate}
    \item Geometric morphisms - basic theory
    \begin{enumerate}[{A4}.1]
      \item Definition and examples
    \end{enumerate}
  \end{enumerate}
  \item $2$-Categorical Aspects of Topos Theory
  \begin{enumerate}[{B}1]
    \item Indexed categories and fibrations
    \begin{enumerate}[{B1}.1]
      \item Review of $2-$categories
      \item Indexed categories
      \item Fibrations
      \item Limits and colimits
      \item Descent conditions and stacks
    \end{enumerate}    
    \item Internal and localy internal categories
    \begin{enumerate}[{B2}.1]
      \item Review of enriched categories
      \item Locally internal categories
      \item Internal categories and diagram categories
      \item The indexed adjoint functor theorem
      \item Discrete opfibrations
      \item Filtered colimits
      \item Internal profunctors
    \end{enumerate}
    \item Toposes over a base
    \begin{enumerate}[{B3}.1]
      \item $S-$Toposes as $S-$indexed categories
      \item Diaconescu's theorem
      \item Giraud's theorem
      \item Colimits in Top
    \end{enumerate}
  \end{enumerate}
  \item Toposes as Spaces
  \begin{enumerate}[{C}1]
    \item Sheaves on a locale
    \begin{enumerate}[{C1}.1]
      \item Frames and nuclei
      \item Locales and spaces
      \item Sheaves, local homeomorphisms and frame-valued sets
      \item Continuous maps
      \item Some topological properties of toposes
    \end{enumerate}
    \item Sheaves on a site
    \begin{enumerate}[{C2}.1]
      \item Sites and coverages
      \item The topos of sheaves
      \item Morphisms of sites
      \item Internal sites and pullbacks
      \item Fibrations of sites
    \end{enumerate}
    \item Classes of geometric morphisms
    \begin{enumerate}[{C3}.1]
      \item Proper maps
      \item Locally connected morphisms
      \item Atomic morphisms
      \item Local maps
    \end{enumerate}
  \end{enumerate}
  \item Toposes as theories
  \begin{enumerate}[{D}1]
    \item First-order categorical logic
    \begin{enumerate}[{D1}.1]
      \item First-order languages
      \item Categorical semantics
      \item First-order logic
      \item Syntactic categories
      \item Classical completeness
    \end{enumerate}
    \item Sketches
    \begin{enumerate}[{D2}.1]
      \item The concept of sketch
      \item Sketches and theories
      \item Sketchable and accessible categories
      \item Properties of model categories
    \end{enumerate}
    \item Classifying toposes
    \begin{enumerate}[{D3}.1]
      \item Classifying toposes via syntactic sites
      \item The object classifier
      \item Coherent toposes
    \end{enumerate}
    \item Higher-order logic
    \begin{enumerate}[{D4}.1]
      \item Predicative type theories
      \item Axioms of choice and Booleanness
      \item Real numbers in a topos
    \end{enumerate}
    \item Aspects of finiteness
    \begin{enumerate}[{D5}.1]
      \item Finite cardinals
      \item Finitary algebraic theories
      \item Kuratowski-finiteness
    \end{enumerate}
  \end{enumerate}
  \item Homotopy and Cohomology
  \begin{enumerate}[{E}1]
    \item Homotopy theory for toposes
    \begin{enumerate}[{E1}.1]
      \item Homotopy theory of toposes
      \item The fundamental groupoid via paths
      \item The fundamental groupoid via coverings
      \item Natural homotopy
    \end{enumerate}
    \item Algebraic homotopy theory
    \begin{enumerate}[{E2}.1]
      \item Quillen model structures
      \item Model structure for simplicial sets
      \item Model structures for sheaves
    \end{enumerate}
    \item Cohomology theory
    \begin{enumerate}[{E3}.1]
      \item Abelian groups and modules in a topos
      \item Cech cohomology
      \item Torsors and non-abelian cohomology
      \item Cohomological applications of descent theory
    \end{enumerate}
  \end{enumerate}
  \item Toposes a Mathematical Universes
  \begin{enumerate}[{F}1]
    \item Synthetic differential geometry
    \begin{enumerate}[{F1}.1]
      \item Properties of the generic ring
      \item Rings of line type
      \item Well-adapted models
      \item Tiny objects
      \item Synthetic integration theory
      \item Intrinsic infinitesimal
    \end{enumerate}
    \item Topos theory and set theory
    \begin{enumerate}[{F2}.1]
      \item Internal sets in a topos
      \item Algebraic set theory
      \item Independence proofs via classifying toposes
      \item Independence of the axiom of choice
    \end{enumerate}
  \end{enumerate}
\end{enumerate}
