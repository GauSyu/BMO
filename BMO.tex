%%%%%%%%%%%%%%%%%%%%%%%%%%%%%%%%%%
%
%                        A Tex File Made by Gau Syu
%                              GauSyu@Gmail.com
%
%###########################################
%%%%%%%%%%%%%%%%%%%%%%%%%%%%%%%%%%
%
%                     Include The Common Settings
%
%%%%%%%%%%%%%%%%%%%%%%%%%%%%%%%%%%
%%%%%%%%%%%%%%%%%%%%%%%%%%%%%%%%%%
%
%               A Common Setting File Made by Gau Syu
%                              GauSyu@Gmail.com
%
%###########################################
% @ Basic Document Packages
%     @@ Typeface
%     @@ Latex Graphics
%     @@ Index and Nomenclatures
%     @@ Define Colors
%     @@ Terminology Format
% @ Theorems and References
%     @@ Change the Indentation
%     @@ Define Theorem Environments
%     @@ Define Cross Reference Names
% @ Chapters and Sections
%%%%%%%%%%%%%%%%%%%%%%%%%%%%%%%%%%
%
%                          Basic Document Packages
%
%%%%%%%%%%%%%%%%%%%%%%%%%%%%%%%%%%
\documentclass[11pt,a4paper,oneside,nocap,fancyhdr,hyperref]{ctexbook}
%nocap means English
\usepackage{appendix} % Allowed Special Appendix Chapters
\usepackage{enumerate} % This package gives the enumerate environment an optional argument which determines the style in which the counter is printed.
%
%                                         Typefaces
%
    \newfontinstance{\Edward}{Edwardian Script ITC}
    \newfontinstance{\Frak}{Euclid Fraktur}
    \newcommand{\Giant}{\fontsize{72pt}{\baselineskip}\selectfont}
    \newcommand{\giant}{\fontsize{27pt}{\baselineskip}\selectfont}
%
%                                     Latex Graphics
%
\usepackage{graphicx}% Able to Insert Pictures
\usepackage{epic} % Extending Latex Graphics
\usepackage[all]{xy} % Able to Draw Diagrams
\xyoption{2cell}
\UseAllTwocells
\usepackage{tikz} % Able to Draw Pictures
%
%                           Index and Nomenclatures
%
\usepackage{makeidx}
  \makeindex
\usepackage[
  symbols,                %list of symbols
%  nonumberlist,       %do not show page numbers
  seeautonumberlist,
  hyperfirst=false,
  toc,                         %show listings as entries in table of contents
  section=chapter, %use section level for toc entries
  counter=section] %countered by section
{glossaries}

  \altnewglossary{categories}{cat}{Categories}
  \glsenablehyper
  \makeglossaries

%%  Usage of glossaryentry
%\newglossaryentry{<\label>}  {name=<\what occurs in the glossary>, description=<\>,  text=<\what occurs in the context>, sort=<\How this term by sorted>, type=<\>}
%%
%\usepackage[refpage,intoc]{nomencl}
%  \def\nomname{Notations}
%  \setlength{\nomlabelwidth}{.20\hsize}
%  \setlength{\nomitemsep}{-\parsep}
%    \makenomenclature

\usepackage{xifthen}% provides \isempty test
\newcommand{\termin}[2][]{%
  \ifthenelse{\isempty{#1}}%
    {\textbf{{#2}}\index{#2}}% if #1 is empty
    {\textbf{{#1}}\index{#2}}% if #1 is not empty
}
%
%                                    Define Colors
%
\usepackage{color}
  \newcommand{\red}{\color{red}}
  \newcommand{\blue}{\color{blue}}

%%%%%%%%%%%%%%%%%%%%%%%%%%%%%%%%%%
%
%                        Define Theorem Environments
%
%%%%%%%%%%%%%%%%%%%%%%%%%%%%%%%%%%
\usepackage{amsmath,amsthm} % the Standard AMS Package
\usepackage{mathtools} %Many tools

%\renewcommand{\proofname}{\textbf{Solution}}

\newtheoremstyle{question}{1.5ex plus 1ex minus .2ex}{1.5ex plus 1ex minus .2ex}{\large\itshape}{}{\bfseries}{}{1em}{}
\theoremstyle{question}
\newtheorem{qst}{Question}[section]
\renewcommand{\theqst}{\arabic{qst}}
\newtheorem{subqst}{Question}[qst]
\newtheorem*{qst*}{Question}

\theoremstyle{theorem}
\newtheorem{thm}{定理}[section]
\newtheorem{lem}[thm]{引理}
\newtheorem{prop}[thm]{命题}
\newtheorem{cor}[thm]{推论}

\theoremstyle{definition}
\newtheorem{defn}[thm]{定义}
\newtheorem{exam}[thm]{例子}
\newtheorem{ex}{}[chapter]
\theoremstyle{remark}
\newtheorem*{rem}{注}
\newtheorem*{rec}{回顾}

\renewcommand{\proofname}{证明}
%%%%%%%%%%%%%%%%%%%%%%%%%%%%%%%%%%
%
%                              Chapters and Sections
%
%%%%%%%%%%%%%%%%%%%%%%%%%%%%%%%%%%
\CTEXsetup[name={{$\S$},}]{section}
\CTEXsetup[name={{\Frak Cap}.,},aftername={\quad},format={\centering}]{chapter}
\CTEXoptions[contentsname={\textbf{Contents}}]
%\CTEXsetup[name={$\mathscr{P}$,}]{part}
\setcounter{secnumdepth}{2}% Depth of Sections
\setcounter{tocdepth}{1}% Depth of Contents
\usepackage{minitoc}

%%%%%%%%%%%%%%%%%%%%%%%%%%%%%%%%%%
%
%                                Letters and Notations
%
%%%%%%%%%%%%%%%%%%%%%%%%%%%%%%%%%%
%%%%%%%%%%%%%%%%%%%%%%%%%%%%%%%%%%
%
%               A Common Setting File Made by Gau Syu
%                              GauSyu@Gmail.com
%
%###########################################
% @ Letters and Notations
%     @@ Letters
%     @@ Math Characters
%     @@ Arrows
%     @@ Functions
%     @@ Terminology
%     @@ Map Descriptions
%     @@ Typical Commutative Diagrams
%     @@ Arrows in Diagram
%     @@ Words
%%%%%%%%%%%%%%%%%%%%%%%%%%%%%%%%%%
\usepackage{amssymb,amsfonts,bbm,mathrsfs,upgreek}
%
%                                           Letters
%
\def\Aa{{\cal A}}
\def\Bb{{\cal B}}
\def\Cc{{\cal C}}
\def\Dd{{\cal D}}
\def\Ee{{\cal E}}
\def\Ff{{\cal F}}
\def\Gg{{\cal G}}
\def\Hh{{\cal H}}
\def\Ii{{\cal I}}
\def\Jj{{\cal J}}
\def\Kk{{\cal K}}
\def\Ll{{\cal L}}
\def\Mm{{\cal M}}
\def\Nn{{\cal N}}
\def\Oo{{\cal O}}
\def\Pp{{\cal P}}
\def\Qq{{\cal Q}}
\def\Rr{{\cal R}}
\def\Ss{{\cal S}}
\def\Tt{{\cal T}}
\def\Uu{{\cal U}}
\def\Vv{{\cal V}}
\def\Ww{{\cal W}}
\def\Xx{{\cal X}}
\def\Yy{{\cal Y}}
\def\Zz{{\cal Z}}

\def\AA{{\mathbb A}}
\def\BB{{\mathbb B}}
\def\CC{{\mathbb C}}
\def\DD{{\mathbb D}}
\def\EE{{\mathbb E}}
\def\FF{{\mathbb F}}
\def\GG{{\mathbb G}}
\def\HH{{\mathbb H}}
\def\II{{\mathbb I}}
\def\JJ{{\mathbb J}}
\def\KK{{\mathbb K}}
\def\LL{{\mathbb L}}
\def\MM{{\mathbb M}}
\def\NN{{\mathbb N}}
\def\OO{{\mathbb O}}
\def\PP{{\mathbb P}}
\def\QQ{{\mathbb Q}}
\def\RR{{\mathbb R}}
\def\SS{{\mathbb S}}
\def\TT{{\mathbb T}}
\def\UU{{\mathbb U}}
\def\VV{{\mathbb V}}
\def\WW{{\mathbb W}}
\def\XX{{\mathbb X}}
\def\YY{{\mathbb Y}}
\def\ZZ{{\mathbb Z}}

\def\Aaa{\mathscr{A}}
\def\Bbb{\mathscr{B}}
\def\Ccc{\mathscr{C}}
\def\Ddd{\mathscr{D}}
\def\Eee{\mathscr{E}}
\def\Fff{\mathscr{F}}
\def\Ggg{\mathscr{G}}
\def\Hhh{\mathscr{H}}
\def\Iii{\mathscr{I}}
\def\Jjj{\mathscr{J}}
\def\Kkk{\mathscr{K}}
\def\Lll{\mathscr{L}}
\def\Mmm{\mathscr{M}}
\def\Nnn{\mathscr{N}}
\def\Ooo{\mathscr{O}}
\def\Ppp{\mathscr{P}}
\def\Qqq{\mathscr{Q}}
\def\Rrr{\mathscr{R}}
\def\Sss{\mathscr{S}}
\def\Ttt{\mathscr{T}}
\def\Uuu{\mathscr{U}}
\def\Vvv{\mathscr{V}}
\def\Www{\mathscr{W}}
\def\Xxx{\mathscr{X}}
\def\Yyy{\mathscr{Y}}
\def\Zzz{\mathscr{Z}}

\def\aa{\mathfrak{a}}
\def\bb{\mathfrak{b}}
\def\cc{\mathfrak{c}}
\def\dd{\mathfrak{d}}
\def\ee{\mathfrak{e}}
\def\ff{\mathfrak{f}}
\def\gg{\mathfrak{g}}  % Knuth uses $\gg$ for ``>>''.
\def\hh{\mathfrak{h}}
\def\ii{\mathfrak{i}}
\def\jj{\mathfrak{j}}
\def\kk{\mathfrak{k}}
\def\ll{\mathfrak{l}}  % Knuth uses $\ll$ for ``<<''.
\def\mm{\mathfrak{m}}
\def\nn{\mathfrak{n}}
\def\oo{\mathfrak{o}}
\def\pp{\mathfrak{p}}
\def\qq{\mathfrak{q}}
\def\rr{\mathfrak{r}}
\def\ss{\mathfrak{s}}
\def\tt{\mathfrak{t}}
\def\uu{\mathfrak{u}}
\def\vv{\mathfrak{v}}
\def\ww{\mathfrak{w}}
\def\xx{\mathfrak{x}}
\def\yy{\mathfrak{y}}
\def\zz{\mathfrak{z}}

\def\BBb{\mathfrak{B}}
\def\CCc{\mathfrak{C}}
\def\EEe{\mathfrak{E}}
\def\FFf{\mathfrak{F}}
\def\GGg{\mathfrak{G}}
\def\IIi{\mathfrak{I}}
\def\PPp{\mathfrak{P}}
\def\SSs{\mathfrak{S}}
\def\UUu{\mathfrak{U}}

\def\aaa{{a}}
\def\bbb{{b}}
\def\ccc{{c}}
\def\ddd{{d}}

%
%                                  Math Characters
%
\def\<{\langle}
\def\>{\rangle}
\def\anti{\mathpzc{S}}
\def\ctimes{\textrm{\c{$\otimes$}}}
\def\sminus{\smallsetminus}
\def\Wedge{\mbox{$\bigwedge$}}
\def\lrtimes{\Join}

\def\Hodge{\widetilde{\Delta}}

%
%                                         Arrows
%
\def\acts{\curvearrowright}
\def\epi{\twoheadrightarrow}
\def\from{\leftarrow}
\def\isom{\overset{\sim}{\to}}
\def\longto{\longrightarrow}
\def\mono{\rightarrowtail}
\def\onto{\twoheadrightarrow}
\def\injection{\hookrightarrow}
\def\then{\Rightarrow}
\def\Then{\Longrightarrow}
\def\To{\longto}
\def\Ot{\longleftarrow}
\def\tofrom{\leftrightarrow}
\def\tto{\rightrightarrows}
\def\down{\downarrow}

%
%                                          Functions
%
\newcommand{\norm}[1]{\lVert #1\rVert} % Norm
\newcommand{\dual}[1]{{#1}^{\wedge}}% Dual
\newcommand{\codual}[1]{{#1}^{\vee}}% Codual
\newcommand{\pfrac}[2]{\frac{\partial{#1}}{\partial{#2}}}

\newcommand{\markar}[1]{\stackrel{{#1}}{\longrightarrow}}
\newcommand{\markal}[1]{\stackrel{{#1}}{\longleftarrow}}
\newcommand{\defen}{\stackrel{\text{def}}{\iff}}
\newcommand{\local}[2]{\left.{#1}\right|_{#2}}%Local #1 at #2

%
%                                        Combinations
%
\newcommand{\exseq}[5]{{#1} \xrightarrow{{#2}} {#3} \xrightarrow{{#4}} {#5}}
\newcommand{\longexseq}[5]{{#1} \markar{{#2}} {#3} \markar{{#4}} {#5}}
\newcommand{\shortexseq}[5]{1\longrightarrow{#1}\xrightarrow{#2}{#3}\xrightarrow{#4}{#5}\longrightarrow{1}}
\newcommand{\myldto}[1]{$\mathop{\rightsquigarrow}\limits_{\mathclap{\text{\scriptsize #1}}}$}
\newcommand{\mysim}[1]{\mathop{\sim}\limits_{\mathclap{#1}}}
\newcommand{\defeq}{\stackrel{\text{def}}{=}}
\newcommand{\iso}[1]{\stackrel{#1}{\cong}}
%
%                                    Map Descriptions
%
\newcommand{\mapdes}[4]
  {
    \begin{align*}
       #1 & \longrightarrow #2 \\
       #3 & \longmapsto #4
    \end{align*}
  }
\newcommand{\longmapdes}[5]
  {
    \begin{align*}
      #1\colon  #2 & \longrightarrow  #3 \\
            #4 & \longmapsto  #5
    \end{align*}
  }
\newcommand{\isodes}[4]
  {
    \begin{align*}
      #1 & \cong  #2 \\
      #3 & \leftrightarrow  #4
    \end{align*}
  }
%
%                        Typical Commutative Diagrams
%
\newcommand{\initial}[6]
{
\begin{displaymath}
   \xymatrix{
     {#1} \ar[r]^{#2} \ar[dr]_{#4} & {#3} \ar@{-->}[d]^{#6} \\
     & {#5}
   }
\end{displaymath}
}
\newcommand{\terminal}[6]
{
\begin{displaymath}
   \xymatrix{
     {#5} \ar[dr]^{#4} \ar@{-->}[d]_{#6} & \\
     {#3} \ar[r]_{#2} & {#1}
   }
\end{displaymath}
}
\newcommand{\functor}[8]
{
\begin{displaymath}
   \xymatrix{
     {#1}\ar[d]_{#2}\ar[r]^-{#4} & {#5}\ar[d]^{#8} \\
     {#3}\ar[r]^-{#6} & {#7}
   }
\end{displaymath}
}

\newcommand{\Functor}[9]
{
\begin{displaymath}
   \xymatrix{
     {#1}\ar[rr]^{#2} & & {#3}\\
     {#4}\ar[dd]^{#5}="a" \ar@{|->}[rr]  & & {#7}\ar[dd]_{#8}="b" \ar@{|->} "a";"b"\\
     & & \\
     {#6}\ar@{|->}[rr] & & {#9}
   }
\end{displaymath}
}
%
%                                    Arrows in Diagram
%
\newdir{ (}{{}*!/-5pt/@^{(}}
\newdir{ >}{{}*!/-5pt/@{>}}
%
%                                              Words
%
\DeclareMathOperator{\ab}{ab}%abelian group
\DeclareMathOperator{\Abtf}{\mathbf{Ab}_{\mathrm{tf}}}%Abelian group with torsion-free
\DeclareMathOperator{\Ad}{Ad}
\DeclareMathOperator{\Add}{\mathbf{Add}}
\DeclareMathOperator{\ad}{ad}
\DeclareMathOperator{\adj}{adj}
\DeclareMathOperator{\Alt}{Alt}
\DeclareMathOperator{\an}{an}%Berkovich analytic space
\DeclareMathOperator{\ann}{ann}
\DeclareMathOperator{\Ann}{Ann}
\DeclareMathOperator{\Aut}{Aut}
\DeclareMathOperator{\Bil}{Bil}%Bilinear
\DeclareMathOperator{\ch}{ch}
\DeclareMathOperator{\Char}{char}
\DeclareMathOperator{\cls}{cls}
\DeclareMathOperator{\coim}{coim}
\DeclareMathOperator{\coker}{coker}
\DeclareMathOperator{\di}{d}
\DeclareMathOperator{\diag}{diag}
\DeclareMathOperator*{\dirlim}{\underrightarrow{\lim}}%direct limit
\DeclareMathOperator{\D}{D}
\DeclareMathOperator{\Der}{Der}%Derivations
\DeclareMathOperator{\End}{End}
\DeclareMathOperator{\ev}{ev}
\DeclareMathOperator{\et}{{\acute{e}t}}%\'{E}tale
\DeclareMathOperator{\Ext}{Ext}
\DeclareMathOperator{\filt}{\mathfrak{f}}%filtration  %\ff
\DeclareMathOperator{\Fct}{\mathbf{Fct}}%Functor
\DeclareMathOperator{\Fix}{Fix}%Fixed points
\DeclareMathOperator{\Forget}{Forget}
\DeclareMathOperator{\Free}{Free}
\DeclareMathOperator{\Fun}{\mathbf{Fun}}
\DeclareMathOperator{\gr}{gr}
\DeclareMathOperator{\Gal}{Gal}
\DeclareMathOperator{\Gr}{Gr}
\DeclareMathOperator{\Hess}{Hess}%Hessian
\DeclareMathOperator{\Hom}{Hom}%Hom bifunctor
\DeclareMathOperator{\Nat}{Nat}%Nat bifunctor
\DeclareMathOperator{\im}{im}
\DeclareMathOperator*{\invlim}{\underleftarrow{\lim}}%inverse limit
%\DeclareMathOperator*{\testsum}{L.M.}%inverse limit
\DeclareMathOperator{\id}{id}%Identity
\DeclareMathOperator{\Id}{\mathbf{I}}%Identity
\DeclareMathOperator{\Inn}{Inn}
\DeclareMathOperator{\Int}{Int}
\DeclareMathOperator{\IR}{IR}
\DeclareMathOperator{\Isw}{Isw}
\DeclareMathOperator{\Is}{\mathfrak{Is}}%Isolator
\DeclareMathOperator{\Lie}{\mathfrak{L}}%lie algebra
\DeclareMathOperator{\Mor}{Mor}
\DeclareMathOperator{\ML}{\mathbf{M.L.}}
\DeclareMathOperator{\nat}{nat}
\DeclareMathOperator{\N}{\mathbb{N}}
\DeclareMathOperator{\ob}{ob}%Object
\DeclareMathOperator{\obj}{\mathcal{T}}
\DeclareMathOperator{\ord}{ord}
\DeclareMathOperator{\op}{op}%opposite ring
\DeclareMathOperator{\Op}{\mathcal{O}\mathfrak{p}\mathrm{ext}}%the sets of congruence classes of extensions
\DeclareMathOperator{\Pic}{Pic}%Picard group
\DeclareMathOperator{\prim}{prim}
\DeclareMathOperator{\Proj}{Proj}
\DeclareMathOperator{\Quot}{Qout}%Quotient functor
\DeclareMathOperator{\rad}{rad}
\DeclareMathOperator{\Rad}{Rad}
\DeclareMathOperator{\Rt}{Rt}
\DeclareMathOperator{\rank}{rank}
\DeclareMathOperator{\sgn}{sgn}
\DeclareMathOperator{\Span}{span} % Annoyingly, \span is already a command in TeX, and redefining it leads to other problems.
\DeclareMathOperator{\Spec}{Spec}
\DeclareMathOperator{\Sw}{Sw} % Swan conductor
\DeclareMathOperator{\Split}{\mathbf{Split}}%Karoubi envelope
\DeclareMathOperator{\supp}{supp}
\DeclareMathOperator{\Tor}{Tor}
\DeclareMathOperator{\tor}{tor}
\DeclareMathOperator{\Tr}{Tr}

\DeclareMathOperator{\GL}{GL}
\DeclareMathOperator{\PGL}{PGL}
\DeclareMathOperator{\PSL}{PSL}
\DeclareMathOperator{\SL}{SL}
\DeclareMathOperator{\SO}{SO}
\DeclareMathOperator{\GO}{O}
\DeclareMathOperator{\SP}{Sp}
\DeclareMathOperator{\Spin}{Spin}
\DeclareMathOperator{\SU}{SU}
\DeclareMathOperator{\GU}{U}
\DeclareMathOperator{\Pt}{Pt}


%%%%%%%%%%%
%         url
%%%%%%%%%%%
\def\nlab{\href{http://ncatlab.org}{{$n$Lab}} }
\def\hrefacc{\href{http://katmat.math.uni-bremen.de/acc/}{\emph{The Joy of Cats}} }

\def\gl{\gg\ll}
\def\sl{\mathfrak{sl}}
\def\so{\ss\oo}
\def\sp{\ss\pp}
\def\su{\ss\uu}

\def\st{\textrm{ s.t. }}
\def\RP{\mathbb{R}\mathbf{P}}
\def\Real{\mathbb{R}}

\DeclareMathOperator{\Br}{Br}
\renewcommand{\mod}{\mathop{\mathrm{mod}}}

%
%                                          Notations
%

\loadglsentries{Glossaries/TableOfCats}
\loadglsentries{Glossaries/TableOfSymbols}
%%%%%%%%%%%%%%%%%%%%%%%%%%%%%%%%%%
%
%                                 PDF File Information
%
%%%%%%%%%%%%%%%%%%%%%%%%%%%%%%%%%%
\hypersetup{
             pdftitle={Seminar notes on Algebra},
             pdfauthor={Gau Syu},
             pdfsubject={algebra},
             pdfkeywords={seminar, notes},
             pdfproducer={XeLaTeX},
    colorlinks=true,
    citecolor=black,
    filecolor=black,
    linkcolor=black,
    urlcolor=black
}
%%%%%%%%%%%%%%%%%%%%%%%%%%%%%%%%%%
%
%                                           Title
%
%%%%%%%%%%%%%%%%%%%%%%%%%%%%%%%%%%
\title{{\giant Seminar notes on}\\ \Giant\textsc{Algebra}}
\author{Gau Syu}
\date{
  \begin{gather*}
    \xymatrix@=5em{
    & Y' \ar@{-->}[dr]^g \ar[ddl]^(0.65){[1]} & \\
    Z' \ar@{-->}[ur]^f \ar[d]_{[1]} & &
        X' \ar[ll]_{j[1] \circ i}^{[1]}|(0.25)\hole|(0.75)\hole
           \ar[ddl]_(0.4)i^(0.4){[1]}|(0.25)\hole|(0.5)\hole \\
    X \ar[rr]^{v \circ u} \ar[dr]_u & &
        Z \ar[u] \ar[uul]\\
   & Y \ar[ur]_v \ar[uul]_(0.6)j|(0.5)\hole|(0.75)\hole &
    }\\
    \\
    \\
  \end{gather*}
\emph{\small Last Update: \today}}

\begin{document}
\frontmatter
\maketitle
\dominitoc
\chapter*{\giant\Edward Preface}
\addstarredchapter{Preface}

这是Gau Syu《Seminar notes on Algebra》的中文版本,除术语与特定表述外,大部分内容采用中文。

这份笔记最初被设计为Serge Lang《Algebra》\cite{lang2002algebra}的习题解答,然而随着我在本笔记中不断引入各种其他来源的资料以及讨论班重点的转移,本笔记事实上已经变成另外的东西了。

除Serge Lang的书外,本笔记还参考了其他资料,特别是维基百科和$n$Lab,详见Bibliography。

\begin{flushright}
  \emph{Gau Syu}
\end{flushright}

\tableofcontents
\mainmatter
\pagestyle{plain}
\setcounter{minitocdepth}{2}
\part{Mathematical Structures}
  “数学结构”的观点出现于N.Bourbaki的《Elements of Mathematics》第一卷《Theory of Sets》第4章“Structures”。借助于公理化集合论,N.Bourbaki给出了一个比较一般性的“数学结构”以及“同构”的概念,尽管这套理论在N.Bourbaki自己的书中都没使用多少。

  “范畴论”在某种意义上是叙述“数学结构”的更自然的语言,对如今的数学领域乃至其他科学领域影响深远。故此在越来越多的现代代数教材中出现了介绍“范畴论”的章节。

  本部分收录一些基本的范畴论知识,它们中大部分可见于各类讲授范畴论的书籍。这方面标准的参考书是MacLane《Categories for the Working Mathematician》\cite{lane1998categories}。 此外,F.W.Lawvere\&S.H.Schanuel《Conceptual Mathematics》\cite{lawvere1997conceptual} 和S.Awodey《Category Theory》\cite{awodey2010category}都是简单易懂的教材,F.Borceux《Handbook of categorical algebra》\cite{borceux} 则是本资料丰富的大全。

  虽然原则上并不必要,但在阅读本部分之前掌握一些基本的代数学常识是有好处的,可参考
  R.戈德门特《代数学教程》\cite{godement1963cours}、N.Jacobson《Basic Algebra》\cite{jacobson1980basic}、M.Artin《Algebra》\cite{artin2011algebra} 或者任何一本大学程度的代数教材。

  \chapter{The Language of Categories}
  In this chapter, we start with the basic vocabulary of categories, functors, natural transformations, monomorphisms, epimorphisms, isomorphisms. The analogies between monomorphisms and epimorphisms, covariant and contravariant functors, lead to the famous duality principle which is, with the Yoneda lemma, one of the key results of the first chapter.
\minitoc
\newpage
\section{Introduction}
\subsection{Why category?}
  Many similar phenomena and constructions with similar properties occur in completely different mathematical fields.
  To describe precisely such phenomena and investigate such constructions simultaneously, the language of categories emerges.

  For past years, categorists have developed a symbolism that allows one quickly to visualize quite complicated facts by means of diagrams.

  Nowadays, category theory become a powerful language provides suitable vehicles that allow one to transport problems from one area of mathematics to another area, where solutions are sometimes easier. Therefore, the language of categories become more and more popular in modern mathematics and other fields like logic, computer science, linguistics and philosophy.

  More history comment can be found in many textbooks about category theory.
%Category Theory is a way of treating metaphor rigorously - in great complex nests, perhaps, but still rigorously.  And of having rigor without losing all meaning.

%These are the points on usefulness of category theory that Graham Hutton once mentioned in a course on category theory:
%Building bridges—exploring relationships between various mathematical objects, e.g., Products and Function
%Unifying ideas - abstracting from unnecessary details to give general definitions and results, e.g., Functors
%High level language - focusing on how things behave rather than what their implementation details are e.g. specification vs implementation
%Type safety - using types to ensure that things are combined only in sensible ways e.g. (f: A -> B  g: B -> C) => (g o f: A -> C)
%Equational proofs—performing proofs in a purely equational style of reasoning
\subsection{What is a category?}
  There are several ways to define what is a category; in the usual foundations of mathematics, these two definitions are equivalent. We now provide a popular one.
  \begin{defn}\label{def:category}
  A \termin{category} $\Cc$ consists of the following data:
  \begin{itemize}
    \item a collection $\ob\Cc$ of \termin[objects]{object (category theory)}.
    \item a collection $\hom\Cc$ of \termin[morphisms]{morphism (category theory)} (or \termin[arrows]{arrow (category theory)}, \termin[maps]{map (category theory)}) between objects.

             Each morphism $f$ has a unique \termin[source]{source (category theory)} object $A$ and \termin[target]{target (category theory)} object $B$.

             We write $f\colon A\To B$, and say ``$f$ is a morphism from $A$ to $B$'', ``$A$ is the \termin[domain]{domain (category theory)} of $f$'' and ``$B$ is the \termin[codomain]{codomain (category theory)} of $f$''.

             We write $\Hom(A, B)$\glsadd{hom} (or $\Hom_{\Cc}(A, B)$ when there may be confusion about to which category $\Hom(A, B)$ refers) to denote the collection of all morphisms from $A$ to $B$. (Some authors write $\Mor(A, B)$ or simply $\Cc(A,B)$ instead.)
    \item for every three objects $A,B$ and $C$, a binary operation
             \begin{equation*}
               \Hom(A, B) \times \Hom(B, C) \To \Hom(A, C)
             \end{equation*}
             called \termin{composition of morphisms}.

             The composition of $f\colon A \To B$ and $g\colon B \To C$ is written as $g\circ f$ or simply $gf$. (Some authors use ``diagrammatic order'', writing $f;g$ or $fg$.)
  \end{itemize}
  subject to the following axioms:
  \begin{description}
    \item[associativity] if $f\colon A \To B, g\colon B \To C$ and $h\colon C \To D$ then
                                 \begin{equation*}
                                   h\circ(g\circ f) = (h\circ g)\circ f
                                 \end{equation*}
    \item[identity] for every object $A$, there exists a morphism $1_A\colon A \To A$ (some times write $\id_A$) called the \termin[identity morphism]{identity morphism} for $A$, such that for every morphism $f\colon A \To B$, we have $1_B \circ f = f = f \circ 1_A$.\glsadd{identityM}
  \end{description}
  From these axioms, one can prove that there is exactly one identity morphism for every object. Some authors use a slight variation of the definition in which each object is identified with the corresponding identity morphism.
  \end{defn}
  \begin{rem}
     To emphasize the category $\Cc$, one often say an object (resp. morphism, arrow, map) in $\Cc$ or a $\Cc-$objects (resp. $\Cc-$morphism, $\Cc-$arrow, $\Cc-$map).
  \end{rem}
\begin{rem}
  Category theory provide a framework to discuss so-called ``categorical properties''. Informally, a \termin{categorical property} is a statement about objects and arrows in a category. More technically, a category provide a two-typed first order language with objects and morphisms as distinct types, together with the relations of an object being the source or target of a morphism and a symbol for composing two morphisms, thus a category property is a statement in such a language.
\end{rem}

  Let us now define a "homomorphism of categories".
  \begin{defn}
    A \termin{functor} $F$ from a category $\Aa$ to a category $\Bb$ consists of the following data:
    \begin{itemize}
      \item a mapping
                 \begin{equation*}
                   \ob\Aa\To\ob\Bb
                 \end{equation*}
                 between the collections of objects of $\Aa$ and $\Bb$; the image of $A\in\ob\Aa$ is written $F(A)$ or just $F A$;
      \item for every pair of objects $A, A'$ of $\Aa$, a mapping
                 \begin{equation*}
                   \Hom_{\Aa}(A,A')\To\Hom_{\Bb}(F(A),F(A'))
                 \end{equation*}
                 the image of $f\in\Hom_{\Aa}(A,A')$ is written $F(f)$ or just $F f$.
    \end{itemize}
    subject to the following axioms:
    \begin{itemize}
      \item for every pair of morphisms $f\in\Hom_{\Aa}(A,A'), g\in\Hom_{\Aa}(A',A'')$,
                 \begin{equation*}
                   F(g\circ f) = F(g)\circ F(f)
                 \end{equation*}
      \item for every object $A\in\ob\Aa$,
                 \begin{equation*}
                   F(1_A) = 1_{F(A)}
                 \end{equation*}
    \end{itemize}
  \end{defn}

  Given two functors $F\colon\Aa\to\Bb$ and $G\colon\Bb\to\Cc$, a pointwise composition immediately produces a new functor $G\circ F\colon\Aa\to\Cc$. This composition law is obviously associative.

  On the other hand, every category $\Cc$ has an identity functor, which can be simply obtained by choosing every mapping in the above definition to be the identity. It is an identity for the previous composition law.

  To summarize, the relationship between categories and functors is just like the one between objects and morphisms in a category. Thus a careless argument could lead to the conclusion that categories and functors constitute a new category, which is doubtful.

\subsection{Foundations and size issues}
    You may find that we use the undefined word ``collection'' instead of ``set'' or ``class''. The chosen of the word depends on the logical foundations we chose: when we use ZFC with one universe, then we choose the word ``set'' and all classical mathematical objects we considered such as sets, groups etc is small respect to the universe; when we use class theory like NBG, then we choose the word ``class'' and all classical mathematical objects are sets. More details can be found in the beginning of \cite{borceux} or section 1.8 of \cite{awodey2010category}.

    However, the foundation of category theory is independent of the axioms of set theory. In fact, category theory can be used to provide the foundations for mathematics as an alternative to set theory. Therefore we can rewrite the sentences of our axioms about what is a category in fully formal logic, which allow us to use the undefined word ``collection''. More details can be found in \nlab or \cite{lane1998categories}.

    Nevertheless, without set theory, the size-discussions such as ``a collection of all XXX'' do not make sense. Indeed, one of the lessons from the Russell's paradox is that the unrestricted usage of quantifiers is very dangerous.

    After choose an axiom system of set theory, the size issues make sense. In usual class theory such as NBG, a class is \emph{small} means it is a set, while in ZFC with one universe, a set is \emph{small} means it is an element of the universe.

    A category $\Cc$ is called \termin[small]{small category} if both $\ob\Cc$ and $\hom\Cc$ are small,
    and \termin[proper large]{proper large category} otherwise.
    A category is called \termin[locally small]{locally small category} if for each pair of objects $A$ and $B$, $\Hom(A, B)$ is small. Many important categories in mathematics (such as the category of sets), although not small, are at least locally small.
    For this reason, people tend to use the word ``category'' instead of ``locally small category''.
    Following this convention, we will simply put a locally small category called ``category''.
    This does not cause ambiguity, since we will still use the word ``\termin{large category}'' if needed.

    Under the above agreements, one can safely claim that all small categories and functors constitute a category, which is usually denoted by $\Cat$.
    On the other hand, the size of the ``category'' of all (locally small) categories and functors are larger than every large category, thus it does not exist under the usual set-theoretic axioms.

    Of course, one can also choose an axiom system of set theory which promise numerous sizes rather than just two: ``small'' and ``large''. In this case, all smaller categories and functors constitute a larger category. For instance, all locally small categories and functors form a ``very large'' (namely, larger than proper large categories) category, called $\CAT$: see \hrefacc i.e.\cite{acc}.

    More details about the foundations and size issues can be found in \nlab and related publications.

\newpage\section{Examples}
  The discussion of small categories and functors provide the first example of category in this book, i.e. $\Cat$. We should introduce some other examples in this section.

  In fact, traditional mathematics has provided a number of examples by different ways.
  \begin{exam}
    Many traditional mathematical structures are obtained by attaching some structures on sets. They provided a lot of obvious examples of categories.
    \begin{itemize}
      \item Sets
      \footnote{By this word we mean the small ones the set theory provided, e.g. small sets if we use ZFC with one universe. In such case, to avoid ambiguity, we use ``large set'' to call the general sets, which may be not small. A usual class theory already provides a word ``class'', thus one can also simply assume we have chose a usual class theory as our foundational set theory.}
       and functions: $\Set$.
      \item Groups and group homomorphisms: $\Grp$.
      \item Rings and ring homomorphisms: $\Ring$.
      \item Real vector spaces and linear mappings: $\Vect_{\RR}$.
      \item Right $R-$modules and module homomorphisms: $\Mod_R$.
      \item Topological spaces and continuous mappings: $\Top$.
      \item Uniform spaces and uniformly continuous functions: $\Uni$.
      \item Differentiable manifolds and smooth mappings: $\Diff$.
      \item Metric spaces and metric mappings: $\Met$.
      \item Real Banach spaces and bounded linear mappings: $\Banb$.
      \item Real Banach spaces and linear contractions: $\Ban$.
    \end{itemize}
    All of these categories encapsulate one ``kind of mathematical structure''. These are often called ``concrete'' categories (we will introduce a technical definition that these examples all satisfy later).
  \end{exam}
  \begin{exam}\label{exam:category2}
    Some mathematical devices can also be viewed as categories.
    \begin{itemize}
      \item $\Mat$: The set of natural numbers $\N$ can be viewed as a category as following: choose as objects the natural numbers and as arrows from $n$ to $m$ the matrices with $n$ rows and $m$ columns; the composition is the usual product of matrices.
      \item Every set $S$ can be viewed as a category whose objects are the elements of $S$ and the only morphisms are identities.

                 In general, a category whose only morphisms are the identities is called a \termin{discrete category}.
      \item A poset $(S,<)$ can be viewed as a category whose objects are the elements of $S$ and the set $\Hom(x,y)$ of morphisms is a singleton when $x < y$ and is empty otherwise.
                The possibility of defining a (unique) composition law is just the transitivity axiom of the partial order; the existence of identities is just the reflexivity axiom.
      \item A monoid $(M,\cdot)$ can be seen as a category $\Mm$ with a single object $\ast$ and $\Hom(\ast,\ast)=M$ as a set of morphisms; the composition law is just the multiplication of the monoid.
    \end{itemize}
  \end{exam}

  Do not worry if some of these examples are unfamiliar to you. Later on, we will take a closer look at some of them. In addition, there are also many examples emerge from outside of mathematics, such as logic, computer science, linguistics and philosophy. Those examples can be found in \cite{awodey2010category} and related publications.

\subsection{Basic examples via a given category}
From a given category $\Cc$, there are various ways to construct new categories. Here are some basic constructions.
\begin{exam}
  Any category $\Cc$ can itself be considered as a new category in a different way: the objects are the same as those in the original category but the arrows are those of the original category reversed. This is called the \termin[dual]{dual category} or \termin{opposite category} and is denoted by $\Cc^{\op}$.\glsadd{opposite}
\end{exam}
Obviously, the double dual category of a category is itself. The concept of dual category implies an important principle, the \emph{duality principle}. We will see soon it in this chapter.
\begin{exam}[\termin{slice category}]
  Let us fix an object $I\in\ob\Cc$. The category $\Cc/I$ of ``arrows over $I$'' is defined by the following.\glsadd{Slice}
  \begin{itemize}
    \item Objects: the arrows of $\Cc$ with codomain $I$.
    \item Morphisms from an object $(f\colon A\to I)$ to another $(g\colon B\to I)$:
               the morphisms $h\colon A\to B$ in $\Cc$ satisfying the ``commutative triangles over $I$''
               \begin{displaymath}
                 \xymatrix@R=0.5cm{
                    A\ar[rr]^{h}\ar[dr]_{f} && B\ar[dl]^{g} \\
                    &I&                }
               \end{displaymath}
               i.e.  $g\circ h = f$.
    \item   The composition law is that induced by the composition of $\Cc$.
  \end{itemize}
\end{exam}
  \begin{rem}
    Notice that in the case of $\Set$, a function $f\colon A\to I$ can be identified with the $I-$indexed family of sets $\{f^{-1}(i)\}_{i\in I}$ so that the previous category is just that of $I-$indexed families of sets and $I-$indexed families of functions.
  \end{rem}

\begin{exam}[\termin{coslice category}]
  Again fixing an object $I\in\ob\Cc$, the category $I/\Cc$ of ``arrows under $I$'' is defined by the following.\glsadd{Coslice}
  \begin{itemize}
    \item Objects: the arrows of $\Cc$ with domain $I$.
    \item Morphisms from an object $(f\colon I\to A)$ to another $(g\colon I\to B)$:
               the morphisms $h\colon A\to B$ in $\Cc$ satisfying the ``commutative triangles under $I$''
               \begin{displaymath}
                 \xymatrix@R=0.5cm{
                    &I\ar[dl]_{f}\ar[dr]^{g}& \\
                    A\ar[rr]_{h} && B   }
               \end{displaymath}
               i.e.  $h \circ f = g$.
    \item   The composition law is that induced by the composition of $\Cc$.
  \end{itemize}
\end{exam}

\begin{exam}
  Now we consider all the arrows of $\Cc$. The category $\Cc^{\to}$  of all arrows is defined by the following. \glsadd{Arr}
  \begin{itemize}
    \item Objects: the arrows of $\Cc$.
    \item A morphism from an object $(f\colon A \to B)$ to another $(g\colon C \to D)$ is a pair $(h\colon A\to C, k\colon B\to D)$ of morphisms of $\Cc$ satisfying a ``commutative square''
               \begin{displaymath}
                 \xymatrix{
                    A\ar[d]_{h}\ar[r]^{f} & B\ar[d]^{k} \\
                    C\ar[r]_{g}&D                }
               \end{displaymath}
    \item   The composition law is induced pointwise by the composition in $\Cc$.
  \end{itemize}
\end{exam}

\subsection{The simplest examples}
We now introduce some simple examples which can be inspired by the axioms of categories immediately.
\begin{exam}
  The category $\mathbf{0}$ has no objects or arrows.

  The category $\one$ has one object and its identity arrow. It looks like
               \begin{displaymath}\glsadd{terminalCate}
                 \xymatrix@R=0.5cm{
                    \bullet              }
               \end{displaymath}

  The category $\mathbf{2}$ has two objects, their required identity arrows, and exactly one arrow between the objects. It looks like
               \begin{displaymath}
                 \xymatrix@R=0.5cm{
                    \bullet\ar[r] & \bullet             }
               \end{displaymath}

  The category $\mathbf{3}$ has three objects, their required identity arrows, exactly one arrow from the first to the second object, exactly one arrow from the second to the third object, and exactly one arrow from the first to the third object (which is therefore the composite of the other two). It looks like
               \begin{displaymath}
                 \xymatrix@R=0.5cm{
                    \bullet\ar[r]\ar[dr] & \bullet\ar[d] \\
                    & \bullet            }
               \end{displaymath}
\end{exam}

One can see these categories look like quivers. Indeed, a category can be viewed a quiver with extra structure. More details can be found in \nlab.

The notation $\one,\mathbf{2},\mathbf{3}$ comes from the fact that these categories are in fact the ordinals regarded as posets and thus categories. We will discuss this in the later part.

\newpage\section{Monic-, epi- and isomorphisms}
The basic notions about functions are injectivity, surjectivity and bijectivity. To generalize them to the context of category theory, one need characterize them without using elements. Early category theorists believed that the cancellation properties provide the correct generalization, thus they define the following notions.
  \begin{defn}
    A morphism $f$ is called \termin{monic}, or a \termin{monomorphism}, if it is left cancellable, that means
    for any morphisms
    \begin{displaymath}
    \xymatrix@1{\cdot\ar@<0.5ex>[r]^{\alpha}\ar@<-0.5ex>[r]_{\beta} &\cdot\ar[r]^{f} &\cdot}
    \end{displaymath}
    $f\alpha=f\beta$ implies $\alpha=\beta$.

    Dually, $f$ is called \termin{epi}, or an \termin{epimorphism}, if it is right cancellable, that means for any morphisms
    \begin{displaymath}
    \xymatrix@1{\cdot\ar[r]^{f} &\cdot\ar@<0.5ex>[r]^{\alpha}\ar@<-0.5ex>[r]_{\beta} &\cdot}
    \end{displaymath}
    $\alpha f=\beta f$ implies $\alpha=\beta$.
  \end{defn}

  The followings are the basic propositions about monomorphisms and epimorphisms.
  \begin{prop}
    every identity morphism is both monic and epi. The composite of two monomorphisms (resp. epimorphisms) is also a monomorphism (resp. epimorphism).
  \end{prop}
  \begin{prop}[Triangle lemma]\label{prop:triangle lemma}
    If the composite $g\circ f$ of two morphisms is monic, then so is $f$; if the composite is epi, then so is $g$.
  \end{prop}

  By the definition, a morphism is monic means it is left cancellable, but this does not implies the existence of its left inverse. For epimorphisms, the argument is similar.
  \begin{defn}
    Consider two morphisms $f\colon A \to B$ and $g\colon B \to A$ in a category. When $g \circ f = 1_A$, $f$ is called a \termin[section]{section (category theory)} of $g$, $g$ is called a \termin[retraction]{retraction (category theory)} of $f$ and $A$ is called a \termin[retract]{retract (category theory)} of $B$.
  \end{defn}
  \begin{prop}
    In a category, every section is a monomorphism and every retraction is an epimorphism.
  \end{prop}
  \begin{defn}
    A morphism $f\colon A\to B$ is called a \termin{split monomorphism}, if it has a retraction.
    Dually, $f$  is called a \termin{split epimorphism}, if it has a section.
    If $f$ is both split monic and split epi, then we say it is an \termin{isomorphism} and $A$ is \termin{isomorphic} to $B$, denoted by $A\approx B$.  \glsadd{isomorphic}
  \end{defn}
  \begin{rem}
    It is clear that any split monomorphism must be monic and any split epimorphism is epi, hence any isomorphism is both monic and epi. However, the converses are not true in general case.

    A morphism which is both monic and epi is traditionally called a \termin{bimorphism}, although this is a bad name cause confusions when we go to higher category theory. A category in which every bimorphism is an isomorphism is called \termin[balanced]{balanced category}.
  \end{rem}

\subsection{Examples}
  \begin{exam}
    In $\Set$, $\Grp$, $\Mod_R$ or $\Top$, monomorphisms (resp. epimorphisms) are precisely those morphisms that have injective (resp. surjective) underlying functions. The verifications will be given in corresponding chapters.
  \end{exam}
  \begin{exam}\label{exam:nonsurjective epi}
    In $\Ring$, monomorphisms are precisely those morphisms that have injective underlying functions. However, epimorphisms are not necessarily surjective. For instance, the inclusion map $\ZZ\hookrightarrow\QQ$ is a non-surjective epimorphism.
    To see this, note that any ring homomorphism on $\QQ$ is determined entirely by its action on $\ZZ$. A similar argument shows that the canonical ring homomorphism from any commutative ring $R$ to any one of its localizations is an epimorphism.
  \end{exam}
  \begin{exam}\label{exam:noninjective mono}
    In the category $\DivAb$ of divisible abelian groups and group homomorphisms between them, there are monomorphisms that are not injective:
    consider, for example, the quotient map $q\colon\QQ\to\QQ/\ZZ$ from additive group of rational numbers. This is obviously not an injective map. Nevertheless, it is a monomorphism in this category.
    Indeed, choose $G$ a divisible abelian group and $f,g\colon G \to\QQ$ two group homomorphisms such that $q\circ f = q\circ g$ Putting $h = f - g$ we have $q\circ h = 0$ and the thesis becomes $h=0$. Given an element $x\in G$, $h(x)$ is an integer since $q \circ h = 0$. If $h(x)\neq0$, then one can easily find a contradiction.
  \end{exam}
  \begin{exam}
    In the category of connected pointed topological spaces and pointed continue maps, every coverings map are monomorphisms although they are usually not injective. This is just the unique lifting property of covering maps, one can find it in a textbook about algebraic topology, for example, \cite{AllenHatcher}.
  \end{exam}
  \begin{exam}
    In $\Set$, $\Grp$, $\Mod_R$, isomorphisms are precisely those morphisms that have bijective underlying functions.
    In $\Top$, isomorphisms are exactly the homeomorphisms. Unfortunately, a continuous bijection is in general not a homeomorphism. For instance, the map from the half-open interval $[0,1)$ to the unit circle $S^1$ (thought as a subspace of the complex plane) which sends $x$ to $e^{2\pi x i}$ is continuous and bijective but not a homeomorphism since the inverse map is not continuous at $1$.
  \end{exam}

  The counter-examples above shows that the notions of monomorphism and epimorphism do not meet the original requirements, thus category theorists develop some variations to fix this. One can find them in \nlab or \hrefacc.

\newpage\section{Natural transformations}
  Just as the study of groups is not complete without a study of homomorphisms, so the study of categories is not complete without the study of functors.
  However, the study of functors is itself not complete without the study of the morphisms between them: the natural transformations.

  \begin{defn}
    Consider two functors $F,G$ from a category $\Aa$ to $\Bb$. A \termin{natural transformation} $\alpha\colon F\then G$ from $F$ to $G$ is a famliy of morphisms\glsadd{naturalTrans}
    \begin{equation*}
    (\alpha_A\colon F(A)\To G(A))_{A\in\ob\Aa}
    \end{equation*}
    of $\Bb$ indexed by the objects in $\Aa$ satisfying the following commutative diagrams for every morphism $f\colon A \to A'$ in $\Aa$
    \begin{displaymath}
      \xymatrix{
         F(A)\ar[r]^{\alpha_A}\ar[d]_{F(f)}&G(A)\ar[d]^{G(f)}\\
         F(A')\ar[r]^{\alpha_{A'}}&G(A')
      }
    \end{displaymath}
    i.e. $\alpha_{A'}\circ F(f) = G(f) \circ \alpha_{A}$.
  \end{defn}

  Let $F,G,H$ be functors from $\Aa$ to $\Bb$ and $\alpha\colon F\then G, \beta\colon G\then H$ be natural transformations. Then the formula
  \begin{equation*}
    (\beta\circ\alpha)_A = \beta_A\circ\alpha_A
  \end{equation*}
  defines a new natural transformation $\beta\circ\alpha\colon F\then H$.

  This composition law is clearly associate and possesses identity for each functor.
  Thus, a careless argument would deduce the existence of a category whose objects are the functors from $\Aa$ to $\Bb$ and whose morphisms are the natural transformations between them. Such a category is called the \termin{functor category} from $\Aa$ to $\Bb$ and usually denoted by $[\Aa,\Bb]$ or $\Bb^{\Aa}$. \glsadd{Fun}
  \begin{rem}
    We say this argument is careless since there is a size issue:

    If $\Aa$ and $\Bb$ are \emph{small}, then $[\Aa,\Bb]$ is also \emph{small}.

    If $\Aa$ is \emph{small} and $\Bb$ is \emph{locally small}, then $[\Aa,\Bb]$ is still \emph{locally small}.

    Even if $\Aa$ and $\Bb$ are \emph{locally small}, if $\Aa$ is not \emph{small}, then $[\Aa,\Bb]$ will usually not be \emph{locally small}.

    As a partial converse to the above, if $\Aa$ and $[\Aa,\Set]$ are \emph{locally small}, then $\Aa$ must be \emph{essentially small}: see \href{http://tac.mta.ca/tac/volumes/1995/n9/1-09abs.html}{\emph{Freyd \& Street (1995)}}.
  \end{rem}

  In the above discussion, we have used a first composition law for natural transformations.
  In fact, there exists another possible type of composition for natural transformations.

  \begin{prop}
    Consider the following situation:
      \begin{displaymath}
        \xymatrix{
           \Aa\rtwocell^{F}_{G}{\alpha} &\Bb\rtwocell^{F'}_{G'}{\beta} & \Cc
        }
      \end{displaymath}
    Where $\Aa,\Bb,\Cc$ are categories, $F,G,F',G'$ are functors and $\alpha,\beta$ are natural transformations.

    First, we have the composite functors $F'F$ and $G'G$ and a commutative square for every object $A\in\ob\Aa$:
     \begin{displaymath}
        \xymatrix{
           F'F(A)\ar[r]^{F'(\alpha_A)}\ar[d]_{\beta_{F(A)}} & F'G(A)\ar[d]^{\beta_{G(A)}}\\
           G'F(A)\ar[r]^{G'(\alpha_A)} & G'G(A)
        }
    \end{displaymath}

    Now define $(\beta\ast\alpha)_A$ to be the diagonal of this square, i.e.
    \begin{equation*}
      (\beta\ast\alpha)_A = \beta_{G(A)}\circ F'(\alpha_A) = G'(\alpha_A)\circ\beta_{F(A)}
    \end{equation*}
    Then $\beta\ast\alpha$ is also a natural transformation, called the \termin{Godement product} of $\alpha$ and $\beta$. \glsadd{Godpord}
  \end{prop}

  Use the naturalities and functorialities, one can easily check this proposition and also the following.

  \begin{prop}[Interchange law]
  Consider this situation
      \begin{displaymath}
        \xymatrix{
          \Aa \ruppertwocell^{}_{}{\alpha} \rlowertwocell^{}_{}{\beta} \ar[r]
          & \Bb\ruppertwocell^{}_{}{\alpha'} \rlowertwocell^{}_{}{\beta'} \ar[r]
          & \Cc
        }
      \end{displaymath}
    Where $\Aa,\Bb,\Cc$ are categories and $\alpha,\beta, \alpha', \beta'$ are natural transformations. Then the following equality holds:
    \begin{equation*}
      (\beta'\circ\alpha')\ast(\beta\circ\alpha) = (\beta'\ast\beta)\circ(\alpha'\ast\alpha)
    \end{equation*}
  \end{prop}

  For the sake of brevity and with the notations of the previous propositions, we shall often write $\beta\ast F$ instead of $\beta\ast1_{F}$ or $G\ast\alpha$ instead of $1_{G}\ast\alpha$.

\subsection{Remarks on naturality}
  You may have seen the word ``\emph{natural}'' on different occasions. But what does this word mean? Intuitively, it makes reference to a description which is independent of any choices.

  Category theory offer a formal definition.

  Recall such a word usually occurs in a satiation where something be transformed into another. To say this process is \emph{natural}, in the sense of category theory, means it can be realized by a \emph{natural transformation}.

  For instance, the term ``naturally isomorphic'' can be formalized by
  \begin{defn}
    Let $\alpha$ be a natural transformation between two functors $F$ and $G$ from the category $\Aa$ to $\Bb$.
    If, for every object $A$ in $\Aa$, the morphism $\alpha_A$ is an isomorphism in $\Bb$, then $\alpha$ is said to be a
    \termin{natural isomorphism}, and $F$ and $G$ are said to be \termin{naturally isomorphic}, denoted by $F\cong G$.
  \end{defn}

  \begin{exam}[Opposite group]
    Statements such as
    \begin{quote}
      ``Every group is \emph{naturally isomorphic} to its opposite group''
    \end{quote}
    abound in modern mathematics.

    What the above statement really means is:
    \begin{quote}
      ``The identity functor $\Id \colon \Grp \To \Grp$ is \emph{naturally isomorphic} to the opposite functor $\op \colon \Grp \To \Grp$.''
    \end{quote}

    Such a translation also automatically provide a proof to the original statement.
  \end{exam}

  \begin{exam}[Double dual]
    Let $k$ be a field, then for every vector space $V$ over $k$ we have a ``natural'' injective linear map $V \To V^{\ast\ast}$ from $V$ into its double dual. These maps are ``natural'' in the following sense: the double dual operation is a functor, and the maps are the components of a natural transformation from the identity functor to the double dual functor.
  \end{exam}

  However, ``unnatural'' isomorphisms also abound in traditional mathematics.
  \begin{exam}[Dual of finite-dimensional vector spaces]
  Origin:
  \href{http://en.wikipedia.org/wiki/Natural_transformation#Example:_dual_of_a_finite-dimensional_vector_space}{\emph{Wikipedia}}
  Revised by:
  \href{http://mathoverflow.net/a/139398/43771}{\emph{MathOverflow}}

   Every finite-dimensional vector space is isomorphic to its dual space, but this isomorphism is not natural.

   One reason, which is given by \emph{Wikipedia}, is that this isomorphism relies on an arbitrary choice of isomorphism. However, this is a completely different matter and has nothing to do with naturality: the linear dual is a contravariant functor while the identity functor on $\FinVect_{k}$ is covariant, thus there is no possibility to compare them via a natural transformation.

   A more acceptable reason comes from a poset in \emph{MathOverflow}. Dan Petersen pointed that if we just consider the category of finite-dimensional vector spaces and linear isomorphisms, temporarily denote it by $\Cc$. Then there are two obvious functors $\Cc\to\Cc^{\op}$: the linear dual, and the natural isomorphism $\Cc\to\Cc^{\op}$ maps each linear isomorphism to its inverse. These functors are \textbf{unnaturally isomorphic}.

   However, take as objects finite-dimensional vector spaces with a nondegenerate bilinear form and maps linear transforms that respect the bilinear form. Then the resulting category has a natural isomorphism from the linear dual to the identity.
  \end{exam}

  To formalize the ideal that some isomorphisms are not natural, one can introduce the notion of \termin{infranatural transformation}, which is just a family of morphisms indexed by the objects in the source category. Thus an \termin{unnatural isomorphism} is just an infranatural isomorphism which is not natural.

  \begin{exam}
    Quote from \href{http://mathoverflow.net/a/139392/43771}{\emph{MathOverflow}}

    Take $\Cc$ to be the category with one object and two morphisms. Then the identity functor is \textbf{unnaturally isomorphic} to the functor that sends both morphisms to the identity map.
  \end{exam}

  In practice, a particular map between individual objects is said to be a \textbf{natural isomorphism}, meaning implicitly that it is actually defined on the entire category, and defines a natural transformation of functors, otherwise, an \textbf{unnatural isomorphism}.
  \begin{rem}
    Some authors distinguish notations, using $\cong$ for natural isomorphisms and $\approx$ for isomorphisms may not be natural, reserving $=$ for equalities.
  \end{rem}

  The examples in \emph{Wikipedia} may not fit, since what they compared are in fact functors with different domains.
   Some right counterexamples can be found in this poset in \href{http://mathoverflow.net/questions/139388/example-of-an-unnatural-isomorphism}{\emph{MathOverflow}}.

\newpage\section{Contravariant functors}
  Sometimes, we will consider a mapping between categories which reverse the arrows, for instance, the inverse image mapping of functions. Such kind of mappings are essentially functors from the dual of the ordinary category, while people used to image them as ``functors'' from the ordinary one for brevity.
  For this reason, categorists introduce the concept of contravariant functors.
  \begin{defn}
    Let $\Aa, \Bb$ be two categories, a \termin{contravariant functor} from $\Aa$ to $\Bb$ is a functor from $\Aa^{\op}$ to $\Bb$.
  \end{defn}
  Ordinary functors are also called \termin{covariant functor} in order to distinguish them from \emph{contravariant} ones.
  \begin{exam}
    A contravariant functor from a category $\Cc$ to $\Set$ is traditionally called a \termin{presheaf} on $\Cc$. The category of presheaves on $\Cc$ is denoted by $\PSh(\Cc)$. More generally, it is frequently to call a contravariant functor from $\Cc$ to $\Dd$ a $\Dd-$valued presheaf on $\Cc$.
  \end{exam}
  \begin{rem}
    For the reason of size issue, people often require the domain of a presheaf to be small.
  \end{rem}
  One define the natural transformations between contravariant functors similarly as the covariant case.
  \begin{defn}
    Consider two contravariant functors $F,G$ from a category $\Aa$ to a category $\Bb$. A \termin{natural transformation} $\alpha\colon F\then G$ from $F$ to $G$ is a famliy of morphisms
    \begin{equation*}
    (\alpha_A\colon F(A)\To G(A))_{A\in\ob\Aa}
    \end{equation*}
    of $\Bb$ indexed by the objects in $\Aa$ satisfying the following commutative diagrams for every morphism $f\colon A \to A'$ in $\Aa$
    \begin{displaymath}
      \xymatrix{
         F(A)\ar[r]^{\alpha_A}&G(A)\\
         F(A')\ar[r]^{\alpha_{A'}}\ar[u]^{F(f)}&G(A')\ar[u]_{G(f)}
      }
    \end{displaymath}
    i.e. $\alpha_{A}\circ F(f) = G(f) \circ \alpha_{A'}$.
  \end{defn}

  All results about functors can be transported to the contravariant case. One can easily check that, or just apply the duality principle to obtain this transposition.

  The similar idea that take both the dual of source and target categories provides the following notion.
  \begin{defn}
    Every functor $F\colon\Aa\To\Bb$ induces the \termin{opposite functor} $F^{\op}\colon \Aa^{\op} \To \Bb^{\op}$ maps objects and morphisms identically to $F$.  \glsadd{oppositeF}
  \end{defn}
  Although $F^{\op}$ works as $F$, it can be distinguished from $F$ since $\Aa^{\op}$ does not coincide as $\Aa$ as categories and similarly for $\Bb^{\op}$. The motivation to define opposite functor is similar as opposite ring, it inverse the order of composition law.

\subsection{Examples of functors}
  Here is some simple examples of functors and contravariant functors.
  \begin{exam}
    Every category $\Cc$ has an identity functor $\Id_{\Cc}$ as its identity under the composition of functors. \glsadd{identityF}
  \end{exam}
  \begin{exam}
    For any ``concrete'' category, for instance $\Grp$, there is a functor from it to $\Set$, called the \termin{forgetful functor}: it maps a group $G$ to the underlying set $G$ and a homomorphism $f$ to the corresponding function $f$. We will introduce the technical definition of concrete category and the forgetful functor later.
  \end{exam}
  \begin{exam} \glsadd{powerF} \glsadd{copowerF}
    The \emph{power-set functor} $\Pp\colon\Set\to\Set$ from the category of sets to itself is obtained by mapping a set $S$ to its power set $\Pp(S)$ and a function $f\colon A\to B$ to the ``direct image mapping'' from $\Pp(A)$ to $\Pp(B)$.

    Its duality, the \emph{contravariant power-set functor} $\Qq$ maps a set $S$ to its power set $\Pp(S)$ and a function $f\colon A\to B$ to the ``inverse image mapping'':
    \mapdes{\Pp(B)}{\Pp(A)}{U}{f^{-1}(U)}
  \end{exam}
  \begin{exam}\glsadd{constantF}
    The functor $\Delta_B\colon\Aa \To \Bb$ which maps every object of $\Aa$ to a fixed object $B$ in $\Bb$ and every morphism in $\Aa$ to the identity morphism on $B$ is called a \termin{constant functor} or \termin{selection functor} to $B$.
  \end{exam}
  \begin{exam}\glsadd{diagonalF}
    The \termin{diagonal functor} $\Delta$ is defined as the functor from $\Bb$ to the functor category $[\Aa,\Bb]$ which sends each object $B$ in $\Bb$ to the constant functor to $B$.
  \end{exam}
  \begin{exam}\label{exam:hom bifunctor}
    $\Hom(-,-)$ can itself be viewed as two functors:

    Fix an object $A$ in $\Cc$, then $X\mapsto\Hom(A,X)$ defined a functor from $\Cc$ to $\Set$ which maps a morphism $f\colon X\to Y$ to the the mapping
    \longmapdes{f_{\ast}}{\Hom(A,X)}{\Hom(A,Y)}{\phi}{f\circ\phi}
    \glsadd{pushforward}

    Fix an object $B$ in $\Cc$, then $X\mapsto\Hom(X,B)$ defined a contravariant functor from $\Cc$ to $\Set$ which maps a morphism $f\colon X\to Y$ to the the mapping
    \longmapdes{f^{\ast}}{\Hom(Y,B)}{\Hom(X,B)}{\phi}{\phi\circ f}
    \glsadd{pullback}

    Moreover, one can prove that, for every morphisms $A\to B$ and $C\to D$ in $\Cc$, the following diagram is commutative:
    \begin{displaymath}
      \xymatrix{
         \Hom(A,C)\ar[r]&\Hom(A,D)\\
         \Hom(B,C)\ar[r]\ar[u]&\Hom(B,D)\ar[u]
      }
    \end{displaymath}
  \end{exam}

\newpage\section{Full and faithful functors}
  One may find that a ``concrete'' category seems can be included in $\Set$ via the forgetful functor. But such a ``inclusion'' is not like coincide those in usual sense because it is not usually injective. For instance, there are in general many different group structures on a same set. In fact, such a functor is what we call ``faithful functor''.
  \begin{defn}
  A Functor $F\colon \Aa\to\Bb$ is said to be
  \begin{enumerate}
%    \setlength{\itemindent}{2ex}
    \item \termin[faithful]{faithful functor} (resp. \termin[full]{full functor}, resp. \termin[fully faithful]{fully faithful functor}) if for any $X,Y\in\ob\Aa$, the map $\Hom_{\Aa}(X,Y)\to\Hom_{\Bb}(F(X),F(Y))$ is injective (resp. surjective, resp. bijective).
    \item \termin[essentially surjective]{essentially surjective functor} if for each $B\in\ob\Bb$, there exists $A\in\ob\Aa$ and an isomorphism $F(A)\approx B$.
  \end{enumerate}
  \end{defn}
  \begin{rem}
  A faithful functor need not to be injective on objects or morphisms. That is, two objects $X$ and $X'$ may map to the same object in $\Bb$ (which is why the range of a fully faithful functor is not necessarily equivalent to $\Aa$),
  and two morphisms $f \colon X\To Y$ and $f' \colon X'\To Y'$ (with different domains and codomains) may map to the same morphism in $\Bb$.

  Likewise, a full functor need not be surjective on objects or morphisms. There may be objects in $\Bb$ not of the form $F(A)$ for some $A$ in $\Aa$. Morphisms between such objects clearly cannot come from morphisms in $\Aa$.
  \end{rem}

  However, we have
  \begin{prop}
    A functor is injective on morphisms if and only if it is faithful and injective on objects.
    Such a functor is called an \termin[embedding]{embedding (category theory)}.
  \end{prop}

  The basic facts about the above notions are the following
  \begin{prop}\label{prop:tri-(full,faithful)}
    Let $F\colon\Aa\to\Bb$ and $G\colon\Bb\to\Cc$ be functors.
    \begin{enumerate}
      \item If $F$ and $G$ are both isomorphisms (resp. embeddings, faithful, or full), then so is $G\circ F$.
      \item If $G\circ F$ is an embedding (resp. faithful), then so is $F$.
      \item If $F$ is essentially surjective and $G\circ F$ is full, then $G$ is full.
    \end{enumerate}
  \end{prop}
  \begin{proof}
    \emph{1} is obvious. Apply the triangle lemma to hom-sets, we get \emph{2}.

    Under the condition of \emph{3}, each object $B$ in $\Bb$ is isomorphic to some $F(A)$ for $A\in\ob\Aa$. Thus a morphism $h\colon G(B)\to G(B')$ in $\Cc$ is isomorphic to a morphism $h'\colon GF(A)\approx G(B)\to G(B')\approx GF(A')$. By the fullness of $G\circ F$, $h'=GF(f)$ for some morphism $f\colon A\to A'$ in $\Aa$, thus we get a morphism $g\colon B\to B'$ such that $G(g)=h$.
  \end{proof}

  Full functors and faithful functors have good properties.
  \begin{prop}
    Let $F\colon\Aa\to\Bb$ be a faithful functor, then
    \begin{enumerate}
      \item it \termin[reflects monomorphisms]{reflect (category theory)}. That is, for every morphism $f$ in $\Aa$, $F(f)$ is monic implies $f$ is monic.
      \item it \textbf{reflects epimorphism}. That is, for every morphism $f$ in $\Aa$, $F(f)$ is epi implies $f$ is epi.
      \item if it is also full, then it \textbf{reflects isomorphisms}.
    \end{enumerate}
  \end{prop}
  \begin{proof}
    Let $F(f)$ be monic, then for every morphisms $g,h$ such that $f\circ g=f\circ h$, we have $F(f)\circ F(g) = F(f)\circ F(h)$, thus $F(g)=F(h)$. Since $F$ is faithful, $g=h$. The proof of \emph{2} is similar.

    Let $F(f)$ be an isomorphism with inverse $g$. Since $F$ is full, $g=F(h)$ for some $\Aa-$morphism $h$. Then $h$ is the inverse of $f$ since $F$ is faithful.
  \end{proof}

  \begin{cor}
    A fully faithful functor $F\colon\Aa\to\Bb$ is necessarily injective on objects up to isomorphism. That is, every objects $A,A'$ in $\Aa$ which are mapped to isomorphic objects in $\Bb$ must be isomorphic.
  \end{cor}

  One can easily check that every functor $F\colon\Aa\to\Bb$ must \termin[preserve isomorphisms]{preserve (category)}, that is, for every isomorphism $f$ in $\Aa$, $F(f)$ is an isomorphism.
  However, the conditions to preserve monomorphisms and epimorphisms are much stronger. In fact, even fully faithful functors can not promise this.
  But if we assume $F$ to be both fully faithful and essentially surjective, one can verify it do preserve monomorphisms and epimorphisms.
  \begin{defn}
    A functor is said to be a \termin{weak equivalence}, if it is fully faithful and essentially surjective.
  \end{defn}

  In fact, weak equivalence preserves and reflects more than just those properties mentioned above, by suitable assumptions on foundations, it keeps every interesting categorical properties, thus plays an important role in modern mathematics.

\subsection{Equivalence of categories}
  To characterize the concept that two categories share the same properties, the most natural way is consider the isomorphisms in $\Cat$ (or $\CAT$, more generally). Thus, we define
  \begin{defn}
    Two categories $\Aa$ and $\Bb$ are said to be \termin[isomorphic]{isomorphic categories}, if there exists two functors $F\colon\Aa\to\Bb$ and $G\colon\Bb\to\Aa$ which are mutually inverse to each other.
  \end{defn}
  One can easily check that the inverse of an isomorphism $F$ must be unique, thus usually denoted by $F^{-1}$.

  \begin{prop}
    A functor is an isomorphism if and only if it is full, faithful, and bijective on objects.
  \end{prop}

  \begin{exam}
  The basic fact in the representation theory of finite groups is that the category of linear representations of a finite group is isomorphic to the category of left modules over the corresponding group algebra.
  \end{exam}

  However, such example is rare since the condition to be isomorphic is too strong to be satisfied in practice.
  A more pragmatic notion is the equivalence of categories.
  \begin{defn}
    Two categories $\Aa$ and $\Bb$ are said to be \termin{equivalent}, if there exists a pair of functors $F\colon\Aa\to\Bb$ and $G\colon\Bb\to\Aa$ such that there are natural isomorphisms $F\circ G\cong \Id_{\Bb}$ and $\Id_{\Aa}\cong G\circ F$. In this case, we say $F$ is an \termin{equivalence} from $\Aa$ to $\Bb$ with a \termin{weak inverse} $G$.
  \end{defn}
  \begin{rem}
    Knowledge of an equivalence is usually not enough to reconstruct its weak inverse and the natural isomorphisms: there may be many choices. (see examples below)
  \end{rem}
  \begin{rem}
    There is no standard notation of equivalentness, $\Aa\equiv\Bb$ and $\Aa\simeq\Bb$\glsadd{equivalent} are frequently used.
  \end{rem}

  The most obvious relationship between equivalence and weak equivalence is the following.
  \begin{prop}
    An equivalence must be a weak equivalence.
  \end{prop}
  \begin{proof}
    Let $F\colon\Aa\to\Bb$ be an equivalence with a weak inverse $G$ and two natural isomorphism $\Id_{\Aa}\iso{\alpha} G\circ F, F\circ G \iso{\beta} \Id_{\Bb}$.
    The essential surjectivity follows from $\beta$, full faithfulness follows from $\alpha$.
  \end{proof}

  Before going forward, we prove some basic properties of weak equivalences.
  \begin{prop}
    If $F\colon\Aa\to\Bb$ and $G\colon\Bb\to\Cc$ are weak equivalences, then so is $G\circ F$.
  \end{prop}
  \begin{proof}
    The fullness and faithfulness follows from Proposition \ref{prop:tri-(full,faithful)}. The essential surjectivity is easy to check.
  \end{proof}

  Assuming the axiom of choice true, we have
  \begin{prop}
    $F\colon\Aa\to\Bb$ is an equivalence if and only if it is a weak equivalence.
  \end{prop}
  \begin{proof}
    Let $F\colon\Aa\to\Bb$ be a weak equivalence, then we constitute a weak inverse $G\colon\Bb\to\Aa$ as following.

    For each $\Bb-$object $B$, CHOICE an $\Aa-$object $A$ such that $F(A)\approx B$, set $G(B)=A$.
    For each $\Bb-$morphism $f\colon B\to B'$, set $G(f)$ to be the inverse image of following composition
    \begin{equation*}
      FG(B)\approx B\markar{f} B'\approx FG(B')
    \end{equation*}
    in $\Aa$, the existence and uniqueness of such an inverse image comes from the fact that $F$ is fully faithful.

    $F\circ G\cong \Id_{\Bb}$ comes from the constitution above. To see $\Id_{\Aa}\cong G\circ F$, just notice that $F$ reflects isomorphisms.
  \end{proof}

  However, if we do not assume the axiom of choice, then there is a related concept weaker than equivalence.
  \begin{defn}
    Two categories $\Aa$ and $\Bb$ are said to be \termin{weak equivalent}, if there exist a category $\Cc$ and two weak equivalence $F\colon\Cc\to\Aa$ and $G\colon\Cc\to\Bb$.
  \end{defn}

\subsection{Examples}
\begin{exam}
  The basic fact in the linear algebra is that $\Mat$ is equivalent to the category $\FinVect$ of finite-dimensional vector spaces and linear mappings. However, they are not isomorphic.
\end{exam}
\begin{exam}
  Let $\Cc$ a category with two object $A,B$, and four morphisms: two are the identities $1_A, 1_B$, two are isomorphisms $f\colon A\to B$, $g\colon B\to A$.
  Then $\Cc$ is equivalent to $\one$ through the functor which maps every objects in $\Cc$ to $\bullet$, and every morphisms in $\Cc$ to the identity.
  However, there are two weak inverse: one maps $\bullet$ to $A$ and the other to $B$.
\end{exam}
\begin{exam}
  Consider a category $\Cc$ with one object $X$, and two morphisms $1_X, f\colon X\to X$, where $f \circ f = 1_X$.
  Of course, $\Cc$ is equivalent to itself and $1_{\Cc}$ is an equivalence. However, there are two natural isomorphisms: one induced from $1_X$ and the other from $f$. This example shows that even the weak inverse is unique, the choices of natural isomorphisms may not.
\end{exam}

\newpage\section{Subcategories}
  Like subsets, subgroups and so on, we now introduce the notion of subcategories in the similar way.
  \begin{defn}
    A \termin{subcategory} of $\Cc$ is a category whose objects and morphisms form subcollections of $\Cc$'s respectively.
  \end{defn}
    The definition of subcategory implicit a functor which works as an inclusion on the collections of objects and morphisms, named the \termin{inclusion functor}.

  There are two different notions formalize the idea that a subcategory which is big enough to reveal the entire category.
  \begin{defn}
    If the inclusion functor is full, then we say the subcategory is \termin[full]{full subcategory}.
    If the inclusion functor is surjective on objects, then we say the subcategory is \termin[wide]{wide subcategory} (or \termin[lluf]{lluf category}).
  \end{defn}

  It is obvious that the inclusion functor is necessarily an embedding.
  Conversely,
  just as subsets of a set $S$ can be identified with isomorphism classes of injective functions into $S$, subcategories of a category $\Cc$ can be identified with isomorphism classes of monic functors into $\Cc$.
  A functor is easily verified to be monic (in $\Cat$, or $\CAT$ more generally) if and only if it is an embedding.

  Consequently, inclusions are (up to isomorphism) precisely the embeddings.
  \begin{prop}
    A functor $F\colon \Aa \to \Bb$ is a (full) embedding if and only if it factors through a (full) subcategory $\Cc$ of $\Bb$ by an isomorphism $G\colon \Aa \to \Cc$ and an inclusion $E\colon \Cc \to \Bb$. That is $F=E\circ G$.
  \end{prop}
  \begin{proof}
    Set $\Cc$ to be the image of $\Aa$ in $\Bb$.
  \end{proof}
  Moreover, inclusions are (up to (weak) equivalence) precisely the faithful functors.
  \begin{prop}
    A functor $F\colon \Aa \to \Bb$ is faithful if and only if it factors through an inclusion $E_1\colon\Aa\to\Cc$, a weak equivalence $G\colon\Cc\to\Dd$, and an inclusion $E_2\colon\Dd\to\Bb$. That is $F=E_2\circ G\circ E_1$.
  \end{prop}
  \begin{proof}
    Set $\Dd$ to be the full subcategory of $\Bb$ whose objects are same as the image of $\Aa$.
    Set $\Cc$ to be the category whose objects are same as $\Aa$, while morphisms as $\Dd$.
  \end{proof}

  Categories can be classified into different equivalence classes, to characterize this, we have
  \begin{defn}
    A category is said to be \termin[skeletal]{skeletal category} if its objects that are isomorphic are necessarily equal.
    Traditionally, a \termin[skeleton]{skeleton (category theory)} of a category $\Cc$ is defined to be a skeletal subcategory of $\Cc$ whose inclusion functor exhibits it as equivalent to $\Cc$.
  \end{defn}
  \begin{rem}
    However, in the absence of the axiom of choice, it is more appropriate to define a skeleton of $\Cc$ to be any skeletal category which is weakly equivalent to $\Cc$.
  \end{rem}
  \begin{defn}
    A category is said to be \termin[essentially small]{essentially small category} if it is equivalent to a small category. Assuming the axiom of choice, this is the same as saying that it has a small skeleton.
  \end{defn}
  \begin{prop}
    Any two skeletons of a category are isomorphic. Conversely, two categories are equivalent if and only if they have isomorphic skeletons.
  \end{prop}
  \begin{rem}
    In the absence of the axiom of choice, the term ``equivalent'' must be replaced by ``weak equivalent''.
  \end{rem}

\subsection{Terminological remark}
    In many fields of mathematics, objects satisfying some ``universal property'' are not unique on the nose, but only \emph{unique up to unique isomorphism}. It can be tempting to suppose that in a skeletal category, where any two isomorphic objects are equal, such objects will in fact be unique on the nose. However, under the most appropriate definition of ``unique'' this is \textbf{not} true (in general), because of the presence of automorphisms.

    More explicitly, consider the notion \emph{cartesian product} (see Definition \ref{def:product}) in a category as an example. Although we colloquially speak of  ``a product'' of objects $A$ and $B$ as being the object $A\times B$, strictly speaking a product is a triple which consists of the object $A\times B$ together with the projections $A\times B\to A$ and $A\times B\to B$ which exhibit its universal property.
    Thus, even if the category in question is skeletal, so that there can be only one object $A\times B$ that is a product of $A$ and $B$, in general this object can still ``be the product of $A$ and $B$'' in many different ways (in the sense that the projection maps are different): those different ways are then related by an automorphism of the object.

    Finally, it is true in a few cases, though, that skeletality implies uniqueness on the nose. For instance, a \emph{terminal object} (see Definition \ref{def:universal-object}) can have no nonidentity automorphisms, so in a skeletal category, a terminal object (if one exists) really is unique on the nose.

\newpage\section{Comma categories}
  We indicate now a quite general process for constructing new categories from given ones. This type of construction will be used very often in this book.

  \begin{defn}
    Suppose that $\Aa$, $\Bb$, and $\Cc$ are categories, and $S$ and $T$ (for source and target) are functors
          \begin{displaymath}
            \xymatrix{
               \Aa\ar[r]^{S} & \Cc & \Bb\ar[l]_{T}                }
          \end{displaymath}
    We can form the \termin{comma category} $(S\down T)$ as follows:  \glsadd{commma}
    \begin{itemize}
      \item The objects are all triples $(A,f,B)$ with $A$ an object in $\Aa$, $B$ an object in $\Bb$, and $f\colon S(A)\To T(B)$ a morphism in $\Cc$.
      \item The morphisms from $(A,f,B)$ to $(A',f',B')$ are all pairs $(g,h)$
                 where $g\colon A\To A'$ and $h\colon B\To B'$ are morphisms in $\Aa$ and $\Bb$ respectively, such that the following diagram commutes:
                 \begin{displaymath}
                   \xymatrix{
                       S(A)\ar[d]_{S(g)}\ar[r]^{f} & T(B)\ar[d]^{T(h)}  \\
                       S(A')\ar[r]^{f'} & T(B')           }
                 \end{displaymath}
      \item Morphisms are composed by taking $(g,h)\circ(g',h')$ to be $(g\circ g',h\circ h')$, whenever the latter expression is defined.
      \item The identity morphism on an object $(A,f,B)$ is $(1_{A},1_{B})$.
    \end{itemize}
  \end{defn}
  \begin{rem}
    Some people prefer the notation $(S/T)$ rather than $(S\down T)$.
  \end{rem}

  The following proposition provide the ``universal property'' of the comma category.
  \begin{prop}\label{prop:comma-uni}
    For each comma category there are two canonical \termin[forgetful functors]{forgetful functor} from it.
    \begin{itemize}
      \item \termin{domain functor}, $U\colon(S\down T)\To\Aa$, which maps:
      \begin{itemize}
        \item objects: $(A,f,B)\mapsto A$;
        \item morphisms: $(g,h)\mapsto g$;
      \end{itemize}
      \item \termin{codomain functor}, $V\colon(S\down T)\To\Bb$, which maps:
      \begin{itemize}
        \item objects: $(A,f,B)\mapsto B$;
        \item morphisms: $(g,h)\mapsto h$;
      \end{itemize}
    \end{itemize}
    Meanwhile, there exists a natural transformation $\alpha\colon S\circ U \then T\circ V$.
                 \begin{displaymath}
                   \xymatrix{
                       (S\down T)\ar[r]^-{V}\ar[d]_-{U}
                       &\Bb\ar[d]^{T}\\
                       \Aa\ar[r]_{S} \ar@{}[ur]^{\alpha}|-{\SelectTips{eu}{}\object@{=>}}
                       &\Cc %\ultwocell\omit
                               }
                 \end{displaymath}

    Moreover, comma category is the universal one respect to the above property.
    That is, if there exist another category $\Dd$ together with two functor $U'\colon\Dd\to\Aa$ and $V'\colon\Dd\to\Bb$ such that there exists a natural transformation $\alpha'\colon S\circ U' \then T\circ V'$.
                 \begin{displaymath}
                   \xymatrix{
                       \Dd\ar[r]^-{V'}\ar[d]_-{U'}
                       &\Bb\ar[d]^{T}\\
                       \Aa\ar[r]_{S} \ar@{}[ur]^{\alpha'}|-{\SelectTips{eu}{}\object@{=>}}
                       &\Cc %\ultwocell\omit
                               }
                 \end{displaymath}
    Then there exist a unique functor $W\colon\Dd\to(S\down T)$ such that
    \begin{equation*}
      U\circ W = U'\qquad V\circ W = V'\qquad \alpha\ast W = \alpha'.
    \end{equation*}
  \end{prop}
  \begin{proof}
    The property follows just from the definition of the comma category, where the natural transformation $\alpha$ is defined componentwise by
    \begin{equation*}
      \alpha_{(A,f,B)}=f
    \end{equation*}

    If there exist another quadruple $(\Dd,U',V',\alpha')$ satisfies the property, then we can define a functor $W\colon\Dd\to (S\down T)$ by
    \begin{align*}
      W(D) &= (U'(D),\alpha'_D ,V'(D)) \\
      W(f) &= (U'(f), V'(f))
    \end{align*}
    It is easy to check that
    \begin{equation*}
      U\circ W = U'\qquad V\circ W = V'\qquad \alpha\ast W = \alpha'.
    \end{equation*}

    Conversely, the above equations enforce any functor satisfies them must be equal to $W$.
  \end{proof}

\subsection{Examples}
  \begin{exam}%[Slice category]
    $(\Id_{\Cc} \down \Delta_I)$, also denoted by $(\Cc \down I)$ is called the \termin{slice category} over $I$ or \emph{the category of objects over $I$}.
  \end{exam}

  \begin{exam}%[Coslice category]
    $(\Delta_I \down \Id_{\Cc})$, also denoted by $(I \down \Cc)$ is called the \termin{coslice category} under $I$ or \emph{the category of objects under $I$}.
  \end{exam}

  \begin{exam}%[Arrow category]
    $(\Id_{\Cc}\down\Id_{\Cc})$ is the \termin{arrow category} $\Cc^{\to}$.
  \end{exam}

  \begin{exam}\glsadd{T-arrow}\glsadd{S-arrow}
    In the case of the slice or coslice category, the identity functor may be replaced with some other functor $F$; this yields a family of categories particularly useful in the study of adjoint functors. Let $s,t$ be given object in $\Cc$.
    An object of $(s\down F)$ is called a \emph{morphism from $s$ to $F$} or a \termin{$F-$structured arrow} with domain $s$ in.
    An object of $(F\down t)$ is called a \emph{morphism from $F$ to $t$} or a \termin{$F-$costructured arrow} with codomain $t$ in.
  \end{exam}

  \begin{exam}\label{exam:Elts}
    Let $F\colon\Cc\to\Set$ be a functor, $1\colon\one\to\Set$ be the functor maps the object of $\one$ to a singleton. The the comma category $(1\down F)$ is called the \termin{category of elements of $F$}, denoted by $\Elts(F)$. It can be explicitly described in the following way.
  \begin{itemize}
    \item Objects: the pairs $(X,x)$, where $X\in\ob\Cc$ and $x\in F(X)$.
    \item Morphisms $f\colon(A,a)\To(B,b)$:
               the morphisms $f\colon A\to B$ in $\Cc$ such that $F(f)(a)=b$.
    \item The composition law is that induced by the composition of $\Cc$.
  \end{itemize}
  The codomain functor of $\Elts(F)$ is also called the \termin{forgetful functor}.
  \end{exam}

  \begin{exam}
    Let $\Delta_{\Aa}$ denote the \termin{constant functor}\glsadd{ConstantF} from $\Aa$ to $\one$ and $\Delta_{\Bb}$ likewise. Then the comma category $(\Delta_{\Aa}\down\Delta_{\Bb})$ is just the \termin[product]{product of categories} $\Aa\times\Bb$ of $\Aa$ and $\Bb$, which can be described as below.
  \begin{itemize}
    \item Objects:
               the pairs of objects $(A, B)$, where $A$ is an object of $\Aa$ and $B$ of $\Bb$;
    \item Morphisms $f\colon(A,B)\To(A',B')$:
               the pairs of arrows $(a,b)$, where $a\colon A\to A'$ is an arrow of $\Aa$ and $b\colon B\to B'$ is an arrow of $\Bb$;
    \item The composition law is the component-wise composition from the contributing categories:
                                    \begin{equation*}
                                      (a', b') \circ (a, b) = (a' \circ a, b' \circ b);
                                    \end{equation*}
  \end{itemize}
  \end{exam}
  The product $\Aa\times\Bb$ comes with two ``\termin[projection]{projection (functor)}'' functors
  \begin{equation*}
    p_{\Aa}\colon\Aa\times\Bb\To\Aa\qquad p_{\Bb}\colon\Aa\times\Bb\To\Bb
  \end{equation*}
  which are defined by the formulas
  \begin{align*}
    p_{\Aa}(A,B)=A,&\quad p_{\Bb}(A,B) = B,\\
    p_{\Aa}(a,b) =a,&\quad p_{\Bb}(a,b) = b.
  \end{align*}

  These data satisfy the following ``universal property''.
  \begin{prop}
    Consider two categories $\Aa$ and $\Bb$. For every category $\Cc$ and every pair of functors $F\colon\Dd\to\Aa, G\colon\Dd\to\Bb$, there exists a unique functor $H\colon\Dd\to\Aa\times\Bb$ such that $p_{\Aa}\circ H=F, p_{\Bb}\circ H=G$.
  \end{prop}
  \begin{proof}
    This follows straightly from Proposition \ref{prop:comma-uni}.
  \end{proof}
  A point of terminology: a functor defined on the product of two categories is generally called a \termin{bifunctor} (a functor of two ``variables''). In practice, something is said to be \textbf{natural} or \textbf{functorial} in $X_1,X_2,\cdots$ implicits it is actually a functor of variables $X_1,X_2,\cdots$.
  \begin{exam}
    $\Hom_{\Cc}(-,-)$ is a bifunctor from $\Cc^{\op}\times\Cc$ to $\Set$.
  \end{exam}


\newpage\section{The duality principle}
  One may have noticed that every result proved for covariant functors has its counterpart for contravariant functors and every result proved for monomorphisms has its counterpart for epimorphisms.
  These facts are just special instances of a very general principle.

  In the remark under Definition \ref{def:category}, we have shown that a category provides a two-sorted first order language and categorical properties are statements in this language.

  Once we have a categorical property $\sigma$, then the dual $\sigma^{\op}$ can be obtained by reversing arrows and compositions. That is
  \begin{enumerate}
    \item   Interchange each occurrence of ``source'' in $\sigma$ with ``target''.
    \item   Interchange the order of composing morphisms.
  \end{enumerate}

  Then one can easily find that a property $\sigma$ in $\Cc$ is logical equivalent to the property $\sigma^{\op}$ in $\Cc^{\op}$.

  Consequently, we have
  \begin{thm}[The duality principle for categories]
  $ $
  \begin{center}
    Whenever a property $\Pp$ holds for all categories,

    then the property $\Pp^{\op}$ holds for all categories.
  \end{center}
  \end{thm}

  \begin{exam}
    A morphism is monic if and only if the reverse morphism in the opposite category is epi.
  \end{exam}

\newpage\section{Yoneda lemma and representable functors}
  In this section, we will prove an important theorem. Before doing this, we give the useful concept of representable functors.

  Representability is one of the most fundamental concepts of category theory, with close ties to the notion of adjoint functor and to the Yoneda lemma. It is the crucial concept underlying the idea of universal property. The concept permeates much of algebraic geometry and algebraic topology.
  \begin{defn}
    For a functor $F\colon\Cc^{\op}\to\Set$ (also called a \emph{presheaf} on $\Cc$ ), a \termin{representative} of $F$ is a specified natural isomorphism
    \begin{equation*}
      \Phi\colon\Hom_{\Cc}(-,X)\To F
    \end{equation*}
    where the object $X$ in $\Cc$ is called a \termin{representing object} (or \termin{universal object}) of $F$.

    If such a representative exists, then we say the functor $F$ is \termin[representable]{representable functor} and is \textbf{represented} by $X$.

    Similarly, a covariant functor $F\colon\Cc\to\Set$ is said to be \textbf{representable}, if it is representable when view it as a presheaf on $\Cc^{\op}$.
  \end{defn}

  Given a category $\Cc$, there exists a functor:
  \begin{equation*}\glsadd{Yoneda}
    \Upsilon\colon\Cc\To\PSh(\Cc)
  \end{equation*}
  sends any object $X\in\ob\Cc$ to the presheaf $\Hom_{\Cc}(-,X)$.

  The Yoneda lemma asserts that the set of morphisms from the presheaf represented by $X$ into any other presheaf $F$ is in natural bijection with the set $F(X)$ that this presheaf assigns to $X$.

  Formally:
  \begin{thm}[Yoneda lemma]
    There is a canonical isomorphism
    \begin{equation*}
      \Hom_{\PSh(\Cc)}(\Upsilon(X),F)\cong F(X)
    \end{equation*}
    natural in both $X$ and $F$.
  \end{thm}
  \begin{rem}
    In some literature it is customary to denote the presheaf represented by $X$ as $h_X$. In that case the above is often written
    \begin{equation*}\glsadd{Nat}
      \Nat(h_X,F)\cong F(X)
    \end{equation*}
    to emphasize that the morphisms of presheaves are natural transformations of the corresponding functors.
  \end{rem}

  \begin{proof}
    The crucial point is that any natural transformation
    \begin{equation*}
      \alpha\colon\Hom_{\Cc}(−,X)\then F
    \end{equation*}
    is entirely fixed by the value $\alpha_X(1_X)\in F(X)$ of its component
    \begin{equation*}
    \alpha_X\colon\Hom(X,X)\to F(X)
    \end{equation*}
    on the identity morphism $1_X$. And every such value extends to a natural transformation $\alpha$.

    To see this, we fix a value $\alpha_X(1_X)\in F(X)$ and consider an arbitrary object $A$ in $\Cc$. If $\Hom(A,X)=\varnothing$, then the component $\alpha_A$ must be the trivial function from empty set. If there exists a morphism $f\colon A\to X$, then by the naturality condition, the following commutative square has already been determined.
          \begin{displaymath}
            \xymatrix{
               \Hom(X,X)\ar[r]^-{\alpha_X}\ar[d]_{f^{\ast}}& F(X)\ar[d]^{F(f)} \\
               \Hom(A,X)\ar[r]^-{\alpha_A}& F(A)               }
          \end{displaymath}
    Consequently, all components of $\alpha$ have been determined.

    Conversely, given a value $a=\alpha_X(1_X)\in F(X)$, define $\alpha$ by components as following:
    \begin{equation*}
      \alpha_A(f):=F(f)(a)\quad\forall A\in\ob\Cc,\forall f\in\Hom(A,X)
    \end{equation*}
    It is easy to check this is a natural transformation.

    The naturalities on $X$ and $F$ is easy to check.
  \end{proof}

\subsection{Corollaries}
  The Yoneda lemma has the following direct consequences. Like the Yoneda lemma, they are as easily established as they are useful and important.
  \begin{cor}
    The functor $\Upsilon$ is a full embedding.
  \end{cor}
  This $\Upsilon$ is customary called the \termin{Yoneda embedding}
  \begin{proof}
    For any $A,B\in\ob\Cc$, by the Yoneda lemma, we have
    \begin{equation*}
      \Hom_{\PSh(\Cc)}(\Upsilon(A),\Upsilon(B))\cong (\Upsilon(B))(A) = \Hom_{\Cc}(A,B)
    \end{equation*}
    Thus $\Upsilon$ is fully faithful. The injectivity on objects is obvious.
  \end{proof}

  \begin{cor}\label{coro:Yoneda2}
    For any $A,B\in\ob\Cc$, we have
    \begin{equation*}
      \Upsilon(A)\cong\Upsilon(B)\iff A\cong B
    \end{equation*}
  \end{cor}
  \begin{proof}
    Since $\Upsilon$ is fully faithful, thus reflects isomorphisms.
  \end{proof}

  \begin{cor}
    Let $F$ be a presheaf on $\Cc$, then a presentation of $F$ is uniquely determined by the universal object $X$ together with an element $u\in F(X)$. Such a pair $(X,u)$ satisfies the following universal property:
    \begin{quote}
      For every pair $(A,a)$, where $A\in\ob\Cc$ and $a\in F(A)$, there is a unique morphism $f\colon A\to X$ such that $F(f)(u)=a$.
    \end{quote}
  \end{cor}
  \begin{proof}
    Notice that, a presentation $\Phi$ of $F$ is nothing but an element in the set $\Hom_{\PSh(\Cc)}(\Upsilon(X),F)$, thus by Yoneda lemma it can be corresponded to an element of $F(X)$ via $\Upsilon$, say $u\in F(X)$. Then $\Phi$ is uniquely determined by $X$ and $u$.

    On the other hand, for each object $A$ in $\Cc$, $\Phi_{A}$ gives a 1-1 corresponding between $\Hom_{\Cc}(A,X)$ and $F(A)$, thus the universal property follows.
  \end{proof}
  \begin{rem}
    Someone may doubt the existence of the morphism in this universal property. It is possible that there is not morphisms from $A$ to $X$. But in this case, Yoneda lemma ensure that $F(A)$ is empty, thus there is no pair $(A,a)$ at all.
  \end{rem}
  \begin{rem}
    The \nlab gives another description of this corollary: the presentation of $F$ is the \emph{terminal object} in the comma category $(\Upsilon\down\Delta_{F})$.
  \end{rem}

\newpage\section{Exercises}
\begin{ex}
  Let $f\colon X\to Y$ be a morphism in $\Cc$. Show that
  \begin{enumerate}
    \item $f$ is monic if and only if the induced function
               \begin{equation*}
                 f_{\ast}\colon\Hom(C,X)\to\Hom(C,Y)
               \end{equation*}
               is injective for every object $C$.
    \item $f$ is epi if and only if the induced function
               \begin{equation*}
                 f^{\ast}\colon\Hom(Y,C)\to\Hom(X,C)
               \end{equation*}
               is injective for every object $C$.
    \item $f$ is split epi if and only if the induced function
               \begin{equation*}
                 f_{\ast}\colon\Hom(C,X)\to\Hom(C,Y)
               \end{equation*}
               is surjective for every object $C$.
    \item $f$ is split monic if and only if the induced function
               \begin{equation*}
                 f^{\ast}\colon\Hom(Y,C)\to\Hom(X,C)
               \end{equation*}
               is surjective for every object $C$.
  \end{enumerate}
\end{ex}
\begin{ex}
  Let $S,T\colon\Aa\to\Bb$ be two functors. Shows that there exists a 1-1 corresponding between the set of natural transformations from $S$ to $T$ and the set of sections of both domain functor and codomain functor.
\end{ex}
\begin{ex}\label{prop:power law for functor}
  For any categories $\Aa,\Bb,\Cc$,
  \begin{enumerate}
    \item $[\Aa,\Bb]^{\op}\simeq[\Aa^{\op},\Bb^{\op}]$
    \item $(\Aa\times\Bb)^{\Cc}\simeq\Aa^{\Cc}\times\Bb^{\Cc}$
    \item $\Cc^{\Aa\times\Bb}\simeq(\Cc^{\Aa})^{\Bb}\simeq(\Cc^{\Bb})^{\Aa}$
  \end{enumerate}
\end{ex}
\begin{rec}
  $\Bb^{\Aa}$ is just another notation of $[\Aa,\Bb]$.
\end{rec}
\begin{ex}
  Show that the functor category $[\Aa,\Bb]$ is functorial in both $\Aa$ and $\Bb$. That means, $[-,-]\colon\Cat^{\op}\times\Cat\to\Cat$ is a bifunctor.
\end{ex}
\begin{ex}
  Use the duality principle to define the notion of covariant representable functor explicitly.
\end{ex}
\begin{ex}
  Show that a covariant representable functor preserves monomorphisms. On the other hand, a contravariant representable functor maps an epimorphism to a monomorphism.
\end{ex}
\begin{ex}
  Prove the covariant Yoneda lemma: There is a canonical isomorphism
    \begin{equation*}
      \Nat(\Hom(X,-),F)\cong F(X)
    \end{equation*}
    natural in both $X$ and $F$.
\end{ex}
\begin{ex}
  Consider the category $\PSh(\Cc)$ of presheaves on a small category $\Cc$, show that a morphism in $\PSh(\Cc)$ is monic if and only if its every component is monic.
  However, if we replace $\PSh(\Cc)$ by an arbitrary functor category $[\Cc,\Dd]$, then the previous statement is no longer valid. Show this by a counterexample.
\end{ex}

  \chapter{范畴的语言}
在本章,我们首先介绍一些基本概念,诸如\emph{范畴(category)、函子(functor)、自然变换(natural transformation)、单射(monomorphism)、满射(epimorphisms)、同构(isomorphism)}.接着,\emph{monomorphism}和\emph{epimorphism}、\emph{协变(covariant)}和\emph{反变(contravariant)}函子之间的相似性带领我们看到\emph{对偶原理(duality principle)}.
\minitoc
\newpage
\section{范畴}
\subsection{为什么研究范畴?}
  在现代数学里,一些看起来完全不相干的领域往往涌现出许多相似的现象和具有类似性质的结构.为了精确地描述这些现象、同步的处理这些结构,范畴论的语言应运而生.

  在过去一些年里,范畴论学者已经发展出了一套符号体系,使得我们可以通过图形快速地识别相当复杂的事实.

  如今,范畴论已经是一套强大的语言.它提供了合适的工具,能将一个领域的数学问题转换到另一个领域里,从而变得更加易于解决.因此,范畴论的语言在现代数学乃至其他学科,如逻辑学、计算机科学、语言学、哲学等,大受好评.

  更多的历史评论可参考任何一本范畴论教材.

\subsection{什么是范畴?}
  定义什么是范畴有好几种方式;在通常的数学基础里,这些定义都是等价的.这里我们提供一个常见的定义.

  \begin{defn}
  一个\emph{范畴}(\termin{category})$\Cc$由以下数据构成:
  \begin{itemize}
    \item 诸\emph{对象}(\termin[objects]{object (category theory)})之\emph{收集}$\ob\Cc$;
    \item 诸\emph{态射}(\termin[morphisms]{morphism (category theory)})或曰\emph{箭头}(\termin[arrows]{arrow (category theory)})或曰\emph{映射} (\termin[maps]{map (category theory)})之\emph{收集}$\hom\Cc$;

          \begin{quote}
             每一态射$f$具有唯一的出发对象(\termin[source]{source (category theory)} object)$A$以及到达对象(\termin[target]{target (category theory)} object)$B$,我们用$f\colon A\To B$来记这个态射,并说“$f$是从$A$到$B$的态射”,“$A$ 是$f$的\termin[domain]{domain (category theory)}” 以及“$B$是$f$的\termin[codomain]{codomain (category theory)}”.

             我们用$\Hom(A, B)$ (或者在必须指明范畴时用$\Hom_{\Cc}(A, B)$)来标记从$A$到$B$的态射全体之收集.(有的作者用$\Mor(A, B)$ 或者$\Cc(A,B)$来标记.)\glsadd{hom}
          \end{quote}
    \item 对于每三个对象$A,B,C$,有一个二元运算
             \begin{equation*}
               \Hom(A, B) \times \Hom(B, C) \To \Hom(A, C)
             \end{equation*}
             称为\emph{态射的复合}(\termin{composition of morphisms}).

             \begin{quote}
             态射$f\colon A \To B$ 和$g\colon B \To C$ 之复合写作$g\circ f$或者$gf$.(也有作者采用“图示顺序”,写作$f;g$或者$fg$.)
             \end{quote}
  \end{itemize}
  并满足下述公理:
  \begin{description}
    \item[结合律(associativity)] 若$f\colon A \To B, g\colon B \To C, h\colon C \To D$,则
                                 \begin{equation*}
                                   h\circ(g\circ f) = (h\circ g)\circ f
                                 \end{equation*}
    \item[单位律(identity)] 每一对象$A$存在一态射$1_A\colon A \To A$(有时也写作$\id_A$)称为$A$的
                                                  \emph{单位态射}(\termin[identity morphism]{identity morphism}),满足:对于每个态射$f\colon A \To B$,有$1_B \circ f = f = f \circ 1_A$.
  \end{description}
  由以上公理,不难验证每个对象有唯一的单位态射.一些作者采用一种轻量级的定义,在这种定义中,对象由其所对应的单位态射取代.
  \end{defn}
  \begin{rem}
     为强调范畴$\Cc$,人们通常称一个$\Cc$中的对象(态射、箭头、映射)为$\Cc-$对象($\Cc-$态射、$\Cc-$ 箭头、$\Cc-$映射).
  \end{rem}
\begin{rem}
  范畴论提供了一个框架用以描述“范畴性质”. 非正式地说,一个\emph{范畴性质}(\termin{categorical property})是一个关于某个范畴中的对象和态射的语句. 更技术化的说法是:一个范畴提供了一个双类型的一阶语言,它以对象和态射为不同的类型,带有关系“(某对象)是(某态射)的domain(或codomain)”以及符号“态射的复合”. 因此一个范畴性质就是在这个一阶语言中的一条语句.
\end{rem}

  接下来我们定义范畴之间的同态,即\emph{函子}.
  \begin{defn}
    一个从范畴$\Aa$到范畴$\Bb$ 的\emph{函子}(\termin{functor}) $ F$由以下数据构成:
    \begin{itemize}
      \item 一个$\Aa$和$\Bb$的对象之间的对应
                 \begin{equation*}
                   \ob\Aa\To\ob\Bb
                 \end{equation*}
                 对象$A\in\ob\Aa$的像记作$ F(A)$或者$ F A$;
      \item 对于$\Aa$中的每对对象$A, A'$,有一个对应
                 \begin{equation*}
                   \Hom_{\Aa}(A,A')\To\Hom_{\Bb}( F(A), F(A'))
                 \end{equation*}
                 态射$f\in\Hom_{\Aa}(A,A')$ 的像记作$ F(f)$或者$ F f$.
    \end{itemize}
    并满足下述公理:
    \begin{itemize}
      \item 对于每对态射$f\in\Hom_{\Aa}(A,A'), g\in\Hom_{\Aa}(A',A'')$,
                 \begin{equation*}
                    F(g\circ f) =  F(g)\circ F(f)
                 \end{equation*}
      \item 对于每个对象$A\in\ob\Aa$,
                 \begin{equation*}
                    F(1_A) = 1_{F(A)}
                 \end{equation*}
    \end{itemize}
  \end{defn}

  任给两个函子$ F\colon\Aa\to\Bb$和$ G\colon\Bb\to\Cc$,一个逐点的复合立刻产生了一个新的函子$ G\circ F\colon\Aa\to\Cc$. 进一步地,这种复合满足结合律.

  另一方面,每个范畴$\Cc$具有一个单位函子——只要在上面定义中将每个对应取作单位即可.这个函子显然上再上述复合下的单位.

  总之,范畴和函子之间的关系正如一个范畴中对象和态射的关系.于是,一不小心就会得出这样一个轻率的结论:所有范畴和函子构成一个新的范畴.

\subsection{逻辑基础与大小问题}
    你可能会注意到这里使用了一个未定义的词\emph{收集},而不是术语\emph{集合}或者\emph{类}.术语的选择依赖于我们选定的逻辑基础:当采用带有universe的ZFC 时,我们采用术语\emph{集合},并且我们所考虑的典型的数学对象,诸如\emph{集合}、\emph{群}等等都只考虑那些相对于universe\emph{小}的对象;当采用类论,例如NBG时,我们采用术语\emph{类}并且所考虑的典型的数学对象从大小上来讲都是集合. 更多的细节可参考\cite{borceux} 的开头或者\cite{awodey2010category}的第一章第8节.

    然而,范畴论的基础是独立于集合论的.事实上,范畴论可以代替集合论作为数学的逻辑基础.这意味着我们可以将前面的公理用严格的形式逻辑写出来,因此这里采用一个未定义的词“收集”并不会出问题.更多的细节可参考\nlab 或者\cite{lane1998categories}.

    不过,离开集合论,有关大小的讨论,例如“全体XXX” 之类的将失去意义.毕竟,罗素悖论带给人们的教训之一就是不加限制地使用量词是十分危险的.

    在选定集合公理后,讨论大小问题是有意义的.在通常的类论,例如NBG中,一个类是\emph{小的}意味着它是一个集合;在带universe的ZFC 中,一个集合是\emph{小的},意味着它同时也是universe中的元素.

    一个范畴$\Cc$称为是\emph{小的}(\termin[small]{small category}),如果$\ob\Cc$ 和$\hom\Cc$都是小的,否则就称为\emph{真大的}(\termin[proper large]{proper large category}).
    一个范畴称为是\emph{局部小的}(\termin[locally small]{locally small category}),如果对于每对对象$A$ 和$B$,$\Hom(A, B)$ 是小的.许多数学中的重要的范畴(例如\emph{集合范畴}),尽管不是小的,但却是局部小的.
    因此,人们往往倾向于用\emph{范畴}一词来称呼局部小的范畴.在本书中,我们也将遵循这一传统.这样做并不会带来歧义——必要时,我们会按照上面的定义来使用\emph{大范畴}(\termin{large category})这个术语.

    在上述设定下,我们才可以安全地宣布所有小范畴及其间函子构成一个范畴.这个范畴通常记作$\Cat$.
    另一方面,全体(局部小)范畴及其间函子所组成的“范畴”甚至比任何大范畴还要大——这在通常的集合论中意味着这样的结构是不存在的.

    当然,你完全可以选择一个允许多重大小而不是仅仅只区分“small”和“large”的集合论公理系统.在这种设定下,较小的范畴及其间函子全体构成一个较大的范畴. 例如,全体(局部小)范畴及其函子构成一个“非常大的”(比大范畴还大)范畴,记作$\CAT$:参考
    \hrefacc 即\cite{acc}.

    更多有关逻辑基础和大小问题的讨论可参考\nlab 以及相关出版物.

\section{例子}
  前面对于小范畴及其函子的讨论提供了在本书中的第一个范畴的例子,即$\Cat$.
  我们在这一节里将介绍更多的例子.

  事实上,传统数学已经通过不同途径提供了相当数量的例子.
  \begin{exam}
    很多传统的数学结构是通过给集合配上额外结构得到的.它们提供了一些明显的例子.
    \begin{itemize}
      \item 集合
      \footnote{这里,\emph{集合}指的是集合论提供的那些小的对象,例如带universe 的ZFC里的\emph{小集合}.在这种情况下,为了避免歧义,我们用\emph{大集合}来称呼一般的集合.通常的类论里已经提供了区别于\emph{集合}的术语\emph{类},因此初学者也可以认为我们选择了一个通常的类论来作为数学基础.}
       以及函数:$\Set$.
      \item 群及群同态:$\Grp$.
      \item 环及环同态:$\Ring$.
      \item 实线性空间及线性映射:$\Vect_{\RR}$.
      \item 右$R$模及模同态:$\Mod_R$.
      \item 拓扑空间及连续映射:$\Top$.
      \item 一致空间及一直连续函数:$\Uni$.
      \item 微分流形及光滑映射:$\Diff$.
      \item 度量空间及度量映射(metric mappings):$\Met$.
      \item 实Banach空间及有界线性算子:$\Banb$.
      \item 实Banach空间及线性收缩(linear contractions):$\Ban$.
    \end{itemize}
    以上这些范畴都封装了“一种数学结构”.它们通常被称为“具体”范畴(``concrete'' categories).我们将在以后给出\emph{具体范畴}的一个技术性的定义.
  \end{exam}
  \begin{exam}
    一些数学构造本身也能看做范畴.
    \begin{itemize}
      \item 自然数集$\N$能以如下方式看做范畴:
                 其对象为自然数,从$n$到$m$的一个态射是一个$n$行$m$列的矩阵;态射的复合就是通常的矩阵乘法.
      \item 每个集合$S$也能被看做一个以$S$中元素作为对象且每个态射都是单位态射的范畴.

                 一般地,像这种每个态射都是单位态射的范畴称为离散范畴(\termin{discrete category}).
      \item 一个偏序集$(S,<)$可看作以$S$中的元素为对象的一个范畴:其中态射集$\Hom(x,y)$当且仅当$x < y$时是单点集,其他情况为空集:定义出唯一的复合的原因是偏序的传递性;而单位态射的存在性则由自反性保证.
      \item 一个幺半群$(M,\cdot)$可以被视为一个范畴$\Mm$,其中只有一个对象$\ast$并且$\Hom(\ast,\ast)=M$就是全部态射;复合就是幺半群的乘法.
    \end{itemize}
  \end{exam}

  如果你发现上面有些例子不太熟悉,不要紧,后面我们会专门讨论的.此外,还有一些来自非数学领域的例子,可以参考
  \cite{awodey2010category}及相关出版物.

\subsection{用范畴构造新范畴的简单例子}
给定一个范畴$\Cc$,我们有许多办法来得到一个新的范畴.这里是一些简单的例子.
\begin{exam}
  任何范畴$\Cc$都能通过如下方式看做一个新的范畴:对象仍然是原来的对象,但全体箭头都倒转了. 这个范畴称为是原来范畴的对偶范畴
    (\termin{dual category}/\termin{opposite category}),记作$\Cc^{\op}$.\glsadd{opposite}
\end{exam}
显然,每个范畴的二次对偶都是其自身。
对偶范畴的概念蕴含了一个重要的原理——对偶原理(\emph{duality principle}). 在本章后面会讲到它.
\begin{exam}[\termin{slice category}]
  首先选定一个对象$I\in\Cc$,“在$I$上的箭头”的范畴$\Cc/I$定义如下.\glsadd{Slice}
  \begin{itemize}
    \item 对象:以$I$为codomain的$\Cc$中的箭头.
    \item 态射:从对象$(f\colon A\to I)$到对象$(g\colon B\to I)$的态射是$\Cc$中的“$I$上的交换三角形”:
               \begin{displaymath}
                 \xymatrix@R=0.5cm{
                    A\ar[rr]^{h}\ar[dr]_{f} && B\ar[dl]^{g} \\
                    &I&                }
               \end{displaymath}
               换句话说就是满足$g\circ h = f$的$\Cc$ 中的态射$h\colon A\to B$.
    \item 复合由$\Cc$中的复合给出.
  \end{itemize}
\end{exam}
  \begin{rem}
    注意到在范畴$\Set$中,一个函数$f\colon A\to I$又可以看成是一个以$I$为指标集的集族$\{f^{-1}(i)\}_{i\in I}$,因此前述范畴正是以$I$为指标集的集族及其函数族之全体.
  \end{rem}

\begin{exam}[\termin{coslice category}]
  同样选定一个对象$I\in\Cc$,“在$I$下的箭头”的范畴$I/\Cc$定义如下.\glsadd{Coslice}
  \begin{itemize}
    \item 对象:以$I$为domain 的$\Cc$中的箭头.
    \item 态射:从对象$(f\colon A\to I)$到对象$(g\colon B\to I)$的态射是$\Cc$中的“$I$下的交换三角形”:
               \begin{displaymath}
                 \xymatrix@R=0.5cm{
                    &I\ar[dl]_{f}\ar[dr]^{g}& \\
                    A\ar[rr]_{h} && B   }
               \end{displaymath}
               换句话说就是满足$h \circ f = g$的$\Cc$中的态射$h\colon A\to B$.
    \item  复合由$\Cc$中的复合给出.
  \end{itemize}
\end{exam}

\begin{exam}
  现在我们来考虑$\Cc$中的全体态射,它们按如下方式构成范畴$\Cc^{\to}$.\glsadd{Arr}
  \begin{itemize}
    \item 对象:$\Cc$中的态射.
    \item 一个从对象$(f\colon A \to B)$到对象$(g\colon C \to D)$的态射是一个$\Cc$ 中的“交换方形”:
               \begin{displaymath}
                 \xymatrix@R=0.5cm{
                    A\ar[d]_{h}\ar[r]^{f} & B\ar[d]^{k} \\
                    C\ar[r]_{g}&D                }
               \end{displaymath}
    \item 复合由$\Cc$中的复合给出.
  \end{itemize}
\end{exam}

\subsection{最简单的例子}
我们最后介绍一些受到公理启发得到的简单的例子:
\begin{exam}
  范畴$\mathbf{0}$:无对象、无态射.

  范畴$\mathbf{1}$只有一个对象及其单位态射.它看起来像是这样:\glsadd{terminalCate}
               \begin{displaymath}
                 \xymatrix@R=0.5cm{
                    \bullet              }
               \end{displaymath}

  范畴$\mathbf{2}$包含两个对象及其单位态射,以及它们之间唯一的态射.它看起来像是这样:
               \begin{displaymath}
                 \xymatrix@R=0.5cm{
                    \bullet\ar[r] & \bullet             }
               \end{displaymath}

  范畴$\mathbf{3}$包含三个对象及其单位态射,以及唯一的从第一个对象到第二个、第二个到第三个及第一个到第三个的态射(于是第三个态射实际上上前两个的复合).它看起来像是这样:
               \begin{displaymath}
                 \xymatrix@R=0.5cm{
                    \bullet\ar[r]\ar[dr] & \bullet\ar[d] \\
                    & \bullet            }
               \end{displaymath}
\end{exam}

这些范畴看起来像是有向图(quivers).事实上,一个范畴可以看做一个一般的quiver配备一些额外结构.更多细节可参考\nlab.

记号$\mathbf{1},\mathbf{2},\mathbf{3}$ 来自这样一个事实:这些范畴正是对应的\emph{序数}(\emph{ordinals})--- 每个序数都可以看做一个偏序集因此是一个范畴.我们将在下一部分讨论序数.

\section{单射、满射和同构}
关于函数的最基本的概念是单射、满射和双射。
为了把它们推广到一般的范畴,我们必须对它们进行``外部的''刻画,即不提及\emph{元素}这个概念。早期的范畴论学者相信\emph{可消去}性质是对一般的单射、满射的正确刻画,于是他们定义了下述概念:
  \begin{defn}
    一个\emph{左消去}的态射$f$称作是\emph{单的}(\termin{monoic}),或者\emph{单态射}(\termin{monomorphism})。左消去的意思是说对任意的态射
    \begin{displaymath}
    \xymatrix@1{\cdot\ar@<0.5ex>[r]^{\alpha}\ar@<-0.5ex>[r]_{\beta} &\cdot\ar[r]^{f} &\cdot}
    \end{displaymath}
    $f\alpha=f\beta$ 蕴含 $\alpha=\beta$.

    对偶地,一个\emph{右消去}的态射$f$称作是\emph{满的}(\termin{epi}),或者\emph{满态射}(\termin{epimorphism})。右消去的意思是说对任意的态射
    \begin{displaymath}
    \xymatrix@1{\cdot\ar[r]^{f} &\cdot\ar@<0.5ex>[r]^{\alpha}\ar@<-0.5ex>[r]_{\beta} &\cdot}
    \end{displaymath}
    $\alpha f=\beta f$ 蕴含 $\alpha=\beta$.
  \end{defn}

  下面是关于单态射和满态射的基本性质。
  \begin{prop}
    每个恒等既单又满。单态射的复合还是单态射,满态射的复合还是满态射。
  \end{prop}
  \begin{prop}[三角形引理]
    如果复合$g\circ f$是单的,则$f$也是;如果复合是满的,则$g$也是.
  \end{prop}

  由定义,一个态射是单的意味着它可左消去,但这并不意味着存在一个左逆元。对于满态射,道理是一样的。
  \begin{defn}
    考虑两个态射$f\colon A \to B$和$g\colon B \to A$。 当$g \circ f = 1_A$时,称$f$为$g$的一个\termin[section]{section (category theory)},$g$为$f$ 的一个\termin[retraction]{retraction (category theory)},并称$A$为$B$的一个\termin[retract]{retract (category theory)}.
  \end{defn}
  \begin{prop}
    在一个范畴中,每个section都是单的,每个retraction都是满的。
  \end{prop}
  \begin{defn}
    一个拥有retraction的态射称为一个\termin{split monomorphism}.
    对偶地,一个拥有section的态射称为一个\termin{split epimorphism}.
    如果态射$f\colon A\to B$既split monoic又split epi,则称之为一个\emph{同构}(\termin{isomorphism}),
    并称$A$\emph{同构于}(\termin{isomorphic})$B$,记作$A\approx B$.  \glsadd{isomorphic}
  \end{defn}
  \begin{rem}
    显然split monomorphism必须是单的,split epimorphism必须是满的,因此同构必然既单且满. 然而,反之不然。

    一个既单且满的态射传统上称为一个\emph{双态射}(\termin{bimorphism}),尽管这不是一个好名字——当我们考虑高阶范畴时,它会导致一些混淆. 一个范畴要是能保证双态射都是同构就称为是\emph{平衡的}(\termin[balanced]{balanced category}).
  \end{rem}

\subsection{例子}
  \begin{exam}
    在范畴$\Set$、$\Grp$、$\Mod_R$或$\Top$中,单态射(相应地,满态射)正是那些作为函数是单的(相应地,满的)态射.
  \end{exam}
  \begin{exam}
    在范畴$\Ring$中,单态射正是那些作为函数是单的态射. 然而,满态射不必是满射. 例如,包含映射$\ZZ\hookrightarrow\QQ$就是一个作为函数不满的满态射.
    为了看清楚这一点,请注意这个事实:任何在$\QQ$上的同态总是由其在$\ZZ$上的取值决定.
    类似的论证表明,从任意交换环到其局部化的典范同态都是满态射。
  \end{exam}
  \begin{exam}
    在\emph{可除交换群}(divisible abelian group)的范畴$\DivAb$中,存在不是单射的单态射。
    例如,考虑典范映射$q\colon\QQ\to\QQ/\ZZ$,它显然不是单射,然而却是这个范畴中的单态射.
    事实上,任取一个可除交换群$G$以及两个群同态$f,g\colon G \to\QQ$使得$q\circ f = q\circ g$. 令$h = f - g$,我们有$q\circ h = 0$,于是问题归结为证明$h=0$. 给定一个元素$x\in G$,因为$q \circ h = 0$,故$h(x)$是一个整数. 如果$h(x)\neq0$,则立刻会产生矛盾.
  \end{exam}
  \begin{exam}
    在固定点的连通空间及其间固定点连续映射构成的范畴中,每个\emph{覆盖映射}都是单态射,然而它们往往不是单射.
    这正是覆盖映射的\emph{唯一提升性}(unique lifting property of covering maps).
    可参考一本代数拓扑的教材,例如\cite{AllenHatcher}.
  \end{exam}
  \begin{exam}
    在范畴$\Set$、$\Grp$或$\Mod_R$中,同构正是双射.
    在范畴$\Top$中,同构正是同胚.
    然而,一个连续映射即使是双射也不一定是同胚映射.
    例如,从半开半闭区间$[0,1)$到单位圆$S^1$的把$x$映到$e^{2\pi x i}$的映射是连续的双射,但不是个同胚:它的逆映射在$1$这一点不连续.
  \end{exam}

  这些反例表明单态射和满态射的概念并没有满足一开始的要求,因此范畴论学者又开发出一些变种来解决这个问题.
  进一步的参考资料是 \nlab 或 \hrefacc.

\section{自然变换}
  正如不研究同态的群论是不完整的,不研究函子的范畴论也是不完整的.然而,如果不研究函子之间的同态,则对函子的研究也是不完整的.

  \begin{defn}
    考虑从范畴$\Aa$到$\Bb$的两个函子$ F, G$.一个从$ F$ 到$ G$的\emph{自然变换}(\termin{natural transformation})$\alpha\colon F\to G$是以$\Aa$中对象为指标的$\Bb$中的一族态射
    \begin{equation*}
    (\alpha_A\colon F(A)\To G(A))_{A\in\ob\Aa}
    \end{equation*}
    满足:对于$\Aa$中的任何态射$f\colon A \to A'$,下图交换
    \begin{displaymath}
      \xymatrix{
          F(A)\ar[r]^{\alpha_A}\ar[d]_{F(f)}& G(A)\ar[d]^{G(f)}\\
          F(A')\ar[r]^{\alpha_{A'}}& G(A')
      }
    \end{displaymath}
    即,$\alpha_{A'}\circ F(f) =  G(f) \circ \alpha_{A}$.
  \end{defn}

  令$ F, G, H$是从范畴$\Aa$到$\Bb$的函子,$\alpha\colon F\to G, \beta\colon G\to H$是自然变换.则公式
  \begin{equation*}
    (\beta\circ\alpha)_A = \beta_A\circ\alpha_A
  \end{equation*}
  定义了一个新的自然变换$\beta\circ\alpha\colon F\to H$.

  这个复合显然是结合的并且对于每个函子有一个单位元.因此,一不小心就会认为存在这样这个范畴:它的对象是从$\Aa$到$\Bb$的函子,态射是自然变换.这个范畴称为从$\Aa$到$\Bb$ 的\emph{函子范畴}(\termin{functor category}),通常记作$[\Aa,\Bb]$,或者$\Bb^{\Aa}$. \glsadd{Fun}
  \begin{rem}
    我们说这样的结论是轻率的,因为这里有一个大小问题:

    如果$\Aa$和$\Bb$都是\emph{小的},则$[\Aa,\Bb]$也是\emph{小的}.

    如果$\Aa$是\emph{小的}而$\Bb$是\emph{局部小的},则$[\Aa,\Bb]$也是\emph{局部小的}.

    麻烦的是,即使$\Aa$和$\Bb$ 都是\emph{局部小的},如果$\Aa$不是\emph{小的},则$[\Aa,\Bb]$通常就不会是\emph{局部小的}.

    反过来说,如果$\Aa$和$[\Aa,\Set]$都是\emph{局部小的},则$\Aa$必须是\emph{本质小的}:参考
    \href{http://tac.mta.ca/tac/volumes/1995/n9/1-09abs.html}{\emph{Freyd \& Street (1995)}}.
  \end{rem}

  在前述讨论中,我们用到了自然变换之间的第一种复合.事实上,还存在着第二种复合.

  \begin{prop}
    考虑如下情形:
      \begin{displaymath}
        \xymatrix{
           \Aa\rtwocell^{F}_{G}{\alpha} &\Bb\rtwocell^{F'}_{G'}{\beta} & \Cc
        }
      \end{displaymath}
    其中$\Aa,\Bb,\Cc$是范畴,$F,G,F',G'$是函子,$\alpha,\beta$是自然变换.

    首先,我们有复合函子$ F' F$和$ G' G$,并且在每个对象$A\in\ob\Aa$上有一个交换图:
     \begin{displaymath}
        \xymatrix{
            F' F(A)\ar[r]^{F'(\alpha_A)}\ar[d]_{\beta_{F(A)}} &  F' G(A)\ar[d]^{\beta_{G(A)}}\\
            G' F(A)\ar[r]^{G'(\alpha_A)} &  G' G(A)
        }
    \end{displaymath}

    现在,我们定义$(\beta\ast\alpha)_A$为上面这个方形的对角线,即
    \begin{equation*}
      (\beta\ast\alpha)_A = \beta_{G(A)}\circ F'(\alpha_A) =  G'(\alpha_A)\circ\beta_{F(A)}
    \end{equation*}
    于是$\beta\ast\alpha$也是一个自然变换,称为$\alpha$ 和$\beta$的\termin{Godement product}.\glsadd{Godpord}
  \end{prop}

  利用自然性和函子性,不难验证上述命题以及下面这个命题:

  \begin{prop}[Interchange law]
    考虑这样这样情形:
      \begin{displaymath}
        \xymatrix{
          \Aa \ruppertwocell^{}_{}{\alpha} \rlowertwocell^{}_{}{\beta} \ar[r]
          & \Bb\ruppertwocell^{}_{}{\alpha'} \rlowertwocell^{}_{}{\beta'} \ar[r]
          & \Cc
        }
      \end{displaymath}
    其中 $\Aa,\Bb,\Cc$ 是范畴, $\alpha,\beta, \alpha', \beta'$ 是自然变换.于是下面的等式成立:
    \begin{equation*}
      (\beta'\circ\alpha')\ast(\beta\circ\alpha) = (\beta'\ast\beta)\circ(\alpha'\ast\alpha)
    \end{equation*}
  \end{prop}

  为简单起见,我们通常用$\beta\ast F$代替$\beta\ast1_{F}$,$G\ast\alpha$代替$1_{G}\ast\alpha$.

\subsection{自然性}
  也许你已经在不同背景的数学资料中见到过``自然''(\emph{natural})这个词.
  但是它究竟是什么意思呢?直观上讲,它意味着一个描述是不依赖于某些选择的.

  范畴论提供的这个概念的一个形式的定义.

  注意到``自然''通常出现在这样的场合:一种数学对象被``自然''地变换为另一种.
  这时,``自然''意味着这一过程可以被实现为一个自然变换.

  例如,术语``自然地同构于''(\emph{naturally isomorphic})就可以被定义如下.
  \begin{defn}
    令$\alpha$为从$\Aa$到$\Bb$的函子$F$和$G$之间的自然变换.
    如果,对任意的$\Aa$中的对象$A$,态射$\alpha_A$是$\Bb$中的一个同构,则称$\alpha$为一个\emph{自然同构}(\termin{natural isomorphism}),并称$F$和$G$是\emph{自然同构的}(\termin{naturally isomorphic}),记作$F\cong G$.
  \end{defn}

  \begin{exam}[反群]
    诸如
    \begin{quote}
      ``每个群自然同构于其反群''
    \end{quote}
    这样的话在现代数学中大量出现.

    上面这句话的真实含义是:
    \begin{quote}
      ``恒等函子$\Id \colon \Grp \To \Grp$与反群函子$\op \colon \Grp \To \Grp$是自然同构的.''
    \end{quote}

    这样一个翻译也自动给出了这句话的证明.
  \end{exam}

  \begin{exam}[二次对偶]
    每个线性空间都有一个\emph{自然的}到其二次对偶的线性单射.
    这些映射之所以称作是``自然的''是因为二次对偶实际上是一个函子,而这些映射其实就组成了从恒等函子到二次对偶函子的自然变换.
  \end{exam}

  然而,``不自然的'' 同构也不少.
  \begin{exam}[有限维线性空间之对偶]
  来自
  \href{http://en.wikipedia.org/wiki/Natural_transformation#Example:_dual_of_a_finite-dimensional_vector_space}{\emph{Wikipedia}}
  ,按照
  \href{http://mathoverflow.net/a/139398/43771}{\emph{MathOverflow}}
  的讨论进行了修正.

   每个有限维线性空间都同构于其对偶,然而这个同构不是自然的.

   \emph{Wikipedia} 给出的解释是这个同构依赖于一些任意的选取.
   然而,这与自然性其实没什么关系:线性对偶是一个反变函子,而恒等函子是协变的,因此本来就不可能同过一个自然变换去比较它们.

   一个更靠谱的理由来自于 \emph{MathOverflow}.
   Dan Petersen指出,如果我们考虑的是有限维线性空间及其间的线性同构组成的范畴,姑且记作$\Cc$, 那么我们就有两个显然的从$\Cc$到$\Cc^{\op}$的函子:一个是线性对偶,另一个则将每个线性同构映到其逆映射.
   这两个函子是真的\emph{不自然同构的}(\textbf{unnaturally isomorphic}).

   然而,正如 \emph{Wikipedia} 所指出的,如果我们选择的范畴中的对象是给定非退化双线性型的有限维线性空间,态射是保持双线性型的线性映射,则在这个范畴里线性对偶和恒等函子是自然同构的.
  \end{exam}

  为了形式化``一个同构不是自然的''这个想法,我们可以引入\emph{准自然变换}(\termin{infranatural transformation})这一概念,它指的是一族以出发范畴对象的指标的态射.
  因此,一个\emph{不自然同构}(\termin{unnatural isomorphism})就是一个不自然的准自然同构(infranatural isomorphism).

  \begin{exam}
    摘自 \href{http://mathoverflow.net/a/139392/43771}{\emph{MathOverflow}}

    令$\Cc$为只有一个对象和两个态射的范畴. 则恒等函子\emph{不自然地同构于}(\textbf{unnaturally isomorphic})将所有态射都映到恒等态射的函子.
  \end{exam}

  在实际使用中,一个具体对象之间的特定映射称为是\emph{自然同构}(\textbf{natural isomorphism}),隐含的意思是它其实是定义在这个范畴上的,并且给出一个函子间的自然同构. 否则,就称为\emph{不自然同构}(\textbf{unnatural isomorphism}).
  \begin{rem}
    一些作者为了区别这些概念,就用$\cong$来表示自然同构,用$\approx$来表示一般的同构,同时还用$=$表示相等.
  \end{rem}

  \emph{Wikipedia} 中还有一些例子,但未必合适:许多时候他们比较的函子事实上具有不同的定义域.
   一些真正的反例能在
   \href{http://mathoverflow.net/questions/139388/example-of-an-unnatural-isomorphism}{\emph{MathOverflow}}
   的帖子里找到.

\section{反变函子}
  有时,我们会考虑范畴之间的反转箭头指向的对应. 例如,集合在函数下的逆像. 尽管人们往往简单地将这类对应看做是原来范畴上的,但它们本质上是原来范畴的对偶上的函子.
  处于这样的考虑,范畴论学者引入了``反变函子''这个概念.
  \begin{defn}
    令$\Aa, \Bb$为两范畴,一个从$\Aa^{\op}$到$\Bb$的函子称为从$\Aa$到$\Bb$的\emph{反变函子}(\termin{contravariant functor}).
  \end{defn}
  为了以示区别,往往将原来的函子称为\emph{协变函子}(\termin{covariant functor}).
  \begin{exam}
   一个从$\Cc$到$\Set$的反变函子传统上叫做一个$\Cc$上的\termin{presheaf}. $\Cc$上的全体presheaf之范畴记作$\PSh(\Cc)$. 更一般地,通常把从$\Cc$到$\Dd$的一个反变函子称作一个$\Cc$上的$\Dd-$valued presheaf.
  \end{exam}
  反变函子间的自然变换之定义与协变的情况类似.
  \begin{defn}
    考虑从$\Aa$到$\Bb$的两个反变函子$F,G$. 一个从$F$到$G$的\emph{自然变换}(\termin{natural transformation})$\alpha\colon F\to G$是一族以$\ob\Aa$为指标的$\Bb$中的态射
    \begin{equation*}
    (\alpha_A\colon F(A)\To G(A))_{A\in\ob\Aa}
    \end{equation*}
    并满足:对任意的$\Aa$中的态射$f\colon A \to A'$,下图交换
    \begin{displaymath}
      \xymatrix{
         F(A)\ar[r]^{\alpha_A}&G(A)\\
         F(A')\ar[r]^{\alpha_{A'}}\ar[u]^{F(f)}&G(A')\ar[u]_{G(f)}
      }
    \end{displaymath}
    即,$\alpha_{A}\circ F(f) = G(f) \circ \alpha_{A'}$.
  \end{defn}

  有关函子的结果都能搬到反变函子上来. 既可以简单验证,也可以用对偶原理来办到这一点.

  类似的想法,考虑从对偶范畴到对偶范畴的函子,则有下面的概念.
  \begin{defn}
    每个函子$F\colon\Aa\To\Bb$诱导出一个\emph{反函子}(\termin{opposite functor})$F^{\op}\colon \Aa^{\op} \To \Bb^{\op}$ 它在对象和态射上作用的效果如同$F$.  \glsadd{oppositeF}
  \end{defn}
  尽管$F^{\op}$和$F$作用相同,但它们还是不同的函子:因为$\Aa^{\op}$和$\Aa$是不同的范畴,对于$\Bb^{\op}$同理. 引入反函子的动机大概与反环之类的类似:毕竟,它颠倒了复合的次序.

\subsection{关于函子的例子}
  这里有一些关于函子和反变函子的简单例子.
  \begin{exam}
    每个范畴$\Cc$都有一个恒等函子(identity functor)$\Id_{\Cc}$,这是它在函子的复合下对应的单位元. \glsadd{identityF}
  \end{exam}
  \begin{exam}
    对每个``具体的''范畴,例如$\Grp$,有一个从它到$\Set$的函子,称为\emph{遗忘函子}(\termin{forgetful functor}):它将每个群$G$ 映到集合$G$,群同态$f$映到函数$f$. 我们将在之后引入关于\emph{具体范畴}(concrete category)和\emph{遗忘函子}(forgetful functor)的技术性定义.
  \end{exam}
  \begin{exam} \glsadd{powerF} \glsadd{copowerF}
    \emph{幂集函子} $\Pp\colon\Set\to\Set$把每个集合$S$映到其幂集$\Pp(S)$,每个函数$f\colon A\to B$映到从$\Pp(A)$到$\Pp(B)$的``顺像映射''.
    \mapdes{\Pp(A)}{\Pp(B)}{U}{f(U)}

    其对偶,\emph{反变幂集函子} $\Qq$ 把每个集合$S$映到其幂集$\Pp(S)$,每个函数$f\colon A\to B$映到其``逆像映射'':
    \mapdes{\Pp(B)}{\Pp(A)}{V}{f^{-1}(V)}
  \end{exam}
  \begin{exam}\glsadd{constantF}
    函子$\Delta_B\colon\Aa \To \Bb$将$\Aa$中对象映到$\Bb$中一给定对象$B$,将$\Aa$中态射映到$B$的单位态射. 这个函子称为到$B$ 的\emph{常值函子}(\termin{constant functor})或者\emph{选择函子}(\termin{selection functor}).
  \end{exam}
  \begin{exam}\glsadd{diagonalF}
    \emph{对角函子}(\termin{diagonal functor})$\Delta$是从$\Bb$到函子范畴$[\Aa,\Bb]$的一个函子,它将每个对象$B$映到到$B$的常值函子.
  \end{exam}
  \begin{exam}
    $\Hom(-,-)$自己就能被视作两个函子:

    固定$\Cc$中一对象$A$,则$X\mapsto\Hom(A,X)$定义了一个从$\Cc$到$\Set$的函子,它把态射$f\colon X\to Y$映到
    \longmapdes{f_{\ast}}{\Hom(A,X)}{\Hom(A,Y)}{\phi}{f\circ\phi}
    \glsadd{pushforward}

    固定$\Cc$中一对象$B$,则$X\mapsto\Hom(X,B)$定义了一个从$\Cc$到$\Set$的反变函子,它把态射$f\colon X\to Y$映到
    \longmapdes{f^{\ast}}{\Hom(Y,B)}{\Hom(X,B)}{\phi}{\phi\circ f}
    \glsadd{pullback}

    进一步地,不难证明对$\Cc$中的任何态射$A\to B$和$C\to D$,下图交换:
    \begin{displaymath}
      \xymatrix{
         \Hom(A,C)\ar[r]&\Hom(A,D)\\
         \Hom(B,C)\ar[r]\ar[u]&\Hom(B,D)\ar[u]
      }
    \end{displaymath}
  \end{exam}

\section{满函子和忠实函子}
  你也许会发现,一个``具体的''范畴看起来像能通过遗忘函子嵌入到$\Set$中去. 但是这种``嵌入''并不和通常意义上的嵌入一致,因为它往往不是单射. 例如,一般来说在一个给定集合上有很多种群结构. 事实上,这样的``嵌入''是一个忠实函子.
  \begin{defn}
  一个函子$F\colon \Aa\to\Bb$被称为
  \begin{enumerate}[a)]
%    \setlength{\itemindent}{2ex}
    \item \emph{忠实的}(\termin[faithful]{faithful functor})/
    \emph{满的}(\termin[full]{full functor})/
    \emph{满忠实的}(\termin[fully faithful]{fully faithful functor}),如果对任何$X,Y\in\ob\Aa$,映射$\Hom_{\Aa}(X,Y)\to\Hom_{\Bb}(F(X),F(Y))$是单射 / 满射 / 双射.
    \item \emph{本质满的}(\termin[essentially surjective]{essentially surjective functor}),如果每个$B\in\ob\Bb$同构于某个$A\in\ob\Aa$的像$F(A)$.
  \end{enumerate}
  \end{defn}
  \begin{rem}
   一个忠实函子不必是对象之收集或者态射之收集上的单射. 也就是说,两个不同的对象$X$和$X'$可能被映到$\Bb$中的同一个对象(从而一个满忠实的函子的值域不必等价于$\Aa$),两个有着不同定义域和值域的态射$f \colon X\To Y$和$f' \colon X'\To Y'$也可能被映到$\Bb$中的同一个态射.

  同样,一个满函子也不必是对象或者上的满射. 在$\Bb$中完全可能有对象不是$\Aa$中对象的像. 这样的对象之间的态射当然不会是$\Aa$中态射的像.
  \end{rem}

  然而,我们有
  \begin{prop}
    一个函子在态射上是单射,当且仅当它不但是忠实的还在对象上是单射.
    这样的函子称为一个嵌入(\termin[embedding]{embedding (category theory)}).
  \end{prop}

  关于上述概念的最基本的命题是:
  \begin{prop}\label{prop:tri-(full,faithful)}
    考虑函子$F\colon\Aa\to\Bb$和$G\colon\Bb\to\Cc$.
    \begin{enumerate}[a)]
      \item 如果$F$和$G$都是isomorphisms / embeddings / faithful functors / full functors,则$G\circ F$也是.
      \item 如果$G\circ F$是embedding / faithful functor,则$F$也是.
      \item 如果$F$是essentially surjective functor,并且$G\circ F$是full functor,则$G$也是full functor.
    \end{enumerate}
  \end{prop}
  \begin{proof}
    a)是显然的. 将三角形引理运用到态射集,我们得到b).

    在c)的条件下,每个$\Bb$中的对象$B$都同构于某个$F(A)$. 因此$\Cc$中的态射$h\colon G(B)\to G(B')$就同构于$h'\colon GF(A)\approx G(B)\to G(B')\approx GF(A')$. 由于$G\circ F$是full functor,存在$\Aa$中的态射$f\colon A\to A'$使得$h'=GF(f)$.
    这样我们就得到的一个态射$g\colon B\to B'$使得$G(g)=h$.
  \end{proof}

  满函子和忠实函子具有很好的性质.
  \begin{prop}
    令$F\colon\Aa\to\Bb$为一个忠实函子,则
    \begin{enumerate}[a)]
      \item 它\emph{反射单态射}(\termin[reflects monomorphisms]{reflect (category theory)}). 也就是说,对于每个$\Aa$中的态射$f$,$F(f)$是单的蕴含$f$是单的.
      \item 它\emph{反射满态射}(\textbf{reflects epimorphism}). 也就是说,对于每个$\Aa$中的态射$f$,$F(f)$是满的蕴含$f$也是.
      \item 如果它还是满的,则它\emph{反射同构}(\textbf{reflects isomorphisms}).
    \end{enumerate}
  \end{prop}
  \begin{proof}
    设$F(f)$是单的,则对每对满足$f\circ g=f\circ h$的态射$g,h$,我们有$F(f)\circ F(g) = F(f)\circ F(h)$,从而$F(g)=F(h)$. 由于$F$是忠实的,故$g=h$. b)的证明是类似的.

    设$F(f)$是一个同构,其逆为$g$. 因为$F$是满的,存在某个$\Aa$中的态射$h$使得$g=F(h)$. 于是由$F$是忠实的可知$h$是$f$的逆.
  \end{proof}

  \begin{cor}
    一个满忠实的函子$F\colon\Aa\to\Bb$必须在同构意义下是对象之收集上的单射. 也就是说,每对$\Aa$中的对象,如果它们在$\Bb$中的像同构,则它们必须同构.
  \end{cor}

  不难验证每个函子$F\colon\Aa\to\Bb$都必须\emph{保持同构}(\termin[preserve isomorphisms]{preserve (category)}),也就是说,如果$f$ 是$\Aa$中的同构,则$F(f)$也是同构.
  然而,保持单态射和满态射的条件要强得多. 事实上,即使是满忠实的函子也不能保证这一点.
  然而如果我们假设$F$既fully faithful又essentially surjective,则可以验证它的确保持单态射和满态射.
  \begin{defn}
    一个既fully faithful又essentially surjective的函子称为一个\emph{弱等价}(\termin{weak equivalence}).
  \end{defn}

  事实上,弱等价的性质远不止上面提到的这些,如果选取适当的数学基础,则它能保持和反射所有有意思的范畴性质,因此在现代数学中地位非凡.

\subsection{范畴之等价}
  为了刻画两个范畴享有相似的性质这一想法,最自然的是考虑在$\Cat$(或者更一般的,$\CAT$)中的同构. 因此,我们定义
  \begin{defn}
    两个范畴$\Aa$和$\Bb$称作是\emph{同构的}(\termin[isomorphic]{isomorphic categories}),如果存在函子$F\colon\Aa\to\Bb$和$G\colon\Bb\to\Aa$互为对方之逆.
  \end{defn}
  不难验证一个同构$F$的逆是唯一的,因此往往记作$F^{-1}$.

  \begin{prop}
    一个函子是同构,当且仅当它满忠实的并且在对象上是双射.
  \end{prop}

  \begin{exam}
    有限群表示论中最基本的事实就是一个有限群的表示范畴同构于其群代数上的左模范畴.
  \end{exam}

  然而,同构的条件太强了,在实际使用中像上面这样的例子是罕见的.
  更务实的概念是考虑范畴的``等价''.
  \begin{defn}
    两个范畴$\Aa$和$\Bb$称作是\emph{等价的}(\termin{equivalent}),如果存在函子$F\colon\Aa\to\Bb$和$G\colon\Bb\to\Aa$以及自然同构$F\circ G\cong \Id_{\Bb}$和$\Id_{\Aa}\cong G\circ F$. 在这种情况下,我们称$F$是一个从$\Aa$到$\Bb$的\emph{等价}(
    \termin{equivalence})而$G$是其\emph{弱逆}(\termin{weak inverse}).
  \end{defn}
  \begin{rem}
    一个等价所提供的信息不足以构造出其弱逆及相应的自然同构:它们还依赖于其他选择.(参考下面的例子)
  \end{rem}
  \begin{rem}
    范畴的等价并没有一个标准记号,$\Aa\equiv\Bb$和$\Aa\simeq\Bb$\glsadd{equivalent}都合理且很常见.
  \end{rem}

  等价和弱等价之间的最明显关系是
  \begin{prop}
    一个等价同时也是一个弱等价.
  \end{prop}
  \begin{proof}
    令$F\colon\Aa\to\Bb$为一个等价,带有弱逆$G$和自然同构$\Id_{\Aa}\iso{\alpha} G\circ F$、$F\circ G \iso{\beta} \Id_{\Bb}$.
    则essential surjectivity来自$\beta$,full faithfulness则来自$\alpha$.
  \end{proof}

  在继续之前,我们先证明一个关于弱等价的命题.
  \begin{prop}
    如果$F\colon\Aa\to\Bb$和$G\colon\Bb\to\Cc$都是弱等价,则$G\circ F$也是.
  \end{prop}
  \begin{proof}
    满和忠实来自命题\ref{leprop:tri-(full,faithful)},而essential surjectivity很容易验证.
  \end{proof}

  在选择公理成立的前提下,我们有
  \begin{prop}
    函子$F\colon\Aa\to\Bb$是等价当且仅当它是弱等价.
  \end{prop}
  \begin{proof}
    令$F\colon\Aa\to\Bb$是一个弱等价,则我们构造一个弱逆$G\colon\Bb\to\Aa$如下:

    对每个$\Bb$中对象$B$,\emph{选择}一个$\Aa$中的对象$A$使得$F(A)\approx B$,令$G(B)=A$.
    对每个$\Bb$中的态射$f\colon B\to\B'$,令$G(f)$为如下复合在$\Aa$中的逆像:
    \begin{equation*}
      FG(B)\approx B\markar{f} B'\approx FG(B')
    \end{equation*}
    这样一个逆像的存在性和唯一性来自于$F$是满忠实的.

    自然同构$F\circ G\cong \Id_{\Bb}$来自于上面的构造. 为说明存在自然同构$\Id_{\Aa}\cong G\circ F$,只需注意到$F$反射同构.
  \end{proof}

  然而,如果不假定选择公理,我们就只能得到一个比等价弱的关系:
  \begin{defn}
    两个范畴$\Aa$和$\Bb$称作是\emph{弱等价的}(\termin{weak equivalent}),如果存在范畴$\Cc$以及弱等价$F\colon\Cc\to\Aa$和$G\colon\Cc\to\Bb$.
  \end{defn}

\subsection{例子}
\begin{exam}
  初等线性代数中的基本事实就是矩阵范畴$\Mat$等价于有限维线性空间范畴$\FinVect$. 然而,它们并不同构.
\end{exam}
\begin{exam}
  令$\Cc$为这样一个范畴,它有两个对象$A,B$和四个态射:两个单位态射$1_A, 1_B$和两个同构$f\colon A\to B, g\colon B\to A$.
  则$\Cc$等价于$\one$,等价函子将$\Cc$中每个对象映到$\bullet$,每个态射映到单位.
  然而,这里出现了两个弱逆:一个将$\bullet$映到$A$,另一个映到$B$.
\end{exam}
\begin{exam}
  考虑范畴$\Cc$,它具有一个对象$X$和两个态射$1_X, f\colon X\to X$,这里$f \circ f = 1_X$.
  当然,$\Cc$等价于它自己,而恒等函子就是一个等价. 然而,这里的自然同构有两种选择:一个来自$1_X$而另一个来自$f$. 这个例子说明即使弱逆是唯一的,自然同构的选取也不一定唯一.
\end{exam}

\section{子范畴}
  就像子集、子群这样的概念一样,对于范畴也有类似的概念.
  \begin{defn}
    范畴$\Cc$的一个\emph{子范畴}(\termin{subcategory})是这样一个范畴:它们对象和态射都在$\Cc$里.
  \end{defn}
    子范畴的定义里隐含了一个函子,它在对象和态射上的作用正如同包含映射,故名\emph{包含函子}(\termin{inclusion functor}).

  我们有两种不同的概念来刻画一个足够大到能揭示整个范畴的子范畴.
  \begin{defn}
    若包含函子是满的,则称该子范畴是\emph{满的}(\termin[full]{full subcategory}).
    若包含函子在对象上是满射,则称该子范畴是\emph{宽的}(\termin[wide]{wide subcategory},\termin[lluf]{lluf category}).
  \end{defn}

  显然包含函子必须是一个嵌入.
  反之,
  正如一个集合$S$的子集能被视为到$S$的单射之等价类,一个范畴$\Cc$的子范畴也能看作是到$\Cc$的``单的''函子之等价类.
  不难验证,一个函子是``单的''(作为$\Cat$或者$\CAT$里的态射),当且仅当它是个嵌入.

  总之,包含函子在同构意义下(up to isomorphism)就是嵌入.
  \begin{prop}
    一个函子$F\colon \Aa \to \Bb$是(满)嵌入当且仅当它通过同构$G\colon \Aa \to \Cc$和包含函子$E\colon \Cc \to \Bb$来factors through一个$\Bb$的(满)子范畴$\Cc$. 也就是说$F=E\circ G$.
  \end{prop}
  \begin{proof}
    令$\Cc$为$\Aa$在$\Bb$中的像.
  \end{proof}
  进一步地,包含函子在等价意义下(up to (weak) equivalence)就是忠实函子.
  \begin{prop}
    一个函子$F\colon \Aa \to \Bb$是忠实的,当且仅当它factors through一个嵌入$E_1\colon\Aa\to\Cc$、一个弱等价$G\colon\Cc\to\Dd$ 和一个包含$E_2\colon\Dd\to\Bb$. 也就是说$F=E_2\circ G\circ E_1$.
  \end{prop}
  \begin{proof}
    令 $\Dd$ 为$\Bb$的一个满子范畴,其中的对象正是$\Aa$在$\Bb$里的像.
    令 $\Cc$ 为这样一个范畴,其对象与$\Aa$相同,但是态射与$\Dd$相同.
  \end{proof}

  范畴能按(弱)等价关系分成不同的等价类,作为刻画这种等价类的工具,我们有
  \begin{defn}
    一个范畴称作是\emph{骨骼的}(\termin[skeletal]{skeletal category}),如果在这个范畴里同构和相等是一回事.
    传统上,一个范畴$\Cc$的\emph{骨架}(\termin[skeleton]{skeleton (category theory)})定义为$\Cc$的一个骨骼的子范畴,它的包含函子是一个到$\Cc$的等价.
  \end{defn}
  \begin{rem}
    然而,离开选择公理,更适当的定义是把$\Cc$的骨架定义为与之若等价的骨骼范畴.
  \end{rem}
  \begin{prop}
    同一范畴的两个骨架同构. 反之,两范畴等价,当且仅当它们的骨架同构.
  \end{prop}
  \begin{rem}
    在缺乏选择公理时,``等价''应该被代之以``弱等价''.
  \end{rem}

\subsection{术语注解}
    在数学的许多分支中,具有``泛性质''的对象往往不是唯一的,但却是\emph{在唯一的同构意义下唯一的}(\emph{unique up to unique isomorphism}). 一个很诱人的想法是考虑骨骼的范畴,因为骨骼范畴中同构的对象必相同,于是上面的对象就在事实上唯一了.
    然而, 在严格的``唯一''意义下,因为自同构的存在,这种通常想法是\textbf{错的}.

    举个例子,考虑\emph{笛卡尔积}(\emph{cartesian product},定义在2.x.x),尽管我们在口头上总是说``$A\times B$是$A$和$B$的积'',但严格来讲笛卡尔积是一个三元组,它包括一个对象$A\times B$以及两个投影态射$A\times B\to A$和$A\times B\to B$满足所要求的泛性质. 因此,即便范畴是骨骼的,于是只可能有唯一的对象$A\times B$成为$A$和$B$的积,但一般来讲它能以不同的方式(也就是说具有不同的投影态射)成为$A$和$B$的积:这些不同的积之间通过自同构而关联.

    最后,骨骼范畴能保证唯一性这样的想法在极少见的情况下可以是对的. 比如一个终对象(\emph{terminal object},定义在2.x.x)不含非平凡自同构,于是在一个骨骼范畴里,终对象必须是唯一的.

\section{Comma范畴}
  我们现在来介绍一种非常一般性的从给定范畴出发构造新范畴的途径. 像这样的的构造在本书中会很常见.

  \begin{defn}
    考虑三个范畴$\Aa$、$\Bb$、$\Cc$和两个函子$S$、$T$(分别代表source和target)
          \begin{displaymath}
            \xymatrix{
               \Aa\ar[r]^{S} & \Cc & \Bb\ar[l]_{T}                }
          \end{displaymath}
    构造\termin{comma category} $(S\down T)$\glsadd{commma} 如下:
    \begin{itemize}
      \item 对象:三元组$(A,f,B)$,其中$A$、$B$分别是$\Aa$和$\Bb$中的对象,而$f\colon S(A)\To T(B)$是$\Cc$里的一个态射.
      \item 从$(A,f,B)$到$(A',f',B')$的态射:对$(g,h)$,
                 其中$g\colon A\To A'$和$h\colon B\To B'$分别是$\Aa$和$\Bb$中的态射,并且满足下面的交换图:
                 \begin{displaymath}
                   \xymatrix{
                       S(A)\ar[d]_{S(g)}\ar[r]^{f} & T(B)\ar[d]^{T(h)}  \\
                       S(A')\ar[r]^{f'} & T(B')           }
                 \end{displaymath}
      \item 态射之复合$(g,h)\circ(g',h')$定义为$(g\circ g',h\circ h')$.
      \item 对象$(A,f,B)$的单位态射是$(1_{A},1_{B})$.
    \end{itemize}
  \end{defn}
  \begin{rem}
    相比$(S\down T)$这个记号,有些学者更偏向于用$(S/T)$.
  \end{rem}

  下面这个命题给出了comma范畴的``泛性质''.
  \begin{prop}\label{prop:comma-uni}
    每个comma category都带有两个函子.
    \begin{itemize}
      \item \termin{domain functor} $U\colon(S\down T)\To\Aa$,其作用为:
      \begin{itemize}
        \item 对象:$(A,f,B)\mapsto A$;
        \item 态射:$(g,h)\mapsto g$;
      \end{itemize}
      \item \termin{codomain functor} $V\colon(S\down T)\To\Bb$,其作用为:
      \begin{itemize}
        \item 对象:$(A,f,B)\mapsto B$;
        \item 态射:$(g,h)\mapsto h$;
      \end{itemize}
    \end{itemize}
    同时,还有自然变换$\alpha\colon S\circ U \to T\circ V$.
                 \begin{displaymath}
                   \xymatrix{
                       (S\down T)\ar[r]^-{V}\ar[d]_-{U}
                       &\Bb\ar[d]^{T}\\
                       \Aa\ar[r]_{S} \ar@{}[ur]^{\alpha}|-{\SelectTips{eu}{}\object@{=>}}
                       &\Cc %\ultwocell\omit
                               }
                 \end{displaymath}

    进一步地,comma category在上述性质下是\emph{万有的}(universal).
    也就是说,如果存在另一个范畴$\Dd$以及两个函子$U'\colon\Dd\to\Aa$和$V'\colon\Dd\to\Bb$使得自然变换$\alpha'\colon S\circ U' \to T\circ V'$存在.
                 \begin{displaymath}
                   \xymatrix{
                       \Dd\ar[r]^-{V'}\ar[d]_-{U'}
                       &\Bb\ar[d]^{T}\\
                       \Aa\ar[r]_{S} \ar@{}[ur]^{\alpha'}|-{\SelectTips{eu}{}\object@{=>}}
                       &\Cc %\ultwocell\omit
                               }
                 \end{displaymath}
    则存在唯一的函子$W\colon\Dd\to(S\down T)$使得
    \begin{equation*}
      U\circ W = U'\qquad V\circ W = V'\qquad \alpha\ast W = \alpha'.
    \end{equation*}
  \end{prop}
  \begin{proof}
    前面的性质直接来自于comma范畴的定义,其中的自然变换$\alpha$定义为
    \begin{equation*}
      \alpha_{(A,f,B)}=f
    \end{equation*}

    如果还存在另一四元组$(\Dd,U',V',\alpha')$满足性质,则我们可以定义出函子$W\colon\Dd\to (S\down T)$:
    \begin{align*}
      W(D) &= (U'(D),\alpha'_D ,V'(D)) \\
      W(f) &= (U'(f), V'(f))
    \end{align*}
    不难验证
    \begin{equation*}
      U\circ W = U'\qquad V\circ W = V'\qquad \alpha\ast W = \alpha'.
    \end{equation*}

    反之,上面的等式就迫使这样一个函子必须是$W$.
  \end{proof}

\subsection{例子}
  \begin{exam}%[Slice category]
    $(\Id_{\Cc} \down \Delta_I)$,也记作$(\Cc \down I)$称为$I$上的\termin{slice category} 或者 \emph{$I$上对象之范畴}.
  \end{exam}

  \begin{exam}%[Coslice category]
    $(\Delta_I \down \Id_{\Cc})$,也记作$(I \down \Cc)$称为$I$下的\termin{coslice category} 或者 \emph{$I$下对象之范畴}.
  \end{exam}

  \begin{exam}%[Arrow category]
    $(\Id_{\Cc}\down\Id_{\Cc})$正是\emph{箭头范畴}(\termin{arrow category})$\Cc^{\to}$.
  \end{exam}

  \begin{exam}\glsadd{T-arrow}\glsadd{S-arrow}
    在构造slice或者coslice范畴时,用其他函子$F$来替代恒等函子,这样就得到了一系列在研究伴随函子时很有用的范畴.
    例如,令$s,t$为$\Cc$中给定对象.
    范畴$(s\down F)$中的对象称为 \emph{从$s$到$F$的态射} 或者以$s$为domain的 \termin{$T-$structured arrow}.
    范畴$(F\down t)$中的对象称为 \emph{从$F$到$t$的态射} 或者以$t$为codomain的 \termin{$S-$costructured arrow}.
  \end{exam}

  \begin{exam}
    令$F\colon\Cc\to\Set$为一个函子,$1\colon\one\to\Set$是把$\one$中对象映成一个单点集的函子. 这个comma范畴$(1\down F)$被称为 \emph{$F$的元素之范畴}(\termin{category of elements of $F$}),记作$\Elts(F)$. 它也能用如下方式明确描述出来.
  \begin{itemize}
    \item 对象:对$(X,x)$,其中$X\in\ob\Cc$,$x\in F(X)$.
    \item 态射$f\colon(A,a)\To(B,b)$:
               $\Cc$中的满足$F(f)(a)=b$的态射$f\colon A\to B$.
    \item 态射复合由$\Cc$所诱导.
  \end{itemize}
  \end{exam}

  \begin{exam}
     令$\Delta_{\Aa}$为从$\Aa$到$\one$的\emph{常值函子}(\termin{constant functor}\glsadd{ConstantF}),$\Delta_{\Bb}$类似.
     则comma范畴$(\Delta_{\Aa}\down\Delta_{\Bb})$正是$\Aa$和$\Bb$的\emph{积}(\termin[product]{product of categories})$\Aa\times\Bb$,它可以被描述为:
  \begin{itemize}
    \item 对象:
               对$(A, B)$,其中$A$和$B$分别是$\Aa$和$\Bb$中的对象;
    \item 态射$f\colon(A,B)\To(A',B')$:
               箭头的对$(a,b)$,其中$a\colon A\to A'$和$b\colon B\to B'$分别是$\Aa$和$\Bb$中的态射;
    \item 态射之复合:
                                    \begin{equation*}
                                      (a', b') \circ (a, b) = (a' \circ a, b' \circ b);
                                    \end{equation*}
  \end{itemize}
  \end{exam}
  积$\Aa\times\Bb$携带有两个``投影''(projection)函子
  \begin{equation*}
    p_{\Aa}\colon\Aa\times\Bb\To\Aa\qquad p_{\Bb}\colon\Aa\times\Bb\To\Bb
  \end{equation*}
  它们可以显式地定义为
  \begin{align*}
    p_{\Aa}(A,B)=A,&\quad p_{\Bb}(A,B) = B,\\
    p_{\Aa}(a,b) =a,&\quad p_{\Bb}(a,b) = b.
  \end{align*}

  这些东西满足如下的``泛性质''.
  \begin{prop}
    考虑两范畴$\Aa$、$\Bb$. 对于每个范畴$\Cc$及每对函子$F\colon\Dd\to\Aa$、$G\colon\Dd\to\Bb$,都存在唯一的函子$H\colon\Dd\to\Aa\times\Bb$使得$p_{\Aa}\circ H=F, p_{\Bb}\circ H=G$.
  \end{prop}
  \begin{proof}
    这直接就是命题\ref{prop:comma-uni}的推论.
  \end{proof}
  一点术语:一个定义在两范畴之积上的函子称为\emph{双函子}(\termin{bifunctor},前缀表这个函子有两个``变元'').
  实际使用中,某物被称为在$X_1,X_2,\cdots$上是\emph{自然的}(\emph{natural})或者\emph{函子的}(\emph{functoral})意味着它可以看成以$X_1,X_2,\cdots$为变元的函子.
  \begin{exam}
    $\Hom_{\Cc}(-,-)$是从$\Cc^{\op}\times\Cc$到$\Set$的双函子.
  \end{exam}

\section{对偶原理}
  你也许已经注意到,每个关于函子的结果都有其对应的反变函子版本,每个关于单态射的结果都有其对应的满态射版本.
  这些事实是一个非常一般性的原理的特例.

  在定义\ref{def:category}下面的注中,我们表示一个范畴提供了一个双类型的一阶语言,而一个范畴性质就是在这个一阶语言中的一条语句.

  于是每当我们有一个范畴性质$\sigma$,则其对偶$\sigma^{\op}$就能通过反转箭头来得到. 也就是说
  \begin{enumerate}
    \item   将$\sigma$中出现的``source''和``target''互换.
    \item   将态射的复合顺序颠倒.
  \end{enumerate}

  然而就不难发现$\Cc$中的一句$\sigma$在逻辑上等价于它在$\Cc^{\op}$中的对偶$\sigma^{\op}$.

  总结上述,我们有
  \begin{thm}[范畴的对偶原理]
  $ $
  \begin{center}
    如果性质$\Pp$对所有范畴成立的,

    那么性质$\Pp^{\op}$也如此.
  \end{center}
  \end{thm}

  \begin{exam}
    一个态射是单的当且仅当它在对偶范畴里对应的态射是满的.
  \end{exam}

\section{Yoneda引理及可表达函子}
  在这一节里,我们将证明一个重要的定理. 在此之前,我们先定义一些有用的概念.

  可表达性是范畴论中最基本的概念之一,与伴随函子和Yoneda引理密切相关. 它也是泛性质这一思想背后的核心概念,弥漫于代数几何和代数拓扑之中.
  \begin{defn}
    函子$F\colon\Cc^{\op}\to\Set$(亦称为$\Cc$上的一个\emph{presheaf})的一个\emph{表示}(\termin{representation})是一个特定的自然同构
    \begin{equation*}
      \Phi\colon\Hom_{\Cc}(-,X)\To F
    \end{equation*}
    其中的$\Cc-$对象$X$称为$F$的一个\emph{表示对象}(\termin{representing object}),或\emph{泛对象} (\termin{universal object}).

    如果这样一个表示存在,则称函子$F$是\emph{可表达的}(\termin[representable]{representable functor})并\emph{被$X$表达}
    (\textbf{represented} by $X$).

    类似的,函子$F\colon\Cc\to\Set$被称为是\emph{余可表达的}(\termin[corepresentable]{corepresentable functor}),如果它看成$\Cc^{\op}$上的presheaf是可表达的.
  \end{defn}

  给定范畴$\Cc$,存在函子:
  \begin{equation*}\glsadd{Yoneda}
    \Upsilon\colon\Cc\To\PSh(\Cc)
  \end{equation*}
  将每个对象$X\in\ob\Cc$映到presheaf $\Hom_{\Cc}(-,X)$.

  Yoneda引理表明从被$X$表达的presheaf到另一个presheaf $F$之间的自然变换之集与集合$F(X)$有自然的一一对应.

  正式的说:
  \begin{thm}[Yoneda引理]
    存在典范同构
    \begin{equation*}
      \Hom_{\PSh(\Cc)}(\Upsilon(X),F)\cong F(X)
    \end{equation*}
    它在$X$和$F$上自然.
  \end{thm}
  \begin{rem}
    在有些著作中,被$X$表达的presheaf习惯上记作$h_X$. 此时上面的式子也写成
    \begin{equation*}\glsadd{Nat}
      \Nat(h_X,F)\cong F(X)
    \end{equation*}
    来强调presheaf之间的态射是自然变换.
  \end{rem}

  \begin{proof}
    关键的一点在于自然变换
    \begin{equation*}
      \alpha\colon\Hom_{\Cc}(−,X)\to F
    \end{equation*}
    由其分量
    \begin{equation*}
    \alpha_X\colon\Hom(X,X)\to F(X)
    \end{equation*}
    在$1_X$处的取值$\alpha_X(1_X)\in F(X)$唯一确定. 同时每个这样的取值都能扩展为一个自然变换$\alpha$.

    为此,我们固定一个取值$\alpha_X(1_X)\in F(X)$并考虑$\Cc$中任意一个对象$A$. 如果$\Hom(A,X)$是空集,则分量$\alpha_A$只能是从空集出发的平凡映射. 如果存在态射$f\colon A\to X$,则由自然性条件,下面的交换方形已经被唯一确定.
          \begin{displaymath}
            \xymatrix{
               \Hom(X,X)\ar[r]^-{\alpha_X}\ar[d]_{f^{\ast}}& F(X)\ar[d]^{F(f)} \\
               \Hom(A,X)\ar[r]^-{\alpha_A}& F(A)               }
          \end{displaymath}
    于是,自然变换$\alpha$的每个分量都被唯一确定了.

    反之,给定值$a=\alpha_X(1_X)\in F(X)$,我们可以定义一个自然变换$\alpha$如下:
    \begin{equation*}
      \alpha_A(f):=F(f)(a)\quad\forall A\in\ob\Cc,\forall f\in\Hom(A,X)
    \end{equation*}

    该同构在$X$和$F$上的自然性不外是一些显然成立的交换方形.
  \end{proof}

\subsection{推论}
  Yoneda引理有以下一些直接推论. 和Yoneda引理一样,它们的条件很容易满足,而用处非常大.
  \begin{cor}
    函子$\Upsilon$是个满嵌入(full embedding).
  \end{cor}
  函子$\Upsilon$习惯上称为\termin{Yoneda embedding}
  \begin{proof}
    对任意的$A,B\in\ob\Cc$,由Yoneda引理,我们有
    \begin{equation*}
      \Hom_{\PSh(\Cc)}(\Upsilon(A),\Upsilon(B))\cong (\Upsilon(B))(A) = \Hom_{\Cc}(A,B)
    \end{equation*}
    因此$\Upsilon$是满忠实的(fully faithful). 它在对象上的单射性是显然的.
  \end{proof}

  \begin{cor}
    对任意的$A,B\in\ob\Cc$,我们有
    \begin{equation*}
      \Upsilon(A)\cong\Upsilon(B)\iff A\cong B
    \end{equation*}
  \end{cor}
  \begin{proof}
    因为$\Upsilon$是满忠实的,故其反射同构(reflects isomorphisms).
  \end{proof}

  \begin{cor}
    设$F$是$\Cc$上的presheaf,则$F$的表示由泛对象$X$及元素$u\in F(X)$唯一确定. 对$(X,u)$还满足下面的泛性质:
    \begin{quote}
      对于任何由$A\in\ob\Cc$和$a\in F(A)$组成的对$(A,a)$,存在唯一的态射$f\colon A\to X$使得$F(f)(u)=a$.
    \end{quote}
  \end{cor}
  \begin{proof}
    注意到$F$的一个表示$\Phi$就是集合$\Hom_{\PSh(\Cc)}(\Upsilon(X),F)$里的一个元素,因此,由Yoneda引理,它通过$\Upsilon$对应于$F(X)$里的一个元素,比如$u\in F(X)$. 于是$\Phi$就由$X$和$u$唯一确定了.

    另一方面,对于$\Cc$中的每个对象$A$,$\Phi_{A}$给出了$\Hom_{\Cc}(A,X)$和$F(A)$之间的一一对应,由此不难得出前述泛性质.
  \end{proof}
  \begin{rem}
    允许有人会怀疑泛性质中提到的态射的存在性. 的确,有可能根本没有从$A$到$X$的态射. 但在这种情况下,Yoneda引理表明集合$F(A)$是空集,于是对$(A,a)$一开始就不可能存在.
  \end{rem}
  \begin{rem}
     \nlab 提供了本推论的另一种描述:$F$之表示是comma范畴$(\Upsilon\down\Delta_{F})$里的\emph{终对象}(\emph{terminal object}).
  \end{rem}

\section{习题}
\begin{ex}
  考虑两函子$S,T\colon\Aa\to\Bb$. 证明从$S$到$T$的自然变换之集合一一对应于domain functor和codomain functor的共同section之集合.
\end{ex}
\begin{ex}
  对任意范畴$\Aa,\Bb,\Cc$,
  \begin{enumerate}[a)]
    \item $[\Aa,\Bb]^{\op}\simeq[\Aa^{\op},\Bb^{\op}]$
    \item $(\Aa\times\Bb)^{\Cc}\simeq\Aa^{\Cc}\times\Bb^{\Cc}$
    \item $\Cc^{\Aa\times\Bb}\simeq(\Cc^{\Aa})^{\Bb}\simeq(\Cc^{\Bb})^{\Aa}$
  \end{enumerate}
\end{ex}
\begin{rec}
  $\Bb^{\Aa}$是$[\Aa,\Bb]$的另一种记号.
\end{rec}
\begin{ex}
  证明函子范畴$[\Aa,\Bb]$在$\Aa$和$\Bb$上自然.
\end{ex}
\begin{ex}
  利用对偶原理写出余可表达函子的明确定义.
\end{ex}
\begin{ex}
  证明协变的余可表达函子保持单态射,而反变的可表达函子则将单态射映成满态射.
\end{ex}
\begin{ex}
  考虑小范畴$\Cc$上的presheaf范畴$\PSh(\Cc)$,证明$\PSh(\Cc)$里的单态射就是每个分量都是单态射的自然变换.
  然而,如果考虑一般的函子范畴$[\Cc,\Dd]$,则前述性质就不成立了,请举出反例.
\end{ex}


\part{Elementary Structures}
\chapter{Set Theory}
\chapter{Order Theory}

\part{Group Theory}
\chapter{Groups}

\chapter{Rings}

\chapter{Modules}

\chapter{Polynomials}
%\chapter{Rings}%Rings
%\chapter{Modules}
\section{Some Definitions}
\subsection{Modules}
  \begin{defn}
    Let $R$ be a ring. A \termin[left module]{module}\index{left!module} over $R$, or a left $R-$module $M$ is an abelian group together with an left operation of $R$ on $M$ satisfies both left and right distributive laws.
  \end{defn}
%  \begin{rem}
%    We usually deal only with left modules and hence call these simply modules.
%  \end{rem}

  \begin{defn}
    Let $R$ be an \emph{entire} ring and let $M$ be a $R-$module. The \termin[torsion submodule]{torsion!submodule} $M_{\tor}$ is the set of \emph{torsion} elements in $M$.
  \end{defn}


  \begin{prop}
    Let $_R\mathfrak{I}$ be the ring of left ideals of $R$, then a $R-$module is also a $_R\mathfrak{I}-$module.
  \end{prop}
  \begin{rem}
    We denote $\mathfrak{I}$ to be the ring of ideals of $R$, then the same proposition holds for $\mathfrak{I}$.
  \end{rem}

\subsection{Algebras}
  \begin{defn}
    By an \termin{$R-$algebra}, we mean a $R-$module together with a bilinear map.
  \end{defn}

  \begin{defn}
    By an \termin{algebra over $A$}, we mean a ring-homomorphism $f\colon A\To B$ such that $f(A)$ is contained in the center of $B$. We say that the algebra is \termin[finitely generated]{finitely generated!algebra} if $B$ is \emph{finitely generated as a ring} over $f(A)$.
  \end{defn}
  \begin{rem}
    In this case, $B$ can be view as an $A-$module by the operation
    \begin{equation*}
      (a,b)\longmapsto f(a)b
    \end{equation*}

    More over, any $B-$module can be view as an $A-$module by the above operation.
  \end{rem}

\newpage\section{Homomorphisms}
In this section, we should assume $R$ is commutative.
\subsection{Exactness}
  \begin{prop}
    The functor $\Hom_R(-,N)$ is \termin[right exact]{right!exact}. Which means each sequence
    \begin{equation*}
      M'\To M\To M''\To 0
    \end{equation*}
    is exact, then the sequence
    \begin{equation*}
      \Hom_R(M',N)\Ot \Hom_R(M,N)\Ot \Hom_R(M'',N)\Ot 0
    \end{equation*}
    is exact.
  \end{prop}
  \begin{prop}
    The functor $\Hom_R(M,-)$ is \termin[left exact]{left!exact}. Which means each sequence
    \begin{equation*}
      0\To N'\To N\To N''
    \end{equation*}
    is exact, then the sequence
    \begin{equation*}
      0\To \Hom_R(M,N')\To \Hom_R(M,N)\To \Hom_R(M,N'')
    \end{equation*}
    is exact.
  \end{prop}

\subsection{Representation}

  Let $M$ be an $A-$module, then $\End_A(M)$ is a ring and $M$ is a module over $\End_A(M)$. Moreover, if $A$ is commutative and $\rho\colon R\To \End_A(M)$ is a ring homomorphism, then $M$ is also a module over $R$.
  \begin{defn}
    Let $R$ be a ring and let $\rho\colon R\To \End_A(M)$ be a ring homomorphism. Then $\rho$ is called a \termin{representation} of $R$ on $M$.
  \end{defn}

  \begin{defn}
    A \emph{morphism} of representation $\rho\colon R\To \End_A(M)$ into another $\rho'\colon R\To \End_A(M')$ is an $A-$module homomorphism $h\colon M\To M'$ such that the following diagram is commutative.
    \begin{displaymath}
      \xymatrix@R=0.5cm{
                &         \End_A(M) \ar[dd]^{[h]}     \\
              R \ar[ur]^{\rho} \ar[dr]_{\rho'}                 \\
                &         \End_A(M')                 }
    \end{displaymath}
    where $[h]$ is defined by $[h]f=h\circ g\circ h^{-1}$.
  \end{defn}

  \begin{defn}
    Let $G$ be a \emph{monoid}. By a \termin[representation]{representation!of monoid} of $G$ on an $A-$module $M$, we mean a homomorphism $\rho\colon G\To \End_A(M)^{\ast}$.
  \end{defn}
  \begin{rem}
    We may extend $\rho$ to be a representation of $A[G]$ on $\End_A(M)$.
  \end{rem}

\newpage\section{Category of Modules}
  \begin{defn}
    A module $M$ is said to be \termin[finitely generated]{finitely generated!module} or of \termin[finite type]{module!of finite type}, or \emph{\red finite} over $R$, if it has a finite number of generators.
  \end{defn}
  \begin{defn}
    An exact sequence of modules $0\To\longexseq{M'}{f}{M}{g}{M''}\To0$ is said to \termin[split]{split!exact sequence}, if it is \termin[equivalent]{equivalent!exact sequences} to the canonical one: $0\To\longexseq{M'}{\imath}{M'\oplus M''}{p}{M''}\To0$. That is there exist an isomorphism $M\To M'\oplus M''$ make the following diagram commutative:
  \begin{displaymath}
      \xymatrix{
         0\ar[r] & M'\ar[r]^{f}\ar@{=}[d] & M\ar[r]^{g}\ar[d]_{\cong} & M''\ar[r]\ar@{=}[d] & 0 \\
         0\ar[r] & M'\ar[r]_-{\imath} & M'\oplus M''\ar[r]_-{p} & M''\ar[r] & 0
      }
  \end{displaymath}
  \end{defn}

  \begin{prop}
    Let $0\To\longexseq{M'}{f}{M}{g}{M''}\To0$ be an exact sequence of modules. Then the following statements are equivalent:
    \begin{enumerate}
      \setlength{\itemindent}{2ex}
      \item this sequence is split;
      \item $f$ is split monomorphism;
      \item $g$ is split epimorphism.
    \end{enumerate}
  \end{prop}
  \begin{rem}
    Such a proposition is true in general abelian category.
  \end{rem}

  \begin{defn}
    In abelian category, a \termin{kernel} of a morphism $f\colon A\To B$, is a morphism $k\colon K\To A$ such that for any object $X$, the following sequence is exact:
    \begin{equation*}
      0 \To \Hom(X,K) \To \Hom(X,A) \To \Hom(X,B) \To 0
    \end{equation*}
  \end{defn}

  Similarly, we have
  \begin{defn}
    In abelian category, a \termin{cokernel} of a morphism $f\colon A\To B$, is a morphism $c\colon B\To C$ such that for any object $X$, the following sequence is exact:
    \begin{equation*}
      0 \To \Hom(A,X) \To \Hom(B,X) \To \Hom(C,X) \To 0
    \end{equation*}
  \end{defn}
  \begin{rem}
    It is clear that these definitions are consist with the definitions in Chapter1.
  \end{rem}

  \begin{thm}
    The category of modules over a ring is an abelian category.
  \end{thm}

\newpage\section{Free Module}
  \begin{defn}
    A non-empty family of elements of $M$ is called a \termin[basis]{basis!of module} of $M$ if it is linearly independent and generates $M$. By a \termin[free]{free!module} module we mean a module admits a basis, or zero module.
  \end{defn}

  \begin{defn}
    Let $S$ be a non-empty set, we define the \emph{free} module over a ring $R$ \emph{generated by} $S$, denoted by $R\<S\>$, to be the unique object in $\mathbf{Mod}_R$ satisfying the following universal property
    \begin{quote}
      For any $R-$module $M$ and any set-map $f\colon S\To M$, there exist exact one $R-$homomorphism $\tilde{f}\colon F\To M$ making the following diagram commutative
      \begin{displaymath}
        \xymatrix@R=0.5cm{
                &         F \ar[dd]^{\tilde{f}}     \\
              S \ar[ur]^{t} \ar[dr]_{f}                 \\
                &         M                 }
      \end{displaymath}
    \end{quote}
  \end{defn}

  These two definitions are consist. If $S$ is a non-empty set then it is consist with a diagram of type $S$ (view as a discrete category) say $\{Rx_i\}_{i\in S}$. For any $R-$module $M$ and any set-map $f\colon S\To M$, $(M,f)$ induces a nature cocone $(M,\phi_i)$ where $\phi_i$ is the composite of canonical maps
  \begin{alignat*}{5}
       Rx_i & \longrightarrow & R & \longrightarrow & M \\
       rx_i & \longmapsto & r & \longmapsto & ry_i
  \end{alignat*}
  where $y_i\in M$ is arbitrary. By such a translation, the solution of the above universal property is exact the colimit of $\{Rx_i\}_{i\in S}$ and hence is $\oplus Rx_i$.

  \begin{prop}
    If sets $S$ and $S'$ have same cardinality, then $R\<S\>\cong R\<S'\>$.
  \end{prop}

  \begin{warn}
     The converse may not be true! If it is true, we say $R$ has \termin[invariant dimension property]{invariant!dimension property}. In this case, we define the \termin{dimension} of a free module $F$ to be the cardinality of a basis of $F$.
  \end{warn}

  \begin{exam}
    Let $F$ be a free $A-$module with countable infinite basis, then $R=\End_A(F)$ is a ring. And as $R-$modules $R\cong R\oplus R\oplus \cdots \oplus R$ for any finite number of summands.
  \end{exam}

  The following propositions about invariant dimension property can be found in \cite{hungerford1974algebra}. We just list them here.

  \begin{prop}
    Let $F$ be a free $R-$module with infinite basis, then every basis of $F$ has same cardinality.
  \end{prop}

  \begin{prop}\label{3.5.2}
    Fields have invariant dimension property.
  \end{prop}

  \begin{prop}
    Let $I$ be an ideal of $R$ and $F$ be a free $R-$module with basis $X$. $\pi\colon F\To F/IF$ is the canonical map. Then $F/IF$ is a free $R/I-$module with basis $\pi(X)$ and $|\pi(X)|=|X|$.
  \end{prop}

  \begin{prop}
    If $f\colon R\To A$ is an epimorphism where $A$ has invariant dimension property, then so does $R$.
  \end{prop}
  \begin{proof}
    Consider $I=\ker f$, then $A\cong R/I$, and the following comes from the above proposition.
  \end{proof}

  \begin{prop}
    Commutative rings have invariant dimension property.
  \end{prop}


  \begin{defn}
    A module $M$ is called \termin[principal]{principal!module} if there exists an element $x\in M$ such that $M=Ax$.
  \end{defn}

\subsection{Vector Spaces}
  \begin{prop}
    If any $A-$module is free, then $A$ is a division ring.
  \end{prop}
  \begin{rem}
    A ring over which every module is projective is called \termin[semisimple]{semisimple!ring}.
  \end{rem}

  \begin{thm}[Kernel and Image]
    If $f\colon V\To W$ is a homomorphism of vector spaces over a field $k$, then
    \begin{equation*}
      \dim V=\dim\ker f+\dim\im f
    \end{equation*}
  \end{thm}

  \begin{ex}
    Let $V$ be a vector spaces over a field $k$, and let $U,W$ be subspaces. Show that
    \begin{equation*}
      \dim U+\dim W=\dim(U+W)+\dim(U\cap W)
    \end{equation*}
  \end{ex}
  \begin{proof}
    Let $\{v_1,v_2,\cdots,v_k\}$ be a basis of $U\cap W$. One can extend it to a basis of $U$, say $\{v_1,v_2,\cdots,v_k,u_{k+1},u_{k+2},\cdots,u_n\}$. Similarly, we get a basis of $W$, say $\{v_1,v_2\cdots,v_k,w_{k+1},w_{k+2},\cdots,w_m\}$.
    Then it is clear that $\{v_1,v_2,\cdots,v_k,u_{k+1},u_{k+2},\cdots,u_n,w_{k+1},w_{k+2},\cdots,w_m\}$ is a basis of $U+W$. Hence we get the formula.
  \end{proof}

  \begin{ex}
    Generalize the dimension statement of \ref{3.5.2} to free modules over a commutative ring.
  \end{ex}
  \begin{proof}
    Notice that $F/\mm F$ is a vector space over $A/\mm$.
  \end{proof}

  \begin{ex}
    Let $R$ be an entire ring containing a field $k$ as subring. Suppose that $R$ is a finite dimensional vector space over $k$ under the ring multiplication. Show that $R$ is a field.
  \end{ex}
  \begin{proof}
    Let $e^1,e^2,\cdots,e^n$ be a basis of $R$, and $e^ie^j=a^{ij}_ke^k$, then for any $\alpha=\alpha_ie^i,\beta=\beta_je^j\neq0$, the equations $\alpha X=\beta$ has a solution $X=\alpha^{-1}\beta$. Otherwise, $\det(\alpha_ia^{ij}_k)=0$, hence $\alpha X=0$ has a non-zero solution, which contradict to the entireness of $R$.
  \end{proof}

\newpage\section{Duality}
  \begin{defn}
    Let $E$ be a free module over a commutative ring $A$. By the \termin[dual module]{dual!module} $\codual{E}$ of $E$ we mean the module $\Hom(E,A)$. Its elements will be called \termin[functionals]{functional}.
  \end{defn}
  \begin{rem}
    If $x\in E,f\in \codual{E}$, we sometimes denote $f(x)$ by $\<x,f\>$. Keeping $x$ fixed, we get a linear map on $\codual{E}$, which is $0$ if and only if $x=0$. Hence we get an injection $E\To E^{\vee\vee}$ which is not always a surjective. If it is, we call $E$ is \termin{reflexive}.
  \end{rem}

  \begin{prop}
    Let $E$ ba a finite free module over the commutative ring $A$, of dimension $n$. Then $\codual{E}$ is also free, and $\dim\codual{E}=n$. If $\{x_1,\cdots,x_n\}$ is a basis for $E$, and $f_i$ is the functional such that $f_i(x_j)=\delta_{ij}$, then $\{f_1,\cdots,f_n\}$ is a basis for $\codual{E}$.
  \end{prop}

  We call such a basis $\{f_i\}$ the \termin[dual basis]{dual!basis} of $\{x_i\}$.

  \begin{cor}
    When $E$ is of finite type, then the natural map $E\To E^{\vee\vee}$ is an isomorphism.
  \end{cor}

  \begin{defn}
    Let $V,V'$ be two vector spaces over a field $K$. Given a bilinear map
    \mapdes{V\times V'}{K}{(x,x')}{\<x,x'\>}
    An element $x\in V$ is said to be \termin{orthogonal} to a subset $s'$ of $V'$ if $\<x,x'\>=0$ for all $x'\in S'$. We define the \termin[kernel]{kernel!of bilinear map} of the bilinear map on the left to be the subspace of $V$ which is orthogonal to $V'$, and similarly for the kernel on the right.
  \end{defn}
  \begin{thm}
    Let $V\times V'\To K$ be a bilinear map, let $W,W'$ be its kernels on the left and right respectively, and assume that $V'/W'$ is finite dimensional. Then the induced homomorphism $V'/W'\To\codual{(V/W)}$ is an iosmorphism.
  \end{thm}

  \begin{rem}
    Let $E$ be a module over a commutative ring $A$, then we may form two types of dual:

    $\dual{E}=\Hom(E,\QQ/\ZZ)$, viewing $E$ as an abelian group;

    $\codual{E}=\Hom_A(E,A)$, viewing $E$ as an $A-$module.

    Both are called dual. If we need to distinguish them, we call $\dual{E}$ the \termin{Pontrjagin dual}.
  \end{rem}

\newpage\section{Modules over Principal Rings}
  \emph{Throughout this section, we assume that $R$ is a principal entire ring. All modules are over $R$ unless otherwise specified.}

  \begin{prop}
    Principal rings have invariant dimension property.
  \end{prop}

  \begin{thm}
    Let $F$ be a free module, and $M$ a submodule. Then $M$ is free, and its dimension is less than or equal to the dimension of $F$.
  \end{thm}
  \begin{proof}
    Let $\{v_i\}_{i\in I}$ be a basis of $F$. For any subset $J$ of $I$, we let $F_J$ to be the free submodule generated by $\{v_j\}_{j\in J}$, and we let $M_J=F_J\cap M$. Let $S$ be the set of all pairs $(M_J,w)$ where $J$ is a subset of $I$, and $w$ is a basis of $M_J$ indexed by a subset $J'$ of $J$. For two such pairs $(M_J,w), (M_K,u)$, we define $(M_J,w)\prec(M_K,u)$ if $J\subset K$ and if the basis $u$ of $M_K$ is a extension of the basis $w$ of $M_J$. Then we will use Zorn's lemma and the rest are easy to prove.
  \end{proof}
  \begin{warn}
    This is not true in general case. For example, $\ZZ/6$ is a free $\ZZ/6-$module, but  its submodule $\{0,2,4\}$ is not free.
  \end{warn}

  \begin{cor}
    Let $E$ be a finitely generated module and $E'$ a submodule. Then $E'$ is finitely generated.
  \end{cor}

  \begin{defn}
    A free $1-$dimensional module over $R$ is called \termin{infinite cyclic}.
  \end{defn}
  \begin{defn}
    Let $E$ be a module. An element $x$ of $E$ is called a \termin[torsion element]{torsion!element} if there exists $a\in R,a\neq0$ such that $ax=0$. The \termin[torsion submodule]{torsion!submodule} $E_{\tor}$ is the set of torsion elements in $E$.
    We say that $E$ is a \termin[torsion module]{torsion!module} if it consists of torsion elements.
    If $E_{\tor}=0$, we say that $E$ is \termin[torsion free]{torsion!free}.
  \end{defn}

  \begin{thm}[Structure Theorem of f.g.Modules over PID]\label{f.g.module over PID}
    Let $E$ be finitely generated. Then $E/E_{\tor}$ is free. There exists a free submodule $F$ of $E$ such that
    \begin{equation*}
      E=E_{\tor}\oplus F
    \end{equation*}
    The dimension of such a submodule $F$ is uniquely determined.
  \end{thm}
  \begin{proof}
    First prove that $E/E_{\tor}$ is torsion free, then use lemma \ref{torsion free is free} and lemma \ref{free summand}.
  \end{proof}
  \begin{lem}\label{torsion free is free}
    A finitely generated torsion free module is free.
  \end{lem}
  \begin{lem}\label{free summand}
    Let $E,E'$ be modules, and assume that $E'$ is free. Let $f\colon E\To E'$ be surjective. Then there exist a free submodule $F$ of $E$ such that $\local{f}{F}$ is an isomorphism, and such that
    \begin{equation*}
      E=F\oplus\ker f
    \end{equation*}
  \end{lem}

  \begin{defn}
    The dimension of the free module $F$ in Theorem \ref{f.g.module over PID} is called the \termin{rank} of $E$.
  \end{defn}

  \begin{defn}
    Let $E$ be a module over $R$. For $x\in E$, the map $a\mapsto ax$ is a homomorphism of $R$ onto the submodule generated by $x$, and the kernel $\ann(x)$ is an ideal called the \termin{annihilator} of $x$. We say that $m\in R$ is a \termin{period} of $x$ if $(m)=\ann(x)$. An element $c\in R,c\neq0$ is said to be an \termin{exponent} for $E$ (resp. for $x$) if $cE=0$ (resp. $cx=0$).

    Let $p$ be a rime element. We denoted by $E(p)$ the submodule of $E$ consisting of all elements having an exponent which is a power $p^r(r\geqslant1)$. A \termin{$p-$submodule} of $E$ is a submodule contained in $E(p)$.

    We select once and for all a system of representatives for the prime elements of $R$ (modulo units).

    Let $m\in R,m\neq0$. We denoted by $E_m$ the kernel of the map $x\mapsto mx$. It consists of all elements of $E$ having exponent $m$.

    A module $E$ is said to be \termin{cyclic} if  it is isomorphic to $R/(a)$ for some element $a\in R$. Without loss of generality if $a\neq0$, one may assume that $a$ is a product of primes in our system of representatives, and then we could say that $a$ is the \termin{order} of the module.
  \end{defn}
  \begin{defn}
    Let $r_1,\cdots,r_s$ be integers $\geqslant1$. A $p-$module $E$ is said to be of \termin{type}
    \begin{equation*}
      (p^{r_1},\cdots,p^{r_s})
    \end{equation*}
    if it is isomorphic to the product of cyclic modules $R/(p^{r_i})$ ($i=1,\cdots,s$). If $p$ is fixed, then one could say that the module is of type $(r_1,\cdots,r_s)$ (relative to $p$).
  \end{defn}

  \begin{defn}
    Let $y_1,\cdots,y_m$ be elements of a module, we say that they are \termin{independent} if whenever we have a relation
    \begin{equation*}
      a_1y_1+\cdots+a_my_m=0
    \end{equation*}
    with $a_i\in R$, then we must have $a_iy_i=0$ for all $i$.
  \end{defn}
  \begin{warn}
    Observe that \emph{independent} does not mean \emph{linearly independent}.
  \end{warn}
  \begin{lem}
    Let $E$ be a torsion module of exponent $p^r(r\geqslant1)$ for some prime element $p$. Let $x_1\in E$ be an element of period $p^r$. Let $\overline{E}=E/(x_1)$. Let $\overline{y}_1,\cdots,\overline{y}_m$ be independent elements of $\overline{E}$. Then for each $i$ there exists a representative $y_i\in E$ of $\overline{y}_i$, such that the period of $y_i$ is the same as the period of $\overline{y}_i$. The elements $x_1,y_1,\cdots,y_m$ are independent.
  \end{lem}

  \begin{thm}
    Let $E$ be a finitely generated torsion module $\neq0$. Then $E$ is the direct sum
    \begin{equation*}
      E=\bigoplus_p E(p)
    \end{equation*}
    taken over all primes $p$ such that $E(p)\neq0$.
    Each $E(p)$ can be written as a direct sum
    \begin{equation*}
      E(p)=R/(p^{r_1})\oplus\cdots\oplus R/(p^{r_s})
    \end{equation*}
    with $1\leqslant r_1\leqslant\cdots\leqslant r_s$. The sequence $r_1,\cdots,r_s$ is uniquely determined.
  \end{thm}

  \begin{thm}
    Let $E$ be a finitely generated torsion module $\neq0$. Then $E$ is isomorphic to a direct sum of non-zero factors
    \begin{equation*}
      R/(q_1)\oplus\cdots\oplus R/(q_r)
    \end{equation*}
    where $q_1,\cdots,q_r$ are non-zero elements of $R$, and $q_1|q_2|\cdots|q_r$. The sequence of ideals $(q_1),\cdots,(q_r)$ is uniquely determined by the above conditions.
  \end{thm}
  \begin{rem}
    The ideals $(q_1),\cdots,(q_r)$ are called the \termin[invariants]{invariant!of module} of $E$.
  \end{rem}

  \begin{thm}[Elementary Divisors]
    Let $F$ be a free module, and let $M$ be a finitely generated submodule $\neq0$. Then there exists a basis $\BBb$ of $F$, elements $e_1,\cdots,e_m$ in this basis, and non-zero elements $a_1,\cdots,a_m\in R$ such that
    \begin{enumerate}[(i)]
      \item The elements $a_1e_1,\cdots,a_me_m$ form a basis of $M$ over $R$.
      \item We have $a_i|a_{i+1}$ for $i=1,\cdots,m-1$.
    \end{enumerate}
    The sequence of ideals $(a_1),\cdots,(a_m)$ is uniquely determined by the preceding conditions.
  \end{thm}
  \begin{rem}
    We call the ideals $(a_1),\cdots,(a_m)$ the \termin[invariants]{invariant!of module} of $M$ in $F$.
  \end{rem}

  \begin{thm}
    Assume that the elementary matrices in $R^{n\times n}$ generate $\GL_n(R)$. Let $(c_{ij})$ be a non-zero matrix with components in $R$. Then with a finite number of row and column operations, it is possible to bring the matrix to the form
    \begin{equation*}
      \diag\{a_1,a_2,\cdots,a_m,0,\cdots,0\}
    \end{equation*}
    with $a_1\cdots a_m\neq0$ and $a_1|a_2|\cdots|a_m$.
  \end{thm}



\subsection{Lattice}
  \begin{ex}
    Let $A$ be an additive subgroup of Euclidean space $\RR^n$, and assume that in every bounded region of space, there is only a finite number of elements of $A$. Show that $A$ is a free abelian group on $\leqslant n$ generators.
  \end{ex}
  \begin{proof}
    Induction on the maximal number of linearly independent elements of $A$ over $\RR$.

    When this number is $1$, since in every bounded region of space, there is only a finite number of elements of $A$, there must one $a\in A$, such that $d(a,0)=\min\limits_{x\in A\setminus\{0\}}d(x,0)$. Then it is clear that $A=\ZZ a$.

    Let $v_1,\cdots,v_m$ be a maximal set of linearly independent elements of $A$ over $\RR$, and let $A_0$ be the subgroup of $A$ contained in $\RR-$space generated by $v_1,\cdots,v_{m-1}$. By induction, one may assume that any element of $A_0$ is a linear integral combination of $v_1,\cdots,v_{m-1}$.

    Let $S$ be the subset of elements $v\in A$ of the form $v=a_1v_1+\cdots+a_mv_m$ with real coefficients $a_i$ satisfying
    \begin{align*}
      &0\leqslant a_i<1  \quad\text{if } i=1,\cdots,m-1 \\
      &0\leqslant a_m\leqslant1
    \end{align*}

    Since $S$ is in a bounded region of space, there must be an element $v'_m$ of $S$ with the smallest $a_m\neq0$. It is clear that $\{v_1,\cdots,v_{m-1},v'_m\}$ is a basis of $A$ over $\ZZ$.
  \end{proof}
  \begin{rem}
    Such $A$ is called a \termin{lattice} in a Euclidean space. The above exercise is applied in algebraic number theory to show that the group of units in the ring of integers of a number field modulo torsion is isomorphic to a lattice in a Euclidean space. See.
  \end{rem}

  \begin{ex}[Artin-Tate]
    Let $G$ be a finite group operating on a finite set $S$. For $w\in S$, denote $1\cdot w$ by $[w]$, so that we have the direct sum
    \begin{equation*}
      \ZZ\<S\>=\sum_{w\in S}\ZZ[w]
    \end{equation*}
    Define an action of $G$ on $\ZZ\<S\>$ by defining $\sigma[w]=[\sigma w]$ (for $w\in S$), and extending $\sigma$ to $\ZZ\<S\>$ by linearity. Let $M$ be a subgroup of $\ZZ\<S\>$ of rank $\#[S]$. Show that $M$ has a $\ZZ-$basis $\{y_w\}_{w\in S}$ such that $\sigma y_w=y_{\sigma w}$ for all $w\in S$.
  \end{ex}
    \begin{proof}
      First, we consider a Euclidean space $E$ generated by $S$. Then $M$ is a lattice of this space, hence there exists a real number $r>0$ such that for any $X\in E$, there exist a $x\in M$ such that
      \begin{equation*}
        \|X-x\|<r
      \end{equation*}

      For any integer $n$, we can find some $\{x_v\}_{v\in S}\subset M$ such that
      \begin{equation*}
        \left\|n[v]-x_v\right\|<r
      \end{equation*}

      For any $w\in S$, define $y_w$ as follow
      \begin{equation*}
        y_w=\sum_{\substack{
                                          \sigma\in G\\
                                          \sigma v=w}}\sigma x_v
      \end{equation*}

      Then it is clear that $\sigma y_w=y_{\sigma w}$. We will show that $\{y_w\}_{w\in S}$ is a $\RR-$basis of $M$.

      If not, then there exist some real numbers $c_w$, such that
      \begin{equation*}
        \sum_{w\in S} c_wy_w=0
      \end{equation*}
      Moreover, we can assume that, some $c_w=1$.

      Let $x_v=n[v]+b_v$, then
      \begin{align*}
        0&=\sum_{w\in S} c_w\sum_{\substack{
                                          \sigma\in G\\
                                          \sigma v=w}}\sigma n[v]+b_v\\
          &=\sum_{w\in S} \sum_{\substack{
                                          \sigma\in G\\
                                          \sigma v=w}} c_wn[w]+b_v
      \end{align*}

      Hence
      \begin{equation*}
        n\leqslant\left\|\sum_{w\in S} \sum_{\substack{
                                          \sigma\in G\\
                                          \sigma v=w}} c_wn[w]\right\|
        =\left\|\sum_{w\in S} \sum_{\substack{
                                          \sigma\in G\\
                                          \sigma v=w}} b_v\right\|<|S||G|r
      \end{equation*}

      However, $n$ can be large enough to make this inequality fail to hold. Thence $\{y_w\}_{w\in S}$ is a $\RR-$basis of $M$, and by multiply some rational numbers, we can get a $\ZZ-$basis of $M$ satisfying the condition.
    \end{proof}

\subsection{Seminorm}
  \begin{ex}\label{seminorm}
    Let $M$ be a finitely generated abelian group. By a \termin{seminorm} on $M$ we mean a real-valued function $v\mapsto|v|$ satisfying the following properties:
    \begin{enumerate}
      \item $|v|\geqslant0$ for all $v\in M$;
      \item $|nv|=|n||v|$ for $n\in\ZZ$;
      \item $|v+w|\leqslant|v|+|w|$ for all $v,w\in M$.
    \end{enumerate}
    By the \emph{kernel} of seminorm we mean the subset of elements $v$ such that $|v|=0$.
    \begin{enumerate}[a)]
      \item Let $M_0$ be the kernel. Show that $M_0$ is a subgroup. If $M_0=\{0\}$, then the seminorm is called a \termin{norm}.
      \item Assume that $M$ has rank $r$. Let $v_1,\cdots,v_r\in M$ be linearly independent over $\ZZ$ mod $M_0$. Prove that there exists a basis $\{w_1,\cdots,w_r\}$ of $M/M_0$ such that
          \begin{equation*}
            |w_i|\leqslant\sum_{j=1}^i|v_j|
          \end{equation*}
    \end{enumerate}
  \end{ex}
  \begin{proof}
    \begin{enumerate}[a)]
      \item For $v\in M_0$, by \emph{2}, $0,-v\in M_0$. For $u,v\in M_0$, by \emph{3}, $u+v\in M_0$.
      \item Without loss of generality, we can assume $M_0=\{0\}$.

               Let $M_1=\<v_1,\cdots,v_r\>$. Let $d$ be the exponent of $M/M_1$.
               Then $dM$ has a finite index in $M_1$ ($d\geqslant[M:dM]=[M:M_1][M_1:dM]$).

               Let $n_{j,j}$ be the smallest positive integer such that there exist integers $n_{j,1},\cdots,n_{j,j-1}$ satisfying
               \begin{equation*}
                 n_{j,1}v_1+\cdots+n_{j,j}v_j=dw_j
               \end{equation*}
               for some $w_j\in M$.

               Without loss of generality, we may assume $0\leqslant n_{j,k}\leqslant d-1$. Then $w_1,\cdots,w_r$ form the desired basis:

               First, $w_1,\cdots,w_r$ obversely form a $\RR-$basis. We need to prove that it is also a $\ZZ-$basis. For any $a\in M$, we have $a=a_1w_1+\cdots+a_rw_r$. Without loss of generality, we can assume $0\leqslant a_j<1$. Hence
               \begin{align*}
                 da & = a_1dw_1+\cdots+a_rdw_r \\
                  & = \sum_{j=1}^r a_j \sum_{i=1}^j n_{j,i}v_i \\
                  & = \sum_{i=1}^r\sum_{j=1}^r a_jn_{j,i}v_i
               \end{align*}

               Notice that $0\leqslant a_rn_{r,r}<n_{r,r}$, by the minimum of $n_{r,r}$, $a_r$ must be $0$, and the equality become
               \begin{equation*}
                 da=\sum_{i=1}^{r-1}\sum_{j=1}^{r-1} a_jn_{j,i}v_i
               \end{equation*}

               Since $0\leqslant a_jn_{j,j}<n_{j,j}$, we can get all $a_j=0$ by induction. Hence $a=0$, which means that $w_1,\cdots,w_r$ a $\ZZ-$basis.

               The desired inequality comes from
               \begin{align*}
                 |w_j| & = \frac{1}{d}|n_{j,1}v_1+\cdots+n_{j,j}v_j| \\
                  & \leqslant \sum_{i=1}^j\frac{n_{j,i}}{d}|v_i|\leqslant\sum_{i=1}^j|v_i|
               \end{align*}
    \end{enumerate}
  \end{proof}

  \begin{ex}
    Consider the multiplicative group $\QQ^{\ast}$ of non-zero rational numbers. For a non-zero rational number $x=a/b$ with $a,b\in\ZZ$ and $(a,b)=1$, define the \emph{height}
    \begin{equation*}
      h(x)=\log\max(|a|,|b|)
    \end{equation*}
    \begin{enumerate}[a)]
      \item Show that $h$ defines a seminorm on $\QQ^{\ast}$, whose kernel consists of $\pm1$ (the torsion group).
      \item Let $M_1$ be a subgroup of $\QQ^{\ast}$, generated by $x_1,\cdots,x_m$. Let $M$ be the subgroup of $\QQ^{\ast}$ consisting of those elements $x$ such that $x^s\in M_1$ for some positive integer $s$. Show that $M$ is finitely generated, and using Exercise \ref{seminorm}, find a bound for the seminorm of a set of generators of $M$ in terms of the seminorms of $x_1,\cdots,x_m$.
    \end{enumerate}
  \end{ex}
  \begin{proof}
    \begin{enumerate}[a)]
      \item \emph{1} is clear. For $x=a/b,(a,b)=1$, if $n>0$, then $(a^n,b^n)=1$, hence
               \begin{equation*}
                  h(x^n)=\log\max(|a^n|,|b^n|)=\log(\max(|a|,|b|))^n=nh(x)
               \end{equation*}
               if $n<0$, then
               \begin{equation*}
                  h(x^n)=\log\max(|b^{|n|}|,|a^{|n|}|)=\log(\max(|b|,|a|))^{|n|}=|n|h(x)
               \end{equation*}
               if $n=0$, then
               \begin{equation*}
                 h(x^0)=\log\max(1,1)=1=h(1)
               \end{equation*}
               Which proves \emph{2}.

               Let $v=a/b,w=c/d,(a,b)=(c,d)=1$, assume $(ac,bd)=e$, then
               \begin{align*}
                 h(vw) &=\log\max(|ac/e|,|bd/e|)\\
                          &\leqslant\log\max(|ac|,|bd|)\\
                          &\leqslant\log\max(|a|,|b|)+\log\max(|c|,|d|)=h(v)+h(w)
               \end{align*}
               Which is \emph{3}.

               Hence $h(x)$ is a seminorm on $\QQ^{\ast}$.

               If $h(a/b)=0$, then $\max(|a|,|b|)=1$, hence $|a|=|b|=1$. So the kernel consists of $\pm1$.
     \item For $x_i=a_i/b_i$, let $a_i$ and $b_i$ be factorized into power of primes $p_{i,j}$ and $q_{i,j}$ respectively.
              Then For any $x=a/b\in M_1$, $a$ (resp. $b$) must be a product of some power of $p_{i,j}$ (resp. $q_{i,j}$).

              For $x=a/b\in M$, assume $x^s\in M_1$, then $a^s$ (resp. $b^s$) must be a product of some power of $p_{i,j}$ (resp. $q_{i,j}$),
              and hence so do $a$ and $b$. Whence $M$ must be a subgroup of the subgroup of $\QQ^{\ast}$ which is generated by those primes.

              Moreover, since for any $x\in M$, there exist a $s$ such that $x^s\in M_1$ and hence can be written by a product of $x_1,\cdots,x_m$. This fact shows that $x_1,\cdots,x_m$ is a maximal linear independent term in $M$.
              By Exercise \ref{seminorm}, the generators of $y_1,\cdots,y_m$ of $M$ must satisfying
              \begin{equation*}
                h(y_i)\leqslant\sum_{j=1}^ih(x_j)
              \end{equation*}
              for $i=1,\cdots,m$. Whence we get a bound of the set of generators of $M$, say $\sum\limits_{i=1}^m h(x_i)$.
    \end{enumerate}
  \end{proof}

\newpage\section{Localization}
  \emph{Throughout this section, we assume that $A$ is a commutative ring. All modules are over $A$ unless otherwise specified.}

  \begin{ex}
    Let $A$ be a commutative ring and let $M$ be an $A-$module. Let $S$ be a multiplicative subset of $A$. Define $S^{-1}M$ in a manner analogous to the one we used to define $S^{-1}A$. Show that
    \begin{enumerate}[a)]
      \item $S^{-1}M$ is an $S^{-1}A-$module.
      \item the functor $M\mapsto S^{-1}M$ is exact.
    \end{enumerate}
  \end{ex}
  \begin{proof}
    \begin{enumerate}[a)]
      \item We consider pairs $(x,s)$ with $x\in M$ and $s\in S$. we define a relation
      \begin{equation*}
        (x,s)\sim(x',s')
      \end{equation*}
      between such pairs, by the condition that there exists an element $s_1\in S$ such that
      \begin{equation*}
        s_1(s'x-sx')=0
      \end{equation*}
      It is then trivially verified that this is an equivalence relation, and the equivalence class containing a pair $(x,s)$ is denoted by $x/s$. The set of equivalence classes is denoted by $S^{-1}M$.

      We define a $S^{-1}A-$module structure on $S^{-1}M$ by
      \begin{equation*}
        (a/s)(x/s')=ax/ss'
      \end{equation*}
      It is trivially verified that this is well defined. For any $a/s_1,a'/s_1'\in S^{-1}A$ and $x/s_2,x'/s_2'\in S^{-1}M$, we have the left and right distributive laws:
      \begin{align*}
        (\frac{a}{s_1}+\frac{a'}{s_1'})\frac{x}{s_2} & = \frac{s_1'a+s_1a'}{s_1s_1'}\frac{x}{s_2} = \frac{s_1'ax+s_1a'x}{s_1s_1's_2} \\
         & = \frac{ax}{s_1s_2}+\frac{a'x}{s_1's_2} = \frac{a}{s_1}\frac{x}{s_2}+\frac{a'}{s_1'}\frac{x}{s_2} \\
        \frac{a}{s_1}(\frac{x}{s_2}+\frac{x'}{s_2'}) & = \frac{a}{s_1}\frac{s_2'x+s_2x'}{s_2s_2'} = \frac{s_2'ax+s_2ax'}{s_1s_2s_2'} \\
         & = \frac{ax}{s_1s_2}+\frac{ax'}{s_1s_2'} = \frac{a}{s_1}\frac{x}{s_2}+\frac{a}{s_1}\frac{x'}{s_2'}
      \end{align*}
      \item For any $A-$homomorphism $f\colon M\To M'$, we define $S^{-1}f\colon S^{-1}M\To S^{-1}M'$ to be
               \begin{equation*}
                 S^{-1}f(x/s)=f(x)/s
               \end{equation*}
               It is trivially verified that this is a $S^{-1}A-$homomophism and the following diagram is commutative:
               \begin{displaymath}
                 \xymatrix{
                    M \ar[r]^-{\varphi_S}\ar[d]_{f} & S^{-1}M\ar[d]^{S^{-1}f}       \\
                    M' \ar[r]_-{\varphi_S}  & S^{-1}M'             }
               \end{displaymath}
               where $\varphi_S$ is the natural map $x\mapsto x/1$. Whence $S^{-1}$ is a functor.

               Given an arbitrary short exact sequence
               \begin{equation*}
                 0\To \longexseq{M'}{f}{M}{g}{M''} \To 0
               \end{equation*}
               We need to show the sequence
               \begin{equation*}
                 0\To \longexseq{S^{-1}M'}{S^{-1}f}{S^{-1}M}{S^{-1}g}{S^{-1}M''} \To 0
               \end{equation*}
               is exact.
               \begin{enumerate}[1)]
                 \item \emph{$S^{-1}f$ is injective:}
                          For any $S^{-1}f(x'/s)=0/1$, there exists a $s'\in S$ such that $s'f(x')=0$. Hence $s'x'\in\ker f$. Since $f$ is injective, hence $s'x'=0$, which means $x'/s=0/1$.
                 \item \emph{$S^{-1}g$ is surjective:}
                          For any $x''/s\in S^{-1}M''$, there exists a $x\in M$ such that $g(x)=x''$ since $g$ is surjective. Hence $S^{-1}g(x/s)=x''/s$.
                 \item \emph{$\ker S^{-1}g=\im S^{-1}f$:}
                          Since $g\circ f=0$ and $S^{-1}$ is a functor, we have $S^{-1}g\circ S^{-1}f=0$.
                          The rest is to show that $\ker S^{-1}g\subset\im S^{-1}f$:
                          For any $x/s\in\ker S^{-1}g$, there exists a $s'\in S$ such that $s'g(x)=0$. Hence $s'x\in\ker g$. Since $\ker g=\im f$, there exist a $x'\in M'$ such that $f(x')=s'x$. Whence $S^{-1}f(x'/s's)=x/s$.
               \end{enumerate}
    \end{enumerate}
  \end{proof}

  \begin{prop}
    If $N,P$ are submodules of an $A-$module $M$, then
    \begin{enumerate}[(i)]
      \item $S^{-1}(N+P)=S^{-1}N+S^{-1}P$;
      \item $S^{-1}(N\cap P)=S^{-1}N\cap S^{-1}P$;
      \item $S^{-1}(M/N)\cong S^{-1}M/S^{-1}N$.
    \end{enumerate}
  \end{prop}
  \begin{proof}
    \emph{(i)} follows from definition. \emph{(ii)} is easy to verify: if $x/s=y/t$ ($x\in N,y\in P, s,t\in S$), then there exist a $u\in S$ such that $u(tx-sy)=0$. Hence $w=utx=usy\in N\cap P$ and $x/s=w/ust\in S^{-1}(N\cap P)$. Consequently, $S^{-1}N\cap S^{-1}P\subset S^{-1}(N\cap P)$, and the reverse inclusion is clear.

    \emph{(iii)}. Apply the functor $S^{-1}$ to the following exact sequence:
    \begin{equation*}
      0 \To N \To M \To M/N \To 0
    \end{equation*}
  \end{proof}
  \begin{cor}
    $S^{-1}(M\oplus N)=S^{-1}M\oplus S^{-1}N$.
  \end{cor}

  \begin{ex}\label{3.ex10}
    Let $\pp$ be a prime ideal, define $M_{\pp}$ in a manner analogous to the one we used to define $A_{\pp}$.
    \begin{enumerate}[a)]
      \item Show that the natural map
               \begin{equation*}
                 M\To\prod M_{\pp}
               \end{equation*}
               of a module $M$ into the direct product of all localizations $M_{\pp}$ where $\pp$ ranges over all \emph{maximal} ideals, is injective.
      \item Show that a sequence
               \begin{equation*}
                 0\To \longexseq{M'}{}{M}{}{M''} \To 0
               \end{equation*}
               is exact if and only if the sequence
               \begin{equation*}
                 0\To \longexseq{M'_{\pp}}{}{M_{\pp}}{}{M''_{\pp}} \To 0
               \end{equation*}
               is exact for all primes $\pp$.
      \item Let $A$ be entire and let $M$ be torsion-free. For each prime $\pp$ of $A$ show that the natural map $M\To M_{\pp}$ is injective.
    \end{enumerate}
  \end{ex}
  \begin{proof}
    \begin{enumerate}[a)]
      \item Suppose the natural map is not injective, and $x$ is a non-zero element in its kernel, let $\aa=\ann(x)$. It is an ideal $\neq(1)$, hence must be contained in some maximal ideal $\mm$. But by the natural map, there exists a $s\notin\mm$ such that $sx=0$ which contradicts with $\aa\subset\mm$. Whence the natural map must be injective.
      \item We leave the proof to proposition \ref{L.P.3}.
      \item The condition implies that $\ann(x)=0$ for any $x\neq0$. Let $x$ be in the kernel, then there exists a $s\notin \pp$ such that $sx=0$, which can be true only when $x=0$.
    \end{enumerate}
  \end{proof}

\subsection{Local Properties}
  See \cite{atiyah1994introduction} and \cite{eisenbud1995commutative}.

  \begin{defn}
    A property $\Pp$ of $A$ (or of an $A-$module $M$) is said to be a \termin[local property]{local!property} if the following is true:
    \begin{quote}
      $A$ (or $M$) has $\Pp$ if and only if $A_{\pp}$ (or $M_{\pp}$) has $\Pp$ for each prime ideal of $A$.
    \end{quote}
  \end{defn}

  \begin{prop}\label{L.P.1}
    Let $M$ be an $A-$module, and $x\in M$. Then the following are equivalent:
    \begin{enumerate}
      \item $x=0$;
      \item $x/1=0/1$ in $M_{\pp}$ for all prime ideals $\pp$ of $A$;
      \item $x/1=0/1$ in $M_{\mm}$ for all maximal ideals $\mm$ of $A$.
    \end{enumerate}
  \end{prop}
  \begin{proof}
    It is clear \emph{1)$\Rightarrow$2)$\Rightarrow$3)}. Suppose \emph{3)}, and $x\neq0$. Let $\aa=\ann(x)$, it is an ideal $\neq(1)$, hence must be contained in some maximal ideal $\mm$. But $x/1=0/1$ in $M_{\mm}$, there exists a $s\notin\mm$ such that $sx=0$ which contradicts with $\aa\subset\mm$.
  \end{proof}
  \begin{cor}\label{L.P.1.1}
    Let $M$ be an $A-$module. Then the following are equivalent:
    \begin{enumerate}
      \item $M=0$;
      \item $M_{\pp}=0$ for all prime ideals $\pp$ of $A$;
      \item $M_{\mm}=0$ for all maximal ideals $\mm$ of $A$.
    \end{enumerate}
  \end{cor}

  \begin{prop}\label{L.P.2}
    Let $f\colon M\To N$ be an $A-$homomorphism. Then the following are equivalent:
    \begin{enumerate}
      \item $f$ is a monomorphism (resp. epimorphism, isomorphism);
      \item $f_{\pp}\colon M_{\pp}\To N_{\pp}$ is a monomorphism (resp. epimorphism, isomorphism) for each prime ideal $\pp$ of $A$;
      \item $f_{\mm}\colon M_{\mm}\To N_{\mm}$ is a monomorphism (resp. epimorphism, isomorphism) for each maximal ideal $\mm$ of $A$.
    \end{enumerate}
  \end{prop}
  \begin{proof}
    We prove only the statements for monomorphisms:

    \emph{1)$\Rightarrow$2)}. $0\to M\to N$ is exact, hence $0\to M_{\pp}\to N_{\pp}$ is exact, i.e. $f_{\pp}$ is a monomorphism.

    \emph{2)$\Rightarrow$3)} because a maximal ideal is prime.

    \emph{3)$\Rightarrow$1)}. Let $M'$ be the kernel of $f$, then $0\to M'\to M\to N$ is exact, hence so is $0\to M'_{\mm}\to M_{\mm}\to N_{\mm}$. Then $M'_{\mm}=0$ because $f_{\mm}$ is injective. Hence $M=0$ by \ref{L.P.1.1}. Which proved $f$ is injective.
  \end{proof}

  \begin{prop}\label{L.P.3}
    Let $0\to M'\to M\to M''\to 0$ be a sequence of $A-$modules. Then the following are equivalent:
    \begin{enumerate}
      \item $0\to M'\to M\to M''\to 0$ is exact;
      \item $0\to M'_{\pp}\to M_{\pp}\to M''_{\pp}\to 0$ is exact for each prime ideal $\pp$ of $A$;
      \item $0\to M'_{\mm}\to M_{\mm}\to M''_{\mm}\to 0$ is exact for each maximal ideal $\mm$ of $A$.
    \end{enumerate}
  \end{prop}
  \begin{proof}
    By the exactness of the functor $S^{-1}$, it is clear \emph{1)$\Rightarrow$2)$\Rightarrow$3)}.

    Denote $M'\To M$ by $f$, and $M\To M''$ by $g$.

    \emph{3)$\Rightarrow$1)}. By proposition \ref{L.P.2}, we only need to prove that $\ker g=\im f$.
    It is clear that $\ker g_{\mm}=(\ker g)_{\mm}$ and $\im f_{\mm}=(\im f)_{\mm}$, hence $(\ker g)_{\mm}=(\im f)_{\mm}$. Hence $\ker g=\im f$ because the natural map $M\To\prod M_{\mm}$ is injective.
  \end{proof}

\newpage\section{Projective Modules}
  \begin{prop}
    Let $P$ be a $R-$module, then the following are equivalent.
    \begin{enumerate}[P1]
      \item Given a homomorphism $f\colon P\To M''$ and surjective homomorphism $g\colon M\To M''$, there exist a homomorphism $h\colon P\To M$ make the following diagram commutative.
        \begin{displaymath}
          \xymatrix{
             & P \ar[dl]_{h}\ar[d]^{f} &        \\
             M \ar[r]_{g} & M'' \ar[r] & 0             }
        \end{displaymath}
      \item Every exact sequence $0\To M'\To M''\To P\To 0$ splits.
      \item There exists a module $M$ such that $P\oplus M$ is free, or in words, $P$ is a direct summand of a free module.
      \item The functor $M\mapsto\Hom_R(P,M)$ is exact.
    \end{enumerate}
  \end{prop}

  \begin{defn}
    A \termin[projective module]{projective!module} is a module satisfying \emph{P1}.
  \end{defn}

  \begin{prop}
    Direct sums and direct summands of projective modules are projective.
  \end{prop}
  \begin{warn}
    Submodules of projective modules need not be projective; a ring R for which every submodule of a projective left module is projective is called left \termin{hereditary}.
  \end{warn}

  \begin{prop}
    In principal domains or local rings, projective modules and free modules coincide.
  \end{prop}
  \begin{proof}
    It is clear that projective is torsion-free, hence free over a principal domain by Theorem \ref{f.g.module over PID}. The local ring case is a deep theorem by Irving Kaplansky, see \cite{kaplansky1958projective}, but the finite generated case is just a corollary to Nakayama's lemma.
  \end{proof}
  \begin{rem}
    There may be other rings over which the similar statement is true. For example, for polynomial rings over a field, any finite projective module is free. This is Quillen-Suslin theorem, see Chapter 21, Theorem 3.7.
  \end{rem}

  \begin{defn}
    A module is called locally free if its every localization is free.
  \end{defn}
  \begin{cor}
    Projective module is locally free.
  \end{cor}
  \begin{proof}
    It is clear that a localization of a projective module is a projective module over a local ring, hence is free.
  \end{proof}
  \begin{warn}
    The converse is not always true.
  \end{warn}



\subsection{Grothendieck Group}
  Let $A$ be a ring. Isomorphism classes of finite\footnote{finitely generated} projective modules form an abelian monoid. The addition is defined by
  \begin{equation*}
    [P]+[Q]\defeq[P\oplus Q]
  \end{equation*}

  The corresponding \emph{Grothendieck group} is denoted by $K(A)$.

  We define an equivalence relation (which is called \termin{stably isomorphic}) $\sim$ as follow: $P\sim P'$ if there exist finite free modules $F,F'$ such that $P\oplus F\cong P'\oplus F'$. Under this equivalence relation we obtain another group denoted by $K_0(A)$.

  \begin{prop}
    $K(A)=K_0(A)$
  \end{prop}
  \begin{proof}
    It suffices to show that $K_0(A)$ satisfying the universal property of $K(A)$.

   Let $M(A)$ be the monoid of isomorphism classes of finite projective modules. Let $i\colon M(A)\To K_0(A)$ be $[M]\mapsto[[M]]$ where the $[[M]]$ denotes the equivalent class contains $M$. This map is well-defined since any isomorphic modules must be stably isomorphic.

    Let $G$ be an arbitrary abelian group, and $f$ be a homomorphism from $M(A)$ to $G$. We need to show that there exists a unique homomorphism $h\colon K_0(A)\To G$ satisfying the commutative diagram:
    \begin{displaymath}
      \xymatrix@R=0.5cm{
                &         K_0(A) \ar[dd]^{h}     \\
              M(A) \ar[ur]^{i} \ar[dr]_{f}                 \\
                &         G                 }
    \end{displaymath}

    \emph{Uniqueness}. Let $h,h'$ be two homomorphisms satisfying the commutative diagram, then for any $[M]\in M(A)$, $h([[M]])=h'([[M]])$.
  \end{proof}
  \begin{warn}
    Since the monoid-homomorphism from an abelian monoid to its Grothendieck group may not be injective in general, two finite projective modules which are isomorphic may not be stably isomorphic.
  \end{warn}

  A family of modules $\FFf$ is called \termin[exact]{exact!family} if for any short exact sequence $0\To M'\To M\To M''\To 0$, $M\in\FFf$ if and only if $M',M''\in\FFf$.

  Let $\Ff$ be the free abelian group generated by isomorphism classes of modules in $\FFf$. Let $\Gamma$ be the subgroup generated by all elements
  \begin{equation*}
    [M']-[M]+[M'']
  \end{equation*}
  for which there exist an exact sequence $0\To M'\To M\To M''\To 0$.

  The factor group $\Ff/\Gamma$ is called the \termin{Grothendieck group} $K(R)$.

  More general, one can consider Grothendieck group of an exact category.

  By an \termin[exact category]{exact!category} $\Aa$, we mean an additive category together with a class of distinguished short sequences $A\To B\To C$ which are called ``exact sequences''.

  It is defined in the same way as before as the abelian group with one generator $[M]$ for each isomorphism class of objects in the category and one relation
  \begin{equation*}
    [A]-[B]+[C]=0
  \end{equation*}
  for each exact sequence $A\To B\To C$.

\subsection{Euler-Poincar\'{e} Maps}

  \begin{defn}
    Let $A$ be a ring and $\Gamma$ is an abelian group. An \termin{Euler-Poincar\'{e} mapping} is a corresponding $\varphi$ from an exact family of $A-$modules, or a Serre subcategory of $\Mod_A$, to $\Gamma$ such that: for any short exact sequence
    \begin{equation*}
      0\To M'\To M\To M''\To 0
    \end{equation*}
    we have
    \begin{equation*}
      \varphi(M)=\varphi(M')+\varphi(M'')
    \end{equation*}
  \end{defn}

  \begin{defn}
    If $M$ is a module, then a sequence of submodules
    \begin{equation*}
      M=M_1\supset M_2\supset\cdots\supset M_r=0
    \end{equation*}
    is called a \termin[finite filtration]{finite!filtration}, and $r$ is called the \termin[length]{length!of filtration} of the filtration.
  \end{defn}

  \begin{defn}
    A module $M$ is said to be \termin[simple]{simple!module} if $M\neq0$ and if it does not contain any submodule other than $0$ and $M$.
    A filtration is called \termin[simple]{simple!filtration} if each $M_i/M_{i+1}$ is simple.
  \end{defn}

  \begin{thm}[Jordan-H\"{o}lder Theorem]
    Two simple filtrations of a module are equivalent.
  \end{thm}

  \begin{defn}
    A module $M$ is said to be \termin[of finite length]{module!of finite length} if it is $0$ or if it admits a simple finite filtration. The length of such a simple filtration is called the \termin[length]{length!of module} of the module.
  \end{defn}


  \begin{thm}
    Let $\varphi$ be an rule which to each simple module associates an element of an abelian group $\Gamma$, and such that if $M\cong M'$ then
    \begin{equation*}
      \varphi(M)=\varphi(M')
    \end{equation*}
    Then $\varphi$ has a unique extension to an Euler-Poincar\'{e} mapping defined on all modules of finite length.
  \end{thm}
  \begin{proof}
    Given a simple filtration
    \begin{equation*}
      M=M_1\supset M_2\supset\cdots\supset M_r=0
    \end{equation*}
    we define
    \begin{equation*}
      \varphi(M)=\sum_{i=1}^{r-1} \varphi(M_i/M_{i+1})
    \end{equation*}
  \end{proof}

\subsection{Projective Modules over Dedekind Rings}
  \emph{Let $\oo$ be a Dedekind ring and $K$ its quotient field.}

  \begin{ex}
    Let $M$ be a finitely generated torsion-free module over $\oo$. Prove that $M$ is projective.
  \end{ex}
  \begin{proof}
    Given a prime ideal $\pp$, the localized module $M_{\pp}$ is finitely generated torsion-free over $\oo_{\pp}$, which is principal. Then $M_{\pp}$ is projective, so if $F$ is finite free over $\oo$, and $f\colon F\to M$ is a surjective homomorphism, then $f_{\pp}\colon F_{\pp}\to M_{\pp}$ has a splitting $g_{\pp}\colon M_{\pp}\to F_{\pp}$. There exists $c_{\pp}\in\oo$ such that $c_{\pp}\notin\pp$ and $c_{\pp}g_{\pp}(M)\subset F$ because $g_{\pp}(M)$ is finitely generated. The family $\{c_{\pp}\}$ generates the unit ideal $\oo$: if not then $\{c_{\pp}\}$ generates an proper ideal hence belongs to some maximal ideal $\mm$, but $c_{\mm}\notin\mm$ which is a contradiction. So there is a finite number of elements $c_{\pp_i}$ and elements $x_i\in\oo$ such that $\sum x_ic_{\pp_i}=1$. Let
    \begin{equation*}
      g=\sum x_ic_{\pp_i}g_{\pp_i}
    \end{equation*}
    Then $g\colon M\to F$ is a homomorphism, and $f\circ g=f\circ(\sum x_ic_{\pp_i}g_{\pp_i})=\sum x_i f\circ c_{\pp_i} g_{\pp_i}=\sum x_ic_{\pp_i} f_{\pp_i}\circ g_{\pp_i}=\sum x_ic_{\pp_i}\id_{\pp_i}=\id$.
  \end{proof}
  \begin{ex}\label{3.ex12}
    \begin{enumerate}[a)]
      \item Let $\aa,\bb$ be ideals. Show that there is an isomorphism:
               \begin{equation*}
                 \aa\oplus\bb\markar{\cong}\oo\oplus\aa\bb
               \end{equation*}
      \item Let $\aa,\bb$ be fractional ideals, and let $f\colon\aa\to\bb$ be an isomorphism (as $\oo-$modules, of course).
               Then $f$ has an extension to a $K-$linear map $f_K\colon K\to K$. Let $c=f_K(1)$. Show that $\bb=c\aa$ and that $f$ is given by the mapping $m_c\colon x\mapsto cx$.
      \item Let $\aa$ be a fractional ideal. For each $b\in\aa^{-1}$ the map $m_b\colon\aa\to\oo$ is an element of the dual $\codual{\aa}$.
               Show that $\aa^{-1}=\codual{\aa}$ under this map, and so $\aa^{\vee\vee}=\aa$.
    \end{enumerate}
  \end{ex}
  \begin{proof}
    \begin{enumerate}[a)]
      \item Assume $\aa,\bb$ are relatively prime. Then $\aa+\bb=\oo$ and $\aa\bb=\aa\cap\bb$.
               Consider the canonical map from $\aa\oplus\bb$ to $\aa+\bb$, we get the following exact sequence:
               \begin{equation*}
                 0 \To \aa\bb \To \aa\oplus\bb \To \oo \To 0
               \end{equation*}
               Since $\oo$ is a free $\oo-$module, $\aa\oplus\bb\cong\oo\oplus\aa\bb$.

               As for the general case, thanks to Exercise 2.19, there exists a $c\in K$ such that $c\aa$ is relatively prime to $\bb$. Hence we have
               \begin{equation*}
                 \aa\oplus\bb\cong c\aa\oplus\bb\cong\oo\oplus c\aa\bb\cong\oo\oplus\aa\bb
               \end{equation*}
      \item For any $x\in\aa$, we have $c=f_K(1)=f_K(x^{-1}x)=x^{-1}f(x)$ since $f_K$ is a $K-$linear map and an extension of $f$.
               Hence $f(x)=cx$, which proved the statements since $f$ is an isomorphism.
      \item The map $b\mapsto m_b$ is clearly injective and a $\oo-$homomorphism. It suffices to show that it is surjective.
               For any $f\in\codual{\aa}$, by \emph{b)}, there exists a $c\in K$ such that $f=m_c$ and $c\aa\subset\oo$. Since $\aa^{-1}=\{c\in K|c\aa\subset\oo\}$ (if $c\aa\subset\oo$, then $c\aa\aa^{-1}\subset\aa^{-1}$, hence $c\in\aa^{-1}$), we have $c\in\aa^{-1}$.
    \end{enumerate}
  \end{proof}

  \begin{ex}
    \begin{enumerate}[a)]
      \item Let $M$ be a projective finite module over the Dedekind ring $\oo$.
               Show that there exist free modules $F$ and $F'$ such that $F\supset M\supset F'$, and $F,F'$ have the same rank, which is called the \termin{rank} of $M$.
      \item Prove that there exists a basis $\{e_1,\cdots,e_n\}$ of $F$ and ideals $\aa_1,\cdots,\aa_n$ such that $M=\aa_1e_1+\cdots+\aa_ne_n$, or in other words, $M=\bigoplus\aa_i$.
      \item Prove that $M\cong\oo^{n-1}\oplus\aa$ for some ideal $\aa$, and that the association $M\mapsto\aa$ induces isomorphism of $K_0(\oo)$ with the group of ideal classes $\Pic(\oo)$.
    \end{enumerate}
  \end{ex}
  \begin{proof}
    \begin{enumerate}[a)]
      \item Let $x_1,\cdots,x_n$ generated $M$, then there exists a maximal linear independent subterm, say $x_1,\cdots,x_k$, then $F'=\<x_1,\cdots,x_k\>\subset M$. For any $x_i, k<i\leqslant n$, assume that $a_ix_i\in F'$, then there exists a $c\in K$ such that $x_i\in F=\<cx_i,\cdots,cx_k\>$ for any $1\leqslant i\leqslant n$. Hence $F'\subset M\subset F$ and $\rank(F)=\rank(F')$.
      \item Let $p_i\colon F\to\oo$ be the projection from $F$ to the $i-$th coefficient, then $p_i(M)=\aa_i$ is an ideal of $\oo$.
               It is then clear that $M=\aa_1e_1\cdots+\aa_ne_n$.
      \item By Exercise \ref{3.ex12}, there exists an ideal $\aa$ such that $M\cong\oo^{n-1}\oplus\aa$, and the map $M\mapsto\aa$ is a homomorphism.
               Since every ideal is a projective module, the map is clearly surjective. It suffices to show that if $M\cong\oo^n\oplus\aa$, $N\cong\oo^m\oplus\bb$ and $[\aa]=[\bb]$ in $\Pic(\oo)$, then $[M]=[N]$ in $K_0(\oo)$.

               $[\aa]=[\bb]$ means there exists a principal fractional ideal $\cc$ such that $\bb=\cc\aa$.
               Hence $N\cong\oo^m\oplus\cc\aa\cong\oo^{m-1}\oplus\cc\oplus\aa$. But $\cc$ is clearly a free module.
               Hence there exist free modules $F$ and $F'$ such that $M\oplus F\cong\oo^r\oplus\aa\cong N\oplus F'$, which means $[M]=[N]$ in $K_0(\oo)$.
    \end{enumerate}
  \end{proof}

\newpage\section{Inverse Limits}

  \begin{thm}
    Inverse limits exist in the category of groups., in the category of modules over a ring, and also in the category of rings.
  \end{thm}

  \begin{ex}
    Prove that the inverse limit of a system of simple groups in which the homomorphisms are surjective is either the trivial group, or a simple group.
  \end{ex}
  \begin{proof}
    For any homomorphism $\phi$ in this system, since $\phi$ is surjective and its domisn is simple, $\phi$ must be isomorphism or trivial. If all homomorphisms are isomorphism, then it is clear that the limit is isomorphic to the groups in system, which are simple, or trivial.
    If not, there must be one trivial homomorphism whose codomain is trivial since it is surjective. Consider the commutative diagram below
        \begin{displaymath}
          \xymatrix{
             \invlim G\ar[r]\ar[dr]^-{0} & G_i\ar[d]^{0}               \\
             \invlim G\ar[r]_-{0}\ar@{-->}[u]^-{1}_-{0} & G_j }
        \end{displaymath}
    where $0$ denote the trivial homomorphisms and $1$ denote the isomorphism. By the universality of $\invlim G$, $1=0$, hence $\invlim G$ is trivial.
  \end{proof}

  \begin{defn}
    Let $A$ be a commutative ring and $I$ a proper ideal. Define a \termin{$I-$Cauchy sequence} $\{x_n\}$ to be a sequence of elements of $A$ satisfying the following condition:

    Given a positive integer $k$, there exists $N$ such that for all $n,m\geqslant N$ we have $x_n-x_m\in I^k$.

    Define a \termin{null sequence} to be a sequence for which given $k$ there exists $N$ such that for all $n\geqslant N$ we have $x_n\in I^k$.
  \end{defn}
  \begin{prop}
    Define addition and multiplication of sequences termwise. Then the $I-$Cauchy sequences form a ring $\Cc$, the null sequences form an ideal $\Nn$.
  \end{prop}
  \begin{defn}
    The factor ring $\Cc/\Nn$ is called the \termin[$I-$adic completion]{$I-$adic!completion} of $A$.
  \end{defn}
  \begin{prop}
    There is a natural isomorphism
    \begin{equation*}
      \Cc/\Nn \cong \invlim A/I^n
    \end{equation*}
  \end{prop}

  \begin{defn}
    Let $p$ be a prime number. For $n\geqslant m$ we have a canonical surjective ring homomorphism
    \begin{equation*}
      \phi^n_m\colon \ZZ/p^n\ZZ \To \ZZ/p^m\ZZ
    \end{equation*}

    The projective limit is called the ring of $p-$adic integers, and denoted by $\ZZ_p$.
  \end{defn}
  \begin{warn}
     $\ZZ_p$ is not isomorphic to $\ZZ_{(p)}$, the localization of $\ZZ$ at $(p)$. In fact, $\ZZ_p$ has the cardinality of the continuum! By the way, $\ZZ_p$ is the completion of $\ZZ_{(p)}$ at the unique maximal ideal $p\ZZ_{(p)}$.
  \end{warn}

  \begin{ex}
    \begin{enumerate}[a)]
      \item Let $n$ range over the positive integers and let $p$ be a prime number. Show that the abelian groups $A_n=\ZZ/p^n\ZZ$ form a projective system under the canonical homomorphisms. Let $\ZZ_p$ be its inverse limit. Show that $\ZZ_p$ maps surjectively on each $\ZZ/p^n\ZZ$; that $\ZZ_p$ has no divisors of $0$, and has a unique maximal ideal generated by $p$. Show that $\ZZ_p$ is factorial, with only one prime, namely $p$ itself.
      \item Next consider all ideals of $\ZZ$ as forming a directed system, by divisibility. Prove that
               \begin{equation*}
                 \invlim_{(a)} \ZZ/(a) = \prod_p \ZZ_p
               \end{equation*}
               where the limit is taken over all ideals $(a)$, and the product is taken over all primes $p$.
    \end{enumerate}
  \end{ex}
  \begin{proof}
    \begin{enumerate}[a)]
      \item We prove the statements one by one.

        \emph{1). The abelian groups $A_n$ form a projective system under the canonical homomorphisms.}

        For $n\geqslant m$, $\phi^n_m$ is the quotient of $\pi_m$ induced by $\pi_n$, where $\pi_n$ is the canonical map from $\ZZ$ to $\ZZ/p^n\ZZ$.
        \begin{displaymath}
          \xymatrix{
             \ZZ\ar[r]^-{\pi_m}\ar[d]_-{\pi_n} & \ZZ/p^m\ZZ            \\
             \ZZ/p^n\ZZ\ar[ur]_{\phi^n_m} & }
        \end{displaymath}
        By the uniqueness of quotient, $\phi^j_k\circ \phi^i_j = \phi^i_k$ for any $i \leqslant j \leqslant k$. Hence the abelian groups $A_n$ form a projective system under the canonical homomorphisms.

        \emph{2). $\ZZ_p$ maps surjectively on each $A_n$.}

        By the discuss above, $(\ZZ,\pi_n)$ is also a cone of $(A_n,\phi^n_m)$, hence there exists a unique morphism from $(\ZZ,\pi_n)$ to the inverse limit $(\ZZ_p,\phi_n)$. Consider an arbitrary commutative diagram in this morphism:
        \begin{displaymath}
          \xymatrix{
             \ZZ\ar[r]^-{\pi_n}\ar[d] & \ZZ/p^n\ZZ            \\
             \ZZ_p\ar[ur]_{\phi_n} & }
        \end{displaymath}
        Since $\pi_n$ is surjective, hence so is $\phi_n$.

        \emph{3). A $p-$adic integer $\alpha$ equals to $0$ if and only if $\phi_n(\alpha)=0$ for all $n$. }

        Let $\<\alpha\>$ be the subring of $\ZZ_p$ generated by $\alpha$. Then $(\<\alpha\>,0)$ is a cone of $(A_n,\phi^n_m)$, hence there exists a unique morohism from $(\<\alpha\>,0)$ to $(\ZZ_p,\phi_n)$. However, both $0$ and the inclusion $i$ satisfying the commutative diagrams, hence $i=0$, i.e. $\alpha=0$.
      \begin{displaymath}
        \xymatrix@R=0.5cm{
               & &         A_n\ar[dd]^{\phi^n_m}     \\
            \<\alpha\>\ar@/^/[urr]^{0}\ar@/_/[drr]_{0}\ar@<0.5ex>[r]^{i}\ar@<-0.5ex>[r]_{0}
               & \ZZ_p \ar[ur]_{\phi_n} \ar[dr]^{\phi_m}                 \\
               & &         A_m                 }
      \end{displaymath}

        \emph{4). $\ZZ_p$ has no divisors of $0$.}

        Let $\alpha,\beta$ be two non-zero $p-$adic integers, by the discuss above, when $n$ is large enough, we have $\phi_n(\alpha)\neq0$ and $\phi_n(\beta)\neq0$.
        Taking $l,m$ to be the largest integers such that $p^l,p^m$ divide $\phi_n(\alpha)$ and $\phi_n(\beta)$ respectively. then for any $N\geqslant n$, the same statements hold for $\phi_N(\alpha)$ and $\phi_N(\beta)$. Therefore, we can assuming $n$ is larger than $l+m$, and hence $\phi_n(\alpha)\phi_n(\beta)\neq0$, which implies $\alpha\beta\neq0$. That shows $\ZZ_p$ has no divisors of $0$.

        \emph{5). Any $p-$adic integer can be uniquely determined by a sequence $x_n\in A_n$ such that}
             \begin{equation*}
                x_{n+1} \equiv x_n \mod p^n
             \end{equation*}

        For any $x_n\in A_n$, since $\ZZ_p$ maps surjectively on each $A_n$, there exist a $p-$adic integer $\alpha$, such that $\phi_n(\alpha)=x_n$. To show that we can choose the same $\alpha$ satisfying $\phi_m(\alpha)=x_m$ for any $m$, it suffices to check that $\phi^n_m(x_n)=(x_m)$ for all $m,n$, which are nothing but the congruences above. The uniqueness comes from \emph{3)}.

        \emph{6). A $p-$adic integer $\alpha$, which is determined by a sequence $x_n\in A_n$, is unit if and only if $x_1\not\equiv 0 \mod p$.}

        Let $\alpha$ be a unit, then there exists a $p-$adic integer $\beta$ such that $\alpha\beta=1$. Let $\beta$ be determined by a sequence $y_n\in A_n$, then
        \begin{equation*}
          x_ny_n \equiv 1\mod p^n
        \end{equation*}
        In particular, $x_1y_1\equiv 1\mod p$, hence $x_1\not\equiv 0 \mod p$.

        Conversely, let $x_1\not\equiv 0\mod p$, then it is easy to show that
        \begin{equation*}
          x_n\equiv x_{n-1}\equiv \cdots \equiv x_1 \not\equiv 0 \mod p
        \end{equation*}

        Therefore, for any $n$ we can find a $y_n$ satisfying $x_ny_n \equiv 1\mod p^n$.
        Since $x_{n+1}\equiv x_n\mod p^n$ and $x_{n+1}y_{n+1}\equiv x_ny_n \mod p^n$, then also $y_{n+1}\equiv y_n\mod p^n$. This means the sequence $y_n\in A_n$ determined a $p-$adic integer $\beta$, and by the congruences above, $\alpha\beta=1$.

        \emph{7). $\ZZ_p$ has a unique maximal ideal generated by $p$.}

        By \emph{6)}, any $\alpha\not\in p\ZZ_p$ is a unit, hence $p\ZZ_p$ is the unique maximal ideal.

        \emph{8). $\ZZ_p$ is factorial, with only one prime, namely $p$ itself.}

        Let $\alpha$ be an arbitrary non-zero $p-$adic integer, we need to show that $\alpha$ can be uniquely written in the form $p^m\varepsilon$, where $\varepsilon$ is a unit.

        If $\alpha$ is a unit, then the statement is true. Assume not, let $\alpha$ be determined by $x_n\in A_n$.
        Then, by \emph{6)}, $x_1\equiv0\mod p$. Since $\alpha\neq0$, by \emph{3)}, the congruences $x_n\equiv0\mod p^n$ can not hold for all $n$.
        Let $m+1$ be the smallest index for which
        \begin{equation*}
          x_{m+1}\not\equiv0\mod p^{m+1}
        \end{equation*}
        Then, for $s>0$
        \begin{equation*}
          x_{m+s}\equiv x_m \mod p^m
        \end{equation*}
        and therefore $y_s=x_{m+s}/p^m$ is an integer.
        From the congruences
        \begin{equation*}
          p^my_{s+1}-p^my_s=x_{m+s+1}-x_{m+s}\equiv0 \mod p^{m+s}
        \end{equation*}
        it follows that
        \begin{equation*}
          y_{s+1}\equiv y_s \mod p^s
        \end{equation*}
        for all $s>0$. Thus the sequence $y_s$ determine a $p-$adic integer $\varepsilon$. Since $y_1=x_{m+1}/p^m\not\equiv0\mod p$, by \emph{6)}, $\varepsilon$ is a unit.
        Finally, from
        \begin{equation*}
          p^my_s= x_{m+s}\equiv x_s \mod p^s
        \end{equation*}
        it follows that $p^m\varepsilon=\alpha$ as desired.

        We assume now $\alpha$ can be also written as $\alpha=p^k\eta$, where $\eta$ is a unit. Let $\eta$ be determined by $z_n\in A_n$, then
        \begin{equation*}
          p^my_s\equiv p^kz_s \mod p^s
        \end{equation*}
        for all $s>0$, and by \emph{6)}, $p$ never divides $y_s$ or $z_s$ since $\varepsilon,\eta$ are unit.
        From the congruence
        \begin{equation*}
          p^kz_{m+1}\equiv p^my_{m+1}\not\equiv0 \mod p^{m+1}
        \end{equation*}
        it follows that $k\leqslant m$. By symmetry, we have $m\leqslant k$, hence $k=m$.
        Therefore
        \begin{equation*}
          p^m\varepsilon=\alpha=p^m\eta
        \end{equation*}
        which implies $\varepsilon=\eta$ by \emph{4)}.

      \item First, we consider the structure of the inverse system of $\ZZ/(a)$.
               For any ideals $(a),(b)$ of $\ZZ$, there exist a homomorphism $\phi^a_b\colon\ZZ/(a)\To\ZZ/(b)$ in this system if and only if it is induced from $\pi_b$ by $\pi_a$:
        \begin{displaymath}
          \xymatrix{
             \ZZ\ar[r]^-{\pi_b}\ar[d]_-{\pi_a} & \ZZ/(b)            \\
             \ZZ/(a)\ar[ur]_{\phi^a_b} & }
        \end{displaymath}
               if and only if $(a)\subset\ker\pi_b$, if and only if $b\mid a$.

               To prove our statement, it suffices to check that $\prod\limits_p \ZZ_p$ satisfying the universal property of $\invlim \ZZ/(a)$. Therefore, we need to show the existence and uniqueness of $t$ for any $T$ satisfying the following commutative diagram:
        \begin{displaymath}
          \xymatrix@R=0.5cm{
               & &         \ZZ/(a)\ar[dd]^{\phi^a_b}     \\
            T\ar@/^/[urr]\ar@/_/[drr]\ar@{-->}[r]^-{t}
               & \prod_p\ZZ_p \ar[ur]_{\phi_a} \ar[dr]^{\phi_b}                 \\
               & &         \ZZ/(b)                 }
        \end{displaymath}

               Let $a=p_1^{n_1}\cdots p_s^{n_s}, b=p_1^{m_1}\cdots p_s^{m_s}$ where $p_1,p_2,\cdots,p_s$ are distinct primes and $n_i\geqslant m_i$ for all $1\leqslant i\leqslant s$.
               Then $(a)=\bigcap(p_i^{n_i})$ and $(b)=\bigcap(p_i^{m_i})$.

               Hence, by the Chinese Remainder Theorem,
               \begin{equation*}
                 \ZZ/(a)=\prod_{i}\ZZ/(p_i^{n_i}) \qquad \ZZ/(b)=\prod_{i}\ZZ/(p_i^{m_i})
               \end{equation*}

               Hence the diagram above is just the product of the following commutative diagrams:
        \begin{displaymath}
          \xymatrix@R=0.5cm{
               & &         \ZZ/(p_i^{n_i})\ar[dd]^{\phi^{n_i}_{m_i}}     \\
            T\ar@/^/[urr]\ar@/_/[drr]\ar@{-->}[r]^-{t_{p_i}}
               & \ZZ_{p_i} \ar[ur]_{\phi_{n_i}} \ar[dr]^{\phi_{m_i}}                 \\
               & &         \ZZ/(p_i^{m_i})                 }
        \end{displaymath}
               Hence $t$ is the product of $t_p$, where each $t_p$ uniquely exists and the product is taken over all primes $p$. Thus $t$ uniquely exists.
    \end{enumerate}
  \end{proof}

\subsection{Mittag-Leffler Condition}
  \begin{defn}
    A \emph{\red morphism} between inverse system is just a natural transformation between them.
  \end{defn}

  \begin{defn}
    In an abelian category $\Cc$, a sequence of inverse systems indexed by the same index $I$
    \begin{equation*}
      0 \To (A_i) \To (B_i) \To (C_i) \To 0
    \end{equation*}
    is said to be \termin[exact]{exact!sequence} if the corresponding sequence of abelian groups is exact for each $i$.
  \end{defn}

  \begin{rem}
     $\Cc^I$ is also abelian, the definition for exact sequence of $I-$system coincide with concept of exact sequence in the abelian category $\Cc^I$.
  \end{rem}

  \begin{prop}
    Taking inverse limit is a left exact functor $\Cc^I\To\Cc$.
  \end{prop}
  \begin{proof}
    Draw a commutative diagram, then the it is trivial to verify.
  \end{proof}
  \begin{warn}
    $\invlim$ is, in general case, not right exact, since you can't descent the limit cokernel to the termwise cokernels.
  \end{warn}

  If $I$ is ordered (not simply partially ordered) and countable, and $\Cc$ is the category $\Ab$ of abelian groups, the \emph{Mittag-Leffler condition} is a condition on the transition morphisms $\phi^i_j$ that ensures the exactness of $\invlim$.

  Specifically, Eilenberg constructed a functor
  \begin{equation*}
    \invlim\nolimits^1\colon\Ab^I\To\Ab
  \end{equation*}
  (pronounced ``lim one'') such that if $(A_i)$, $(B_i)$, and $(C_i)$ are three projective systems of abelian groups, and
    \begin{equation*}
      0 \To (A_i) \To (B_i) \To (C_i) \To 0
    \end{equation*}
  is a short exact sequence of inverse systems, then
    \begin{equation*}
      0 \To \invlim A_i \To \invlim B_i \To \invlim C_i \To \invlim\nolimits^1 A_i
    \end{equation*}
  is an exact sequence in $\Ab$.

  \begin{defn}
    If the ranges of the morphisms of the inverse system of sets $(A_n, \phi^m_n)$ are \emph{stationary},
    that is, for every $n$ there exists $m \geqslant n$ such that for all $l \geqslant m$, $\im \phi^l_n=\im \phi^m_n$ one says that the system satisfies the \termin{Mittag-Leffler condition}
    \footnote{\emph{Why the Mittag-Leffler condition is so named?} This condition comes from Bourbaki's Algebra \cite{bourbaki1998algebra}, but the name first appear in Bourbaki's General Topology, Chapter II, section 3.5 \cite{bourbaki1998general}. The main theorem is attributed to Mittag-Leffler, and is concerned with inverse systems of ``complete Hausdorff uniform spaces''.
    The Mittag Leffler condition mentioned there says the functions in the system have dense image. The usual theorem about inverse limits is a corollary, for sets with the ``discrete uniformity''.
    Classical Mittag-Leffler is given as an example of the main theorem. The spaces there are essentially holomorphic functions on balls centred at $0$, continuous on the boundary, with the uniform metric.
    --- \url{http://mathoverflow.net/questions/14717/mittag-leffler-condition-whats-the-origin-of-its-name}}.
     Denote $\ML$ to simplify the expressed.
  \end{defn}

  In fact, by the commutative diagram below, ($l\geqslant m\geqslant n$)
        \begin{displaymath}
          \xymatrix{
               A_l\ar[r]^{\phi^l_m}\ar[dr]_{\phi^l_n} & A_m\ar[d]^{\phi^m_n} \\
               & A_n                 }
        \end{displaymath}
  It is clear that $\im \phi^l_n \subset \im \phi^m_n$ is always true, hence form a \emph{descending chain}:
  \begin{equation*}
    A_n=\im \phi^n_n \supset \im \phi^{n+1}_n \supset \cdots \supset \im \phi^m_n \supset \cdots
  \end{equation*}
  Then $\ML$ says nothing but the chains are finite for all $n$.

  Suppose $(A_n)$ satisfying $\ML$ and each $A_n$ is non-trivial. Let $A'_n=\bigcap\limits_{m\geqslant n}\im \phi^m_n$.
  Then $(A'_n)$ is also an inverse system with non-trivial terms and $\invlim A'_n \cong \invlim A_n$. Moreover, each $\phi^m_n\colon A'_m\To A'_n$ is surjective.
  Hence $\invlim A'_n$ must be non-trivial.

  \begin{exam}
    The following situations are examples satisfying $\ML$:
    \begin{itemize}
      \item a system in which the morphisms $\phi^m_n$ are surjective.
      \item a system of finite-dimensional vector spaces.
      \item a system of finite-length $A-$modules.
    \end{itemize}
  \end{exam}

  Let's return to consider the inverse system of \textbf{abelian groups}, here is an important theorem:
  \begin{thm}
    If $(A_i)$ satisfying $\ML$, then $\invlim^1 A_i=0$.
  \end{thm}
  \begin{proof}
    Consider an arbitrary exact sequence:
    \begin{equation*}
      0 \To (A_i) \markar{f_i} (B_i) \markar{g_i} (C_i) \To 0
    \end{equation*}
    It suffices to show that $\invlim g_i$ is surjective.

    For any $c\in\invlim C_i$, let $(c_i)$ be its corresponding sequence.
    By the short exact sequence of $i-$terms, each $D_i=g_i^{-1}(c_i)$ is non-trivial, moreover, a coset of $A_i$ in $B_i$.
    Therefore $(D_i)$ also form an inverse system (of sets) satisfying $\ML$. Hence $\invlim D_i$ is non-trivial.
    But any element of $\invlim D_i$ lies in $\invlim B_i$ and is mapped to $c$, this proves $\invlim g_i$ is surjective as desired.
  \end{proof}

  \begin{exam}
    Taking $I$ to be the non-negative integers, letting $A_i = p^i\ZZ, B_i = \ZZ$, and $C_i = \ZZ/p^i\ZZ$. Then
    \begin{equation*}
      \invlim\nolimits^1 A_i = \ZZ_p/\ZZ
    \end{equation*}
    This shows that $(p^i\ZZ)$ dissatisfies $\ML$, in fact, $\invlim p^i\ZZ=0$.
  \end{exam}

  \begin{prop}
    If $(A_i)$ satisfying $\ML$, then we have an exact sequence
    \begin{equation*}
      0 \To \invlim A_i \To \prod A_i \markar{1-\phi} \prod A_i \To 0
    \end{equation*}
    where the map $\phi$ is the product of $\phi^i_{i-1}$.
  \end{prop}
  \begin{proof}
    For any $N$ large enough, the sequence below is exact:
    \begin{equation*}
      0 \To \invlim_{1\leqslant i\leqslant N} A_i \To \prod_{i=1}^N A_i \markar{1-\phi} \prod_{i=1}^N A_i \To 0
    \end{equation*}
    Moreover, the left terms satisfies $\ML$ since $(A_i)$ satisfies. Hence the sequence in proposition is exact.
  \end{proof}

  \begin{ex}
    \begin{enumerate}[a)]
      \item Let $(A_n)$ be an inverse system of commutative rings, and let $(M_n)$ be an inverse system of modules, each $M_n$ is a module over $A_n$ such that the following diagram is commutative:
          \begin{displaymath}
            \xymatrix@1{
               A_{n+1}\ar@{}[r]|{\times}\ar[d] & M_{n+1}\ar[r]\ar[d] & M_{n+1}\ar[d] \\
               A_n\ar@{}[r]|{\times} & M_n\ar[r] & M_n                 }
          \end{displaymath}
          Show that $\invlim M_n$ is a module over $\invlim A_n$.
      \item Let $M$ be a $p-$divisible group. Show that $T_p(M)$ is a module over $\ZZ_p$.
      \item Let $M,N$ be $p-$divisible groups. Show that $T_p(M\oplus N) = T_p(M) \oplus T_p(N)$, as modules over $\ZZ_p$.
    \end{enumerate}
  \end{ex}
  \begin{proof}
    \begin{enumerate}[a)]
      \item It is clear.
      \item $M[p^n]$ is a module over $\ZZ/p^n\ZZ$. Then by \emph{a)}, $T_p(M)=\invlim M[p^n]$ is a module over $\ZZ_p$.
      \item It is easy to check that $M \oplus N [p^n] = M[p^n] \oplus N[p^n]$. Then $T_p(M\oplus N) = T_p(M) \oplus T_p(N)$.
    \end{enumerate}
  \end{proof}

\newpage\section{Direct Limit}

  The direct limit of the direct system $(A_i,\phi^i_j)$ is can be constructed by the coproduct of the $A_i$ modulo a certain subobject:
  \begin{equation*}
    \dirlim A_i = \left.\coprod A_i \right/N
  \end{equation*}
  Here, $N$ is generated by all $x^i_j$, $i\leqslant j$, defined below:

  For any $x_i\in A_i$, $x^i_j=\imath(x_i)-\imath(\phi^i_j(x_i))$, where $\imath$ is the canonical injection $A_i\injection \coprod A_i$.

  Hence, if $x_i\in A_i$ and $x_j\in A_j$, then $\imath(x_i) = \imath(x_j)$ if there is some $k$ such that $\phi^i_k(x_i)=\phi^j_k(x_j)$.
  Heuristically, two elements in the coproduct are equivalent if and only if they ``eventually become equal'' in the direct system.
  \begin{ex}\label{3.ex19}
    Let $(A_i, \phi^i_j)$ be a directed system of modules. Let $a_k\in A_k$ for some $k$, and suppose that the image of $a_k$ in the direct limit $A$ is $0$. Show that there exists some index $j\geqslant k$ such that $\phi^k_j(a_k) = 0$.
  \end{ex}

  \begin{ex}
    Let $I,J$ be two directed sets, and give the product $I\times J$ as the product of categories. Let $A_{ij}$ be a $I\times J-$system of abelian groups.
    Show that the direct limits
    \begin{equation*}
      \dirlim_i\dirlim_j A_{ij} \qquad \text{and} \qquad \dirlim_j\dirlim_i A_{ij}
    \end{equation*}
    are naturally isomorphic. State and prove the same result for inverse limits.
  \end{ex}
  \begin{proof}
    This follows immediately from Proposition \ref{product and power}, i.e.
    \begin{equation*}
      \Ab^{I\times J}\simeq(\Ab^J)^I\simeq(\Ab^I)^J
    \end{equation*}
  \end{proof}

  \begin{ex}
    The functor $\dirlim\colon\Mod_R^I\To\Mod_R$ is exact.
  \end{ex}
  \begin{proof}
    Consider an arbitrary exact sequence:
    \begin{equation*}
      0 \To (M'_i) \markar{f_i} (M_i) \markar{g_i} (M''_i) \To 0
    \end{equation*}
    Only the injectivity of $f=\dirlim f_i$ is non-trivial.

    Assume $f(x)=0$, we need to show $x=0$. Let $(x_i)$ be the corresponding sequence of $x$,
    Then, the image of $f_i(x_i)$ in $\dirlim M_i$ is $0$, by Exercise \ref{3.ex19}, there exists some index $j\geqslant i$ such that $\phi^i_j(f_i(x_i)) = 0$.
    Therefore $f_j(\phi^i_j(x_i)) = 0$. But $f_j$ is injective, hence $\phi^i_j(x_i)=0$. Therefore $\imath(x_i)=\imath(x_i)-\imath(\phi^i_j(x_i))\in N$, which shows $x=0$ as desired.
  \end{proof}

  \begin{ex}
    \begin{enumerate}[a)]
      \item Let $\{M_i\}$ be a family of modules. For any module $N$ show that
                \begin{equation*}
                  \Hom(\bigoplus M_i,N) = \prod\Hom(M_i,N)
                \end{equation*}
      \item Show that
               \begin{equation*}
                 \Hom(N,\prod M_i) = \prod\Hom(N,M_i)
               \end{equation*}
    \end{enumerate}
  \end{ex}
  \begin{proof}
    \emph{a)} is a special case of Proposition \ref{Hom-dir.lim}, while \emph{b)} is a special case of Proposition \ref{Hom-inv.lim}, here $\Jj$ is the discrete index set of $\{M_i\}$.
  \end{proof}

  \begin{ex}
    Let $(M_i)$ be an inverse system of modules. For any module $N$ show that
    \begin{equation*}
      \Hom(N,\invlim M_i) = \invlim\Hom(N,M_i)
    \end{equation*}
  \end{ex}
  \begin{proof}
    This is again a special case of Proposition \ref{Hom-inv.lim}, here $\Jj$ is the index set (which is a FPOS) of $(M_i)$.
  \end{proof}

  \begin{ex}\label{3.ex24}
    Show that any module is a direct limit of finitely generated submodules.
  \end{ex}
  \begin{proof}
    The finitely generated submodules of a module $M$ form a directed system under the order of containing.
    Then $M$ induces a cocone of this system by the inclusions, hence there exist a unique homomorphism from $M$ to the direct limit, say $f$.

    $f$ is injective: for $f(x)=0$, there must be a finitely generated submodule $N$ such that $x\in N$ and hence its image in the direct limit is $0$. By Exercise \ref{3.ex19}, there exist a $N'\supset N$, such that $\phi^N_{N'}(x)=0$, which means $x=0$.

    $f$ is surjective: for any $a$ in the direct limit, let $x$ be it's preimage, then by the definitions, $f(x)=a$.
  \end{proof}

  \begin{defn}
    A module $M$ is called \termin{finitely presented} if there is an exact sequence
    \begin{equation*}
      F_1\To F_0\To M\To 0
    \end{equation*}
    where $F_0,F_1$ are free with finite basis. The image of $F_1$ in $F_0$ is said to be the submodule of \termin{relations}, among the free basis elements of $F_0$.
  \end{defn}
  \begin{rem}
    Any module is \emph{presented}, in the sense that there is an exact sequence
    \begin{equation*}
      F_1\To F_0\To M\To 0
    \end{equation*}
    where $F_0,F_1$ are free.

    First, for any module $M$ is a quotient of a free module $F_0$, hence we have the an exact sequence
    \begin{equation*}
      0\To R\To F_0\To M\To 0
    \end{equation*}
    Then $R$ is also a quotient of a free module $F_1$, hence we get an epimorphism $F_1\To R$.
    By Lemma \ref{factor through}, we obtain the require exact sequence.
  \end{rem}

  \begin{prop}
    A module $M$ is finitely presented if and only if there is an exact sequence
    \begin{equation*}
      0\To R\To F\To M\To 0
    \end{equation*}
    where $F$ is free with finite basis, $R$ is finitely generated.
  \end{prop}

  \begin{ex}
    Show that any module is a direct limit of finitely presented modules (not necessarily submodules). In other words, given $M$, there exists a directed system $(M_i)$ with $M_i$ finitely presented for all $i$ such that
    \begin{equation*}
      M\cong\dirlim M_i
    \end{equation*}
  \end{ex}
  \begin{proof}
    First, we have an exact sequence
    \begin{equation*}
      0\To R\To F\To M\To 0
    \end{equation*}
    By Exercise \ref{3.ex24}, $R$ is the direct limit of its finitely generated submodules. Denote this directed system by $(R_i)$, then it is clear that $R\To F$ is injective. Let $M_i=F/R_i$, then we have a family of exact sequence
    \begin{equation*}
      0\To R_i\To F\To M_i\To 0
    \end{equation*}
    where each $M_i$ is finitely presented. The connection homomorphisms in $(R_i)$ naturally induce directed systems for $F$ and $\{M_i\}$ of the same type with $(R_i)$. Therefore we have an exact sequence
    \begin{equation*}
      0\To (R_i)\To (F)\To (M_i)\To 0
    \end{equation*}
    Since $\dirlim$ is exact, we obtain an exact sequence of the limits:
    \begin{equation*}
      0\To \dirlim(R_i)\To \dirlim(F)\To \dirlim(M_i)\To 0
    \end{equation*}
    where $\dirlim(R_i)=R, \dirlim(F)=F$. Compare it with the first exact sequence, we have $M=\dirlim(M_i)$ as desired.
  \end{proof}

  \begin{ex}
    Let $E$ be a module. Let $(M_i)$ be a directed system of modules. If $E$ is finitely generated, show that the natural homomorphism
    \begin{equation*}
      \dirlim \Hom(E,M_i) \To \Hom(E,\dirlim M_i)
    \end{equation*}
    is injective. If $E$ is finitely presented, show that this homomorphism is an isomorphism.
  \end{ex}
  \begin{proof}
    We finish the proof step by step.
    \begin{enumerate}
      \item Any connection homomorphism $\phi^i_j$ in $(M_i)$ induces a homomorphism
                 \longmapdes{\Phi^i_j}{\Hom(E,M_i)}{\Hom(E,M_j)}{f}{\phi^i_j\circ f}
                 Thus $(\Hom(E,M_i),\Phi^i_j)$ form a directed system of the same type with $(M_i)$.

                 Any homomorphism $\phi^i\colon M_i\To\dirlim M_i$ induces a homomorphism
                 \longmapdes{\Phi^i}{\Hom(E,M_i)}{\Hom(E,\dirlim M_j)}{f}{\phi^i\circ f}

                 Then for any $\phi^i=\phi^j\circ\phi^i_j$, $\Phi^i=\Phi^j\circ\Phi^i_j$. Therefore $(\Hom(E,\dirlim M_i),\Phi^i)$ is a cocone of $(\Hom(E,M_i),\Phi^i_j)$. Hence there exist a unique morphism
                 \begin{equation*}
                    \dirlim \Hom(E,M_i) \To \Hom(E,\dirlim M_i)
                 \end{equation*}

                 We denoted the natural homomorphism by $\Psi$.

                 Notice that for any $i,j\in I$, there exist $k\in I$ such that $i\leqslant k,j\leqslant k$,
      \item \emph{If $E$ is free with finite basis, then $\Psi$ is an isomorphism.}

                 Let $b_1,\cdots,b_n$ be a basis of $E$. We check that $\Hom(E,\dirlim M_i)$ satisfies the universal property of $\dirlim \Hom(E,M_i)$.

                 For any cocone $(S,g_i)$, we need to show there exist a unique morphism $\Hom(E,\dirlim M_i)\To S$.

                 For any $f\in\Hom(E,\dirlim M_i)$, consider the preimage $x_1,\cdots,x_n$ of $b_1,\cdots,b_n$ in $M_i$, then the map $b_j\mapsto x_j$ induces a unique homomorphism $E\To M_i$.
                 In this way, we get a map $\Hom(E,\dirlim M_i)\To \Hom(E,M_i)$ hence a map $\Hom(E,\dirlim M_i)\To S$. The commutativity is easy to check.

                 Assume there exist two morphisms
                 \begin{equation*}
                   h_1,h_2\colon\Hom(E,\dirlim M_i)\To S
                 \end{equation*}

                 For any $f\in\Hom(E,\dirlim M_i)$, we need to show $h_1(f)=h_2(f)$.

                 Consider $f_i\in g_i^{-1}(h_1(f)), f_i'\in g_i^{-1}(h_2(f))$. For $1\leqslant j\leqslant n$, there exists $k_j\geqslant i$ such that $\phi^i_{k_j}(f_i(b_j))=\phi^i_{k_j}(f_i'(b_j))$. Therefore there exists $k\geqslant i$ such that $\phi^i_{k}(f_i(b_j))=\phi^i_{k}(f_i'(b_j))$ for any $1\leqslant j\leqslant n$.
                 But the map $b_j\mapsto\phi^i_{k}(f_i(b_j))$ induces a unique homomorphism $f_k\in\Hom(E,M_k)$, hence
                 $h_1(f)=g_i(f_i)=g_k(f_k)=g_i(f_i')=h_2(f)$.
      \item \emph{If $E$ is finitely presented by an exact sequence
                 \begin{equation*}
                   F_1\To F_0\To M\To 0
                 \end{equation*}
                 then $\Psi$ is an isomorphism.}

                 Consider the following commutative diagram with exact rows:
                 \begin{displaymath}
                   \xymatrix@1{
                      0\ar[r] & \dirlim \Hom(E,M_i)\ar[r]\ar[d]_{\Psi} & \dirlim \Hom(F_0,M_i)\ar[r]\ar[d]_{\Psi_0} & \dirlim \Hom(F_1,M_i)\ar[d]_{\Psi_1} \\
                      0\ar[r] & \Hom(E,\dirlim M_i)\ar[r] & \Hom(F_0,\dirlim M_i)\ar[r] & \Hom(F_1,\dirlim M_i)              }
                 \end{displaymath}
                 Here $\Psi_0,\Psi_1$ are isomorphisms by 2.
                 Hence, by Five Lemma, $\Psi$ is an isomorphism.
      \item \emph{If $E$ is finitely generated by an epimorphism
                 \begin{equation*}
                   F\To M\To 0
                 \end{equation*}
                 then $\Psi$ is injective.}

                 Consider the following commutative diagram with exact rows:
                 \begin{displaymath}
                   \xymatrix@1{
                      0\ar[r] & \dirlim \Hom(E,M_i)\ar[r]\ar[d]_{\Psi} & \dirlim \Hom(F,M_i)\ar[d]_{\Psi_0}     \\
                      0\ar[r] & \Hom(E,\dirlim M_i)\ar[r] & \Hom(F,\dirlim M_i)              }
                 \end{displaymath}
                 Here $\Psi_0$ is an isomorphisms by 2.
                 By $\triangle$ Lemma, $\Psi$ is injective.
    \end{enumerate}
  \end{proof}

\newpage\section{Graded Algebras}

  \begin{defn}
    Let $A$ be an algebra over a field $k$. By a \termin{filtration} of $A$ we mean a sequence of $k-$vector spaces $A_i (i=0,1,\cdots)$ such that
    \begin{equation*}
      A_0\subset A_1\subset A_2\subset \cdots \qquad\quad \text{and} \qquad\qquad \bigcup A_i=A
    \end{equation*}
    and $A_iA_j\subset A_{i+j}$ for all $i,j\geqslant 0$, In particular, $A$ is an $A_0-$algebra. We then call $A$ a \termin{filtered algebra}. Let $R$ be an algebra. We say that $R$ is \termin[graded]{graded!algebra} if $R$ is a direct sum $R=\bigoplus R_i$ of subspaces such that $R_iR_j\subset R_{i+j}$ for all $i,j\geqslant0$.
  \end{defn}

  \begin{ex}
    Let $A$ be filtered algebra. Define $R_i$ for $i\geqslant0$ by $R_i=A_i/A_{i-1}$. By definition, $A_{-1}=\{0\}$. Let $R=\bigoplus R_i$, and $\gr_i(A)=R_i$. Define a natural product on $R$ making $R$ into a graded algebra, denoted by $\gr(A)$, and called the \termin{associated graded algebra}.
  \end{ex}
  \begin{proof}
    The multiplication $\gr(A)\times\gr(A)\To\gr(A)$ is combined from the natural maps
    \mapdes{A_i/A_{i-1}\times A_j/A_{j-1}}{A_{i+j}/A_{i+j-1}}{(x+A_{i-1},y+A_{j-1})}{xy+A_{i+j-1}}
    The multiplication is well defined and endows $\gr(A)$ with the structure of a graded algebra, with gradation $\gr_i(A)$.
  \end{proof}

  \begin{ex}
    Let $A,B$ be filtered algebras, $A=\bigcup A_i$ and $B=\bigcup B_i$. Let $L\colon A\To B$ be an $(A_0,B_0)-$linear map preserving the filtration, that is $L(A_i)\subset B_i$ for all $i$, and $L(ca)=L(c)L(a)$ for $c\in A_0$ and $a\in A_i$ for all $i$.
    \begin{enumerate}[a)]
      \item Show that $L$ induces an $(A_0,B_0)-$linear map
                 \begin{equation*}
                   \gr_i(L)\colon \gr_i(A)\To\gr_i(B) \qquad \text{for all }i.
                 \end{equation*}
      \item Suppose that $\gr_i(L)$ is an isomorphism for all $i$. Show that $L$ is an $(A_0,B_0)-$isomorphism.
    \end{enumerate}
  \end{ex}
  \begin{proof}
    Consider the following $(A_0,B_0)-$linear map
    \begin{equation*}
      A_i\markar{L}B_i\markar{\pi}B_i/B_{i-1}
    \end{equation*}
    its kernel contain $A_{i-1}$ since $L(A_{i-1})\subset B_{i-1}$, therefore it induces an quotient map $A_i/A_{i-1}\To B_i/B_{i-1}$.

    Consider the following commutative diagrams and using $5-$lemma, by induction on $i$, it is clear that if $\gr_i(L)$ are isomorphisms for all $i$, then $L$ is an $(A_0,B_0)-$isomorphism.
                 \begin{displaymath}
                   \xymatrix@1{
                      0\ar[r] & A_{i-1}\ar[r]\ar[d]_{L} & A_i\ar[r]\ar[d]_{L} & \gr_i(A)\ar[r]\ar[d]_{\gr_i(L)} & 0     \\
                      0\ar[r] & B_{i-1}\ar[r] & B_i\ar[r] & \gr_i(B)\ar[r] & 0            }
                 \end{displaymath}
  \end{proof}

  \begin{ex}
    Suppose $k$ has characteristic $0$. Let $\nn$ be the set of all strictly upper triangular matrices of a given size $n\times n$ over $k$.
    \begin{enumerate}[a)]
      \item For a given matrix $X\in\nn$, let $D_1(X),\cdots,D_n(X)$ be its diagonals, so $D_1=D_1(X)$ is the main diagonal, and is $0$ by the definition of $\nn$. Let $\nn_i$ be the subset of $\nn$ consisting of those matrices whose diagonals $D_1,\cdots,D_{n-i}$ are $0$. Thus $\nn_0=\{0\}$, $\nn_1$ consists of all matrices whose components are $0$ except possibly for $x_{1n}$; $\nn_2$ consists of all matrices whose components are $0$ except possibly those in the last two diagonals; and so forth. Show that each $\nn_i$ is an algebra, and its elements are nilpotent (in fact the $(i+1)-$th power of its elements is $0$).
      \item Let $U$ be the set of elements $I+X$ with $X\in\nn$. Show that $U$ is a multiplicative group.
      \item Let $\exp$ be the exponential series defined as usual. Show that $\exp$ defines a polynomial function on $\nn$ (all but a finite number of terms are $0$ when evaluated on a nilpotent matrix), and establishes a bijection
          \begin{equation*}
            \exp\colon\nn\To U
          \end{equation*}
          Show that the inverse is given by the standard $\log$ series.
    \end{enumerate}
  \end{ex}
  \begin{proof}
    It is clear that $xy\in\nn_i$ for any $x,y\in\nn_i$, therefore $\nn_i$ is a subalgebra of $\nn$. Moreover, $xy\in\nn_{i-1}$, therefore the $(i+1)-$th power of $\nn_i$'s elements is $0$.

    The identity of $U$ is $I$, for any $I+X\in U$, its inverse is $\sum(-X)^i$.

    \begin{align*}
      \exp(X) & = \sum_{k=0}^{\infty} \frac{X^k}{k!} \\
      \log(X) & = - \sum_{k=1}^{\infty} \frac{(I-X)^k}{k}
    \end{align*}
  \end{proof}

\newpage\section{Some Obvious Module Structure}
\begin{prop}
  Let $R,S$ be rings, $ _RA_S, _RB_S$ are double modules. Then
  \begin{enumerate}
    \item $\Hom_R(A,-)$ is a covatiant functor from $ _R\Mod$ to $ _S\Mod$. The left $S-$module structure is given by $(sf)a=f(as)$.
    \item $\Hom_R(-,B)$ is a contravariant functor from $ _R\Mod$ to $\Mod_S$. The right $S-$module structure is given by $(fs)a=(f(a))s$.
  \end{enumerate}
\end{prop}

  Let $A$ be a $R-$module, then $A\cong\Hom_R(R,A)$. $A^{\ast}\defeq\Hom_R(A,R)$ is called the
  \termin[dual module]{dual!module} of $A$. If $A\cong A^{\ast\ast}$, then we say it is \termin{reflexive}.

\begin{prop}
  Let $R,S$ be rings, $ _SA_R, _RB_S$ are double modules. Then
  \begin{enumerate}
    \item $A\otimes_R-$ is a covatiant functor from $ _R\Mod$ to $ _S\Mod$.
    \item $-\otimes_RB$ is a covariant functor from $\Mod_R$ to $\Mod_S$.
  \end{enumerate}
\end{prop}
%Modules
%\chapter{Polynomials}
\section{Basic Properties for Polynomials in One Variable}

%Polynomials
%Historic comment
\chapter{Categories}


\part{Commutative Objects}
\chapter{Abelian Groups}
\chapter{Abelian Category}
\chapter{Commutative Algebra}
\chapter{Algebraic Number Theory}
\chapter{Algebraic Geometry}
\chapter{Arithmetic Geometry}

\part{Associative Objects}
\chapter{Groups in Category}

\chapter{Associative Algebra}

\chapter{Field Extension}


\part{Galois Objects}
\chapter{Galois Theory}

\part{Lie Objects}

\part{Homology Objects}

%\part{Weird Objects}

%\part{Galois Theory}
%\chapter{Algebraic Extensions}
\section{Finite and Algebraic Extensions}
\begin{exam}[Counterexample]
  Let $\alpha$ be an algebraic element over $K$, $L$ is an extension of $K$, then $[L(\alpha):L]$ integer divide $[K(\alpha):K]$.

  $\alpha=2^{\frac{1}{3}}(-\frac{1}{2}+i\frac{\sqrt 3}{2}), K=Q, L=Q(2^{\frac{1}{3}})$.
\end{exam}

%Algebraic Extensions
%\chapter{Galois Theory}
\cite{morandi1996field}
\section{Galois Extensions}
  \begin{defn}
    Let $K$ be a field and let $G$ be a group of automorphisms of $K$. The subset of $K$ consisting of all elements which is fixed under all $\sigma\in G$ is a field. It is called the \termin{fixed field} of $G$, and denoted by $K^G$.
  \end{defn}

  \begin{defn}
    An algebraic extension $K$ of a field $k$ is called \termin[Galois]{Galois Extension} if it is both normal and separable.
  \end{defn}

  \begin{defn}
    For an extension $K/k$, the group of automorphisms of $K$ over $k$ is denoted by $\Aut(K/k)$. When the extension is Galois, $\Aut(K/k)$ is called the \termin{Galois group} of $K$ over $k$, and is denoted by $\Gal(K/k)$ or $G(K/k)$.
  \end{defn}

  The main theorem in this section is
  \begin{thm}[Galois Connection]
    Let $K$ be a Galois extension of $k$, with Galois group $G$. Denote the set of intermediate field of $K/k$ by $\Int(K/k)$, and the set of subgroups of $G$ by $\Sub(G)$. Then there exists a bijection between them:
    \mapdes{\Sub(G)}{\Int(K/k)}{H}{E=K^H}

    The field $E$ is Galois if and only if $H$ is normal in $G$. If that is the case, then the map $\sigma\mapsto\local{\sigma}{E}$ induces an isomorphism of $G/H$ onto the Galois group of $E$ over $k$.
  \end{thm}

  Lang gives the proofs step by step.

  \begin{thm}
    Let $K$ be a Galois extension of $k$, with Galois group $G$, then $k=K^G$. Denote the set of intermediate field of $K/k$ by $\Int(K/k)$, and the set of subgroups of $G$ by $\Sub(G)$. If $F\in\Int(K/k)$, then $K$ is Galois over $F$. The map
    \mapdes{\Int(K/k)}{\Sub(G)}{F}{G(K/F)}
    is injective.
  \end{thm}

  \begin{defn}
    We say a subgroup $H$ of $G$ \emph{\red belongs} to an intermediate field $F$ if $H=G(K/F)$.
  \end{defn}

  \begin{cor}
    Let $K/k$ be Galois with group $G$. Let $F,F'$ be two intermediate fields, and let $H,H'$ be the subgroups of $G$ belonging to $F,F'$ respectively. Then
    \begin{enumerate}[a)]
      \item  $H\cap H'$ belongs to $FF'$;
      \item  The fixed field of the smallest subgroup of $G$ containing $H,H'$ is $F\cap F'$;
      \item  $F\subset F'$ if and only if $H'\subset H$.
    \end{enumerate}
  \end{cor}
  Such results can be represented by the corresponding between the following two \emph{Hasse diagrams}
                 \begin{displaymath}
                   \xymatrix@!0{
                      & <H\cup H'> &\ar[ddrr]&&& FF' & \\
                      H\ar@{-}[ur]\ar@{-}[dr] && H'\ar@{-}[ul]\ar@{-}[dl] && F\ar@{-}[ur]\ar@{-}[dr] && F'\ar@{-}[ul]\ar@{-}[dl] \\
                      & H\cap H' &\ar[uurr]&&& F\cap F' &            }
                 \end{displaymath}

  \begin{cor}
    Let $E$ be a finite separable extension of a field $k$. Let $K$ be the smallest normal extension of $k$ containing $E$. Then $K$ is finite Galois over $k$, and $\Int(E/k)$ is finite.
  \end{cor}

  \begin{lem}
    Let $E$ be an algebraic separable extension of $k$. Assume that there is an integer $n\geqslant1$ such that every element of $E$ is of degree $\leqslant n$ over $k$. Then $[E:k]\leqslant n$.
  \end{lem}

  Now, we have $K^{G(K/F)}=F$, but how about $G(K/K^H)$.

  \begin{thm}[Artin]
    Let $K$ be a field and let $G$ be a finite group of automorphisms of $K$, of order $n$. Let $k=K^G$ be the fixed field. Then $K$ is a finite Galois extension of $k$, and its Galois group is $G$. We have $[K:k]=n$.
  \end{thm}

  \begin{cor}
    Let $K$ be a finite Galois extension of $k$ and let $G$ be its Galois group. Then every subgroup of $G$ belongs to some $F\in\Int(K/k)$.
  \end{cor}
  \begin{warn}
    This statement is not true when $K$ is an infinite Galois extension of $k$.
  \end{warn}

  \begin{lem}
    Let $K$ be a Galois extension of $k$. Let
    \begin{equation*}
       \lambda\colon K\To \lambda K
    \end{equation*}
    be an isomorphism, then
    \begin{equation*}
      G(\lambda K/\lambda k)^{\lambda}=G(K/k)
    \end{equation*}
    i.e.
    \begin{equation*}
      G(\lambda K/\lambda k)=\lambda G(K/k) \lambda^{-1}
    \end{equation*}
  \end{lem}

  \begin{thm}
    Let $K$ be a Galois extension of $k$ with group $G$. Let $F\in\Int(K/k)$, and let $H=G(K/F)$. Then $F$ is normal over $k$ if and only if $H$ is normal in $G$. If that is the case, then the map $\sigma\mapsto\local{\sigma}{E}$ induces a homomorphism of $G$ onto the Galois group of $F$ over $k$, whose kernel is $H$.
  \end{thm}

  \begin{defn}
    A Galois extension is said to be \termin[abelian]{abelian extension} (resp. \termin[cyclic]{cyclic extension}) if its Galois group is \emph{abelian} (resp. \emph{cyclic}).
  \end{defn}

  \begin{cor}
    Let $K/k$ be abelian (resp. cyclic). If $F\in\Int(K/k)$, then $F/k$ is also abelian (resp. cyclic).
  \end{cor}

  \begin{thm}
    Let $K/k$ be Galois, $F/k$ be an arbitrary extension, then $KF/F, K/(K\cap F)$ are Galois. Moreover, we have an isomorphism
    \isodes{\Gal(KF/F)}{\Gal(K/(K\cap F))}{\sigma}{\local{\sigma}{K}}
                 \begin{displaymath}
                   \xymatrix@1{
                      && KF\ar@{-}[dr] &\\
                      K\ar@{-}[urr] &&& F\ar@{-}[dll] \\
                      & K\cap F\ar@{-}[ul] &&\\
                      & k\ar@{-}[u] &&            }
                 \end{displaymath}
  \end{thm}

  \begin{cor}
    Let $K/k$ be finite Galois, $F/k$ be an arbitrary extension. Then $[KF:F]$ divides $[K:k]$.
  \end{cor}
  \begin{warn}
    The assertion of above corollary is not usually valid if $K/k$ is not Galois.

    Indeed, let
    \begin{equation*}
      \alpha=2^{\frac{1}{3}}\quad\zeta=\frac{-1+\sqrt{-3}}{2}\quad\beta=\alpha\zeta
    \end{equation*}
    and $k=\QQ,K=\QQ(\beta),F=\QQ(\alpha)$. Then
    \begin{equation*}
      [KF:F]=2\quad [K:k]=3
    \end{equation*}
  \end{warn}

  \begin{thm}
    Let $\Gal(K_1/k)=G_1, \Gal(K_2/k)=G_2$, then $K_1K_2/k$ is Galois. Let $G$ be its Galois group, then
    \mapdes{G}{G_1\times G_2}{\sigma}{(\local{\sigma}{K_1},\local{\sigma}{K_2})}
    is injective. If $K_1\cap K_2=k$, then it is an isomorphism.
                 \begin{displaymath}
                   \xymatrix@1{
                      & K_1K_2\ar@{-}[dr] &\\
                      K_1\ar@{-}[ur] && K_2\ar@{-}[dl] \\
                      & K_1\cap K_2\ar@{-}[ul] &\\
                      & k\ar@{-}[u] &           }
                 \end{displaymath}
  \end{thm}

  \begin{cor}
    Let $K_1,\cdots,K_n$ be Galois extensions of $k$ with Galois groups $G_1,\cdots, G_n$. Assume $K_{i+1}\cap(K_1\cdots K_i)=k$, then
    \begin{equation*}
      \Gal(K_1\cdots K_n/k)\cong G_1\times\cdots\times G_n
    \end{equation*}
  \end{cor}

  \begin{cor}
    Let $K$ be finite Galois over $k$ with group $G$, and assume $G=G_1\times\cdots\times G_n$, let $K_i$ be the fixed field of
    \begin{equation*}
      G_1\times\cdots\times 1 \times\cdots\times G_n
    \end{equation*}
    then $K_i$ is Galois over $k$ and $K_{i+1}\cap(K_1\cdots K_i)=k, K=K_1\cdots K_n$.
  \end{cor}

  \begin{thm}
    Assume all fields contained in some common field.
    \begin{enumerate}[(i)]
      \item If $K,L$ are abelian over $k$, then so is $KL$;
      \item If $K/k$ is abelian and $F/k$ is arbitrary, then $KF/F$ is abelian;
      \item IF $K/k$ is abelian and $F\in\Int(K/k)$, then $K/F,F/k$ are abelian.
    \end{enumerate}
  \end{thm}

  \begin{warn}
    The converse of last statement may not be true.
  \end{warn}

  \begin{defn}
    The composite of all abelian extensions of $k$ in $k^{\ac}$ is called the \termin{abelian closure} of $k$ and denoted by $k^{\ab}$.
  \end{defn}

\newpage\section{Examples and Applications}

\newpage\section{Norm and Trace}
  Let $E/k$ be finite, $[E:k]_s=r$, $\alpha\in E$. We define the \termin{norm} and \termin{trace} of $\alpha$ to be
  \begin{align*}
    N_{E/k} &= N^E_k(\alpha) = \prod_{v=1}^{r}\sigma_v\alpha^{[E:k]_i} = \left(\prod_{v=1}^r\sigma_v\alpha\right)^{[E:k]_i} \\
    \Tr_{E/k} &= \Tr^E_k(\alpha) = [E:k]_i\sum_{v=1}^r\sigma_v\alpha
  \end{align*}

  If $E/k$ is separable, then
  \begin{align*}
    N^E_k(\alpha) & =\prod_{\sigma}\sigma\alpha \\
    \Tr^E_k(\alpha) & = \sum_{\sigma}\sigma\alpha
  \end{align*}

  \begin{thm}
    Let $E/k$ be finite, then
    \begin{enumerate}[(i)]
      \item $N^E_k(\alpha)$ is a multiplicative homomorphism of $E^{\ast}$ into $k^{\ast}$ and  $\Tr^E_k(\alpha)$ is an additive homomorphism of $E$ into $k$.
      \item If $E\supset F\supset k$ is a tower, then
                 \begin{equation*}
                   N^E_k=N^F_k\circ N^E_F\quad\text{and}\quad \Tr^E_k=\Tr^F_k\circ\Tr^E_F
                 \end{equation*}
      \item If $E=k(\alpha)$, and $f(X)=\Irr(\alpha,k,X)=X^n+a_{n-1}X^{n-1}+\cdots+a_0$, then
                 \begin{equation*}
                   N^{k(\alpha)}_k(\alpha)=(-1)^na_0\quad\text\quad\Tr^{k(\alpha)}_k(\alpha)=-a_{n-1}
                 \end{equation*}
    \end{enumerate}
  \end{thm}

  \begin{thm}
    Let $E/k$ be finite separable. Then $\Tr^E_k$ is a nonzero functional. The map
    \mapdes{E\times E}{k}{(x,y)}{\Tr(xy)}
    is bilinear, and identifies $E$ with its dual space $\codual{E}$.
  \end{thm}

  \begin{cor}
    Let $\omega_1,\cdots,\omega_n$ be a basis of $E/k$. Then there exists a basis $\omega'_1,\cdots,\omega'_n$ of $E/k$ such that $\Tr(\omega_i\omega'_j)=\delta_{ij}$.
  \end{cor}

  \begin{cor}
    Let $E/k$ be finite separable, and let $\sigma_1,\cdots,\sigma_n$ be the distinct embeddings. Let $w_1,\cdots,w_n$ be elements of $E$. Then the vectors
    \begin{gather*}
      \xi_1=(\sigma_1w_1,\cdots,\sigma_1w_n) \\
      \cdots \\
      \xi_n=(\sigma_nw_1,\cdots,\sigma_nw_n)
    \end{gather*}
    are linearly independent over $E$ if $w_1,\cdots,w_n$ form a basis of $E/k$.
  \end{cor}

  \begin{rem}
    In characteristic $0$, one sees much more trivially that the trace is not identically $0$. Indeed, if $c\in k$ and $c\neq0$, then $\Tr(c)=nc$ where $n=[E;k]$, and $n\neq0$. This argument also holds in characteristic $p$ where $n$ is prime to $p$.
  \end{rem}

  \begin{prop}
    Let $E=k(\alpha)$ be separable. Let
    \begin{equation*}
      f(X)=\Irr(\alpha,k,X)
    \end{equation*}
    and let $f'(X)$ be its derivative. Let
    \begin{equation*}
      \frac{f(X)}{(X-\alpha)}=\beta_0+\beta_1X+\cdots+\beta_{n-1}X^{n-1}
    \end{equation*}
    with $\beta_i\in E$. Then the dual basis of $1,\alpha,\cdots,\alpha^{n-1}$ is
    \begin{equation*}
      \frac{\beta_0}{f'(\alpha)},\cdots,\frac{\beta_{n-1}}{f'(\alpha)}
    \end{equation*}
  \end{prop}

  Define
  \longmapdes{m_{\alpha}}{E}{E}{x}{\alpha x}

  \begin{prop}
    Let $E/k$ be finite and let $\alpha\in E$. Then
    \begin{equation*}
      \det(m_{\alpha})=N_{E/k}(\alpha)\quad\text{and}\quad\Tr(m_{\alpha})=\Tr_{E/k}(\alpha)
    \end{equation*}
  \end{prop}

\newpage\section{Cyclic Extensions}

  \begin{thm}[Hilbert's Theorem 90]
    Let $K/k$ be cyclic of degree $n$ with Galois group $G$. Let $\sigma$ be a generator of $G$. Let $\beta\in K$. The norm $N(\beta)=1$ if and only if there exists an element $\alpha\neq0$ in $K$ such that $\beta=\alpha/\sigma\alpha$.
  \end{thm}

  \begin{thm}
    Let $k$ be a field, $n$ an integer $>0$ prime to the characteristic of $k$, and assume that there is a primitive $n-$th root of unity in $k$.
    \begin{enumerate}[(i)]
      \item Let $K$ be a cyclic extension of degree $n$. Then there exists $\alpha\in K$ such that $K=k(\alpha)$, and $\alpha$ satisfies an equation $X^n-a=0$ for some $a\in k$.
      \item Conversely, let $a\in k$. Let $\alpha$ be a root of $X^n-a$. Then $k(\alpha)$ is cyclic over $k$, of degree $d$, $d\mid n$, and $\alpha^d$ is an element of $k$.
    \end{enumerate}
  \end{thm}

  \begin{thm}[Hilbert's Theorem 90, Additive Form]
    Let $K/k$ be cyclic of degree $n$ with Galois group $G$. Let $\sigma$ be a generator of $G$. Let $\beta\in K$. The trace $\Tr(\beta)=0$ if and only if there exists an element $\alpha\neq0$ in $K$ such that $\beta=\alpha-\sigma\alpha$.
  \end{thm}

  \begin{thm}[Artin-Schreier]
    Let $k$ be a field of characteristic $p$.
    \begin{enumerate}[(i)]
      \item Let $K$ ba a cyclic extension of $k$ of degree $p$. Then there exists $\alpha\in K$ such that $K=k(\alpha)$ and $\alpha$ satisfies an equation $X^p-X-a=0$ with some $a\in k$.
      \item Conversely, given $a\in k$, the polynomial $f(X)=X^p-X-a$ either has one root in $k$, in which case all its roots are in $k$, or it is irreducible. In this latter case, if $\alpha$ is a root then $k(\alpha)$ is cyclic of degree $p$ over $k$.
    \end{enumerate}
  \end{thm}

\newpage\section{Solvable and Radical Extensions}

  \begin{defn}
    A Galois extension is called \termin[solvable]{solvable extension} if its Galois group is solvable. A finite extension $E/k$ is said to be \termin[solvable]{solvable extension} if the smallest Galois extension $K$ of $k$ containing $E$ is solvable.
  \end{defn}
  \begin{rem}
     This is equivalent to saying that there exist a solvable Galois extension $L$ of $k$ containing $E$.
  \end{rem}

  \begin{prop}
    Solvable extension form a distinguished class.
  \end{prop}

  \begin{defn}
    A simple extension $k(\alpha)/k$ is called \termin{solvable by radicals} if it is one of the following type:
    \begin{enumerate}
      \item It is obtained by adjoining a root of unity.
      \item It is obtained by adjoining a root of a polynomial $X^n-a$ with $a\in k$ and $n$ prime to the characteristic.
      \item It is obtained by adjoining a root of an equation $X^p-X-a$ with $a\in k$, if $p$ is the character $>0$.
    \end{enumerate}

    A finite extension $F/k$ is said to be \termin{solvable by radicals} if it is separable and if there exists a finite extension $E/k$ containing $F$, and admitting a tower
    \begin{equation*}
      k=E_0\subset E_1\subset E_2\subset\cdots\subset E_m=E
    \end{equation*}
    such that each step is a simple extension solvable by radicals.
  \end{defn}

  \begin{prop}
    The class of extensions which are solvable by radical is distinguished.
  \end{prop}
  \begin{proof}
    Using lifting.
  \end{proof}

  \begin{thm}
    Let $E/k$ be separable, then $E$ is solvable by radical if and only if $E/k$ is solvable.
  \end{thm}

  \begin{defn}
    A polynomial $f\in k[X]$ is said to be \termin{solvable by radicals} if its splitting field is solvable by radicals.
  \end{defn}

  \begin{defn}
    Let $k$ be a field. The general polynomial of degree $n$ over $k$ is defined to be
    \begin{equation*}
      f(X)=X^n-t_1X^{n-1}+t_2X^{n-2}+\cdots+(-1)^nt_n\in k(t_1,t_2,\cdots,t_n)[X]
    \end{equation*}
    where $t_1,t_2,\cdots,t_n$ are algebraic independent over $k$.
  \end{defn}

  \begin{cor}
    The general polynomial of degree $n$ is solvable by radicals only if $n\leqslant4$.
  \end{cor}

\newpage\section{Abelian Kummer Theory}

  \begin{defn}
    A Galois extension $K/k$ with group $G$ is said to be of \termin{exponent} $m$ if $\sigma^m=1$ for all $\sigma\in G$.
  \end{defn}

    Let $m$ be prime to the characteristic of $k$. We assume $k$ contains a primitive $m-$th root of unity.

    Let $a\in k$, and $\alpha$ be an $m-$th root of $a$. The field $k(\alpha)$ is independent of the choice of $\alpha$ and hence denoted by $k(a^{1/m})$.

    We denote by $k^{\ast m}$ the image of $k^{\ast}$ under the homomorphism $x\mapsto x^m$.

    Let $k^{\ast m}\subset B< k^{\ast}$. We denoted by $k(B^{1/m})$ or $K_B$ the composite of all fields $k(a^{1/m})$ with $a\in B$.

    $K_B/k$ is Galois, let $G$ be its Galois group. Let $\sigma\in G$. Then $\sigma\alpha=\omega_{\sigma}\alpha$ for some $m-$th root of unity $\omega_{\sigma}\in\root_m\subset k^{\ast}$.

    There is a homomorphism:
    \mapdes{G}{\root_m}{\sigma}{\omega_{\sigma}}

    We may write $\omega_{\sigma}=\sigma\alpha/\alpha$, but it is independent of the choice of $\alpha$.

    We denote $\omega_{\sigma}$ by $\<\sigma,\alpha\>$. The map
    \mapdes{G\times B}{\root_m}{(\sigma,a)}{\<\sigma,a\>}
    is bilinear.

  \begin{thm}
    Let $k$ be a field, $m$ an integer $>0$ prime to the characteristic of $k$, and assume that a primitive $m-$root of unity lies in $k$.
    \begin{enumerate}[(i)]
      \item Let $k^{\ast m}\subset B< k^{\ast}$, $K_B=k(B^{1/m})$, then $K_B/k$ is Galois and abelian of exponent $m$.
      \item Consider the bilinear map
                 \mapdes{G\times B}{\root_m}{(\sigma,a)}{\<\sigma,a\>}
                 The kernel of left is $1$, and the kernel of right is $k^{\ast m}$.
      \item $K_B/k$ is finite if and only if $(B:k^{\ast m})$ is finite. If that is the case, then
                 \begin{equation*}
                   B/k^{\ast m}\cong \dual{G}
                 \end{equation*}
                 and
                 \begin{equation*}
                   [K_B:k]=(B:k^{\ast m})
                 \end{equation*}
    \end{enumerate}
  \end{thm}

  \begin{thm}
    There is a $1-1$ corresponding:
    \isodes{\Sub(k^{\ast};k^{\ast m})}{\{\text{abelian extension of $k$ of exponent $m$}\}}{B}{K_B}
  \end{thm}

  \begin{thm}
      Let $k$ be a field with characteristic $p$, we define the operator $\wp$ by
      \begin{equation*}
        \wp(x)=x^p-x
      \end{equation*}
      Then $\wp$ is an additive automorphism of $k$.

      For $\wp(k)\subset B\subset k$, Let $K_B=k(\wp^{-1}B)$.
    \begin{enumerate}[(i)]
      \item There is a $1-1$ corresponding
                 \isodes{\Sub(k:\wp(k))}{\{\text{abelian extension of $k$ of exponent $p$}\}}{B}{K_B}
      \item For $\sigma\in G, a\in B$ and $\alpha\in K_B$ such that $\wp\alpha=a$. Let $\<\sigma,a\>=\sigma\alpha-\alpha$.
                 There is a bilinear map
                 \mapdes{G\times B}{\ZZ/p}{(\sigma,a)}{\<\sigma,a\>}
                 The kernel of left is $1$, and the kernel of right is $\wp(k)$.
      \item $K_B/k$ is finite if and only if $(B:\wp(k))$ is finite. If that is the case, then
                 \begin{equation*}
                   [K_B:k]=(B:\wp(k))
                 \end{equation*}
    \end{enumerate}
  \end{thm}

\newpage\section{The Equation $X^n-a=0$}
  In this section, the roots of unity are not in the ground field.

  \begin{thm}
    Let $k$ be a field and $n$ an integer $\geqslant2$. Let $a\in k,a\neq0$. Assume that for all $p\mid n$, we have $a\in k^{p}$, and if $4\mid n$ then $a\notin-4k^4$. Then $X^n-a$ is irreducible in $k[X]$.
  \end{thm}

  \begin{cor}
    $a\in k^{\ast}$, and $a\notin k^p$ for some $p$. If $p$ is equal to the characteristic of $k$ or $p$ is odd, then for all $r\geqslant1$, the polynomial $X^{p^r}-a$ is irreducible over $k$.
  \end{cor}

  \begin{cor}
    If $k^{\ac}/k$ is finite of degree $>1$, then $k^{\ac}=k(i)$ and its characteristic is $0$.
  \end{cor}
  
  \begin{exam}
    Let $k$ be a field with characteristic not dividing $n$. Let $a\in k^{\ast}$ and $K$ be the splitting field of $X^n-a$. Let $\alpha$ be a root of $X^n-a$ and let $\zeta$ be a primitive $n-$th root of unity. Then
    \begin{equation*}
      K=k(\alpha,\zeta)=k(\alpha,\root_n)
    \end{equation*}
    
    Let $\sigma\in G_{K/k}$. Then there exists some integer $b(\sigma)$ uniquely determined mod $n$, such that
    \begin{equation*}
      \sigma(\alpha)=\alpha\zeta^{b(\sigma)}
    \end{equation*}
    
    There exists an integer $d(\sigma)$ uniquely determined mod $n$, such that
    \begin{equation*}
      \sigma(\zeta)=\zeta^{d(\sigma)}
    \end{equation*}
    
    Let $G(n)$ be the subgroup of $\GL_2(\ZZ/n)$ consisting of all matrices 
    \begin{equation*}
      M=\begin{pmatrix}
          1 & 0 \\
          b & d \\
        \end{pmatrix}
    \end{equation*}
    where $b\in\ZZ/n,d\in(\ZZ/n)^{\ast}$.
    
    Observe that $|G(n)|=n\varphi(n)$, and we obtain an injective map:
    \mapdes{G_{K/k}}{G(n)}{\sigma}{M(\sigma)}
  \end{exam}
  
  \begin{thm}
    Let $k$ be a field. Let $n$ be an odd integer prime to the characteristic, and assume that $[k(\root_n):k]=\varphi(n)$. Let $a\in k$, and $a\notin k^p$ for all prime $p\mid n$. Let $K$ be the splitting field of $X^n-a$ over $k$.
    
    Then the above homomorphism $\sigma\mapsto M(\sigma)$ is an isomorphism. The commutator group is $\Gal(K/k(\root_n))$, so $k(\root_n)$ is the maximal abelian subextension of $K$. 
  \end{thm}
  \begin{warn}
    When $n$ is even, there are some complications.
  \end{warn}%Galois Theory
%infinite Galois Theory

\part{Dedekind Domain}
\section{23}


\begin{appendices}
%\renewcommand{\thepart}{}%\Alph{part}.}
\CTEXsetup[name={,},number={}]{part}
\part{Appendix}
%\renewcommand{\thechapter}{}
%\chapter{Category Theory}
\section{Categories}
  \begin{defn}
  A \termin{category} $\Cc$ consists of
  \begin{itemize}
    \item a class $\ob\Cc$ of \termin[objects]{object}.
    \item a class $\hom\Cc$ of \termin[morphisms]{morphism (category theory)}, or \termin[arrows]{arrow (category theory)}, or \termin[maps]{map (category theory)}, between the objects.

             Each morphism $f$ has a unique source object $A$ and target object $B$ where $A$ and $B$ are in $\ob\Cc$.

             We write $f\colon A\To B$, and we say ``$f$ is a morphism from $A$ to $B$''.

             We write $\Hom(A, B)$ (or $\Hom_{\Cc}(A, B)$ when there may be confusion about to which category $\Hom(A, B)$ refers) to denote the hom-class of all morphisms from $A$ to $B$. (Some authors write $\Mor(A, B)$ or simply $\Cc(A,B)$ instead.)
    \item for every three objects $A,B$ and $C$, a binary operation
             \begin{equation*}
               \Hom(A, B) \times \Hom(B, C) \To \Hom(A, C)
             \end{equation*}
             called \termin{composition of morphisms}.

             The composition of $f\colon A \To B$ and $g\colon B \To C$ is written as $g\circ f$ or simply $gf$. (Some authors use ``diagrammatic order'', writing $f;g$ or $fg$.)
  \end{itemize}
  such that the following axioms hold:
  \begin{description}
    \item[associativity] if $f\colon A \To B, g\colon B \To C$ and $h\colon C \To D$ then
                                 \begin{equation*}
                                   h\circ(g\circ f) = (h\circ g)\circ f
                                 \end{equation*}
    \item[identity] for every object $A$, there exists a morphism $1_A\colon A \To A$ (some times write $\id_A$) called the \termin[identity morphism]{identity (morphism)} for $A$, such that for every morphism $f\colon A \To B$, we have $1_B \circ f = f = f \circ 1_A$.
  \end{description}
  From these axioms, one can prove that there is exactly one identity morphism for every object. Some authors use a slight variation of the definition in which each object is identified with the corresponding identity morphism.
  \end{defn}

  \begin{defn}
    A category $\Cc$ is called \termin[small]{small category} if both $\ob\Cc$ and $\hom\Cc$ are actually sets,
    and \termin[large]{large category} otherwise.
    A \termin[locally small category]{locally!small} is a category such that for all objects $A$ and $B$, the hom-class $\Hom(A, B)$ is a set,
    called a hom-set. Many important categories in mathematics (such as the category of sets), although not small, are at least locally small.
  \end{defn}

  \begin{defn}
    Any category $\Cc$ can itself be considered as a new category in a different way: the objects are the same as those in the original category but the arrows are those of the original category reversed. This is called the \termin[dual]{dual!category} or \termin[opposite category]{opposite!category} and is denoted $\Cc^{\op}$.
  \end{defn}

  \begin{defn}
    The \termin{product category} $\Cc\times\Dd$ consists of:
    \begin{itemize}
      \item \emph{objects:}
               pairs of objects $(A, B)$, where $A$ is an object of $\Cc$ and $B$ of $\Dd$;
      \item \emph{arrows from $(A_, B_1)$ to $(A_2, B_2)$:}
               pairs of arrows $(f, g)$, where $f\colon A_1 \To A_2$ is an arrow of $\Cc$ and $g\colon B_1 \To B_2$ is an arrow of $\Dd$;
      \item \emph{compositions:}
               component-wise composition from the contributing categories:
                                    \begin{equation*}
                                      (f_2, g_2) \circ (f_1, g_1) = (f_2 \circ f_1, g_2 \circ g_1);
                                    \end{equation*}
      \item \emph{identities:}
               pairs of identities from the contributing categories:
                               \begin{equation*}
                                 1_{(A, B)} = (1_A, 1_B).
                               \end{equation*}
    \end{itemize}
  \end{defn}

\subsection{Subcategories}
  \begin{defn}
    Let $\Cc$ be a category. A \termin{subcategory} $\Ss$ of $\Cc$ is given by
    \begin{itemize}
      \item a subcollection of objects of $\Cc$, denoted $\ob\Ss$,
      \item a subcollection of morphisms of $\Cc$, denoted $\hom\Ss$.
    \end{itemize}
    such that
    \begin{enumerate}
      \item for every $X$ in $\ob\Ss$, the identity morphism $\id_X$ is in $\hom\Ss$,
      \item for every morphism $f\colon X\To Y$ in $\hom\Ss$, both the source $X$ and the target $Y$ are in $\ob\Ss$,
      \item for every pair of morphisms $f$ and $g$ in $\hom\Ss$ the composite $f\circ g$ is in $\hom\Ss$ whenever it is defined.
    \end{enumerate}
  \end{defn}
  \begin{rem}
    These conditions ensure that $\Ss$ is a category in its own right: the collection of objects is $\ob\Ss$, the collection of morphisms is $\hom\Ss$, and the identities and composition are as in $\Cc$. There is an obvious faithful functor $I\colon \Ss\To\Cc$, called the \termin{inclusion functor} which takes objects and morphisms to themselves.
  \end{rem}

  \begin{defn}
    Let $\Ss$ be a subcategory of a category $\Cc$. We say that $\Ss$ is a \termin[full subcategory]{full!subcategory} of $\Cc$ if the inclusion functor is fully faithful.
  \end{defn}

  \begin{defn}
    A subcategory $\Ss$ of $\Cc$ is said to be \termin{isomorphism-closed} or \termin{replete} if every isomorphism $k\colon X\To Y$ in $\Cc$ such that $Y$ is in $\Ss$ also belongs to $\Ss$. A isomorphism-closed full subcategory is said to be \termin[strictly full]{strictly full!subcategory}.
  \end{defn}

  \begin{defn}
    A subcategory of $\Cc$ is said to be \termin{wide} or \termin{lluf} if it contains all the objects of $\Cc$.
  \end{defn}

\subsection{Reflective Subcategory}

\subsection{Comma Categories}
  \begin{defn}
    Suppose that $\Aa$, $\Bb$, and $\Cc$ are categories, and $S$ and $T$ (for source and target) are functors
          \begin{displaymath}
            \xymatrix{
               \Aa\ar[r]^{S} & \Cc & \Bb\ar[l]_{T}                }
          \end{displaymath}
    We can form the \termin{comma category} $(S\down T)$ as follows:
    \begin{itemize}
      \item The objects are all triples $(\alpha,\beta,f)$ with $\alpha$ an object in $\Aa$, $\beta$ an object in $\Bb$, and $f\colon S(\alpha)\To T(\beta)$ a morphism in $\Cc$.
      \item The morphisms from $(\alpha,\beta,f)$ to $(\alpha',\beta',f')$ are all pairs $(g,h)$
                 where $g\colon \alpha\To\alpha'$ and $h\colon \beta\To\beta'$ are morphisms in $\Aa$ and $\Bb$ respectively, such that the following diagram commutes:
                 \begin{displaymath}
                   \xymatrix{
                       S(\alpha)\ar[r]^{S(g)}\ar[d]_{f} & S(\alpha')\ar[d]^{f'}  \\
                       T(\beta)\ar[r]_{T(h)} & T(\beta')           }
                 \end{displaymath}
      \item Morphisms are composed by taking $(g,h)\circ(g',h')$ to be $(g\circ g',h\circ h')$, whenever the latter expression is defined.
      \item The identity morphism on an object $(\alpha,\beta,f)$ is $(\id_{\alpha},\id_{\beta})$.
    \end{itemize}
  \end{defn}

  \begin{exam}
    \textbf{Slice category}.

    When $\Aa=\Cc$, $S$ is the identity functor, and $\Bb=\one$ (the category with one object $\ast$ and one morphism).
    Then $T(\ast)=A$ for some object $A$ in $\Cc$. In this case, the comma category is written $(\Cc\down A)$, and is often called the \termin{slice category} over $A$ or the category of \emph{\red objects over $A$}.
    The objects $(\alpha,\ast,f)$ can be simplified to pairs $(\alpha,f)$, where $f\colon \alpha\To A$.
    Sometimes, $f$ is denoted $\pi_{\alpha}$.
    A morphism from $(B,\pi_B)$ to $(B',\pi_{B'})$ in the slice category is then an arrow $g\colon B\To B'$ making the following diagram commute:
          \begin{displaymath}
            \xymatrix@R=0.5cm{
               B\ar[rr]^{g}\ar[dr]_{\pi_B} && B'\ar[dl]^{\pi_{B'}} \\
               &A&                }
          \end{displaymath}
  \end{exam}

  \begin{exam}
    \textbf{Coslice category}.

    The dual concept to a slice category is a coslice category. Here, $S$ has domain $\one$ and $T$ is an identity functor. In this case, the comma category is often written $(A\down\Cc)$, where $A$ is the object of $\Cc$ selected by $S$. It is called the \termin{coslice category} with respect to $A$, or the category of \emph{\red objects under $A$}.
    The objects are pairs $(B,i_B)$ with $i_B\colon A\To B$. Given $(B,i_B)$ and $(B',i_{B'})$, a morphism in the coslice category is a map $h\colon B\To B'$ making the following diagram commute:
          \begin{displaymath}
            \xymatrix@R=0.5cm{
               &A\ar[dl]_{i_B}\ar[dr]^{i_{B'}}&  \\
               B\ar[rr]_{h} && B'                }
          \end{displaymath}
  \end{exam}

  \begin{exam}
    \textbf{Arrow category}.

    $S$ and $T$ are identity functors on $\Cc$ (so $\Aa=\Bb=\Cc$).
    In this case, the comma category is the \termin{arrow category} $\Cc^{\to}$. Its objects are the morphisms of $\Cc$, and its morphisms are commuting squares in $\Cc$.
  \end{exam}

  \begin{exam}
    In the case of the slice or coslice category, the identity functor may be replaced with some other functor; this yields a family of categories particularly useful in the study of adjoint functors. Let $s,t$ be given object in $\Cc$.
    An object of $(s\down T)$ is called a \emph{morphism from $s$ to $T$} or a \termin{$T-$structured arrow} with domain $s$ in.
    An object of $(S\down t)$ is called a \emph{morphism from $S$ to $t$} or a \termin{$S-$costructured arrow} with codomain $s$ in.
  \end{exam}

  \begin{prop}
    For each comma category there are forgetful functors from it.
    \begin{itemize}
      \item \termin{domain functor}, $(S\down T)\To\Aa$, which maps:
      \begin{itemize}
        \item objects: $(\alpha,\beta,f)\mapsto\alpha$;
        \item morphisms: $(g,h)\mapsto g$;
      \end{itemize}
      \item \termin{codomain functor}, $(S\down T)\To\Bb$, which maps:
      \begin{itemize}
        \item objects: $(\alpha,\beta,f)\mapsto\beta$;
        \item morphisms: $(g,h)\mapsto h$;
      \end{itemize}
    \end{itemize}
  \end{prop}

  \begin{exam}
    The category of \textbf{pointed sets} is a comma category $(\bullet\down\Set)$, with $\bullet$ being (a functor selecting) any singleton set, and $\Set$ (the identity functor of) the category of sets.
    Each object of this category is a set, together with a function selecting some element of the set: the ``\textbf{basepoint}''. Morphisms are functions on sets which map basepoints to basepoints. In a similar fashion one can form the category of \textbf{pointed spaces} $(\bullet\down\Top)$.
  \end{exam}

  \begin{exam}
    The category of \textbf{graphs} is $(\Set\down D)$, with $D\colon\Set\To\Set$ the functor taking a set $s$ to $s\times s$.
    The objects $(a,b,f)$ then consist of two sets and a function; $a$ is an \textbf{indexing set}, $b$ is a set of \textbf{nodes}, and $f\colon a\To (b\times b)$ chooses pairs of elements of $b$ for each input from $a$. That is, $f$ picks out certain edges from the set $b\times b$ of possible edges.
    A morphism in this category is made up of two functions, one on the indexing set and one on the node set. They must ``agree'' according to the general definition above, meaning that $(g,h)\colon(a,b,f)\To(a',b',f')$ must satisfy $f'\circ g=D(h)\circ f$. In other words, the edge corresponding to a certain element of the indexing set, when translated, must be the same as the edge for the translated index.
  \end{exam}

  \begin{exam}
    Colimits in comma categories may be ``inherited''. If $\Aa$ and $\Bb$ are cocomplete, $S\colon\Aa\To\Cc$ is a cocontinuous functor, and $T\colon\Bb\To\Cc$ another functor (not necessarily cocontinuous), then the comma category $(S\down T)$ produced will also be cocomplete.

    If $\Aa$ and $\Bb$ are complete, and both $S\colon\Aa\To\Cc$ and $T\colon\Bb\To\Cc$ are continuous functors, then the comma category $(S\down T)$ is also complete, and the projection functors $(S\down T)\To\Aa$ and $(S\down T)\To\Bb$ are limit preserving.
  \end{exam}

  \begin{exam}
    \textbf{Adjunctions}.

    Lawvere showed that the functors $F\colon\Cc\To\Dd$ and $G\colon\Dd\To\Cc$ are adjoint if and only if the comma categories $(F\down \id_{\Dd})$ and $(\id_{\Cc}\down G)$, with $\id_{\Dd}$ and $\id_{\Cc}$ the identity functors on $\Dd$ and $\Cc$ respectively, are isomorphic, and equivalent elements in the comma category can be projected onto the same element of $\Cc\times\Dd$. This allows adjunctions to be described without involving sets, and was in fact the original motivation for introducing comma categories.
  \end{exam}

  \begin{exam}
    \textbf{Natural transformations}.

    A natural transformation $\eta\colon S\To T$, with $S,T\colon\Aa\To\Cc$, corresponds to a functor $\Aa\To(S\down T)$ which maps each object $\alpha$ to $(\alpha,\alpha,\eta_{\alpha})$ and maps each morphism $g$ to $(g,g)$.
    This is a bijective correspondence between natural transformations $S\To T$ and functors $\Aa\To(S\down T)$ which are sections of both forgetful functors from $(S\down T)$.
  \end{exam}

\newpage\section{Morphisms}
\subsection{Monomorphisms, Epimorphisms and Zero Morphisms}
  \begin{defn}
    A morphism $f$ is called a \termin{monomorphism}, or \termin{monoic}, if for any morphisms
    $\xymatrix@1{\cdot\ar@<0.5ex>[r]^{\alpha}\ar@<-0.5ex>[r]_{\beta} &\cdot\ar[r]^{f} &\cdot}$, $f\alpha=f\beta$ implies $\alpha=\beta$.
    Dually, $f$ is called an \termin{epimorphism}, or \termin{epi}, if for any morphisms
    $\xymatrix@1{\cdot\ar[r]^{f} &\cdot\ar@<0.5ex>[r]^{\alpha}\ar@<-0.5ex>[r]_{\beta} &\cdot}$, $\alpha f=\beta f$ implies $\alpha=\beta$.
    If $f$ is both a monomorphism and an epimorphism, then we say it is a \termin{bimorphism}.
  \end{defn}
  \begin{defn}
    A morphism $f$ is called a \termin[split monomorphism]{split!monomorphism}, if it has a left inverse, names \termin{retraction}.
    Dually, a morphism $f$  is called a \termin[split epimorphism]{split!epimorphism}, if it has a right inverse, names \termin{section}.
    If $f$ is both a split monomorphism and a split epimorphism, then we say it is an \termin{isomorphism}.
  \end{defn}
  \begin{rem}
    It is clear that any split monomorphism must be monoic and any split epimorphism is epi, hence any isomorphism is a bimorphism. However, the converse is not true in general case.
  \end{rem}

  A bijective morphism may fail to be an isomorphism:
  \begin{exam}
    In $\mathbf{Top}$, the map from the half-open interval $[0,1)$ to the unit circle $S^1$ (thought of as a subspace of the complex plane) which sends $x$ to $e^{2��ix}$ is continuous and bijective but not a \emph{\red  homeomorphism}\index{homeomorphism} since the inverse map is not continuous at $1$.
  \end{exam}
  \begin{rem}
    This counterexample also shows why the concept of \emph{subobject} does not correspond subspace in $\mathbf{Top}$.
  \end{rem}

  An epimorphism may fail to be surjective:

  \begin{exam}
    In the category of rings, $\mathbf{Ring}$, the inclusion map $\ZZ\hookrightarrow\QQ$ is a non-surjective epimorphism; to see this, note that any ring homomorphism on $\QQ$ is determined entirely by its action on $\ZZ$. A similar argument shows that the natural ring homomorphism from any commutative ring $R$ to any one of its localizations is an epimorphism.
  \end{exam}
  \begin{rem}
    This is also a counterexample shows that a quotient object may not be a quotient.
  \end{rem}

  \begin{defn}
    A morphism $f$ is called a \termin[constant morphism]{constant!morphism} (or sometimes \termin[left zero morphism]{left!zero morphism}) if for any morphisms
    $\xymatrix@1{\cdot\ar@<0.5ex>[r]^{\alpha}\ar@<-0.5ex>[r]_{\beta} &\cdot\ar[r]^{f} &\cdot}$, $f\alpha=f\beta$.
    Dually, $f$ is called a \termin{coconstant morphism} (or sometimes \termin[right zero morphism]{right!zero morphism}) if for any morphisms
    $\xymatrix@1{\cdot\ar[r]^{f} &\cdot\ar@<0.5ex>[r]^{\alpha}\ar@<-0.5ex>[r]_{\beta} &\cdot}$, $\alpha f=\beta f$.
     A \termin[zero morphism]{zero!morphism} is one that is both a constant morphism and a coconstant morphism.
  \end{defn}
  \begin{defn}
    A \termin[category with zero morphisms]{category!with zero morphisms} is one where, for every two objects $A$ and $B$ in $\Cc$, there is a fixed morphism $0_{AB} \colon A \To�� B$ such that for all objects $X, Y, Z$ in C and all morphisms $f \colon X \To Y, g \colon Y \To Z$, the following diagram commutes:
      \begin{displaymath}
        \xymatrix{
             X\ar[r]^{0_{XY}}\ar[d]_{f}\ar[dr]|{0_{XZ}} & Y\ar[d]^{g} \\
             Y\ar[r]_{0_{YZ}} & Z                    }
      \end{displaymath}
  \end{defn}
  \begin{rem}
    The morphisms $0_{XY}$ necessarily are zero morphisms and form a \emph{compatible system} of zero morphisms. If $\Cc$ is a category with zero morphisms, then the collection of $0_{XY}$ is unique.
  \end{rem}
  \begin{rem}
    If a category has zero morphisms, then one can define the notions of \emph{kernel} and \emph{cokernel} for any morphism in that category.
  \end{rem}

\subsection{Factorization}
\cite{freyd1972categories}
  \begin{defn}
    If a morphism $f\colon X\To Y$ can be written as a composition $f=g\circ h$ with $g\colon Z\To Y$ and $h\colon X\To Z$,
    then $f$ is said to \termin{factor through} any (and all) of $Z$, $g$, and $h$. We also say $f$ is \emph{\red factorized} as $h$ followed by $g$.
  \end{defn}

  \begin{defn}
    A \termin[factorization system]{factorization!system} $(E, M)$ for a category $\Cc$ consists of two classes of morphisms $E$ and $M$ of $\Cc$ such that:
    \begin{enumerate}
      \item $E$ and $M$ both contain all isomorphisms of $\Cc$ and are closed under composition.
      \item Every morphism $f$ of $\Cc$ can be factored as $f=m\circ e$ for some morphisms $e\in E$ and $m\in M$.
      \item The factorization is \emph{functorial}: if $u$ and $v$ are two morphisms such that $vme=m'e'u$ for some morphisms $e,e'\in E$ and $m,m'\in M$, then there exists a unique morphism $w$ making the following diagram commute:
          \begin{displaymath}
            \xymatrix{
               \ar[r]^{e}\ar[d]_{u} & \ar[r]^{m}\ar@{-->}[d]_{w} & \ar[d]^{v} \\
               \ar[r]_{e'} & \ar[r]_{m'} &                    }
          \end{displaymath}
    \end{enumerate}
  \end{defn}

  \begin{defn}
    Two morphisms $e$ and $m$ are said to be \termin[orthogonal]{orthogonal!morphism}, if for every pair of morphisms $u$ and $v$ such that $ve=mu$ there is a unique morphism $w$ such that the diagram
          \begin{displaymath}
            \xymatrix{
               \ar[r]^{e}\ar[d]_{u} & \ar@{-->}[dl]^-{w}\ar[d]^{v} \\
               \ar[r]_{m} &                     }
          \end{displaymath}
    commutes. If so, denote by $e\down m$,

    This notion can be extended to define the orthogonals of sets of morphisms by
     \begin{equation*}
       H^{\uparrow}\defeq\left\{ e \mid \forall h\in H, e\down h \right\}
     \end{equation*}
     and
     \begin{equation*}
       H^{\down}\defeq\left\{ m \mid \forall h\in H, h\down m \right\}
     \end{equation*}
    Since in a factorization system $E\cap M$ contains all the isomorphisms, the condition 3. of the definition is equivalent to 3':
    \begin{equation*}
      E\subset M^{\uparrow} \qquad M\subset E^{\down}
    \end{equation*}
  \end{defn}

  \begin{prop}
    The pair $(E,M)$ of classes of morphisms of $\Cc$ is a factorization system if and only if it satisfies the following conditions:
    \begin{enumerate}
      \item Every morphism $f$ of $\Cc$ can be factored as $f=m\circ e$ for some morphisms $e\in E$ and $m\in M$.
      \item $E\subset M^{\uparrow}$ and $M\subset E^{\down}$.
    \end{enumerate}
  \end{prop}

\subsection{Endomorphisms}
\cite{balmer2001idempotent}
  \begin{defn}
    An \termin{endomorphism} is a morphism whose domain and co-domain coincide. An \termin{automorphism} is a morphism that is both an endomorphism and an isomorphism.
  \end{defn}
  \begin{defn}
    An \termin{idempotent} $e$ is an endomorphism such that $e\circ e=e$.
    An endomorphism $e$ is said to \termin[split]{split!endomorphism} if it is idempotent, and if there are two morphisms $f,g$ such that $e = g f$ and $\id = f g$.
  \end{defn}
  \begin{defn}
    A category is called \termin[idempotent complete]{idempotent!complete}, if every idempotent splits.
  \end{defn}
  \begin{defn}
    Let $\Cc$ be a category, the \termin{Karoubi envelope} of $\Cc$, sometimes written $\Split(\Cc)$, is the category whose objects are pairs of the form $(A, e)$ where $A$ is an object of $\Cc$ and $e\colon A\To A$ is an idempotent of $\Cc$, and whose morphisms are triples of the form
    \begin{equation*}
      (e,f,e')\colon (A,e) \To (A',e')
    \end{equation*}
    where $f\colon A\To A'$ is a morphism of $\Cc$ satisfying $e'\circ f=f=f\circ e$ (or equivalently $f=e'\circ f\circ e$).

    Composition in $\Split(\Cc)$ is as in $\Cc$, but the identity morphism on $(A,e)$ in $\Split(\Cc)$ is $(e,e,e)$, rather than the identity on $A$.
  \end{defn}

  \begin{prop}
    The Karoubi envelope $\Split(\Cc)$ of $\Cc$ is the \termin[idempotent completion]{idempotent!completion} of $\Cc$, which means that
    $\Cc$ can be fully faithfully embedded into $\Split(\Cc)$, and the embedding $\imath\colon\Cc\To\Split(\Cc)$ satisfying the following universal property:
    \begin{quote}
      For any functor $F\colon\Cc\To\Dd$ with $\Dd$ is idempotent complete, there is a unique functor $F'\colon \Split(\Cc) \To \Dd$ such that the following diagram commutes:
      \begin{displaymath}
        \xymatrix@R=0.5cm{
                &         \Split(\Cc) \ar[dd]^{F'}     \\
              \Cc \ar[ur]^-{\imath} \ar[dr]_{F}                 \\
                &         \Dd                 }
      \end{displaymath}
    \end{quote}
  \end{prop}

\subsection{Initial and Terminal Morphisms}
  \begin{defn}
    Suppose that $U\colon\Dd \To \Cc$ is a functor from a category $\Dd$ to a category $\Cc$, and let $X$ be an object of $\Cc$.
    An \termin[initial morphism]{initial!morphism} from $X$ to $U$
    is an initial object in the category $(X\down U)$ of morphisms from $X$ to $U$.
    A \termin[terminal morphism]{terminal!morphism} from $U$ to $X$
    is a terminal object in the comma category $(U\down X)$ of morphisms from $U$ to $X$.
  \end{defn}
  \begin{rem}
  The term \termin[universal morphism]{universal!morphism} refers either to an initial morphism or a terminal morphism.
  \end{rem}

  \begin{prop}
    Given a functor $U$ and an object $X$ as above, there may or may not exist an initial morphism from $X$ to $U$.
    However, if an initial morphism does exist then it is unique up to a unique isomorphism.
  \end{prop}

  \begin{prop}
    Let $U$ be a functor from $\Dd$ to $\Cc$, and let $X$ be an object of $\Cc$.
    Then the following statements are equivalent:
    \begin{enumerate}[a.]
      \item $(A, \phi)$ is an initial morphism from $X$ to $U$;
      \item $(A, \phi)$ is an initial object of the comma category $(X \down U)$;
      \item $(A, \phi)$ is a representation of $\Hom_{\Cc}(X, U(-))$.
    \end{enumerate}

    The dual statements are also equivalent:
    \begin{enumerate}[a'.]
      \item $(A, \phi)$ is a terminal morphism from $U$ to $X$;
      \item $(A, \phi)$ is a terminal object of the comma category $(U \down X)$;
      \item $(A, \phi)$ is a representation of $\Hom_{\Cc}(U(-), X)$.
    \end{enumerate}
  \end{prop}

  Suppose $(A_1, \phi_1)$ is an initial morphism from $X_1$ to $U$ and $(A_2, \phi_2)$ is an initial morphism from $X_2$ to $U$.
  By the initial property, given any morphism $h\colon X_1 \To X_2$ there exists a unique morphism $g\colon A_1 \To A_2$ such that the following diagram commutes:
      \begin{displaymath}
        \xymatrix{
                X_1\ar[r]^{\phi_1}\ar[d]_{h} & U(A_1)\ar@{-->}[d]^{U(g)} & A_1\ar@{-->}[d]^{g} \\
                X_2\ar[r]_{\phi_2} & U(A_2) & A_2                 }
      \end{displaymath}

  If every object $X_i$ of $\Cc$ admits an initial morphism to $U$, then the assignment $X_i\mapsto A_i$ and $h\mapsto g$ defines a functor $V$ from $\Cc$ to $\Dd$. The maps $\phi_i$ then define a natural transformation from $\id_{\Cc}$ to $UV$. The functors $(V, U)$ are then a pair of adjoint functors, with $V$ left-adjoint to $U$ and $U$ right-adjoint to $V$.

  Similar statements apply to the dual situation of terminal morphisms from $U$. If such morphisms exist for every $X$ in $\Cc$ one obtains a functor $V\colon \Cc \To \Dd$ which is right-adjoint to $U$ (so $U$ is left-adjoint to $V$).

  Indeed, all pairs of adjoint functors arise from \emph{universal constructions} in this manner. Let $F$ and $G$ be a pair of adjoint functors with unit $\eta$ and co-unit $\varepsilon$.
  Then we have a universal morphism for each object in $\Cc$ and $\Dd$:
  \begin{itemize}
    \item For each object $X$ in $\Cc$, $(F(X), \eta_X)$ is an initial morphism from $X$ to $G$.
               That is, for all $f\colon X \To G(Y)$ there exists a unique $g\colon F(X) \To Y$ for which the following diagrams commute.
    \item For each object $Y$ in $\Dd$, $(G(Y), \varepsilon_Y)$ is a terminal morphism from $F$ to $Y$.
               That is, for all $g\colon F(X) \To Y$ there exists a unique $f\colon X \To G(Y)$ for which the following diagrams commute.
  \end{itemize}
      \begin{displaymath}
        \xymatrix{
                X\ar[r]^-{\eta_X}\ar[dr]_{f} & GF(X)\ar@{-->}[d]^{G(g)} & F(X)\ar@{-->}[d]_{F(f)}\ar[dr]^{g} & \\
                & G(Y) & FG(Y)\ar[r]_-{\varepsilon_Y} & Y               }
      \end{displaymath}

  \emph{Universal constructions} are more general than adjoint functor pairs: a universal construction is like an optimization problem; it gives rise to an adjoint pair if and only if this problem has a solution for every object of $\Cc$ (equivalently, every object of $\Dd$).

\newpage\section{Functors}
  \begin{defn}
    Let $\Cc, \Dd$ be two categories, a \termin{functor} $F\colon\Cc\To\Dd$ is a corresponding from $\ob\Cc$ into $\ob\Dd$ and $\hom\Cc$ into $\hom\Dd$ such that
    \begin{enumerate}[(i)]
      \item $F(1_X)=1_{F(X)}$ for every object $X$;
      \item $F(g\circ f)=F(g)\circ F(f)$ for all morphisms $f\colon A\To B, g\colon B\To C$.
    \end{enumerate}
  \end{defn}

  \begin{defn}
    Let $\Cc, \Dd$ be two categories, a \termin{contravariant functor} is a functor from $\Cc^{\op}$ to $\Dd$.
  \end{defn}
  Ordinary functors are also called \termin[covariant functor]{covariant functors} in order to distinguish them from \emph{contravariant} ones.

  \begin{defn}
    Every functor $F\colon\Cc\To\Dd$ induces the \termin[opposite functor]{opposite!functor} $F^{\op}\colon \Cc^{\op} \To \Dd^{\op}$ maps objects and morphisms identically to $F$.
  \end{defn}

  \begin{defn}
    A \termin{bifunctor} is a functor whose domain is a product category.
  \end{defn}

  \begin{exam}
    Consider $S_2\To S_3\To S_2$ in $\Grp$, it is not difficult to show that there is no functor $\Grp\To\Ab$ sending each group to its center.
  \end{exam}

  \begin{exam}
    The functor $\Cc \To \Dd$ which maps every object of $\Cc$ to a fixed object $X$ in $\Dd$ and every morphism in $\Cc$ to the identity morphism on $X$. Such a functor is called a \termin[constant]{constant!functor} or \termin{selection functor}.
  \end{exam}

  \begin{exam}
    The \termin[diagonal functor]{diagonal!functor} is defined as the functor from $\Dd$ to the functor category $[\Cc,\Dd]$ which sends each object in $\Dd$ to the constant functor at that object.
  \end{exam}

\begin{defn}
  A Functor $F\colon \mathcal{C}\to\mathcal{D}$ is said to be
  \begin{enumerate}[a)]
    \setlength{\itemindent}{2ex}
    \item \termin[faithful]{faithful!functor} (resp. \termin[full]{full!functor}, resp. \termin{ fully faithful}) if for any $X,Y\in\ob\mathcal{C}$, the map $\Hom_{\mathcal{C}}(X,Y)\to\Hom_{\mathcal{D}}(F(X),F(Y))$ is injective (resp. surjective, resp. bijective).
    \item \termin{essentially surjective} if for each $Y\in\ob\mathcal{D}$, there exists $X\in\ob\mathcal{C}$ and an isomorphism $F(X)\cong Y$.
    \item \termin{conservative} if any morphism $f\colon X\to Y$ in $\mathcal{C}$ is an isomorphism as soon as $F(f)$ is an isomorphism.
  \end{enumerate}
\end{defn}
\begin{warn}
  A faithful functor need not be injective on objects or morphisms. That is, two objects $X$ and $X'$ may map to the same object in $\Dd$ (which is why the range of a fully faithful functor is not necessarily equivalent to $\Cc$),
  and two morphisms $f \colon X\To Y$ and $f' \colon X'\To Y'$ (with different domains/codomains) may map to the same morphism in $\Dd$.

  Likewise, a full functor need not be surjective on objects or morphisms. There may be objects in $\Dd$ not of the form $F(X)$ for some $X$ in $\Cc$. Morphisms between such objects clearly cannot come from morphisms in $\Cc$.
\end{warn}
\begin{prop}
  \begin{enumerate}[1)]
    \setlength{\itemindent}{2ex}
    \item Let $F\colon \mathcal{C}\to \mathcal{D}$ be a faithful functor and let $f$ be a morphism in $\mathcal{C}$. Then if $F(f)$ is a monomorphism (resp. an epimorphism), then $f$ is a monomorphism (resp. an epimorphism).
    \item Moreover, assume that $F$ is fully faithful. Then if $F(f)$ is an isomorphism, then $f$ is an isomorphism. In other words, fully faithful functors are conservative.
  \end{enumerate}
\end{prop}
\begin{proof}
For any $\xymatrix@1{T\ar@<0.5ex>[r]^{\alpha}\ar@<-0.5ex>[r]_{\beta} &X\ar[r]^{f} &T}$ in $\mathcal{C}$ such that $f\alpha=f\beta$, then $F(f)F(\alpha)=F(f)F(\beta)$. Since $F(f)$ is injective, $F(\alpha)=F(\beta)$. Since $F$ is faithful, $\alpha=\beta$. Hence $f$ is also a monomorphism. The epimorphism case is similar.

Let $\varphi$ be the inverse morphism of $F(f)\colon F(X)\to F(Y)$, since $F$ is fully faithful, there exist a $g\in\Hom_{\mathcal{C}}(Y,X)$ such that $F(g)=\varphi$, moreover, it is the inverse morphism of $f$.
\end{proof}
\begin{cor}
  A fully faithful functor is necessarily injective on objects up to isomorphism.
\end{cor}

\subsection{Natural Transformations and Functor categories}
\begin{defn}
  Let $F,G$ be two functors from $\Cc$ to $\Dd$. A morphism (or \termin[natural transformation]{natural!transformation}) of functors $\pi\colon F\to G$ is the data for all $X\in\ob\Cc$ of a morphism $\pi(X)\colon F(X)\to G(X)$ such that for all $f\colon X\to Y$ in $\Cc$ , the following diagram commutes:
 \begin{displaymath}
      \xymatrix{
         F(X)\ar[r]^{\pi_X}\ar[d]_{F(f)}&G(X)\ar[d]^{G(f)}\\
         F(Y)\ar[r]^{\pi_Y}&G(Y)
      }
\end{displaymath}

Hence, by considering the family of functors from $\Cc$ to $\Dd$ and the morphisms of such functors, we get a new category, denoted by $\Fct(\Cc,\Dd)$ or simply $[\Cc,\Dd]$, or $\Dd^{\Cc}$.
\end{defn}
  \begin{rem}
    If $\Cc$ and $\Dd$ are both \emph{preadditive} categories, then we can consider the category of all additive functors from $\Cc$ to $\Dd$, denoted by $\Add(\Cc,\Dd)$.
  \end{rem}
  \begin{exam}
    Any ring $R$ can be considered as a one-object preadditive category; the category of left modules over $R$ is the same as the additive functor category $\Add(R,\Ab)$, and the category of right $R-$modules is $\Add(R^{\op},\Ab)$. Because of this example, for any preadditive category $\Cc$, the category $\Add(\Cc,\Ab)$ is sometimes called the ``category of \termin[left modules]{module!over a preadditive category} over $\Cc$'' and $\Add(\Cc^{\op},\Ab)$ is the category of \emph{\red right modules} over $\Cc$.
  \end{exam}

  Most constructions that can be carried out in $\Dd$ can also be carried out in $[\Cc,\Dd]$ by performing them ``componentwise'', separately for each object in $\Cc$.

  For instance, if any two objects $X$ and $Y$ in $\Dd$ have a product $X\times Y$, then any two functors $F$ and $G$ in $[\Cc,\Dd]$ have a product $F\times G$, defined by $(F\times G)(c) = F(c)\times G(c)$ for every object $c$ in $\Cc$.

  Similarly, if $\eta_c\colon F(c)\To G(c)$ is a natural transformation and each $\eta_c$ has a kernel $K_c$ in the category $\Dd$, then the kernel of $\eta$ in the functor category $[\Cc,\Dd]$ is the functor $K$ with $K(c) = K_c$ for every object $c$ in $\Cc$.

  As a consequence we have the general rule of thumb that the functor category $[\Cc,\Dd]$ shares most of the ``nice'' properties of $\Dd$:
  \begin{itemize}
    \item if $\Dd$ is complete (or cocomplete), then so is $[\Cc,\Dd]$;
    \item if $\Dd$ is an abelian category, then so is $[\Cc,\Dd]$;
    \item if $\Cc$ is any small category, then the category $[\Cc,\Set]$ of presheaves is a \emph{topos}.
  \end{itemize}
  \begin{prop}
    the categories of directed graphs, $G-$sets and presheaves on a topological space $X$ are all complete and cocomplete topoi, and that the categories of representations of $G$, modules over the ring $R$, and presheaves of abelian groups on a topological space $X$ are all abelian, complete and cocomplete.
  \end{prop}

\begin{prop}
  Every natural transformation $\pi\colon F\To G$ defines a function which sends each arrow $f\colon A\To B$ of $\Cc$ to an arrow $\pi_f\colon F(A)\To G(B)$ of $\Dd$ in such a way that
  \begin{equation*}
    G(g)\circ\pi_f =\pi_{gf} = \pi_g\circ F(f)
  \end{equation*}
  for each composable pair $g,f$.
  Conversely, every such function $\pi$ comes from a unique natural transformation with $\pi_X=\pi_{1_X}$.
\end{prop}
\begin{rem}
  This gives an arrows only description of a natural transformation.
\end{rem}

  \begin{defn}
    An \termin{infranatural transformation} $\eta$ from $F$ to $G$ is simply a family of morphisms $\eta_X\colon F(X) \To G(X)$.
    Thus a natural transformation is an infranatural transformation for which $\eta_Y \circ F(f) = G(f) \circ \eta_X$ for every morphism $f\colon X\To Y$. The \termin{naturalizer} of $\eta$, $\nat(\eta)$, is the largest subcategory of $\Cc$ containing all the objects of $\Cc$ on which $\eta$ restricts to a natural transformation.
  \end{defn}

  \begin{defn}
    If, for every object $X$ in $\Cc$, the morphism $\eta_X$ is an isomorphism in $\Dd$, then $\eta$ is said to be a
    \termin[natural isomorphism]{natural!isomorphism}.
    Two functors $F$ and $G$ are said to be \termin{naturally isomorphic} if there exists a natural isomorphism from $F$ to $G$.
  \end{defn}

  \begin{exam}
    Statements such as
    \begin{quote}
      ``Every group is \emph{naturally isomorphic} to its opposite group''
    \end{quote}
    abound in modern mathematics.

    The content of the above statement is:
    \begin{quote}
      ``The identity functor $\Id \colon \Grp \To \Grp$ is \emph{naturally isomorphic} to the opposite functor $\op \colon \Grp \To \Grp$.''
    \end{quote}
  \end{exam}

  \begin{exam}
    If $K$ is a field, then for every vector space $V$ over $K$ we have a ``natural'' injective linear map $V \To V^{\ast\ast}$ from the vector space into its double dual. These maps are ``natural'' in the following sense: the double dual operation is a functor, and the maps are the components of a natural transformation from the identity functor to the double dual functor.
  \end{exam}

  \begin{defn}
    A particular map between particular objects may be called an \termin{unnatural isomorphism} (or ``this isomorphism is not natural'') if the map cannot be extended to a natural transformation on the entire category.
  \end{defn}
  \begin{rem}
    Some authors distinguish notationally, using $\cong$ for a natural isomorphism and $\approx$ for an unnatural isomorphism, reserving $=$ for equality (usually equality of maps).
  \end{rem}
  \begin{exam}
    In group theory or module theory, a given decomposition of an object into a direct sum is ``not natural'', or rather ``not unique'', as automorphisms exist that do not preserve the direct sum decomposition
  \end{exam}

    \begin{exam}
    \textbf{fundamental group of torus}

    As an example of the distinction between the \emph{\red functorial statement} and \emph{\red individual objects}, consider homotopy groups of a product space, specifically the fundamental group of the torus.
    \begin{quote}
      The homotopy groups of a product space are \emph{naturally} the product of the homotopy groups of the components,
      \begin{equation*}
        \pi_n((X,x_0)\times(Y,y_0))\cong\pi_n(X,x_0)\times\pi_n(Y,y_0)
      \end{equation*}
      with the isomorphism given by projection onto the two factors, fundamentally because maps into a product space are exactly products of maps into the components.

      \emph{This is a \red functorial statement.}
    \end{quote}

    However, given the torus, which is abstractly a product of two circles, and thus has fundamental group isomorphic to $\ZZ^2$,
    but the splitting $\pi_1(T,t_0)\approx\ZZ\times\ZZ$ is not natural. Note the use of $\approx$, $\cong$, and $=$:
    \begin{equation*}
      \pi_1(T,t_0)\approx\pi_1(S^1,x_0)\times\pi_1(S^1,y_0)\cong\ZZ\times\ZZ=\ZZ^2
    \end{equation*}

    This abstract isomorphism with a product \emph{is \red not natural}, as some isomorphisms of $T$ do not preserve the product:
    the self-homeomorphism of $T$ (thought of as the quotient space $\RR^2/\ZZ^2$) given by
    $\begin{pmatrix}
      \begin{smallmatrix}
      1 & 1 \\
      0 & 1 \\
      \end{smallmatrix}
    \end{pmatrix}$
     (geometrically a \emph{Dehn twist} about one of the generating curves) acts as this matrix on $\ZZ^2$ (it's in the general linear group $\GL(\ZZ, 2)$ of invertible integer matrices), which does not preserve the decomposition as a product because it is not diagonal.

     However, if one is given the torus as a product --- equivalently, given a decomposition of the space --- then the splitting of the group follows from the general statement earlier.

     In categorical terms, the relevant category (preserving the structure of a product space) is ``maps of product spaces, namely a pair of maps between the respective components''.

     \emph{\red Naturality is a categorical notion}, and requires being very precise about exactly what data is given --- the torus as a space that happens to be a product (in the category of spaces and continuous maps) is different from the torus presented as a product (in the category of products of two spaces and continuous maps between the respective components).
    \end{exam}

    \begin{exam}
    \textbf{dual of a finite-dimensional vector space}

    Every finite-dimensional vector space is isomorphic to its dual space, but this isomorphism relies on an arbitrary choice of isomorphism (for example, via choosing a basis and then taking the isomorphism sending this basis to the corresponding dual basis).
    There is in general no natural isomorphism between a finite-dimensional vector space and its dual space.

    However, related categories (with additional structure and restrictions on the maps) do have a natural isomorphism.

    In this category (finite-dimensional vector spaces with a \emph{nondegenerate bilinear form}, maps linear transforms that respect the bilinear form), the dual of a map between vector spaces can be identified as a transpose.

    Often for reasons of geometric interest this is specialized to a subcategory, by requiring that the nondegenerate bilinear forms have additional properties, such as being \emph{symmetric} (\emph{orthogonal matrices}), \emph{symmetric} and \emph{positive definite} (\emph{inner product space}), \emph{symmetric sesquilinear} (\emph{Hermitian spaces}), \emph{skew-symmetric} and \emph{totally isotropic} (\emph{symplectic vector space}), etc. --- in all these categories a vector space is naturally identified with its dual, by the nondegenerate bilinear form.
    \end{exam}

\begin{defn}
  A functor $F\colon\mathcal{C}\to\mathcal{D}$ is an \termin[equivalence]{equivalence (functor)} of categories if there exists $G\colon\mathcal{D}\to\mathcal{C}$ such that:
  \begin{align*}
    G\circ F&\cong\id_{\mathcal{C}}\\
    F\circ G&\cong\id_{\mathcal{D}}
  \end{align*}
  If such a functor exist, say $\mathcal{C}$ and $\mathcal{D}$ are \termin[equivalent]{equivalent!category}, denoted by $\mathcal{C}\simeq\mathcal{D}$.
\end{defn}
\begin{rem}
  If two categories are equivalent, all results and concepts in one of them have their counterparts in the other one. This is why this notion of equivalence of categories plays an important role in Mathematics.
\end{rem}

The following properties are easy to check
\begin{thm}
  The functor $F\colon\mathcal{C}\to\mathcal{D}$ is an equivalence of categories if and only if $F$ is fully faithful and essentially surjective.
\end{thm}

\begin{prop}
For any categories $\mathcal{C},\mathcal{D},\mathcal{E}$,
  \begin{enumerate}[1)]
    \setlength{\itemindent}{2ex}
    \item $[\Cc,\Dd]^{\op}\simeq[\Cc^{\op},\Dd^{\op}]$
    \item $[\Cc\times\Dd,\Ee]\simeq[\Dd,[\Cc,\Ee]]\simeq[\Cc,[\Dd,\Ee]]$
  \end{enumerate}
\end{prop}

\subsection{Category of All Categories}

    Given functors and natural transformations:
      \begin{displaymath}
        \xymatrix{
           \Cc \ar@<1.5ex>[r]^{F}="1" \ar@<-1.5ex>[r]_{G}="2" & \Bb \ar@<1.5ex>[r]^{F'}="3" \ar@<-1.5ex>[r]_{G'}="4" & \Aa
           \ar"1";"2"^{\pi} \ar"3";"4"^{\tau}
        }
      \end{displaymath}
    we first have the composite functors $F'F$ and $G'G$ and a commutative square
     \begin{displaymath}
        \xymatrix{
           F'F(X)\ar[r]^{F'(\pi_X)}\ar[d]_{\tau_{F(X)}} & F'G(X)\ar[d]^{\tau_{G(X)}}\\
           G'F(X)\ar[r]^{G'(\pi_X)} & G'G(X)
        }
    \end{displaymath}

    Now define $(\tau\circ\pi)_X$ to be the diagonal of this square. Then $\tau\circ\pi$ is also a natural transformation. (Which will not be confused with the original composition of natural transformations: first, they are compositions in different categories; second, we will always use different nations. Indeed, we will use $\tau\pi$ to denote the original one.)

    It is easy to check that all functors\footnote{Of course, we need a restriction. For instance, the functors between small categories.} form a category $\Cat$ under this composition (the horizontal composition). Moreover, for any functor $F\colon\Cc\To\Dd$, the identity at $F$ in $\Cat$ coincide with in $[\Cc,\Dd]$.

    Apart from this, For any functors and natural transformations:
      \begin{displaymath}
        \xymatrix{
           \Cc \ar@<4ex>[r]^{}="1" \ar[r]^{}="3"_{}="2" \ar@<-4ex>[r]_{}="4"
           & \Bb \ar@<4ex>[r]^{}="5" \ar[r]^{}="7"_{}="6" \ar@<-4ex>[r]_{}="8" & \Aa
           \ar"1";"2"^{\pi} \ar"3";"4"^{\tau} \ar"5";"6"^{\pi'} \ar"7";"8"^{\tau'}
        }
      \end{displaymath}
    there is a identity (\termin{interchange law}):
    \begin{equation*}
      (\tau'\pi')\circ(\tau\pi) = (\tau'\circ\tau)(\pi'\circ\pi)
    \end{equation*}

    Conclusively, we have
    \begin{thm}
      $\Cat$ has two compositions satisfying the interchange law and share the same identities.
    \end{thm}

    \begin{cor}
      The horizontal composition $\circ$ is a functor
      \begin{equation*}
        \circ\colon \Aa^{\Bb}\times\Bb^{\Cc}\To\Aa^{\Cc}
      \end{equation*}
    \end{cor}

    More general,  we define \termin{double category} to be a category with two compositions satisfying interchange law. Further, a \termin{$2-$category} is a double category which the two compositions share the same identities.

    \begin{exam}
      Let $G$  be a topological group with identity element $e$, while $\sigma,\sigma',\tau,\tau'$ are continuous loops in $G$ at $e$.
      Define $\tau\circ\sigma$ to be the path $\sigma$ followed by the path $\tau$.
      Define $\tau\sigma$ to be the pointwise product of $\tau$ and $\sigma$.
      Then they form a $2-$category.
    \end{exam}

    \begin{thm}[Hilton-Eckmann]
      Let $S$ be a set with two binary operations
      \begin{equation*}
        \cdot\colon S\times S\To S \qquad   \circ\colon S\times S\To S
      \end{equation*}
      which both have the same unit element $e$ and satisfying the interchange law. Then $\cdot$ and $\circ$ are equal, and each is commutative.
    \end{thm}
    \begin{cor}
      The fundamental group of a topological group is abelian.
    \end{cor}

    \begin{prop}
      The functor category $\Dd^{\Cc}=[\Cc,\Dd]$ is itself a bijective $\Cat^{\op}\times\Cat\To\Cat$. The arrow function sends a pair of functors $F\colon\Dd\To\Dd'$ and $G\colon\Cc'\To\Cc$ to the functor
      \begin{equation*}
        F^G\colon \Dd^{\Cc} \To \Dd'^{\Cc'}
      \end{equation*}
      which defined on objects $S\in\Dd^{\Cc}$ as $F^G(S)=F\circ S\circ G$ and on arrows $\tau\colon S\To T$ in $\Dd^{\Cc}$ as $F^G(\tau)=F\circ\tau\circ G$.
      \begin{displaymath}
        \xymatrix{
           \Cc' \ar[r]^{G} & \Cc \ar@<1.5ex>[r]^{S}="3" \ar@<-1.5ex>[r]_{T}="4"
           & \Dd \ar[r]^{F} & \Dd'
           \ar"3";"4"^{\tau}
        }
      \end{displaymath}
    \end{prop}

    \begin{prop}\label{product and power}
      For categories $\Aa,\Bb,\Cc$ establish natural isomorphisms:
      \begin{equation*}
        (\Aa\times\Bb)^{\Cc}\simeq\Aa^{\Cc}\times\Bb^{\Cc}  \qquad  \Cc^{\Aa\times\Bb}\simeq(\Cc^{\Bb})^{\Aa}\simeq(\Cc^{\Aa})^{\Bb}
      \end{equation*}
    \end{prop}

\subsection{Yoneda Lemma}
\begin{defn}
  For a category $\mathcal{C}$, one can define two categories:
  \begin{align*}
    \dual{\Cc} & \defeq [\Cc^{\op},\mathbf{Set}] \\
    \codual{\Cc} & \defeq [\Cc,\mathbf{Set}]^{\op}
  \end{align*}
  and two functors:
  \begin{center}
  \parbox{0.40\linewidth}{\longmapdes{\mathcal{M}^{*}}{\mathcal{C}}{\dual{\mathcal{C}}}{X}{\mathcal{M}^{X}}}  \parbox{0.40\linewidth}{\longmapdes{\mathcal{M}_{*}}{\mathcal{C}}{\codual{\mathcal{C}}}{X}{\mathcal{M}_{X}}}
  \end{center}
  where $\mathbf{Set}$ denoted the category of sets, $\mathcal{M}^X$ denoted the functor from $\mathcal{C}$ to $\mathbf{Set}$ which maps $Y\in\ob\mathcal{C}$ to the set $\Hom(Y,X)$, and $\mathcal{M}_X$ is similar.
\end{defn}
\begin{thm}[Yoneda Lemma]\index{Yoneda!Lemma}
  For $F\in\ob\dual{\mathcal{C}}$ and $X\in\ob\mathcal{C}$, there is an isomorphism
  \begin{equation*}
    \Hom_{\dual{\mathcal{C}}}(\mathcal{M}^{X},F)\cong F(X)
  \end{equation*}
  which, moreover, is natural in both $F$ and $X$.
\end{thm}
\begin{proof}
  For any $\alpha\in\Hom_{\dual{\mathcal{C}}}(\mathcal{M}^{X},F)$, let $\Phi(\alpha)$ be $\alpha_X(\id_X)$. Which defined a map to $F(X)$. Conversely, for any $a\in F(X)$, let $\Psi(a)$ be a natural transformation such that for any $Y\in\ob\mathcal{C}$ and morphism $f\colon Y\to X$, $\Psi(a)_Y(f)=F(f)(a)$.

  To show that $\Psi(a)$ is natural, consider the following diagram for each $f\colon Y\to X$
 \begin{displaymath}
      \xymatrix{
         \mathcal{M}^X(X)\ar[r]^-{\Psi(a)_X}\ar[d]_{\mathcal{M}^X(f)}&F(X)\ar[d]^{F(f)}\\
         \mathcal{M}^X(Y)\ar[r]^-{\Psi(a)_Y}&F(Y)
      }
\end{displaymath}
  Let$g\in\mathcal{M}^X(X)$, then
  \begin{align*}
    F(f)\Psi(a)_X(g) &=F(f)F(g)(a) \\
    &=F(gf)(a) \\
    &=\Psi(a)_Y(gf) \\
    &=\Psi(a)_Y(\mathcal{M}^X(f)(g))
  \end{align*}
  Hence the diagram commutes and $\Psi(a)$ is natural.

  For each $\alpha\in\Hom_{\dual{\mathcal{C}}}(\mathcal{M}^{X},F)$, consider the commutative diagram below
 \begin{displaymath}
      \xymatrix{
         \mathcal{M}^X(X)\ar[r]^-{\alpha_X}\ar[d]_{\mathcal{M}^X(f)}&F(X)\ar[d]^{F(f)}\\
         \mathcal{M}^X(Y)\ar[r]^-{\alpha_Y}&F(Y)
      }
\end{displaymath}
  It tells us that
  \begin{align*}
    \Psi(\Phi(\alpha))_Y(f) &=F(f)(\Phi(\alpha)) \\
    &=F(f)(\alpha_X(\id_X)) \\
    &=\alpha_Y(\mathcal{M}^X(f)(\id_X)) \\
    &=\alpha_Y(f)
  \end{align*}
  Thus $\Psi(\Phi(\alpha))=\alpha$. Hence $\Psi\circ\Phi=\id_{\Hom_{\dual{\mathcal{C}}}(\mathcal{M}^{X},F)}$.
  \begin{align*}
    \Phi(\Psi(a))&=\Psi(a)_X(\id_X) \\
    &=F(\id_X)(a) \\
    &=\id_{F(X)}(a)=a
  \end{align*}
  Hence $\Phi\circ\Psi=\id_{F(X)}$. Then, $\Phi$ is isomorphism.

  To show this isomorphism is natural in $F$, consider the diagram below for any natural transformation $\pi\colon F\to G$
   \begin{displaymath}
      \xymatrix{
         \Hom_{\dual{\mathcal{C}}}(\mathcal{M}^{X},F)\ar[r]^-{\Phi}\ar[d]_{\Hom_{\dual{\mathcal{C}}}(\mathcal{M}^{X},\pi)}&F(X)\ar[d]^{\pi_X}\\
         \Hom_{\dual{\mathcal{C}}}(\mathcal{M}^{X},G)\ar[r]^-{\Phi}&G(X)
      }
  \end{displaymath}
  For any $\alpha\in\Hom_{\dual{\mathcal{C}}}(\mathcal{M}^{X},F)$, we have
  \begin{align*}
    \pi_X\Phi(\alpha)&=\pi_X\alpha_X(\id_X) \\
    &=(\pi\alpha)_X(\id_X) \\
    &=\Phi(\pi\alpha) \\
    &=\Phi\Hom_{\dual{\mathcal{C}}}(\mathcal{M}^{X},\pi)(\alpha)
  \end{align*}
  Thus the diagram commutes and $\Phi$ is natural in $F$.

  Similar, for each $f\colon Y\to X$, consider the diagram below
  \begin{displaymath}
      \xymatrix{
         \Hom_{\dual{\mathcal{C}}}(\mathcal{M}^{X},F)\ar[r]^-{\Phi}\ar[d]_{\Hom_{\dual{\mathcal{C}}}(\mathcal{M}^{f},F)}&F(X)\ar[d]^{F(f)}\\
         \Hom_{\dual{\mathcal{C}}}(\mathcal{M}^{Y},F)\ar[r]^-{\Phi}&F(Y)
      }
  \end{displaymath}
  For any $\alpha\in\Hom_{\dual{\mathcal{C}}}(\mathcal{M}^{X},F)$, we have
  \begin{align*}
    F(f)\Phi(\alpha)&=F(f)\alpha_X(\id_X) \\
    &=\alpha_Y\mathcal{M}^{X}(f)(\id_X) \\
    &=\alpha_Y(\id_Xf) \\
    &=\alpha_Y(f)
  \end{align*}
  On the other hand
  \begin{align*}
    \Phi\Hom_{\dual{\mathcal{C}}}(\mathcal{M}^{f},F)(\alpha)&=\Phi(\alpha\mathcal{M}^{f}) \\
    &=(\alpha\mathcal{M}^{f})_Y(\id_Y) \\
    &=\alpha_Y\mathcal{M}^{f}_Y(\id_Y) \\
    &=\alpha_Y(f\id_Y) \\
    &=\alpha_Y(f)
  \end{align*}
  Thus the diagram commutes and $\Phi$ is natural in $X$.
\end{proof}
\begin{cor}
  The functor $\mathcal{M}^{*}$ is fully faithful.
\end{cor}
\begin{proof}
  For any $X,Y\in\ob\mathcal{C}$, we have
 \begin{align*}
   \Hom_{\dual{\mathcal{C}}}(\mathcal{M}^{X},\mathcal{M}^{Y})&\cong \mathcal{M}^{Y}(X)\\
   &= \Hom_{\mathcal{C}}(X,Y)
 \end{align*}
\end{proof}
\begin{rem}
One calls $\mathcal{M}^{*}$ the \termin[Yoneda embedding]{Yoneda!embedding}, sometimes denoted by $Y_{\Cc}$. Hence, one may consider $\mathcal{C}$ as a full subcategory of $\dual{\mathcal{C}}$. In particular, for $X\in\ob\mathcal{C}$, $\mathcal{M}^{X}$ determines $X$ up to unique isomorphism, that is, an isomorphism $\mathcal{M}^{X}\cong\mathcal{M}^{Y}$ determines a unique isomorphism $X\cong Y$.
\end{rem}
  \begin{rem}
     Some authors define an \termin{embedding} to be a fully faithful functor. Such a functor is necessarily injective on objects up-to-isomorphism. For instance, the Yoneda embedding is an embedding in this sense.
  \end{rem}
\begin{exam}
  If $\mathcal{C}$ has products and coproducts, then there is a canonical isomorphism
  \begin{equation*}
    (A\times B)+(A\times C)\cong A\times(B+C).
  \end{equation*}
  To prove this, by the remark above, it is enough to prove
  \begin{equation*}
    \Hom(X,(A\times B)+(A\times C))\cong\Hom(X,A\times(B+C))
  \end{equation*}
  for each $X\in\ob\mathcal{C}$ and this isomorphism is natural in $X$. Which is easy to check.
\end{exam}
\begin{cor}
  Let $\mathcal{C}$ be a category and let $f\colon X\to Y$ be a morphism in $\mathcal{C}$.
  \begin{enumerate}[1)]
    \setlength{\itemindent}{2ex}
    \item Assume that for any $Z\in\mathcal{C}$, the map $\Hom_{\mathcal{C}}(Z,X)\markar{f\circ}\Hom_{\mathcal{C}}(Z,Y)$ is bijective. Then $f$ is an isomorphism.
    \item Assume that for any $Z\in\mathcal{C}$, the map $\Hom_{\mathcal{C}}(X,Z)\markar{\circ f}\Hom_{\mathcal{C}}(Y,Z)$ is bijective. Then $f$ is an isomorphism.
  \end{enumerate}
\end{cor}

  \begin{rem}
    If $\Cc$ is a preadditive category, then the Yoneda's lemma yields a full embedding of $\Cc$ into the functor category $\Add(\Cc^{\op},\Ab)$.
    So $\Cc$ naturally sits inside an abelian category.
  \end{rem}

\subsection{Representable Functors}
\begin{defn}
  One says that a functor $F\colon\op{\mathcal{C}}\to\mathbf{Set}$ is \emph{\red  representable}\index{representable functor} if there exists $X\in\ob\mathcal{C}$ such that $F(Y)\cong\Hom_{\mathcal{C}}(Y,X)$ functorially in $Y\in\mathcal{C}$. In other words, $F\cong\mathcal{M}^X$ in $\dual{\mathcal{C}}$. Such an object $X$ is called a \emph{\red  representative}\index{representative} of $F$. Similarly, a functor $G\colon \mathcal{C}\to\mathbf{Set}$ is \emph{\red  representable} if there exists $X\in\ob\mathcal{C}$ such that $G(Y)\cong\Hom_{\mathcal{C}}(X,Y)$ functorially in $Y\in\mathcal{C}$.
\end{defn}
\begin{rem}
  It is important to notice that the isomorphisms above determine $X$ up to unique isomorphism. More precisely, given two isomorphisms $F\markar{\cong}\mathcal{M}^X$ and $F\markar{\cong}\mathcal{M}^{X'}$ there exists a unique isomorphism $\theta:X\markar{\cong}X'$ making the following diagram commutative:
 \begin{displaymath}
      \xymatrix{
         &F\ar[ld]_{\cong}\ar[rd]^{\cong}&\\
         \mathcal{M}^X\ar[rr]^{\mathcal{M}^{*}(\theta)}_{\cong}&&\mathcal{M}^{X'}
      }
\end{displaymath}
\end{rem}

  \begin{defn}
    Let $\Vv$ be a category. A $\Vv-$valued presheaf $\Fff$ on a category $\Cc$ is a functor $\Fff\colon\Cc^{\op}\To\Vv$.
     Often presheaf is defined to be a $\Set-$valued presheaf.
    A morphism of presheaves is defined to be a natural transformation of functors.
    This makes the collection of all presheaves into a category, often written $\widehat{\Cc}$.
    A functor into $\widehat{\Cc}$ is sometimes called a \termin{profunctor}.
  \end{defn}
  \begin{prop}
    A locally small category $\Cc$ embeds fully and faithfully into the category $\widehat{\Cc}$ of $\Set-$valued presheaves via the Yoneda embedding $Y_{\Cc}$ which to every object $A$ of $\Cc$ associates the hom-set $\Hom_{\Cc}(-,A)$.
  \end{prop}
  \begin{prop}
    The presheaf category $\widehat{\Cc}$ is (up to equivalence of categories) the free colimit completion of the category $\Cc$.
  \end{prop}

\newpage\section{Objects}
\subsection{Initial and Terminal Objects}
\begin{defn}
  Let $\Cc$ be a category. An \termin[initial object]{initial!object} of $\Cc$ is an object $I$ in $\Cc$ such that for every object $X$ in $\Cc$, there exists precisely one morphism $I\to X$. Dually, an object $T$ is \termin[terminal object]{terminal!object} if for every object $X$ in $\Cc$, there exists a single morphism $X\to T$. If an object is both initial and terminal, it is called a \termin[zero object]{zero!object} or \termin{null object}.
\end{defn}
\begin{rem}
  It is easy to see that the initial object and terminal object are unique up to isomorphism. Such universal properties will be detail in the limit theory later.
\end{rem}
  \begin{rem}
    If $\Cc$ has a zero object $0$, then given two objects $X$ and $Y$ in $\Cc$, there are canonical morphisms $f \colon 0 \To X$ and $g \colon Y \To 0$.
    Then, $f\circ g$ is a zero morphism in $\Hom_{\Cc}(Y, X)$. Thus, every category with a zero object is also a category with zero morphisms given by the composition $0_{XY} \colon X \To 0 \To Y$.
  \end{rem}

  Not every category has terminal objects, for example:
  \begin{exam}
    The category of infinite groups do not have a terminal object: given any infinite group $G$ there are infinitely many morphisms $\ZZ\to G$, so $G$ cannot be terminal.
  \end{exam}

\subsection{Subobjects and Quotient Objects}
\begin{defn}
  Let $A,B\in$ be objects in $\mathcal{C}$. If there exist a monomorphism $f\colon A\to B$, then we call $(A,f)$ a \emph{\red  subobject}\index{subobject} of $B$. If there exist an epimorphism $f\colon B\to A$, then we call $(A,f)$ a \emph{\red  quotient object}\index{quotient object} of $B$.
\end{defn}
\begin{warn}
  Notice that the notation of subobject and quotient object may not be suitable abstract of  sub- and quotient in usual sense. For example, Consider $\mathbf{Top}$, the subobjects of an object are not just the subspaces, this concept mixes others. The same story happened in quotient objects.
\end{warn}

\begin{exam}
  In $\mathbf{Top}$, every epimorphism is surjective. However, a quotient object may still not be a quotient space. In fact, for every topological space, the identity map from itself to the trivial topological space on the same underlying set is epimorphism.
\end{exam}

\begin{defn}
  An \emph{\red  extremal monomorphism}\index{extremal monomorphism} is a monomorphism that cannot be nontrivially factored through an epimorphism. In another word, if $m=g\circ e$ with $e$ an epimorphism, then $e$ is an isomorphism. A subobject composed by an object with an extremal monomorphism is called an \emph{\red  extremal subobject}.
\end{defn}
\begin{exam}
  The extremal subobject in $\mathbf{Top}$ is just the subspace with its inclusion map.
\end{exam}
\begin{rem}
  Notice that, in category theory, when we use the word ``is'', is actually under the meaning of ``up to isomorphism''. However, a bijective morphism may not be isomorphism. Which makes lots trouble, especially in epi- case: the concept of quotient object in the category of rings and topological spaces totally lose shape. Even consider special type of quotient objects like extremal ones may not work. The epimorphisms may be very mysterious.
\end{rem}
\begin{rem}
  Given two subobjects $(A,f),(A',f')$ of $B$, the morphism $g$ from $(A,f)$ to $(A',f')$ is the morphism $g$ (in fact, it is unique) in $\mathcal{C}$ such that the following diagram commutes:
  \begin{displaymath}
      \xymatrix{
         A\ar[rd]_{f}\ar[rr]^{g}&&A'\ar[ld]^{f'}\\
         &B&
      }
  \end{displaymath}
  Thus we get a category $\Sub_{\mathcal{C}}(B)$. Similar, we get $\Quot_{\mathcal{C}}(B)$.
\end{rem}
\subsection{Free Objects and Generators}
\begin{defn}
  A \termin{concrete category} $(\mathcal{C},U)$ is a category $\mathcal{C}$ together with a faithful functor $U\colon\mathcal{C}\to\mathbf{Set}$, named
  \termin{forgetful functor}.
  A category $\mathcal{C}$ is called \termin{concretizable} if there exists such a forgetful functor.
\end{defn}
Unlike the literal meaning, a concrete category may be very abstract. In fact, we have
\begin{exam}
 Let $\mathcal{C}$ be any small category, then there exists a faithful functor
 \mapdes{P\colon\dual{\mathcal{C}}}{\mathbf{Set}}{X}{\coprod_{c \in \mathrm{ob}C} X(c)}
 By composing this with the Yoneda embedding $\mathcal{M}^{*}\colon\mathcal{C}\to\dual{\mathcal{C}}$, one obtains a faithful functor $\mathcal{C}\to\mathbf{Set}$.
\end{exam}
Not every category, whose objects are based on sets, are concretizable. For example
\begin{exam}
 The homotopy category of topological spaces $\mathbf{hTop}$, which has same objects as $\mathbf{Top}$ but its morphisms are homotopy classes of continuous functions, is an example of a category that is not concretizable. The fact that there does not exist any faithful functor from $\mathbf{hTop}$ to $\mathbf{Set}$ was first proven by Peter Freyd, see \cite{Freyd70homotopyis}. In the same article, Freyd cites an earlier result that the category of ``small categories and natural equivalence-classes of functors'' also fails to be concretizable.
\end{exam}
A category $\mathcal{C}$ may admit several faithful functors into $\mathbf{Set}$. Hence there may be several concrete categories $(\mathcal{C},U)$ all corresponding to the same category $\mathcal{C}$.
\begin{exam}
  For technical reasons, the category $\mathbf{Ban}_1$ of Banach spaces and linear contractions is often equipped not with the ``obvious'' forgetful functor but the functor $U_1\colon \mathbf{Ban}_1\to\mathbf{Set}$ which maps a Banach space to its (closed) unit ball.
\end{exam}
Notice that, the forgetful functor may map different objects to the same set and, if this occurs, it will also map different morphisms to the same function, which is not contradictory to faithful.
\begin{exam}
  A set $X$ can be equipped different topologies, hence become different objects in $\mathbf{Top}$, and their identity maps are different morphisms. However, the usual forgetful functor maps them to the same set $X$ and their identity maps to one $\id_X$.
\end{exam}
\begin{defn}
  A left adjoint functor $F$ of a forgetful functor $U$ is called the \termin[free functor]{free!functor}. Let $S$ be a set, then $F(S)$ is called the \termin[free object]{free!object} generated by $S$.
\end{defn}
\begin{rem}
  Since $F$ is the adjoint functor of $U$, there must be a natural transformation $\eta\colon\id_{\mathbf{Set}}\to U\circ F$. More explicitly, $F$ is, up to isomorphisms in $\mathcal{C}$, characterized by the following universal property:
\begin{quote}
Whenever $T\in\ob\mathcal{C}$, and $f\colon S\to U(T)$ is a function, then there is a unique $\mathcal{C}-$morphism $g\colon F(S)\to T$ such that $U(g)\circ\eta(S)=f$.
\end{quote}
\end{rem}
Since the last section of Chapter I of Lang's textbook has discussed the free functor of $\mathbf{Grp}$ in detail, we will not repeat them here.

A related but different concept is the generator
\begin{defn}
  A \termin[generator]{generator!of category} (or \termin{separator}) of a category $\mathcal{C}$ is an object $G$, such that for any two different morphisms $f,g\colon X\to Y$, there exist one morphism $h\colon G\to H$ such that $f\circ h\neq g\circ h$.
\end{defn}
\begin{exam}
  $\ZZ$ is a generator in $\mathbf{Ab}$. Similarly, the one-point set is a generator for $\mathbf{Set}$.
\end{exam}
\newpage\section{Limit Theory}
\subsection{Cones and Limits}
\begin{defn}
  Let $\Jj$ and $\Cc$ be categories. A \termin{diagram of type $\Jj$} or a \termin{$\Jj-$diagram} in $\Cc$ is a functor $D\colon \Jj\to \Cc$.
  The category $\Jj$ is called the \termin{index category} or the \termin{scheme} of the diagram $D$.
  For $j$ in the index category, we will write $D(j)$ in the form $D_j$.

  A \termin{cone} to a diagram $D$ consists of an object $C$ in $\Cc$ and a family of arrows in $\Cc$,
  \begin{equation*}
    c_j\colon C\longrightarrow D_j, \forall j\in\ob\mathcal{J}
  \end{equation*}
  such that for each arrow $\alpha\colon i\to j$ in $\Jj$, the following triangle commutes.
  \begin{displaymath}
      \xymatrix{
         &C\ar[ld]_{c_i}\ar[rd]^{c_j}&\\
         D_i\ar[rr]^{D_{\alpha}}&&D_j
      }
  \end{displaymath}

  A morphism of cones
  \begin{equation*}
    \vartheta\colon (C,c_j)\longrightarrow (C',c'_j)
  \end{equation*}
  is an arrow $\vartheta$ in $\Cc$, making each triangle
  \begin{displaymath}
      \xymatrix{
         C\ar[rd]_{c_j}\ar[rr]^{\vartheta}&&C'\ar[ld]^{c'_j}\\
         &D_j&
      }
  \end{displaymath}
  commute.

  Finally, cones to $D$ with morphisms between them form a category $\mathbf{Cone}(D)$ (or denote $\Delta\downarrow D$).
\end{defn}
\begin{defn}
  A \termin{limit} for a diagram $D\colon \mathcal{J}\to \mathcal{C}$ is a terminal object in $\mathbf{Cone}(D)$. In particular, a finite limit is a limit for a diagram on a finite index category $\mathcal{J}$.
\end{defn}
\begin{rem}
One often denote a limit in the form
\begin{equation*}
  p_i\colon\mathop{\underleftarrow{\lim}}\limits_{j}D_j \longrightarrow D_i
\end{equation*}
When $\{p_i\}$ are obvious, one may simply call the object as the limit.
\end{rem}
\begin{exam}
  Let $\mathbf{0}$ be the empty category, then in any category $\mathcal{C}$, there is only one diagram of type $\mathbf{0}$: the empty one. A cone to the empty diagram is essentially just an object of
   $\mathcal{C}$. The limit of the empty diagram is just the \emph{\red terminal} object in $\mathcal{C}$.
\end{exam}
\begin{exam}
  Let $\mathcal{C}$ be a small category, $\id_{\mathcal{C}}\colon\mathcal{C}\to\mathcal{C}$ is the identity functor. If $\mathcal{C}$ has initial object $I$, then $I$ is the limit of $\id_{\mathcal{C}}$. Conversely, if $\{p_B\colon A\to B\mid B\in\ob\mathcal{C}\}$ is the limit of $\id_{\mathcal{C}}$, then its easy to see that it is the \emph{\red initial} object in $\mathcal{C}$.
\end{exam}
\begin{exam}
  Take $\mathcal{J}=\{1,2\}$ the discrete category with two objects and no nonidentity arrows. A diagram $D\colon\mathcal{J}\to\mathcal{C}$ hence is a pair of objects $D_1, D_2\in\mathcal{C}$. A cone to $D$ is an object $C$ equipped with arrows
  \begin{displaymath}
      \xymatrix{
         D_1&C\ar[l]_-{c_1}\ar[r]^-{c_2}&D_2
      }
  \end{displaymath}
  The limit of $D$ is just the \emph{\red  product} of $D_1$ and $D_2$ in $\mathcal{C}$.
\end{exam}
\begin{exam}\label{limit.2}
  Take $\mathcal{J}$ to be the following category:
  \begin{displaymath}
      \xymatrix{
         1\ar@<0.5ex>[r]^{\alpha}\ar@<-0.5ex>[r]_{\beta} &2
      }
  \end{displaymath}
  Hence a diagram $D$ of type $\mathcal{J}$ looks like
  \begin{displaymath}
      \xymatrix{
         D_1\ar@<0.5ex>[r]^{D_{\alpha}}\ar@<-0.5ex>[r]_{D_{\beta}} &D_2
      }
  \end{displaymath}
  The limit of $D$ is the \emph{\red  equalizer} of $D_{\alpha},D_{\beta}$.
\end{exam}
\begin{exam}
  Let $(I,\leqslant)$ be a \termin{filtered partially ordered set} (FPOS), which means that for each two elements $i,j$, there exist an element $k\in I$ such that $k\leqslant i,k\leqslant j$. Treat $I$ as a category, in any category $\mathcal{C}$, a diagram $D\colon I\to\mathcal{C}$ satisfy that for any $i\leqslant j\leqslant k$ in $I$, $D_{j\leqslant k}D_{i\leqslant j}=D_{i\leqslant k}$ is called an \termin[inverse system]{inverse!system} and its limit is called an \termin[inverse limit]{inverse!limit}, or \termin[projective limit]{projective!limit}.
\end{exam}
The limit of a diagram sometimes works like monomorphism, although each arrow in the cone may not be injective.
\begin{prop}\label{collective inj}
  Let $\{p_j\colon A\to D_j\mid j\in\ob\mathcal{J}\}$ be the limit of diagram $D\colon\mathcal{J}\to\mathcal{C}$. For any $f,g\colon B\to A$, if $p_jf=p_jg, \forall j\in\ob\mathcal{J}$, then $f=g$.
\end{prop}
\begin{exam}
  Let $\{(A_i,f_i)\mid i\in I\}$ be a family of subobject of $A$ in $\mathcal{C}$. Treat $\{f_i\colon A_i\to A\mid i\in I\}$ as a subcategory of $\mathcal{C}$. hence its inclusion functor is a diagram in $\mathcal{C}$. If such a diagram has limit $C$, then then arrow from $C$ to $A$ can be determined by each arrow $\alpha_i$ from $C$ to $A_i$. Use the proposition above, it is easy to check that $f_i\alpha_i\colon C\to A$ is injective, hence $(C,f_i\alpha_i)$ is also a subobject of $A$, called the \termin[intersection]{intersection!of subobjects} of $\{(A_i,f_i)\mid i\in I\}$.
\end{exam}
\subsection{Co-cones and Colimits}
Dually, we have corresponding concepts
\begin{defn}
  A \termin{co-cone} to a diagram $D$ consists of an object $C$ in $\mathcal{C}$ and a family of arrows in $\mathcal{C}$,
  \begin{equation*}
    c_j\colon D_j\longrightarrow C, \forall j\in\ob\mathcal{J}
  \end{equation*}
  such that for each arrow $\alpha\colon i\to j$ in $\mathcal{J}$, the following triangle commutes.
  \begin{displaymath}
      \xymatrix{
         D_i\ar[rr]^{D_{\alpha}}\ar[rd]_{c_i}&&D_j\ar[ld]^{c_j} \\
         &C&
      }
  \end{displaymath}

  A morphism of co-cones
  \begin{equation*}
    \vartheta\colon (C,c_j)\longrightarrow (C',c'_j)
  \end{equation*}
  is an arrow $\vartheta$ in $\mathcal{C}$, making each triangle
  \begin{displaymath}
      \xymatrix{
         &D_j\ar[ld]_{c_j}\ar[rd]^{c'_j}& \\
         C\ar[rr]^{\vartheta}&&C'
      }
  \end{displaymath}
  commute.

  Finally, co-cones to $D$ with morphisms between them form a category $\mathbf{Cocone}(D)$ (or denote $D\downarrow\Delta$).
\end{defn}
\begin{defn}
  A \termin{colimit} for a diagram $D\colon \mathcal{J}\to \mathcal{C}$ is a initial object in $\mathbf{Cocone}(D)$. In particular, a finite colimit is a colimit for a diagram on a finite index category $\mathcal{J}$.
\end{defn}
\begin{rem}
One often denote a colimit in the form
\begin{equation*}
  k_i\colon\mathop{\underrightarrow{\lim}}\limits_{j}D_j \longleftarrow D_i
\end{equation*}
When $\{k_i\}$ are obvious, one may simply call the object as the colimit.
\end{rem}
\begin{exam}
  Let $\mathcal{C}$ be a small category, the colimit of the identity functor is just the \emph{\red terminal} object in $\mathcal{C}$.
\end{exam}
\begin{exam}
  Take $\mathcal{J}=\{1,2\}$ the discrete category with two objects and no nonidentity arrows. A diagram $D\colon\mathcal{J}\to\mathcal{C}$ hence is a pair of objects $D_1, D_2\in\mathcal{C}$. A co-cone to $D$ is an object $C$ equipped with arrows
  \begin{displaymath}
      \xymatrix{
         D_1\ar[r]^-{c_1}&C&D_2\ar[l]_-{c_2}
      }
  \end{displaymath}
  The colimit of $D$ is just the \emph{\red  coproduct} of $D_1$ and $D_2$ in $\mathcal{C}$.
\end{exam}
\begin{exam}
  Take noations as in \ref{limit.2}, the colimit of $D$ is the \emph{\red  coequalizer} of $D_{\alpha},D_{\beta}$.
\end{exam}
\begin{exam}
  Let $(I,\leqslant)$ be a \termin{directed partially ordered set} (DPOS), which means that for each two elements $i,j$, there exist an element $k\in I$ such that $i\leqslant k,j\leqslant k$. Treat $I$ as a category, in any category $\mathcal{C}$, a diagram $D\colon I\to\mathcal{C}$ satisfy that for any $i\leqslant j\leqslant k$ in $I$, $D_{j\leqslant k}D_{i\leqslant j}=D_{i\leqslant k}$ is called an \termin[direct system]{direct!system} and its colimit is called an \termin[direct limit]{direct!limit}, or \termin{inductive limit}.
\end{exam}

The colimit of a diagram sometimes works like epimorphism, although each arrow in the cone may not be surjective.
\begin{prop}\label{collective surj}
  Let $\{k_j\colon D_j\to A\mid j\in\ob\mathcal{J}\}$ be the limit of diagram $D\colon\mathcal{J}\to\mathcal{C}$. For any $f,g\colon A\to B$, if $fk_j=gk_j, \forall j\in\ob\mathcal{J}$, then $f=g$.
\end{prop}
\begin{exam}
  Let $\{(A_i,f_i)\mid i\in I\}$ be a family of quotient object of $A$ in $\mathcal{C}$. Treat $\{f_i\colon A\to A_i\mid i\in I\}$ as a subcategory of $\mathcal{C}$. hence its inclusion functor is a diagram in $\mathcal{C}$. If such a diagram has colimit $C$, then then arrow from $A$ to $C$ can be determined by each arrow $\alpha_i$ from $A_i$ to $C$. Use the proposition above, it is easy to check that $\alpha_if_i\colon A\to C$ is surjective, hence $(C,f_i\alpha_i)$ is also a quotient object of $A$, called the \termin{cointersection} of $\{(A_i,f_i)\mid i\in I\}$.
\end{exam}

\begin{prop}\label{Hom-dir.lim}
  Colimits are linked to limits via
  \begin{equation*}
    \Hom(\dirlim_{\Jj} X_i, Y) = \invlim_{\Jj^{\op}} \Hom(X_i,Y)
  \end{equation*}
\end{prop}
  \begin{proof}
    For any connection morphism $\phi^i_j\colon X_i\To X_j$ in $(X_i)$, the corresponding connection map in $(\Hom(X_i,Y))$ is
    \mapdes{\Hom(X_j,Y)}{\Hom(X_i,Y)}{f}{f\circ\phi^i_j}
    Use this corresponding, the statement is easy to verify.
  \end{proof}

  A similar proposition is
  \begin{prop}\label{Hom-inv.lim}
    \begin{equation*}
      \Hom(X,\invlim_{\Jj} Y_i) = \invlim_{\Jj} \Hom(X,Y_i)
    \end{equation*}
  \end{prop}
  \begin{proof}
    For any connection morphism $\phi^i_j\colon Y_i\To Y_j$ in $(Y_i)$, the corresponding connection map in $(\Hom(X,Y_i))$ is
    \mapdes{\Hom(X,Y_i)}{\Hom(X,Y_j)}{f}{\phi^i_j\circ f}
    Use this corresponding, the statement is easy to verify.
  \end{proof}

\subsection{Kernels and Cokernels}
  \begin{defn}
    Let $\{f_i\}$ be a family of parallel morphisms, which can be view as a diagram. the \termin{equalizer} is the limit of the diagram.
    Dually, the \termin{coequalizer} is the colimit.
  \end{defn}
  \begin{rem}
    The correct diagram for the degenerate case with \emph{no morphisms} is slightly subtle:
    one might initially draw the diagram as consisting of the two objects $X,Y$ and no morphisms.
    This is incorrect, however, since the limit of such a diagram is the product of these two objects, rather than the equalizer.
    (And indeed products and equalizers are different concepts: the set-theoretic definition of them are different.)

    Instead, the appropriate insight is that every equalizer diagram is fundamentally concerned with the $X$, including $Y$ only because $Y$ is the codomain of morphisms which appear in the diagram.

    With this view, we see that if there are no morphisms involved, $Y$ does not make an appearance and the equalizer diagram consists of $X$ alone. The limit of this diagram is then any isomorphism to $X$.

    Similarly, the correct coequalizer diagram of \emph{no morphisms} consists of the codomain alone.
  \end{rem}

  \begin{defn}
    An equalizer of exactly two morphisms is sometimes called the \termin[difference kernel]{difference!kernel} of them.
  \end{defn}

  \begin{prop}
    Any equalizer is a monomorphism. Dually, any coequalizer is an epimorphism.
  \end{prop}
  \begin{rem}
    A monomorphism is said to be \termin[regular]{regular!monomorphism} if it is an equalizer of some set of morphisms.
    Dually, an epimorphism is said to be \termin[regular]{regular!epimorphism} if it is a coequalizer of some set of morphisms.
  \end{rem}

  \begin{defn}
    The \termin{kernel} of a morphism $f$ is the equaliser of $f$ and the parallel zero morphism. Dually, The \termin{cokernel} is the coequalizer of $f$ and the parallel zero morphism.
  \end{defn}

  \begin{defn}
    A monomorphism is called \termin[norm]{norm!monomorphism}, if it is a kernel of some morphism. Dually, an epimorphism is called \termin[norm]{norm!epimorphism}, if it is a cokernel of some morphism.
  \end{defn}

  \begin{exam}
    Coequalisers can be large: There are exactly two functors from the category $\mathbf{1}$ having one object and one identity arrow, to the category $\mathbf{2}$ with two objects and exactly one non-identity arrow going between them. The coequaliser of these two functors is the monoid of natural numbers under addition, considered as a one-object category. In particular, this shows that while every coequalising arrow is epic, it is not necessarily surjective.
  \end{exam}

  \begin{prop}
    The kernel of the limit is also the limit of the kernels. Dually, the cokernel of the colimit is also the colimit of the cokernels.
  \end{prop}
  \begin{proof}
     We start with a commutative diagram:
         \begin{displaymath}
            \xymatrix{
               \cdot\ar@{-->}[rd]|{u}\ar@/^/[rrd]^{t}\ar@{-->}@/_/[rdd]_{u_n}&&&\\
                & \cdot\ar[r]^{\ker}\ar[d]_p & \cdot\ar[r]^{f}\ar[d]_p & \cdot\ar[d]_p \\
                & \cdot\ar[r]^{\ker f_n} & \cdot\ar[r]^{f_n} & \cdot             }
          \end{displaymath}
    Here, $f$ is the limit of $f_n$, and $\ker$ is the limit of the kernels, we want to show that it is also the kernel of $f$.

    First, by the collective injectivity of limit (ref. \ref{collective inj}), the composition of $f$ and $\ker$ is equal to $0$.

    Then, for any $t$ such that $f\circ t=0$, we have $f_n\circ p\circ t=0$,
    hence there exist a unique morphism $u_n$ such that $\ker f_n\circ u_n = p\circ t$.
    By the definition of limit, there exist a unique morphism $u$ such that $u_n=p\circ u$.
    Hence $p\circ\ker\circ u=p\circ t$. By the collective injectivity again, $\ker\circ u=t$, which proves $\ker$ is the kernel of $f$.
  \end{proof}

\subsection{Products and Coproducts}
\begin{defn}
  Consider a diagram of type $\mathcal{J}$, where $\mathcal{J}$ is a discrete category. It looks like a family of objects without arrows between them. The limit of such a diagram is called the \termin{product} of these objects. Dually, the colimit of such a diagram is called the \termin{coproduct} of these objects.
\end{defn}
\begin{rem}
  We often denote the product and the coproduct of a family of objects $\{A_i\}_{i\in I}$ as $\prod A_i$ and $\coprod A_i$.
\end{rem}
\begin{exam}
  Since the nullary discrete category is the empty category, the \termin[nullary product]{nullary!product} is just the terminal object. Similar, the \termin[unary product]{unary!product} of any object is itself. Dually, we get \termin[nullary coproduct]{nullary!coproduct} and \termin[unary coproduct]{unary!coproduct}.
\end{exam}
\begin{prop}
  A category has finite products, which means that each family of finite objects has product, if and only if it has binary product and terminal object.
\end{prop}
\begin{proof}
  For $n\geqslant 2$, it is clear that $(\cdots((A_1\times A_2)\times A_3)\cdots\times A_n)$ is the $n-$ary product of $A_1,A_2\cdots,A_n$. Notice that the word ``finite'' include the nullary case, so we still need the existence of terminal object.
\end{proof}
\begin{defn}
  Let $\{f_i\colon A_i\to B_i\}_{i\in I}$ be family of morphisms in a category has products, the product of them is the only morphism $\Pi f_i\colon\Pi A_i\to\Pi B_i$ make the following diagram commutative
  \begin{displaymath}
      \xymatrix{
         \Pi A_i\ar@{-->}[r]^{\Pi f_i}\ar[d]&\Pi B_i\ar[d]\\
         A_j\ar[r]^{f_j}&B_j
      }
  \end{displaymath}
\end{defn}
\begin{prop}
  Product is a functor from $[I,\mathcal{C}]$ to $\mathcal{C}$.
\end{prop}
\begin{proof}
  It's easy to check by definition.
\end{proof}
\begin{rem}
  Similar, we can define coproduct of morphisms and check coproduct is a functor. More general, limit and colimit are functors.
\end{rem}
\begin{prop}
  Monomorphisms are stable under product, which means the product of a family of monomorphisms is also a monomorphism.
\end{prop}
\begin{prop}
  In a category has products, for any $i\in I$, let $(E_i,e_i)$ be the equalizer of $f_i,g_i\colon A_i\to B_i$, then, $(\Pi E_i,\Pi e_i)$ is the equalizer of $\Pi f_i,\Pi g_i\colon \Pi A_i\to\Pi B_i$.
\end{prop}
\begin{prop}
   Regular monomorphisms are stable under product, and so are isomorphisms.
\end{prop}
Dually, we have similar propositions for coproduct.

\subsection{Pullback and Pushout}
\begin{defn}
  Let $\Jj$ be
  \begin{displaymath}
      \xymatrix{
         \cdot\ar[r]&\cdot&\cdot\ar[l]
      }
  \end{displaymath}
  a limit for a $\Jj-$diagram is of the form
  \begin{displaymath}
      \xymatrix{
         A\ar[r]^{f}&C&B\ar[l]_{g}
      }
  \end{displaymath}
  which can be view as a commutative square in $\Cc$:
  \begin{displaymath}
      \xymatrix{
         P\ar[r]^{\overline{f}}\ar[d]_{\overline{g}}&B\ar[d]^{g}\\
         A\ar[r]^{f}&C
      }
  \end{displaymath}
  we call it a \termin[pullback square]{pullback!square} or \termin{cartesian diagram}, and say $\overline{g}$ is the \termin{pullback} of $g$ through $f$, $\overline{f}$ is the \emph{\red  pullback} of $f$ through $g$. We also call this limit the \termin[fibre product]{fibre!product} of $A$ and $B$ over $C$, and denoted by $A\times_CB$.
\end{defn}
Sometime, we consider such kind of category, in which every pullback exist, we call it a \termin[category with pullbacks]{category!with pullbacks} or say it \emph{\red  has} pullbacks.
\begin{prop}\label{pullback}
  Let $\xymatrix@1{A\ar[r]^{f}&C&B\ar[l]_{g}}$ be a pair of morphisms in $\mathcal{C}$, and $\xymatrix@1{A&A\times B\ar[r]^-{p_B}\ar[l]_-{p_A}&B}$ be the product of $A$ and $B$, $e\colon E\to A\times B$ is the equalizer of $fp_A$ and $gp_B$. Then the following diagram is cartesian:
  \begin{displaymath}
      \xymatrix{
         E\ar[r]^{p_Be}\ar[d]_{p_Ae}&B\ar[d]^{g}\\
         A\ar[r]^{f}&C
      }
  \end{displaymath}
\end{prop}
\begin{prop}
  Monomorphisms are \termin{stable} under pullback, which means that the pullback of a monomorphism is also a monomorphism. Moreover, regular monomorphisms are also stable, and so are isomorphisms.
\end{prop}
\begin{warn}
  Epimorphisms may not be stable under pullback in any category with pullbacks.
\end{warn}
The following proportion is a good exercise for diagram chase:
\begin{prop}[Two-pullbacks]\label{Two-pullbacks}
  Consider a commutative diagram in a category with pullbacks as below:
  \begin{displaymath}
      \xymatrix{
         \cdot\ar[r]\ar[d]&\cdot\ar[r]\ar[d]&\cdot\ar[d]\\
         \cdot\ar[r]&\cdot\ar[r]&\cdot
      }
  \end{displaymath}
  \begin{enumerate}[1)]
    \setlength{\itemindent}{2ex}
    \item  If the two small squares are pullbacks, so is the outer rectangle.
    \item  If the right square and the outer rectangle are pullbacks, so is the left square.
  \end{enumerate}
\end{prop}
\begin{cor}
  The pullback of a commutative triangle is a commutative triangle.
  \begin{displaymath}
      \xymatrix{
         \cdot\ar[rr]\ar[dd]\ar[dr]&&\cdot\ar[dd]\ar[dr]&\\
         &\cdot\ar[dl]&&\cdot\ar[dl]\\
         \cdot\ar[rr]&&\cdot
      }
  \end{displaymath}
\end{cor}

The dual concept of pullback is pushout.
\begin{defn}
  Let $\Jj'$ be
  \begin{displaymath}
      \xymatrix{
         \cdot&\cdot\ar[l]\ar[r]&\cdot
      }
  \end{displaymath}
  a limit for a $\Jj'-$diagram is of the form
  \begin{displaymath}
      \xymatrix{
         B&C\ar[l]_{g}\ar[r]^{f}&A
      }
  \end{displaymath}
  which can be view as a commutative square in $\Cc$:
  \begin{displaymath}
      \xymatrix{
         C\ar[r]^{f}\ar[d]_{g}&A\ar[d]^{\overline{g}}\\
         B\ar[r]^{\overline{f}}&P
      }
  \end{displaymath}
  we call it a \termin[pushout square]{pushout!square} or \termin{cocartesian diagram}, and say $\overline{g}$ is the \termin{pushout} of $g$ through $f$, $\overline{f}$ is the \emph{\red  pushout} of $f$ through $g$. We also call this limit the \termin[fibre coproduct]{fibre!coproduct} of $A$ and $B$ over $C$, and denoted by $A\amalg_CB$.
\end{defn}
By the duality principle, the duality of the properties of pullback are also true.

\begin{defn}
  Let $\overline{f}$ be the pullback of $f$ through $g$, it is called \termin[descendable]{descendable pullback} if $f$ is also the pushout of $\overline{f}$ through $g$.
\end{defn}

\begin{prop}\label{Fun.pull-push}
  Let $\Cc$ be a category with pullbacks (resp, pushouts), then taking pullback (resp. pushout) is a functor from $[\Jj,\Cc]$ to $[\Jj',\Cc]$ (resp. from $[\Jj',\Cc]$ to $[\Jj,\Cc]$).
\end{prop}

\subsection{Complete Categories}
\begin{defn}
  A category is said to be \termin[complete]{complete category}, if every diagram in it has a limit. Similar, a \termin[finite complete category]{finite!complete category} is such a category, in which every finite diagram has a limit. Dually, we have concepts of \termin{cocomplete category} and \termin[finite cocomplete category]{finite!cocomplete category}.
\end{defn}
\begin{thm}
  Let $\mathcal{C}$ be a category, the following statements are equivalent:
  \begin{enumerate}[a)]
    \setlength{\itemindent}{2ex}
    \item $C$ is finite complete.
    \item $C$ has finite products and equalizers.
    \item $C$ has pullbacks and terminal object.
  \end{enumerate}
\end{thm}

\newpage\section{Exactness}
\subsection{Exact Categories}
  \begin{defn}
    An \termin[exact category]{exact!category} $\Ee$ is an additive category possessing a class $\EEe$ of ``short exact sequences'': triples of objects connected by arrows
    \begin{equation*}
      M'\To M\To M''
    \end{equation*}
    satisfying the following axioms inspired by the properties of short exact sequences in an abelian category:
    \begin{enumerate}
      \item $\Ee$ is closed under isomorphisms and contains the split exact sequences:
               \begin{equation*}
                 M'\To M'\oplus M''\To M''
               \end{equation*}
      \item Suppose $M\to M''$ occurs as the second arrow of a sequence in $\EEe$ (it is called an \termin[admissible epimorphism]{admissible!epimorphism}) and $N\to M''$ is any arrow in $\Ee$. Then their \emph{pullback} exists and its \emph{projection} to $N$ is also an admissible epimorphism.

          Dually, if $M'\to M$ occurs as the first arrow of a sequence in $\EEe$ (it is called an \termin[admissible monomorphism]{admissible!monomorphism}) and $M'\to N$ is any arrow, then their \emph{pushout} exists and its \emph{coprojection} from $N$ is also an admissible monomorphism.

          \emph{In other words, the admissible epimorphisms are ``stable under pullback'', resp. the admissible monomorphisms are ``stable under pushout''.}
      \item Admissible monomorphisms are \emph{kernels} of their corresponding admissible epimorphisms, and dually.
               The composition of two admissible monomorphisms is admissible (likewise admissible epimorphisms);
      \item Suppose $M\to M''$ is a map in $\Ee$ which admits a kernel in $\Ee$, and suppose $N\to M$ is any map such that the composition
               $N\to M\to M''$ is an admissible epimorphism. Then so is $M\to M''$.

               Dually, if $M'\to M$ admits a cokernel and $M\to N$ is such that $M'\to M\to N$ is an admissible monomorphism, then so is $M'\to M$.
    \end{enumerate}
  \end{defn}
  \begin{rem}
    Admissible monomorphisms are generally denoted $\mono$ and admissible epimorphisms are denoted $\epi$. These axioms are not minimal; in fact, the last one has been shown by Bernhard Keller \cite{keller1990chain} to be redundant.
  \end{rem}

  \begin{defn}
    An \termin[exact functor]{exact!functor} $F$ from an exact category $\Dd$ to another one $\Ee$ is an additive functor such that if
    \begin{equation*}
      M'\mono M\epi M''
    \end{equation*}
    is exact in $\Dd$, then
    \begin{equation*}
      F(M')\mono F(M) \epi F(M'')
    \end{equation*}
    is exact in $\Ee$.
  \end{defn}

  \begin{defn}
    A subcategory $\Dd$ of $\Ee$ is called an \termin[exact subcategory]{exact!subcategory} if the inclusion functor is fully faithful and exact.
  \end{defn}

  \begin{defn}
    A \termin{Serre subcategory} is a non-empty full subcategory $\Ss$ of an abelian category $\Aa$ such that for all short exact sequences
    \begin{equation*}
      0\To M'\To M\To M''\To 0
    \end{equation*}
    in $\Aa$, $M$ belongs to $\Ss$ if and only if both $M'$ and $M''$ do. This notion arises from Serre's C-theory.
  \end{defn}

  \begin{exam}
    Exact categories come from abelian categories in the following way.
    Suppose $\Aa$ is abelian and let $\Ss$ be any Serre subcategory.
    We can take the class $\EEe$ to be simply the sequences in $\Ss$ which are exact in $\Aa$; that is,
    \begin{equation*}
      M'\To M\To M''
    \end{equation*}
    is in $\EEe$ iff
    \begin{equation*}
      0\To M'\To M\To M''\To 0
    \end{equation*}
    is exact in $\Aa$. Then $\Ss$ is an exact category.
  \end{exam}
  \begin{rem}
    The condition Serre subcategory can be weakened to be a strictly full additive subcategory which is closed under taking \emph{\red extensions} in the sense that given an exact sequence
    \begin{equation*}
      0\To M'\To M\To M''\To 0
    \end{equation*}
    in $\Aa$, then if $M',M''$ are in E, so is $M$.
  \end{rem}

  \begin{exam}
    The category $\Abtf$ of torsion-free abelian groups is exact.
  \end{exam}

  \begin{exam}
    The category $\Ab_{\tor}$ of abelian groups with torsion (and also the zero group) is exact.
  \end{exam}

\subsection{Exact Functors}
  \begin{defn}
    Let $\Aa,\Bb$ be two abelian categories, $F\colon\Aa\To\Bb$ is an additive functor. Let
    \begin{equation*}
      0\To A\To B\To C\To 0
    \end{equation*}
    be a short exact sequence in $\Aa$. We say that $F$ is
    \begin{itemize}
      \item \termin{half exact} if $F(A)\To F(B)\To F(C)$ is exact.
      \item \termin[left exact]{left!exact} if $0\To F(A)\To F(B)\To F(C)$ is exact.
      \item \termin[right exact]{right!exact} if $F(A)\To F(B)\To F(C)\To 0$ is exact.
      \item \termin[exact]{exact!functor} if $0\To F(A)\To F(B)\To F(C)\To 0$ is exact.
    \end{itemize}
  \end{defn}

  For contravariant functor, the definition is similar.

  \begin{prop}
    A covariant (not necessarily additive) functor is left exact if and only if it turns finite limits into limits; a covariant functor is right exact if and only if it turns finite colimits into colimits; a contravariant functor is left exact if and only if it turns finite colimits into limits; a contravariant functor is right exact if and only if it turns finite limits into colimits. A functor is exact if and only if it is both left exact and right exact.
  \end{prop}
  \begin{prop}
    If the functor $F$ is left adjoint to $G$, then $F$ is right exact and $G$ is left exact.
  \end{prop}

  The degree to which a left exact functor fails to be exact can be measured with its right derived functors; the degree to which a right exact functor fails to be exact can be measured with its left derived functors.

\newpage\section{Diagram Lammas in Abelian Categories}
\emph{Throughout of this section, the category is assumed to be abelian unless otherwise specified.}
\subsection{Abelian Category}

The concept of abelian category has been introduced in Chapter 1. We discuss some properties of abelian categories.

\begin{prop}\label{monoic epi 1}
  A morphism $f\colon A\To B$ is monoic (resp. epi) if and only if $\ker f=0$ (resp. $\coker f=0$) if and only if $\coim f=1_A$ (resp. $\im f=1_B$) if and only if $f=\im f$ (resp. $f=\coim f$).
\end{prop}
\begin{proof}
  We prove the monoic case only, the epi case is dual to it.

  (monoic $\Longrightarrow$ $\ker f=0$.) We have $f\circ\ker f=0=f\circ 0$, since$f$ is monoic, $\ker f=0$.

  ($\ker f=0$ $\Longrightarrow$ $\coim f=1_A$.) It is easy to check that $1_A$ satisfying the universal property of $\coim f$.

   ($\coim f=1_A$ $\Longrightarrow$ $f=\im f$.) By the natural isomorphism $\Psi$.

  ($f=\im f$ $\Longrightarrow$ monoic.) A kernel is always monoic.
\end{proof}

\begin{prop}
  Any bimorphism must be a isomorphism.
\end{prop}
\begin{proof}
  Suppose $f\colon A\To B$ is a bimorphism, which means that $f$ is both monoic and epi, hence $\coim f=1_A$ and $\im f=1_B$ by Proposition \ref{monoic epi 1}. Whence the standard factorization become $f=\Psi$ which is a isomorphism in an abelian category.
\end{proof}

\begin{cor}\label{epi-mono}
  Every morphism $f$ can be uniquely factorized as an epimorphism $e$ followed by a monomorphism $m$. Moreover, $e=\coim f$, $m=\im f$.
\end{cor}

\begin{lem}\label{factor through}
  Let $f$ be factorized as $h$ followed by $g$, then
  \begin{enumerate}[(i)]
    \item If $g$ is monoic, then $\ker f=\ker h$;
    \item If $h$ is epi, then $\coker f=\coker g$.
  \end{enumerate}
\end{lem}
\begin{proof}
  By check the definition, the statements are clearly true and hold in any category where related concepts make sense.
\end{proof}

\begin{defn}
  A monomorphism is called \termin[norm]{norm!monomorphism}, if it is a kernel of some morphism. Dually, an epimorphism is called \termin[norm]{norm!epimorphism}, if it is a cokernel of some morphism.
\end{defn}

We now introduce some equivalent conditions of abelian category.

\begin{prop}
  A pre-abelian category becomes abelian if and only if all monomorphisms and epimorphisms are normal.
\end{prop}
\begin{proof}
  The ``only if'' comes from Proposition \ref{monoic epi 1}.

  Conversely, if all monomorphisms and epimorphisms are normal, then for every morphism $f$,
\end{proof}

\subsection{Cartesian Diagrams}
\begin{lem}
  Consider the following diagram:
  \begin{displaymath}
      \xymatrix{
         E\ar[r]^{f'}\ar[d]_{g'}&B\ar[d]^{g}\\
         A\ar[r]^{f}&C
      }
  \end{displaymath}
  \begin{enumerate}[(i)]

    \item The diagram is commutative if and only if the composition
             \begin{equation*}
               \longexseq{E}{\<f',g'\>}{A\oplus B}{\<f,-g\>}{C}
             \end{equation*}
             is equal to $0$. Where $\<f',g'\>$ is the unique morphism such that $f'=p_2\<f',g'\>$ and $g'=p_1\<f',g'\>$, $\<f,-g\>$ is the unique morphism such that $f=\<f,-g\>i_1$ and $-g=\<f,-g\>i_2$.
    \item This diagram is cartesian if and only if $\<f',g'\>=\ker\<f,-g\>$.
    \item This diagram is cocartesian if and only if $\<f,-g\>=\coker\<f',g'\>$.
  \end{enumerate}
\end{lem}
\begin{proof}
  \begin{enumerate}[(i)]
    \item By the definition, we have
             \begin{equation*}
               fg'-gf'=\<f,-g\>i_1p_1\<f',g'\>+\<f,-g\>i_2p_2\<f',g'\>=\<f,-g\>\<f',g'\>
             \end{equation*}
             Hence $fg'=gf'$ if and only if $\<f,-g\>\<f',g'\>=0$.
    \item It follows from Proposition \ref{pullback}.
    \item It is the dual of \emph{(ii)}.
  \end{enumerate}
\end{proof}

\begin{prop}\label{bicartesian}
  In a cartesian diagram, if $f$ is an epimorphism, then so is $f'$, and the diagram is also cocartesian.
\end{prop}
\begin{proof}
  If $f$ is epimorphism, then so is $\<f,-g\>$: let $u\colon C\To T$ be an arbitrary morphism such that $u\<f,-g\>=0$, then $uf=u\<f,-g\>i_1=0$, which implies $u=0$. Thus we get an exact sequence:
  \begin{equation*}
    0\To \longexseq{E}{\<f',g'\>}{A\oplus B}{\<f,-g\>}{C} \To 0
  \end{equation*}
  Hence $\<f,-g\>=\coker\<f',g'\>$, and the diagram is cocartesian.

  Let $v\colon B\To T$ be an arbitrary morphism such that $vf'=0$, then $vp_2\<f',g'\>=0$, hence there exists a morphism $w\colon C\To T$ such that $vp_2=w\<f,-g\>$.
  \begin{displaymath}
      \xymatrix{
          E\ar[r]^-{\<f',g'\>} & A\oplus B\ar[r]^-{\<f,-g\>}\ar[dr]_{vp_2} & C\ar@{-->}[d]^{w}  \\
          & & T
      }
  \end{displaymath}

  We then have (notice that $p_2i_1=0$)
  \begin{equation*}
    0=vp_2i_1=w\<f,-g\>i_1=wf
  \end{equation*}
  hence $w=0$ and therefore $v=0$.
\end{proof}

\begin{prop}
  In a cartesian diagram, let $k\colon K\To A$ be the kernel of $f$. Then $k$ can be factor as $k=g'k'$ where $k'$ is a kernel of $f'$.
\end{prop}
\begin{proof}
  First, we show that $k$ can be factor as $k=g'k'$:   Since $\<f',g'\>=\ker\<f,-g\>$ and
  \begin{equation*}
    \<f,-g\>i_1k=fk=0
  \end{equation*}
  there exist a unique morphism $k'\colon K\To E$ such that
  \begin{equation*}
     \<f',g'\>k'=i_1k
  \end{equation*}
  \begin{displaymath}
      \xymatrix{
          E\ar[r]^-{\<f',g'\>} & A\oplus B\ar[r]^-{\<f,-g\>} & C  \\
          K\ar[ur]_{i_1k}\ar@{-->}[u]^{k'} & &
      }
  \end{displaymath}
  Hence
  \begin{equation*}
    g'k'=p_1\<f',g'\>k'=p_1i_1k=k
  \end{equation*}

  We now prove that $k'$ is a kernel of $f'$:
  Let $t\colon T\To E$ be an arbitrary morphism such that $f't=0$. Then $fg't=gf't=0$, hence there exists a unique morphism $u$ such that $ku=g't$.
  \begin{displaymath}
      \xymatrix{
          T\ar[r]^{t}\ar@{-->}[d]_{u} & E\ar[r]^{f'}\ar[d]^{g'} & B\ar[d]^{g}  \\
          K\ar[ur]_{k'}\ar[r]_{k} & A\ar[r]_{f} & C
      }
  \end{displaymath}
  Notice that
  \begin{equation*}
    f'k'u=p_2\<f',g'\>k'u=p_2i_1ku=0=f't
  \end{equation*}
  and $g'k'u=ku=g't$. Hence $k'u=t$. Which proved $k'$ is a kernel of $f'$.
\end{proof}

\subsection{Snake Lemma}
\emph{When we drawing diagrams contain many kernels and cokernels, to simplify the notation, we denoted the domain of the kernel of $f$ by $\ker f$ while the codomain of the cokernel by $\coker f$. $\im f$ and $\coim f$ are similar. This will not make ambiguity since they are just be used as notations of objects in this situation.}

\begin{thm}[Weak Snake Lemma]\label{w.snake}
  The short exact sequences of morphisms $\alpha,\beta,\gamma$, which means a commutative diagram like below
\begin{displaymath}
      \xymatrix{
         0\ar[r] & A\ar[r]\ar[d]_{\alpha} & B\ar[r]\ar[d]_{\beta} & C\ar[r]\ar[d]_{\gamma} & 0 \\
         0\ar[r] & A'\ar[r] & B'\ar[r] & C'\ar[r] & 0
      }
\end{displaymath}
  induces an exact sequence relating kernels and cokernels
    \begin{equation*}
      0\To\ker\alpha\To\ker\beta\To\ker\gamma\markar{\delta}\coker\alpha\To\coker\beta\To\coker\gamma\To0
    \end{equation*}
\end{thm}

\begin{proof}
  We proceed in steps:
  \begin{enumerate}
    \item
    The morphisms between kers are clearly given by the efinition and the commutative of diagram. By chasing the diagram and use the definition, one can verify that the sequence of kers is exact. For the cokers, the argument is similar. Let's proof that $m_0\colon\ker\alpha\To\ker\beta$ is the kernel of $e_0\colon\ker\beta\To\ker\gamma$ as an example:
    \begin{displaymath}
      \xymatrix{
         T\ar@{-->}[rd]|{t}\ar@/^/[rrd]^{\tau}\ar@{-->}@/_/[rdd]_{\tau'}&&&\\
         &\ker\alpha\ar[r]^{m_0}\ar[d]_{i} & \ker\beta\ar[r]^-{e_0}\ar[d]_{j} & \ker\gamma\ar[d]_{k} \\
         &A\ar[r]^{m}\ar[d]_{\alpha} & B\ar[r]^{e}\ar[d]_{\beta} & C\ar[d]_{\gamma}\\
         &A'\ar[r]^{m'} & B'\ar[r]^{e'} & C'
      }
    \end{displaymath}

    By choosing any object $T$ with morphism $\tau$ such that $e_0\tau=0$, we have $ke_0\tau=0$, hence $ej\tau=0$. Since $m$ is the kernel of $e$, there exist a unique $\tau'\colon T\To A$ such that $j\tau=m\tau'$.

    We have
    \begin{equation*}
      m'\alpha\tau'=\beta m\tau'=\beta j\tau=0
    \end{equation*}

    Which implies $\alpha\tau'=0$ since $m'$ is monoic. Hence there exist a unique $t\colon T\To\ker\alpha$ such that $it=\tau'$, hence
     \begin{equation*}
       jm_0t=mit=m\tau'=j\tau
     \end{equation*}

    Which implies $m_0t=\tau$ since $j$ is monoic.
    Which shows that $m_0$ is a kernel of $e_0$ and the sequence is exact by lemma \ref{exactsq}.
    \item
      The morphism $\delta$ can be obtained by this way:

      Consider the following diagram
    \begin{displaymath}
      \xymatrix{
         A\ar@{-->}[r]^{s}\ar@{=}[d] & D\ar[r]^-{u}\ar[d]_{k'}      & \ker\gamma\ar[d]_{k} \\
         A\ar[r]^{m}\ar[d]_{\alpha}       & B\ar[r]^{e}\ar[d]_{\beta}   & C\ar[d]_{\gamma}        \\
         A'\ar[r]^{m'}\ar[d]_{c}              & B'\ar[r]^{e'}\ar[d]_{c'}     & C'\ar@{=}[d]                \\
         \coker\alpha\ar[r]_-{v}           & D'\ar@{-->}[r]_{t}            & C'
      }
    \end{displaymath}
    where the upper-right square is cartesian and the lower-left square is cocartesian. Since $e$ is epi, then $u$ is epi and $u=\coker s$. Similarly,  $v$ is a monoic and $v=\ker t$.

    By the commutativity, $tc'\beta k'=\gamma ku=0$, hence there exists a unique morphism $d\colon D\To \coker\alpha$ such that
    \begin{equation*}
      vd=c'\beta k'
    \end{equation*}

    Similarly, $vds=c'\beta k's=vc\alpha=0$. Then $ds=0$ because $v$ is monoic. Hence there exists a unique morphism
    \begin{equation*}
      \delta\colon\ker\gamma\To\coker\alpha
    \end{equation*}
    such that $\delta u=d$.
    Moreover,
    \begin{equation*}
      v\delta u=c'\beta k'
    \end{equation*}
    \item
      We now prove that the sequence
    \begin{equation*}
      \ker\beta\markar{e_0}\ker\gamma\markar{\delta}\coker\alpha\markar{m_1}\coker\beta
    \end{equation*}
     is exact, we prove the exactness at $\ker\gamma$ only, the case at $\coker\alpha$ is dual to this.

    Let $x\colon\ker\delta\To\ker\gamma$ be the kernel of $\delta$, and $t\colon\im e_0\To \ker\gamma$ be the image of $e_0$. We need to show the are equivalent. Since both $x$ and $t$ are monoic, it suffices to show that they can factor through each other.

    (\emph{1})
    First, we prove that $e_0\delta=0$.
    \begin{displaymath}
      \xymatrix{
         \ker\beta\ar@{-->}[rd]|{\tau}\ar@/^/[rrd]^{e_0}\ar@/_/[rdd]_{j}&&\\
         &D\ar[r]^{u}\ar[d]_{k'} & \ker\gamma\ar[d]^{k} \\
         & B\ar[r]_{e} & C
      }
    \end{displaymath}
    Since $ej=ke_0$ there exist a unique $\tau\colon\ker\beta\To D$ such that $u\tau=e_0$ and $k'\tau=j$. Therefore
    \begin{equation*}
      v\delta e_0=v\delta u\tau=c'\beta k'\tau=c'\beta j=0
    \end{equation*}
    Which implies $\delta e_0=0$ since $v$ is monoic.

    (\emph{2})
    Therefore $t$ can factor through $x$ by a unique morphism $\mu$. To get a factorization of $x$ through $t$, we pullback $t$ through $x$ and obtain a monomorphism $t'$:
    \begin{displaymath}
      \xymatrix{
         \cdot\ar[r]^{t'}\ar[d]_{x'}   & \ker\delta\ar[d]^{x}        \\
         \im e_0 \ar[r]_-{t}\ar@{-->}[ur]|{\mu}     & \coker\gamma
      }
    \end{displaymath}

    If $t'$ is an isomorphism, then it is clearly that $x=tx't'^{-1}$. To prove this, it suffices to show $t'$ is epi.

    (\emph{3})
    Consider the diagram in 2. Since $e/\beta k'=\gamma ku=0$, there exist a unique $f\colon D\To A'$ such that $m'f=\beta k'$. We have
    \begin{equation*}
      vcf=c'm'f=c'\beta k'=v\delta u
    \end{equation*}
    which implies $cf=\delta u$ because $v$ is monoic.

    Pullback $x$ through $u$:
    \begin{displaymath}
      \xymatrix{
         Y\ar[r]^-{y_1}\ar[d]_{y}        & \ker\delta\ar[d]^{x}       \\
         D\ar[r]_-{u}                          & \ker\gamma
      }
    \end{displaymath}

    We have
    \begin{equation*}
      cfy=\delta uy=\delta xy_1=0
    \end{equation*}
    Hence there exist a unique $f'\colon Y\To\im\alpha=\ker c$ such that
    \begin{equation*}
      \alpha_0f'=fy
    \end{equation*}

    We pullback through the $f'$ the epimorphism $\alpha_1$:
    \begin{displaymath}
      \xymatrix{
         Z\ar[r]^{z_1}\ar[d]_{z}        & Y\ar[d]^{f'}       \\
         A\ar[r]_-{\alpha_1}                          & \im\alpha
      }
    \end{displaymath}

    In order to see more clearly,  we put these morphisms in the following diagram which may not commutative at the upper-left square:
    \begin{displaymath}
      \xymatrix{
         Z\ar@{-->}[r]^{z_1}\ar@{-->}[d]_{z}
         & Y\ar@{-->}[r]^{y_1}\ar@{-->}[d]^{y}\ar@{-->}[ddl]_{f'}
         & \ker\delta\ar[d]^{x} \\
         A\ar[dr]^{m}\ar[d]_{\alpha_1}   & D\ar[r]^-{u}\ar[d]^{k'}\ar[ddl]^{f}      & \ker\gamma\ar[d]^{k} \\
         \im\alpha\ar[d]_{\alpha_0}        & B\ar[r]^{e}\ar[d]_{\beta}   & C\ar[d]_{\gamma}        \\
         A'\ar[r]^{m'}\ar[d]_{c}               & B'\ar[r]^{e'}                     & C'                \\
         \coker\alpha           &              &
      }
    \end{displaymath}

    (\emph{4})
    To measure the non-commutativity, define
    \begin{equation*}
      \Delta=k'yz_1-mz
    \end{equation*}
    Since
    \begin{equation*}
      \beta k' y z_1 = m' f y z_1 = m' \alpha_0 f' z_1 = m' \alpha_0 \alpha_1 z = \beta m z
    \end{equation*}
    We have $\beta\Delta=0$ and hence there exist a unique $\theta\colon Z\To\ker\beta$ such that $j\theta=\Delta$.

    On the other hand,
    \begin{equation*}
      e\Delta = ek'yz_1-emz = ek'yz_1-0 = kxy_1z_1
    \end{equation*}
    Hence $ej\theta=kxy_1z_1$.

    Consider the diagram below:
    \begin{displaymath}
      \xymatrix{
         Z\ar@{-->}[dr]_{\tau}\ar@/^/[drr]^{y_1z_1}\ar[dd]_{\theta} & & \\
                & \cdot\ar[r]_-{t'}\ar[d]_{x'}   & \ker\delta\ar[d]^{x}        \\
         \ker\beta\ar[r]^-{e_1}\ar[d]_{j}     & \im e_0\ar[r]^{t}        & \ker\gamma\ar[d]^{k}  \\
         B\ar[rr]_{e}           & & C
      }
    \end{displaymath}
    We have
    \begin{equation*}
      kte_1\theta = ke_0\theta=ej\theta=kxy_1z_1
    \end{equation*}
    which implies $te_1\theta=xy_1z_1$ because $k$ is monoic.

    By the universality of pullback, there exist a unique morphism $\tau$, such that $t'\tau=y_1z_1$ which is an epimorphism, hence so is $t'$.
  \end{enumerate}
\end{proof}

\begin{thm}[Snake Lemma]
  The following commutative diagram of exact sequences
\begin{displaymath}
      \xymatrix{
         & A\ar[r]^{f}\ar[d]_{\alpha} & B\ar[r]^{g}\ar[d]_{\beta} & C\ar[d]_{\gamma}\ar[r] & 0 \\
         0\ar[r] & A'\ar[r]_{f'} & B'\ar[r]_{g'} & C' &
      }
\end{displaymath}
  induces an exact sequence relating kernels and cokernels
    \begin{equation*}
      \ker\alpha\longrightarrow\ker\beta\longrightarrow\ker\gamma\markar{\delta}\coker\alpha\longrightarrow\coker\beta\longrightarrow\coker\gamma
    \end{equation*}
\end{thm}
\begin{proof}
First, we prove that the kernel of $\alpha$ is a pullback of the kernel of $\beta$ through $f$:

    \begin{displaymath}
      \xymatrix{
         T\ar@{-->}[rd]|{\tau}\ar@/^/[rrd]^{t'}\ar@/_/[rdd]_{t}&&\\
         & \ker\alpha\ar[r]^{f_0}\ar[d]_{i} & \ker\beta\ar[d]^{j} \\
         & A\ar[r]^{f}\ar[d]_{\alpha} & B\ar[d]^{\beta} \\
         & A'\ar[r]_{f'} & B'
      }
    \end{displaymath}

  Let $T$ be an arbitrary object with morphisms $t,t'$ such that $ft=jt'$. Then
  \begin{equation*}
    f'\alpha t=\beta ft=\beta jt'=0
  \end{equation*}
  which implies $\alpha t=0$ because $f'$ is monoic. Hence there exist a unique $\tau\colon T\To\ker\alpha$ such that $i\tau=t$.
  Then
  \begin{equation*}
    jf_0\tau=fi\tau=ft=jt'
  \end{equation*}
  which implies $f_0\tau=t'$ because $j$ is monoic. Hence $i$ is the pullback of $j$ through $f$.

  Let $f_m\colon K\To B$ be the kernel of $g$ and $g_e'\colon B'\To K'$ cokernel of $f'$. Then we get a commutative diagram
  \begin{displaymath}
      \xymatrix{
         A\ar[r]^{f_e}\ar[dr]_-{\alpha} & K\ar[r]^{f_m}\ar@{-->}[d]^{a} & B\ar[r]^{g}\ar[d]^{\beta} & C\ar[dr]^{\gamma}\ar[r]\ar@{-->}[d]_{c} & 0 \\
         0\ar[r] & A'\ar[r]_{f'} & B'\ar[r]_{g'_e} & K'\ar[r]_{g'_m} & C'
      }
  \end{displaymath}
  where $(f_e,f_m), (g'_e,g'_m)$ are the \emph{epi-mono factorizations}\footnote{Notice that the sequence is exact in $B$ means that $\im f=\ker g$ and, dually, $\coker f=\coim g$. Hence we get the epi-mono factorization.} of $f$ and $g$ respectively, and $a,c$ determined uniquely by the fact that $A'=\ker g'$ and $C=\coker f$.

  Then By the Weak Snake Lemma (\ref{w.snake}), the diagram induces an exact sequence:
    \begin{equation*}
      0\To\ker a\To\ker\beta\To\ker c\markar{\delta}\coker a\To\coker\beta\To\coker c\To0
    \end{equation*}

    Since $f_e$ is epi, by Lemma \ref{factor through}, the cokernel of $a$ and $\alpha$ coincide. Dually, the kernel of $c$ and $\gamma$ coincide.

    Hence it suffices to show that $\ker a\To\ker\beta$ is the image of $\ker\alpha\To\ker\beta$, and the cokernel case is dual to this.

    Consider the following diagram, where $e$ exists and make the diagram commutative since $af_ei=fi=0$.
  \begin{displaymath}
      \xymatrix{
         \ker\alpha\ar[r]^{e}\ar[d]_{i} & \ker a\ar[r]^{m}\ar[d]_{j} & \ker\beta\ar[d]_{k} \\
         A\ar[r]_{f_e} & K\ar[r]_{f_m} & B
      }
  \end{displaymath}
  By the universal property of kernel, $me$ is the morphism $\ker\alpha\To\ker\beta$ we discussed above.

  By the statements we proved at first, the right square and the outer rectangle are pullbacks, so is the left square by Proposition \ref{Two-pullbacks}.
  Therefore $e$ is a pullback of $f_e$ and hence epi. By the uniqueness of epi-mono factorization (\ref{epi-mono}), $m$ is the image of $\ker\alpha\To\ker\beta$.
\end{proof}

  \begin{rem}
    The morphism $\delta$ is called \termin{connection morphism}, while the long exact sequence is called \termin{snake sequence}.
  \end{rem}
  \begin{prop}
    The the snake sequence is \emph{natural} in the sense that if
    \begin{displaymath}
        \xymatrix@!0{
          &&& \cdot\ar[rr]\ar'[d]^{\alpha'}[dd] && \cdot\ar[rr]\ar'[d]^{\beta'}[dd] && \cdot\ar[rr]\ar'[d]^{\gamma'}[dd] && 0 \\
          && \cdot\ar[rr]\ar[dd]^(.3){\alpha}\ar[ru] && \cdot\ar[rr]\ar[dd]^(.3){\beta}\ar[ru] && \cdot\ar[rr]\ar[dd]^(.3){\gamma}\ar[ru] && 0 & \\
          & 0\ar'[r][rr] && \cdot\ar'[r][rr] && \cdot\ar'[r][rr] && \cdot && \\
          0\ar[rr] && \cdot\ar[rr]\ar[ru] && \cdot\ar[rr]\ar[ru] && \cdot\ar[ru] &&&
        }
    \end{displaymath}
    is a commutative diagram with exact rows, then the snake lemma can be applied twice, to the ``front'' and to the ``back'',
    yielding two snake sequences; these are related by a commutative diagram of the form:
    \begin{displaymath}
        \xymatrix@!0{
          &\ker\alpha'\ar[rr] && \ker\beta'\ar[rr] && \ker\gamma'\ar[rr]^-{\delta'} && \coker\alpha'\ar[rr] && \coker\beta'\ar[rr] && \coker\gamma' \\
          \ker\alpha\ar[rr]\ar[ur] && \ker\beta\ar[rr]\ar[ur] && \ker\gamma\ar[rr]^-{\delta}\ar[ur] && \coker\alpha\ar[rr]\ar[ur] && \coker\beta\ar[rr]\ar[ur] && \coker\gamma\ar[ur] &
        }
    \end{displaymath}
  \end{prop}
  \begin{proof}
    The commutativity of between kers (resp. cokers) are clear. It suffices to check commutativity at the connection morphisms.

    For $\alpha',\beta',\gamma'$, we have the same diagrams as $\alpha,\beta,\gamma$ in the construction of connection morphism.
    Without causing ambiguity, we can use the same labels and denote the morphisms from the ``front'' to the ``back'' by $\sigma$.

    Since taking pullback and pushout are functors (\ref{Fun.pull-push}), we have
    \begin{equation*}
      v\delta'\sigma u = v\delta'u\sigma = c'\beta'k' \sigma = \cdots = \sigma c'\beta k' = \sigma v\delta u = v\sigma \delta u
    \end{equation*}
    Hence $\delta'\sigma=\sigma\delta$.
  \end{proof}

  \begin{thm}[Short Five Lemma]
    Consider the following commutative diagram of exact sequences:
    \begin{displaymath}
        \xymatrix{
          & \cdot\ar[r]^{f}\ar[d]_{\alpha} & \cdot\ar[r]^{g}\ar[d]_{\beta} & \cdot\ar[d]_{\gamma}\ar[r] & 0 \\
          0\ar[r] & \cdot\ar[r]_{f'} & \cdot\ar[r]_{g'} & \cdot &
        }
    \end{displaymath}
    Then,  if $\alpha,\gamma$ are monoic (resp. epi), then so is $\beta$. Moreover, assume $f$ is monoic and $g'$ is epi, then any two of $\alpha,\beta,\gamma$ are isomorphisms implies so is the third.
  \end{thm}
  \begin{proof}
    Check the snake sequence, then the statements are obvious.
  \end{proof}

  \begin{thm}[Five Lemma]
    Consider the following exact sequences of five morphisms:
    \begin{displaymath}
        \xymatrix{
          \cdot\ar[r]^{a}\ar[d]_{f_1} & \cdot\ar[r]^{b}\ar[d]_{f_2} & \cdot\ar[r]^{c}\ar[d]_{f_3} & \cdot\ar[r]^{d}\ar[d]_{f_4} & \cdot\ar[d]_{f_5} \\
          \cdot\ar[r]_{a'} & \cdot\ar[r]_{b'} & \cdot\ar[r]_{c'} & \cdot\ar[r]_{d'} & \cdot
        }
    \end{displaymath}
    Then
    \begin{enumerate}[a)]
      \item If $f_1$ is epi and $f_2,f_4$ are monoic, then $f_3$ is monoic;
      \item If $f_5$ is monoic and $f_2,f_4$ are epi, then $f_3$ is epi;
      \item If $f_1,f_2,f_4,f_5$ are isomorphisms, then so is $f_3$.
    \end{enumerate}
  \end{thm}
  \begin{proof}
    First of all, notice that any exact sequence of morphisms can be factored into short exact sequences. For example, our exact sequence above
    can be factored as
    \begin{displaymath}
        \xymatrix@1{
          \cdot\ar[r]^-{a_e}\ar[d]_{f_1} & \im a \ar[r]^-{a_m}\ar[d]_{f_a}
          & \cdot\ar[r]^-{b_e}\ar[d]_{f_2} & \im b \ar[r]^-{b_m}\ar[d]_{f_b}
          & \cdot\ar[r]^-{c_e}\ar[d]_{f_3} & \im c \ar[r]^-{c_m}\ar[d]_{f_c}
          & \cdot\ar[r]^-{d_e}\ar[d]_{f_4} & \im d \ar[r]^-{d_m}\ar[d]_{f_d}
          & \cdot\ar[d]_{f_5} \\
          \cdot\ar[r]_-{a'_e} & \im a' \ar[r]_-{a'_m}
          & \cdot\ar[r]_-{b'_e} & \im b' \ar[r]_-{b'_m}
          & \cdot\ar[r]_-{c'_e} & \im c' \ar[r]_-{c'_m}
          & \cdot\ar[r]_-{d'_e} & \im d' \ar[r]_-{d'_m}
          & \cdot
        }
    \end{displaymath}

    We prove only \emph{a)} as a example:

    Since $a'_ef_1$ is epi, so is $f_a$. Consider the short exact sequence of $f_a,f_2,f_b$, by snake lemma, we have the snake sequence:
    \begin{equation*}
      \ker f_a\To\ker f_2\To\ker f_b\markar{\delta}\coker f_a\To\coker f_2\To\coker f_b
    \end{equation*}
    Since $f_2$ is monoic and $f_a$ is epi, the exact sequence at $\ker f_b$ become
    \begin{equation*}
      0\To\ker f_b\markar{\delta}0
    \end{equation*}
    Hence $f_b$ is monoic.

    Since $f_4c_m$ is monoic, so is $f_c$. By the \emph{short five lemma}, $f_3$ is monoic as desired.
  \end{proof}


\newpage\section{Appendix: Some Counterexamples}



\subsection*{About Category Theory}
  In this chapter, I try to introduce some concepts in ``category theory''.

  Nowadays, there are numerous books introducing category theory, like \cite{lawvere1997conceptual} and \cite{awodey2010category} which are easily readable books. Of course, The standard textbook is \cite{lane1998categories}.
  Unless otherwise specified, most of the contents in this chapter comes from them.
  One can also find them in a modern homological algebra textbook.

%\chapter{Category}
\end{appendices}

$\Cat$

\newpage
\backmatter

%Chapter=Table of Categories
\printglossaries

\newpage
%Chapter=Index
\phantomsection
\addcontentsline{toc}{chapter}{Index}
\printindex

\newpage
%Chapter=Bibliography
\phantomsection
\addcontentsline{toc}{chapter}{Bibliography}
\bibliographystyle{alpha}
\bibliography{bib}

\end{document}
