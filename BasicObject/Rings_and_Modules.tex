\chapter{Rings and Modules}
\minitoc
\newpage
\section{Basic definitions}
\subsection{Rings and commutative rings}
  A \termin{ring} is \emph{a monoid internal to abelian groups}. That means, a ring is an abelian group (whose operation is called \emph{addition}) equipped with a monoid structure (whose operation is called \emph{multiplication}).
  \begin{rem}
    Usually, we will use $+$ to denote the addition in a ring and $0$ to denote the \emph{additive zero}. Also, we will use juxtaposition of ring elements to denote the multiplication and $1$ to denote the \emph{multiplicative identity}.
  \end{rem}
  The homomorphisms of rings are the homomorphisms of abelian groups which preserve the monoid structures on them. The category of rings and their homomorphisms is denoted by $\Ring$.
  \begin{rem}
    For some reasons, many authors follow an alternative convention in which a ring is not required to have a multiplicative identity and therefore is merely \emph{a semigroup internal to abelian groups}. Such a structure will be called a \termin{rng} in this book.
  \end{rem}

  A ring is said to be \termin[commutative]{commutative ring} if its monoid structure is abelian. The category of commutative rings and their homomorphisms is denoted by $\CRing$.

  A subset of a ring is said to be a \termin{subring} if the inclusion map is a ring homomorphism.

\subsection{Modules and ideals}
  An abelian group $M$ equipped with a left monoid action of a ring $R$ on $M$ is called a \emph{left module over $R$}, or \emph{left} \termin[$R-$module]{module}.
  \begin{rem}
    Here the ring $R$ is called the \termin{ground ring}, and the action is called \emph{scalar multiplication} or simply \emph{multiplication} and is usually written by juxtaposition.
  \end{rem}
  The homomorphisms of left $R-$modules (usually called \emph{left} \termin[$R-$linear maps]{linear map}) are those homomorphisms of abelian groups preserving the actions of the ground ring $R$. The category of left $R-$modules and their homomorphisms is denoted by $_R\Mod$.

  Likewise, there are \emph{right $R-$modules} and \emph{right $R-$linear maps}, and the category of them is denoted by $\Mod_R$.
  \begin{rem}
    A right $R-$module can be viewed as a left module over the \termin{opposite ring} of $R$. For a given ring $R$, its opposite (denoted by $R^{\op}$) is another ring with the same elements and addition operation, but with the multiplication performed in the reverse order.

    When $R$ is commutative, then its opposite ring is isomorphic to itself. In this case, a left $R-$module is then isomorphic to a right $R-$module and will be simply called an \emph{$R-$module}. The category of $R-$modules and $R-$linear maps is denoted by $R\Mod$.
  \end{rem}

  An \termin[$R-S-$bimodule]{bimodule} is a module that is both a left $R-$module and a right $S-$module such that the two multiplications are compatible. An $R-R-$bimodule is also known as an \emph{$R-$bimodule}. The homomorphisms of bimodules are defined in the natural way. The category of $R-S-$bimodules and their homomorphisms is denoted by $_R\Mod_S$.

  Let $M$ be a left $R-$module and $N$ is a subgroup of $M$. Then $N$ is called a \termin{submodule} (or \emph{$R-$submodule}, to be more explicit) if it is invariant under the action of $R$. For a subset $E$ of $M$, the smallest submodule of $M$ containing $E$ is called the submodule \termin[generated]{generated submodule} by $E$ and is denoted by $RE$. For an element $x$ of $M$, we usually denote $R\{x\}$ simply by $Rx$, such kind of submodules are usually called \termin[principal]{principal submodule}.

  Likewise, one can define the submodule of a right module or bimodule. Note that the notations $RE$ introduced above should be written as $ER$ if we are talking about a right module.

  The ground ring $R$ itself is a left $R-$module under the left multiplication. A submodule of this module is called a \termin{left ideal} of $R$. \termin[Right ideals]{right ideal} and \termin[two-sided ideals]{two-sided ideal} are defined likewise. Note that a two-sided ideal is usually called an \termin{ideal} for short since it is more frequently used.

  For a subset $E$ of $R$, one can also define the left/right/two-sided ideal \termin[generated]{generated ideal} by $E$. We usually denote the two-sided ideal generated by $E$ as $\<E\>$.
  If $R$ is commutative, then both left ideals and right ideals are two-sided ideals, in this case, we also denote the ideal generated by $E$ as $RE$.
  Moreover, the left/right/two-sided ideals generated by one element is called \termin[principal]{principal submodule}.

  Given a ring $R$ and a two-sided ideal $I$ of $R$, the \termin{quotient ring} $R/I$ is the set of cosets of $I$ in $R$ equipped with a ring structure such that the canonical map $R\twoheadrightarrow R/I$ is a ring homomorphism.

\newpage
\section{Category of rings}
  The category of rings and their homomorphisms is denoted by $\Ring$.

\subsection{Completeness}

  If $0=1$ in a ring $R$, then $R$ has only one element, and is called the \termin{zero ring} (denoted by $\mathbf{0}$).
  The zero ring is obviously the terminal object in $\Ring$ but is NOT the initial object (in fact, it is $\mathbb{Z}$). Thus $\Ring$ can NOT be \emph{pre-additive}, a fortiori \emph{abelian}.

  Given a family of rings $\{R_i\}_{i\in I}$, the the cartesian product $\prod_{i\in I}R_i$ of them can be turned into a ring by defining the operations coordinate-wise. The resulting ring is called a \termin[direct product]{direct product of rings} of the rings $R_i$. The direct product of rings are just the \emph{product} in $\Ring$. That means, for every $i$ in $I$ we have a surjective ring homomorphism $p_i\colon \prod_{i\in I}R_i\to R_i$ which projects the product on the $i$th coordinate. The product $\prod_{i\in I}R_i$, together with the \emph{projections} $p_i$, has the following universal property:
  \begin{quote}
    if $T$ is any ring and $f_i\colon T\to R_i$ is a ring homomorphism for every $i$ in $I$, then there exists precisely one ring homomorphism $f\colon T\to \prod_{i\in I}R_i$ such that $p_i\circ f = f_i$ for every $i$ in  $I$.
  \end{quote}
  Or, represented this universal property by $\Hom$,
  \begin{equation*}
    \Hom(T,\prod_{i\in I}R_i) \cong \prod_{i\in I}\Hom(T,R_i).
  \end{equation*}
  \begin{rem}
    Some authors call the direct product of finitely many rings as the direct sum, but this should be avoided since it is NOT the \emph{coproduct} in $\Ring$. Indeed, the inclusion maps $R_i\hookrightarrow R$ are NOT ring homomorphisms since they do not map $1$ to the identity in $R$.
    The true coproducts are something like free products of groups, we will not discuss the details here.
  \end{rem}

  Given two parallel morphisms $f,g\colon R\tto S$, their \termin{equalizer} $\Ker(f,g)$ is the subring of $R$ consists of those elements sharing same images in $S$. That means, this subring $\Ker(f,g)$ together with the inclusion $\imath\colon \Ker(f,g) \hookrightarrow R$, has the following universal property:
  \begin{quote}
    if $T$ is any ring and $h\colon T\to R$ is a ring homomorphism such that $f\circ h = g\circ h$, then there exists precisely one ring homomorphism $u\colon T\to \Ker(f,g)$ such that $\imath\circ u = h$.
  \end{quote}
  Or, represented this universal property by $\Hom$,
  \begin{equation*}
    \Hom(T,\Ker(f,g)) \cong \left\{ h\in\Hom(T,R) \middle|\, f\circ h = g\circ h \right\}.
  \end{equation*}

  The \termin{coequalizer} of $f,g$ is the quotient ring $\Coker(f,g)=S/I$ of $S$, where $I$ is the ideal generated by the image of the abelian group homomorphism $f-g$. That means, this quotient ring $S/I$, together with the canonical map $p\colon S\hookrightarrow S/I$, has the following universal property:
  \begin{quote}
    if $T$ is any ring and $h\colon S\to T$ is a ring homomorphism such that $h\circ f = h\circ g$, then there exists precisely one ring homomorphism $u\colon S/I\to T$ such that $u\circ p = h$.
  \end{quote}
  Or, represented this universal property by $\Hom$,
  \begin{equation*}
    \Hom(\Coker(f,g),T) \cong \left\{ h\in\Hom(S,T) \middle|\, h\circ f = h\circ g \right\}.
  \end{equation*}

  Therefore, $\Ring$ has products and equalizers thus, by the general theorem in category theory, it is \emph{complete}. That means, every categorical limit exists in $\Ring$. Since we can also define coproducts and coequalizers, $\Ring$ is actually also \emph{cocomplete}, which means every categorical colimit exists in $\Ring$.

  As examples, we introduce a kind of important limits we will use later.
    \begin{displaymath}
      \xymatrix{
         R\ar[r]^{f'}\ar[d]_{g'}&B\ar[d]^{g}\\
         A\ar[r]^{f}&C
      }
    \end{displaymath}

  Let's look at the above square, where $A,B,C$ are given objects and $f,g$ are given morphisms. Among all objects $R$ equipped with two morphisms $f',g'$ making the square commutative, there exists a unique terminal one, it is the \termin{fibre product} of $A$ and $B$ over $C$, and is denoted by $A\times_CB$. In layman's word, $A\times_CB$, together with the two morphisms $f',g'$, has the following universal property:
  \begin{quote}
    if $T$ is an object and $f''\colon T\to A$ and $g''\colon T\to B$ are two morphisms such that $f\circ g'' = g\circ f''$, then there exists precisely one morphism $u\colon T\to A\times_CB$ such that $f'\circ u = f''$ and $g'\circ u =g''$.
  \end{quote}
  Or, represented this universal property by $\Hom$,
  \begin{equation*}
    \Hom(T,A\times_CB) \cong \left\{ (f'',g'')\in\Hom(T,B)\times\Hom(T,A) \middle|\, f\circ g'' = g\circ f'' \right\}.
  \end{equation*}

  The fibre product $A\times_CB$ in $\Ring$ can be defined as the subring of $A\times B$ containing those elements $(a,b)$ satisfying $f(a)=g(b)$, $f',g'$ are just the two projections of $A\times B$ restricted in $A\times_CB$.
  \begin{rem}
    The previous square is called a \termin{cartesian diagram} when the triple $(R,f',g')$ is taken to make be the fibre product. In this case, we say $g'$ is the \termin{pullback} of $g$ through $f$ and $f'$ is the \termin{pullback} of $f$ through $g$.
  \end{rem}

  Dually, there are \emph{cofibre products}, \emph{cocartesian diagrams} and \emph{pushouts}. Although they are not easy to describe in detail, their existence in $\Ring$ just follows the cocompleteness.


\subsection{Monomorphisms and kernel pairs}
  The monomorphisms in $\Ring$ are precisely the injective homomorphisms. One may remember that a homomorphism $f$ is injective if and only if its \emph{kernel} $\ker{f}$ is trivial, i.e. equal to $0$.

  However, nonzero kernel is not a ring but an ideal, thus does not exist in $\Ring$. To characterize this property in a category theoretic way, we introduce a special kind of fibre products as the analogy of kernels in $\Ring$

  In a general category, the \termin{kernel pair} of a morphism $f$ is the fibre product of $f$ with itself. In layman's word, the kernel pair $A\times_BA$, together with the two morphisms $g,h\colon A\times_BA\tto A$, has the following universal property:
  \begin{quote}
    if $T$ is an object and $g',h'\colon T\tto B$ is a pair of two parallel morphisms such that $f\circ g' = f\circ h'$, then there exists precisely one morphism $u\colon T\to A\times_BA$ such that $g\circ u = g'$ and $h\circ u = h'$.
  \end{quote}
  \begin{rem}
    It is obvious that the kernel pair morphisms must be \emph{split epi}, which means there exists a morphism $u$ such that $g\circ u = h\circ u = 1_A$
  \end{rem}

  In a category with zero morphisms and kernels, such as the category $\Ab$ of abelian groups, every kernel pair $g,h$ admits a kernel morphism $g-h$ while every kernel morphism $k$ admits a kernel pair $k,0$.

  In $\Ring$, the kernel pair of a homomorphism $f\colon A\to B$ is taken to be
  \begin{equation*}
    A\times_BA := \left\{(a,a')\in A\times A \middle|\, f(a)=f(a')\right\}
  \end{equation*}
  with the two obvious projections $p_1,p_2$. The different of the two projections, i.e. $p_1-p_2$, is not a ring homomorphism. However, the set-theoretic image of this map is nothing but the kernel of $f$.

  We have a general proposition about the relationship between kernel pairs and monomorphisms.
  \begin{prop}\label{prop:kernel pair and monic}
    A morphism $f\colon A\to B$ is a monomorphism if and only if its kernel pair is trivial, i.e. isomorphic to $A$ together with two identity morphisms.
  \end{prop}
  \begin{proof}
    If $f\colon A\to B$ is a monomorphism, then for any parallel morphisms $g,h$, $f\circ g = f\circ h$ implies $g=h$, thus $A$ together with two identity morphisms satisfies the universal property of kernel pair. Conversely, If the kernel pair of $f$ is trivial, then for any parallel morphisms $g,h$ such that $f\circ g = f\circ h$, there exists precisely one morphism $u\colon T\to A\times_BA$ such that $1_A\circ u = g$ and $1_A\circ u = h$, thus $g=h$ and $f$ is a monomorpism.
  \end{proof}

  A mentionable property of kernel pair is:
  \begin{prop}\label{prop:kernel pair}
    If a kernel pair has a coequalizer, then it is a kernel pair of its coequalizer. Conversely, if a coequalizer has a kernel pair, then it is a coequalizer of its kernel pair.
  \end{prop}
  \begin{proof}
    Consider the following diagram
    \begin{displaymath}
      \xymatrix{
         &X\ar@<-0.5ex>[d]_{x}\ar@<0.5ex>[d]^{y}\ar[dl]_{z}&\\
         P\ar@<0.5ex>[r]^{\alpha}\ar@<-0.5ex>[r]_{\beta}&A\ar[r]^{f}\ar[d]_{g}&B\ar[dl]^{h}\\
         &C&
      }
    \end{displaymath}
    where, $\alpha,\beta$ is the kernel pair of $g$ and $f=\Coker(\alpha,\beta)$. Since $g\circ\alpha=g\circ\beta$, we get a unique factor $h$ of $g$ through $f=\Coker(\alpha,\beta)$.
    We need to show that $\alpha,\beta$ is the kernel pair of $f$.

    To do this, consider two parallel morphisms $x,y$ such that $f\circ x=f\circ y$. Then $g\circ x=h\circ f\circ x=h\circ f\circ y=g\circ y$. Thus there exists a unique morphism $z$ such that $\alpha\circ z=x, \beta\circ z=y$.

    Consider the following diagram
    \begin{displaymath}
      \xymatrix{
         &X\ar@<-0.5ex>[d]_{x}\ar@<0.5ex>[d]^{y}\ar[dl]_{z}&\\
         P\ar@<0.5ex>[r]^{\alpha}\ar@<-0.5ex>[r]_{\beta}&A\ar[r]^{f}\ar[d]_{g}&B\ar[dl]^{h}\\
         &C&
      }
    \end{displaymath}
    where, $f=\Coker(x,y)$ and $\alpha,\beta$ is the kernel pair of $f$. Hence there exists a unique morphism $z$ such that $\alpha\circ z=x, \beta\circ z=y$.
    We need to show that $f=\Coker(\alpha,\beta)$.

    To do this, consider an arbitrary morphism $g$ such that $g\circ\alpha=g\circ\beta$. Then $g\circ x=g\circ\alpha\circ z=g\circ\beta\circ z=g\circ y$. Thus we get a unique factor $h$ of $g$ through $f=\Coker(x,y)$.
  \end{proof}

\subsection{Regular images}
  The coequalizer of the kernel pair of a morphism $f$ is called the \termin{regular coimage} of $f$ and is denoted by $\Coim{f}$. In $\Ring$, it is not difficult to see that the regular coimage of a homomorphism $f\colon A\to B$ can be taken to be the quotient ring $A/\ker{f}$.

  Dually, we have \termin{cokernel pairs} for which the dual version of Proposition \ref{prop:kernel pair} and \ref{prop:kernel pair and monic} hold (Note that a cokernel pair is trivial means it is isomorphic to the codomain together with two identity morphisms). The equalizer of the cokernel pair of a morphism $f$ is called the \termin{regular image} of $f$ and is denoted by $\Image{f}$.

  One can see that, in a \emph{pre-abelian category} such as the category $\Ab$ of abelian groups, the regular coimage is just the cokernel of the kernel while the regular image is just the kernel of the cokernel.

  We know that in an abelian category, the regular coimage and regular image coincide and provide an \emph{Epi-mono factorization} through it. That means every morphism in an abelian category can be factored as an epimorphism followed by a monomorphism. Morevoer, such a factorization is unique up to unique isomorphism.

  In general, the regular image and regular coimage of a morpdism admit a ternary factorization from the universal properties of them.
  \begin{thm}[Ternary factorization]\label{thm:ternary factorization}
    If a morphism $f\colon A\to B$ admits a regular image and a regular coimage, then there exists a unique morphism
    \begin{equation*}
      u\colon\Coim{f}\to\Image{f}
    \end{equation*}
    such that the morphism $f$ itself can be factored into the following three morphisms:
    \begin{equation*}
      A \To \Coim{f} \markar{u} \Image{f} \To B.
    \end{equation*}

    Moreover, when $u$ is an isomorphism, the factorization become an Epi-mono factorization which is unique up to unique isomorphism.
  \end{thm}
  \begin{proof}
    By the universal property of $\Coim{f}$, there exists a unique morphism $\Coim{f}\to B$ such that $f$ can be factored as $A\to\Coim{f}\to B$. By the universal property of $\Image{f}$, there exists a unique morphism $\Coim{f}\to \Image{f}$ such that $\Coim{f}\to B$ can be factored as $\Coim{f}\to\Image{f}\to B$. Then this morphism is the desired unique morphism $u$.

    If $u$ is an isomorphism, let's denote the above factorization by $f = i\circ p$. Consider another factorization of $f = m\circ e$ where $e$ is an epimorphism and $m$ is a monomorphism. Then by the universal property of $\Coim{f}$ and $\Image{f}$, there exist a unique morphism $g$ such that $e = g\circ p$ and a unique morphism $h$ such that $m = i\circ h$. Therefore
    \begin{equation*}
      i\circ h\circ g\circ p = m\circ e = f = i\circ p.
    \end{equation*}
    Thus $h\circ g = 1$.
    Moreover,
    \begin{equation*}
      m\circ g\circ h\circ e = i\circ h\circ g\circ h\circ g\circ p = i\circ p = f = m\circ e.
    \end{equation*}
    Thus $g\circ h = 1$. Therefore the factorization $f = m\circ e$ is isomorphic to $f = i\circ p$ through the unique isomorphism $h$.
  \end{proof}
  If this $u$ is an isomorphism, then we say that $f$ is a \termin{strict morphism}.  For example, in an abelian category, every morphism is strict.

  In $\Ring$, the regular image of a homomorphism $f\colon A\to B$ can be taken to be the \emph{joint equalizer} of all pairs of homomorphisms $g,h\colon B\tto C$ satisfying $g\circ f=h\circ f$, i.e. the following subring of $B$:
  \begin{equation*}
    \left\{b\in B \middle|\, \parbox{0.6\textwidth}{for any pair of homomorphisms $g,h\colon B\tto C$ satisfying $g\circ f=h\circ f$, we have $g(b)=h(b)$} \right\}
  \end{equation*}

  However, this is usually not the set-theoretic image of $f$. For instance, the the set-theoretic image of the inclusion $\mathbb{Z}\hookrightarrow\mathbb{Q}$ is $\mathbb{Z}$ while the regular image is the whole $\mathbb{Q}$.

  The problem comes from the trouble that although every surjective ring homomorphisms are epimorphisms in $\Ring$, the converse is NOT true. The inclusion $\mathbb{Z}\hookrightarrow\mathbb{Q}$ is a counterexample. To see this, note that any ring homomorphism on $\mathbb{Q}$ is determined entirely by its action on $\mathbb{Z}$ thus the inclusion is a non-surjective epimorphism.

  As a result, $\mathbb{Z}\hookrightarrow\mathbb{Q}$ fails to be strict. Since it is both a monomorphism and an epimorphism, its kernel pair and cokernel pair are trivial. Then its regular coimage and regular image are isomorphic to the domain and codomain respectively thus not isomorphic to each other.


\subsection{Epimorphisms}
  To characterize the surjective homomorphisms in a categorical way, we will introduce some special kinds of epimorphisms here. We should define them in terms of category theory.

  Except the whole epimorphisms, the most familiar class is the split epimorphisms. Recall that a \termin{split epimorphism} is a morphism $f\colon A\to B$ having a \emph{section}. That means, there exists a morphism $g\colon B\to A$ such that $f\circ g = 1_B$.

  However, being split is too strong for surjective ring homomorphisms, thus in $\Ring$ the class of surjective homomorphisms are between epimorphisms and split epimorphisms. So we need to found some classes of epimorphisms between them.

\subsubsection{Extremal epimorphisms and strong epimorphisms}

  We know that, being an \termin{epi-monomorphism}, which means both an epimorphism and a monomorphism, is not sufficient to implies being an isomorphism. The inclusion $\mathbb{Z}\hookrightarrow\mathbb{Q}$ is such an example in $\Ring$.

  This leads us to define the notion of extremal epimorphisms. An epimorphism $f\colon A\to B$ is said to be an \termin{extremal epimorphism} if it cannot be factored through a nontrivial \emph{subobject} of $B$. That means, if $f=m\circ g$ with $m$ a monomorphism, then $m$ is an isomorphism.

  For extremal epimorphisms, we have the desired property:
  \begin{prop}\label{prop:extremal+mono=iso}
    A morphism which is both a monomorphism and an extremal epimorphism is an isomorphism.
  \end{prop}
  \begin{proof}
    Let $f\colon A\to B$ be both a monomorphism and an extremal epimorphism. Consider the factorization $f = f\circ 1_A$, then since $f$ is also a monomorphism, it is an isomorphism.
  \end{proof}

  Note that in the factorization $f = m\circ g$ where $f$ is an epimorphism and $m$ is a monomorphism, we can always say that $m$ is also an epimorphism and thus an epi-monomorphism. But when we assume $f$ to be extremal, then it become an isomorphism. It seems that making $f$ to be extremal will also make $m$ extremal, and this is true.
  \begin{prop}
    If a composite $g\circ f$ is an extremal epimorphism, then so is $g$.
  \end{prop}
  \begin{proof}
    Assume $g\circ f$ is an extremal epimorphism and $g = m\circ h$ where $m$ is a monomorphism, then $g\circ f = m\circ (h\circ f)$. Thus $m$ is an isomorphism as desired.
  \end{proof}
  However, the composite of two extremal epimorphisms may not be extremal again. The problem comes from that when consider a commutative diagram like below
        \begin{displaymath}
          \xymatrix{
            A\ar[r]^{f}\ar[dr]&B\ar[r]^{g}&C\\
            &D\ar[ur]_{m}&
          }
        \end{displaymath}
  where $f$ and $g$ are extremal epimorphism and $m$ is a monomorphism, we need there exists a morphism $B\to D$ to reduce the case from the factorization of $g\circ f$ to the factorization of $g$.

  This leads us to define the notion of strong epimorphisms. An epimorphism $f\colon A\to B$ is said to be an \termin{strong epimorphism} if it is \emph{left orthogonal} to any monomorphism. That means, in any commutative square
        \begin{displaymath}
          \xymatrix{
            A\ar[r]^{f}\ar[d]&B\ar[d]\\
            C\ar[r]^{m}&D
          }
        \end{displaymath}
  where $m$ is an arbitrary monomorphism, there exists a unique morphism $B\to C$ making both triangles commute:
        \begin{displaymath}
          \xymatrix{
            A\ar[r]^{f}\ar[d]&B\ar[d]\ar@{-->}[dl]\\
            C\ar[r]^{m}&D
          }
        \end{displaymath}

  For strong epimorphisms, composition is not a problem.
  \begin{prop}
    The composite of two strong epimorphisms is also a strong epimorphism. Conversely, if a composite $g\circ f$ is a strong epimorphism, then so is $g$.
  \end{prop}
  \begin{proof}
    Consider the following diagram
        \begin{displaymath}
          \xymatrix{
            A\ar[r]^{f}\ar[d]&B\ar[r]^{g}\ar@{-->}[dl]|-{h}&C\ar[d]\ar@{-->}[dll]|-{u}\\
            D\ar[rr]^{m}&&E
          }
        \end{displaymath}
    where $f,g$ are two strong epimorphisms and $m$ is an arbitrary monomorphism. In square $ABED$, since $f$ is strong epi, there exists a unique morphism $h$ making both triangles commute. In square $BCED$, since $g$ is strong epi, there exists a unique morphism $u$ making  both triangles commute. Then one can verify that $u$ is the desired unique morphism.

    Conversely, assume $g\circ f$ is a strong epimorphism and consider the following diagram
        \begin{displaymath}
          \xymatrix{
            A\ar[r]^{f}\ar@{-->}[dr]&B\ar[d]\ar[r]^{g}&C\ar[d]\ar@{-->}[dl]|-{h}\\
            &D\ar[r]^{m}&E
          }
        \end{displaymath}
    where $m$ is an arbitrary monomorphism. Then in square $ACED$, there exists a unique morphism $h$ making both triangles commute. This $h$ is the desired unique morphism in square $BCED$.
  \end{proof}

  Moreover, we have
  \begin{prop}
    Every strong epimorphism is an extremal epimorphism.
  \end{prop}
  \begin{proof}
    Consider the following commutative diagram
        \begin{displaymath}
          \xymatrix{
            A\ar[r]^{f}\ar[d]&B\ar@{=}[d]\ar@{-->}[dl]\\
            C\ar[r]^{m}&B
          }
        \end{displaymath}
    where $f$ is a strong epimorphism and $m$ is a monomorphism. Then we need to show that $m$ is an isomorphism. This is obvious since it is both a monomorphism and a split epimorphism.
  \end{proof}
  Conversely, the previous discussion shows that an extremal epimorphism may not be strong. However, we have
  \begin{prop}\label{prop:extremal=strong}
    In a category with pullbacks, every extremal epimorphism is strong.
  \end{prop}
  \begin{proof}
    Consider the following commutative diagram
        \begin{displaymath}
          \xymatrix{
            A\ar[r]^{f}\ar[d]_{g}&B\ar[d]_{h}\ar@{-->}[dl]|{u}\\
            C\ar[r]^{m}&D
          }
        \end{displaymath}
    where $f$ is an extremal epimorphism and $m$ is a monomorphism, we need to show that there exists a unique morphism $u\colon B\to C$ such that $u\circ f = g$ and $m\circ u = h$.

    Taking the pullback of $m$ and $h$, then we get the following commutative diagram
        \begin{displaymath}
          \xymatrix{
            A\ar[rr]^{f}\ar[dd]_{g}\ar@{-->}[dr]|{v}&&B\ar[dd]_{h}\ar@/^1pc/@{-->}[ddll]|{u'}\\
            &E\ar@/_/[dl]|{h'}\ar@/^/[ur]|{m'}&\\
            C\ar[rr]^{m}&&D
          }
        \end{displaymath}
    where $v$ is the unique morphism satisfying $h'\circ v=g$ and $m'\circ v=f$. Since $m'$ is a pullback of a monomorphism, it must be a monomorphism again, thus an isomorphism since $f$ is an extremal epimorphism. Then we get a morphism $u:=h'\circ{m'}^{-1}$ and
    \begin{align*}
      u\circ f &= h'\circ{m'}^{-1}\circ f = h'\circ v = g,\\
      m\circ u &= m\circ h'\circ{m'}^{-1} = h.
    \end{align*}
    Thus $u$ is the desired morphism. To see the uniqueness, assume $u'$ is another morphism such that $u'\circ f = g$ and $m\circ u' = h$. Then since $m$ is a monomorphism, $m\circ u' = h = m\circ u$ implies $u'=u$.
  \end{proof}

  Therefore, in a category with pullbacks such as $\Ring$, extremal epimorphisms and strong epimorphisms coincide. One may expect that they also coincide with surjective homomorphisms. Let's leave this thought aside and introduce another family of epimorphisms first.

\subsubsection{Regular epimorphisms}
  Recall that in the ternary factorization of a morphism $f\colon A\to B$ (cf.\ref{thm:ternary factorization}), the regular coimage morphism must be an epimorphism. But not every epimorphism can serves as a regular coimage morphism. For example, one can see that in $\Ring$, the regular coimage morphisms are quotient maps and thus surjective homomorphisms while we already know there exists epimorphisms which are not surjective.

  When a morphism can serve as a regular coimage morphism, it should be called a \termin{regular epimorphism}. However, for some reasons, this name belongs to a bit more general notion, that is a morphism which can serve as a \emph{coequalizer} of some parallel pair of morphisms. While a regular coimage morphism is called an \termin{effective epimorphism}.

  One can see that (by Proportion \ref{prop:kernel pair}), in a category with pullbacks, regular epimorphism are just effective epimorphism.

  Let's consider in such a category, among all regular epimorphisms, there are a special kind of them worth to mention, they are the \emph{universal regular epimorphisms}. A regular epimorphism is said to be \termin[universal]{universal regular epimorphism} if its every pullback is again a regular epimorphism.
  A universal regular epimorphism is also called a \termin{descent morphism}. This name comes from the descent theory in algebraic geometry and has an alternative definition (cf. \cite{Vistoli}).


  It is not difficult to see that universal regular epimorphisms are \emph{stable under pullbacks}. That means every pullback of a universal regular epimorphism is again a universal regular epimorphism. This is because that a pullback of a pullback of a morphism is again a pullback of the original morphism.

  The universal regular epimorphisms are also closed under compositions. That means a composite of two universal regular epimorphisms is again a universal regular epimorphism. To see this, we need a lemma
  \begin{lem}\label{lem:fibre morphism of descent morphism}
    Let $f\colon A\to B$ be a universal regular epimorphism and $g\colon B\to C$ be an arbitrary morphism. Then the \emph{fibre morphism}
    \begin{equation*}
      f\times_Cf\colon A\times_CA\To B\times_CB
    \end{equation*}
    uniquely exists and is an epimorphism.
  \end{lem}
  \begin{proof}
    Taking the kernel pair of $g$ we get the lower right cartesian square. The three other cartesian squares exist and the sides of the upper left square are all epimorphisms since $f$ is a universal regular epimorphism.
    \begin{displaymath}
      \xymatrix{
         A\times_CA\ar[r]\ar[d] & B\times_CA\ar[r]\ar[d] & A\ar@{->>}[d]_{f} \\
         A\times_CB\ar[r]\ar[d] & B\times_CB\ar[r]\ar[d] & B\ar[d]_{g} \\
         A\ar@{->>}[r]^{f} & B\ar[r]^{g} & C
      }
    \end{displaymath}
    Then $f\times_Cf$ is just the composite of two epimorphisms, thus again an epimorphism.
  \end{proof}
  \begin{prop}
    A composite of two universal regular epimorphisms is again a universal regular epimorphism.
  \end{prop}
  \begin{proof}
    Consider the following commutative diagram
    \begin{displaymath}
      \xymatrix{
         &A\times_BA\ar@<0.5ex>[d]^-{}\ar@<-0.5ex>[d]_-{}\ar[dl]_{h}&\\
         A\times_CA\ar@<0.5ex>[r]^-{}\ar@<-0.5ex>[r]_-{}\ar[d]_{f\times_Cf}
         &A\ar[r]^-{g\circ f}\ar[d]_-{f}&C\ar@{=}[d]\\
         B\times_CB\ar@<0.5ex>[r]^-{}\ar@<-0.5ex>[r]_-{}&B\ar[r]^-{g}&C
      }
    \end{displaymath}
    where $f\times_Cf$ is the fibre morphism and $h$ is the unique morphism making the upper left triangles commute. We only need to show that $g\circ f$ is the coequalizer of its kernel pair $A\times_CA\tto A$.

    Let $t\colon A\to T$ be an arbitrary morphism equalizing $A\times_CA\tto A$. We need to show that there exists a unique morphism $u\colon C\to T$ such that $u\circ (g\circ f) = t$.

    By composted by $h$, it equalize $A\times_BA\tto A$. Thus there exists a unique morphism $v\colon B\to T$ such that $v\circ f = t$. Then $v$ equalize the composites $A\times_CA\tto A\markar{f} B$ and thus the composites $A\times_CA\to B\times_CB\tto B$ by the lower left commutative squares. Since the fibre morphism $f\times_Cf$ is an epimorphism, we get that $v$ equalize $B\times_CB\tto B$. Thus there exists a unique morphism $u\colon C\to T$ such that $u\circ g = v$. Then this $u$ is the desired unique morphism.
  \end{proof}

  What's the relationship between regular epimorphisms and strong epimorphisms? It is easy to verify that
  \begin{prop}
    Every regular epimorphism is a strong epimorphism.
  \end{prop}
  \begin{proof}
    Consider the following commutative diagram
        \begin{displaymath}
          \xymatrix{
            A\ar[r]^{f}\ar[d]_{g}&B\ar[d]_{h}\ar@{-->}[dl]|{u}\\
            C\ar[r]^{m}&D
          }
        \end{displaymath}
    where $f$ is a regular epimorphism and $m$ is a monomorphism. $f$ is regular thus a coequalizer of some parallel morphisms, say $x,y$. Then $m\circ g = h\circ f$ equalize $x,y$. Since $m$ is a monomorphism, we get that $g$ equalize $x,y$. Then the desired unique morphism $u\colon B\to C$ comes from the universal property of $f$ as the coequalizer of $x,y$.
  \end{proof}

  Since $\Ring$ is complete, the epimorphisms in $\Ring$ can be classified into five classes:
  \begin{enumerate}
    \item Epimorphisms;
    \item Extremal/strong epimorphisms;
    \item Regular/effective epimorphisms;
    \item Descent morphisms;
    \item Split epimorphisms.
  \end{enumerate}

  We have already know the the class of surjective homomorphisms is between the whole epimorphisms and split epimorphisms. We will prove in the next subsection that the rest three classes are all the same and equivalent to surjective homomorphisms.

\subsection{Regular category and the image factorization}
  The main step in the proof of the our assertion at the end of previous subsection is to show $\Ring$ is a \emph{regular category}.

\subsubsection{Regular category}
  A \termin{regular category} is a \emph{finitely complete} category in which
  \begin{enumerate}
    \item every kernel pair has a coequalizer;
    \item every regular epimorphism is universal.
  \end{enumerate}

  A very important property of regular category is the existence and uniqueness of \emph{regularEpi-mono factorizations}.
  \begin{thm}\label{thm:regular factorization}
    In a regular category, every morphism can be factored as a regular epimorphism followed by a monomorphism. This factorization is unique up to unique isomorphism.
  \end{thm}
  \begin{proof}
    Let $f\colon A\to B$ be a morphism in a regular category. Let $g,h\colon K\tto A$ be its kernel pair and $e\colon A\to\coim{f}$ be its coimage morphism. Then there exists a unique morphism $i\colon\coim{f}\to B$ such that
    \begin{equation*}
      f = (A \markar{e} \coim{f} \markar{i} B).
    \end{equation*}
    Where $e$ is of course a regular epimorphism. We only need to show that $i$ is a monomoprhism and the uniqueness.

    Consider the kernel pair $g',h'\colon K'\tto\coim{f}$ of $i$.
    \begin{displaymath}
      \xymatrix{
         K\ar@<0.5ex>[r]^-{g}\ar@<-0.5ex>[r]_-{h}\ar@{-->}[d]_{e\times_Be}&A\ar[r]^{f}\ar[d]_{e}&B\\
         K'\ar@<0.5ex>[r]^-{g'}\ar@<-0.5ex>[r]_-{h'}&\coim{f}\ar[ru]_{i}&
      }
    \end{displaymath}
    Then by Lemma \ref{lem:fibre morphism of descent morphism}, the fibre morphism $e\times_Be\colon K\to K'$ exists and is an epimorphism. Then
    \begin{equation*}
      g'\circ (e\times_Be) = e\circ g = e\circ h = h'\circ (e\times_Be).
    \end{equation*}
    Thus $g'=h'$ and $i$ is monic.

    To see the uniqueness, assume that there are another factorization
    \begin{equation*}
      f = (A \markar{e'} E \markar{i'} B),
    \end{equation*}
    where $e'$ is a regular epimorphism and $i'$ is a monomorphism. Then
    \begin{equation*}
      i'\circ e'\circ g = f\circ g = f\circ h = i'\circ e'\circ h,
    \end{equation*}
    and thus $e'\circ g = e'\circ h$. By the universal property of coequalizer, there exists a unique morphism $u\colon\coim{f}\to E$ such that $u\circ e = e'$. Since both $e$ and $e'$ are extremal epimorphisms, so is $u$. Note that
    \begin{equation*}
      i'\circ u\circ e = i'\circ e' = i\circ e,
    \end{equation*}
    thus $i'\circ u = i$. Then $u$ is also monic since both $i$ and $i'$ are. Therefore, by Proposition \ref{prop:extremal+mono=iso}, $u$ is an isomorphism as desired.
  \end{proof}

  The axioms of regular category requires every regular morphism being universal. Indeed, we have more:
  \begin{cor}\label{cor:regularCat}
    In a regular category, every extremal epimorphism is a universal regular epimorphism.
  \end{cor}
  \begin{proof}
    Let $f\colon A\to B$ be an extremal epimorphism in a regular category. By Theorem \ref{thm:regular factorization}, it can be written as $f=m\circ e$ where $e\colon A\to\coim{f}$ is the coimage morphism of $f$ and $m$ is a monomorphism. Since $f$ is an extremal epimorphism, $m$ must be an isomorphism. Thus $f$ is a regular epimorphism hence universal.
  \end{proof}

\subsubsection{Image factorizations}
  Know, we come to the main theorem of this subsection
  \begin{thm}\label{thm:RingEpi}
    Let $f$ be a ring homomorphism in $\Ring$, the followings are equivalent:
    \begin{enumerate}
      \item $f$ is surjective;
      \item $f$ is an extremal epimorphism;
      \item $f$ is a strong epimorphism;
      \item $f$ is a regular epimorphism;
      \item $f$ is an effective epimorphism;
      \item $f$ is a universal regular epimorphism.
    \end{enumerate}

    Moreover, $\Ring$ is a regular category and thus every homomorphism can be factored as a regular epimorphism followed by a monomorphism. This factorization is unique up to unique isomorphism..
  \end{thm}
  \begin{proof}
    We only need to show that 1) every regular epimorphism in $\Ring$ is surjective, 2) every surjective homomorphism is a regular epimorphism and 3) every surjective homomorphism is stable under pullback.

    1) is obvious since a coequalizer is the canonical map to a quotient ring by the discussion in previous subsection.

    2) is also not difficult since one can verify that every surjective homomorphism is the regular coimage of itself.

    To see 3), one only need to notice that in a cartesian square
    \begin{displaymath}
      \xymatrix{
         A\times_CB\ar[r]\ar[d]&B\ar[d]\\
         A\ar[r]&C
      }
    \end{displaymath}
    whenever two elements in $A$ and $B$ respectively share the same image in $C$, then they have the same preimage in $A\times_CB$.
  \end{proof}

  From theorem \ref{thm:RingEpi}, one can see that, for any ring homomorphism $f\colon A\to B$, the regular coimage is the smallest subring of $B$ through which $f$ factors. In other words, if we write $f$ as $i\circ p$ where $p\colon A\to\Coim{f}$ is the regular coimage morphism, then $i\colon\Coim{f}\to B$ is a monomorphism having the following universal property:
  \begin{enumerate}
    \item There exists a unique homomorphism $g$ such that $i\circ g = f$.
    \item If $m\colon C\to B$ is a monomorphism and $h\colon A\to C$ is a homomorphism such that $m\circ h = f$, then there exists a unique homomorphism $u\colon \im{f}\to C$ such that $m\circ u = i$.
  \end{enumerate}
  In general, a \emph{subobject} of the \emph{codomain} of a morphism having the above universal property is called the \termin{image} of $f$ and is denoted by $\im{f}$.
  One can verify that in $\Ring$, this concept coincides with the \emph{set-theoretic image}.

  Thus we conclude:
  \begin{cor}[First isomorphism theorem]
    In $\Ring$, for any homomorphism $f$, we have $\im{f}=\Coim{f}$. Moreover, It can be uniquely factored as a regular epimorphism followed by a monomorphism through the image $\im{f}$.
  \end{cor}
  For this reason, the regularEpi-mono factorizations in $\Ring$ are usually cited as \termin[image factorizations]{image factorization}.

\subsection{Characteristic}
  The unique morphism from a ring to the terminal object $\mathbf{0}$ is surjective thus an epimorphism. However, the unique morphism from the initial object $\mathbb{Z}$ to some ring $R$ may not be a monomorphism in general. If it is, we will say $R$ has \termin{characteristic} $0$.

  In general, the morphism $\mathbb{Z}\to R$ can be uniquely factored into a regular epimorphism and a monomorphism, say $\mathbb{Z}\twoheadrightarrow K\hookrightarrow R$.  Note that, the ring $K$ is actually not unique, but unique up to unique isomorphism. As a subring of $R$, it is the smallest one and will be called the \termin{characteristic subring} of $R$, it is the \emph{category theoretical} definition of the characteristic and is denoted by $\Char{R}$.  Moreover, $K$ is isomorphic to a quotient ring of $\mathbb{Z}$, which is $\mathbb{Z}/n\mathbb{Z}$ where $n$ is a natural number. This $n$ is called the \termin{characteristic} of $R$, and denoted by $\Char{R}$. (There is no ambiguities on using same notation to denoted the characteristic subring and the characteristic, just as we use $0$ to denote both the zero ring and zero element.)

  Here we list some basic properties of characteristic:
  \begin{prop}
    Let $f\colon R\to S$ be a ring homomorphism, then the characteristic of $S$ divides the characteristic of $R$. If $f$ is a monomorphism, then those two characteristics are equal.
  \end{prop}
  \begin{proof}
    By the image factorization, the homomorphism $f\colon R\to S$ induces the following commutative diagram
        \begin{displaymath}
          \xymatrix{
            \mathbb{Z}\ar@{->>}[r]\ar@{->>}[d]&\Char{R}\ar@{ (->}[d]\ar@{-->}[dl]|{u}\\
            \Char{S}\ar@{ (->}[r]&S
          }
        \end{displaymath}

    Then since the upper and left morphisms are strong, so there exists a unique morphism $u\colon\Char{R}\to\Char{S}$ making the triangles commute and moreover, it is also a strong epimorphism.

    If $f\colon R\to S$ is a monomorphism, then the right morphism in the above diagram is a composite of two monomorphisms thus also a monomorphism. Then $u$ is a monomorphism and thus an isomorphism.

    We know that a surjective homomorphism of quotients $\mathbb{Z}/n\mathbb{Z}\twoheadrightarrow\mathbb{Z}/m\mathbb{Z}$ of $\mathbb{Z}$ is corresponding to the division $m\mid n$ of integers. From this, we get the desired conclusions.
  \end{proof}
  Therefore, the chains of divisions of integers can be viewed as a sketch of the chains of  morphisms in $\Ring$. Thus, we can classified rings by characteristics and study rings sharing same characteristic while between them there is nothing but divisions of integers.

  \begin{prop}
    Let $R$ be a domain, then its characteristic should be either $0$ or prime.
  \end{prop}
\textbf{【?】}

\newpage
\section{Category of commutative rings}
  The category of commutative rings and their homomorphisms is denoted by $\CRing$. It is a full subcategory of $\Ring$ since the homomorphisms of commutative rings are just the ring homomorphisms.

  Since both $\mathbb{Z}$ and the zero ring $\mathbf{0}$ is commutative, they should be the initial object and terminal object in $\CRing$ respectively. Thus $\CRing$ can NOT be \emph{pre-additive}, a fortiori \emph{abelian}.

  The embedding $\CRing\hookrightarrow\Ring$ preserves \emph{products}, \emph{equalizers} and \emph{coequalizers}. That means those limits and colimits in $\CRing$ are the same as in $\Ring$. In other words, the direct product of a family of commutative rings is commutative, the equalizer and coequalizer of a homomorphism in $\CRing$ is also commutative. However, The embedding does NOT preserve \emph{coproducts}.

  Leaving the trouble of coproducts aside, we have that $\CRing$ is complete and we have kernel pairs and regular coimages as defined in $\Ring$.
  Moreover, the monomorphisms and all kinds of epimorphisms in $\CRing$ coincide in $\Ring$, thus the whole discussion about epimorphisms and image factorizations still hold in $\CRing$.


%\newpage
%\section{Advanced}
%\subsection{Algebras over a ring}
%  NEED RMod being abelian.
%\subsection{Domains and fields}
%  In a given ring, except $0$ and $1$, there are other special elements should be noticed. A \termin{unit} is an element admitting an inverse. A \termin{zero divisor} is an element that divide $0$.










