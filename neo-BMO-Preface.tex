\chapter*{\giant\Edward Preface}
\addstarredchapter{Preface}

This is a note for a reading group organized by me. The main topic of this note is category theory.

In 2012, I start to read Serge Lang's \emph{Algebra}, i.e. \cite{lang2002algebra}, with Zhu, Yiyi and Zhang, Hanbin. Dissatisfied with element-based proof, I turn to initiate a project aim to give category-style proof. implicated by this conceit of mine, the seminar finally stumped after one year, when we finished the seven chapter of Lang's book. After that, I began to organize our notes and found that many category facts are not so obvious and that there are many ways to uniform notions from and beyond category theory. Fascinated by $n$Lab and its philosophy, I dropped myself into an endless waste of time.

The first thing I did is to recheck details in category theory. But how to organize them had really got me there. The final decision is to follow Borceux's \emph{Handbook of Categorical Algebra}, i.e. \cite{borceux}, since my purpose is to organize my proofs so that i can refer them easily. What's more, there are many classical examples in Borceux's book, which may gives me great help. That's how \emph{BMO} born.

But this plan turns out to be a folly. For one thing, even I haven written all the proofs by myself, what I have done is nothing but rewriting Borceux's book. On the other hand, since I try to cram everything I found in $n$Lab under the topics into the corresponding sections, I actually spent much more time on what I should and seriously implicated my major study project. Once I realize this, I suspended \emph{BMO} and then quickly Borceux's book (only first volume, of course). Then, I was busy to apply for PhD programs and hardly bashed into a wall.

Now I have no time and will to continue this note. But a little modification may be good. That is the \emph{neo-BMO}.

Thanks to all.

\begin{flushright}
  \emph{Gao, Xu}
\end{flushright}
