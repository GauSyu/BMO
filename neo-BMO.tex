%%%%%%%%%%%%%%%%%%%%%%%%%%%%%%%%%%
%
%                        A Tex File Made by Gau Syu
%                              GauSyu@Gmail.com
%
%###########################################
%%%%%%%%%%%%%%%%%%%%%%%%%%%%%%%%%%
%
%                     Include The Common Settings
%
%%%%%%%%%%%%%%%%%%%%%%%%%%%%%%%%%%
%%%%%%%%%%%%%%%%%%%%%%%%%%%%%%%%%%
%
%               A Common Setting File Made by Gau Syu
%                              GauSyu@Gmail.com
%
%###########################################
% @ Basic Document Packages
%     @@ Typeface
%     @@ Latex Graphics
%     @@ Index and Nomenclatures
%     @@ Define Colors
%     @@ Terminology Format
% @ Theorems and References
%     @@ Change the Indentation
%     @@ Define Theorem Environments
%     @@ Define Cross Reference Names
% @ Chapters and Sections
%%%%%%%%%%%%%%%%%%%%%%%%%%%%%%%%%%
%
%                          Basic Document Packages
%
%%%%%%%%%%%%%%%%%%%%%%%%%%%%%%%%%%
\documentclass[11pt,a4paper,oneside,nocap,fancyhdr,hyperref]{ctexbook}
%nocap means English
\usepackage{appendix} % Allowed Special Appendix Chapters
\usepackage{enumerate} % This package gives the enumerate environment an optional argument which determines the style in which the counter is printed.
%
%                                         Typefaces
%
    \newfontinstance{\Edward}{Edwardian Script ITC}
    \newfontinstance{\Frak}{Euclid Fraktur}
    \newcommand{\Giant}{\fontsize{72pt}{\baselineskip}\selectfont}
    \newcommand{\giant}{\fontsize{27pt}{\baselineskip}\selectfont}
%
%                                     Latex Graphics
%
\usepackage{graphicx}% Able to Insert Pictures
\usepackage{epic} % Extending Latex Graphics
\usepackage[all]{xy} % Able to Draw Diagrams
\xyoption{2cell}
\UseAllTwocells
\xyoption{frame}
\usepackage{tikz} % Able to Draw Pictures
%
%                           Index and Nomenclatures
%
\usepackage{makeidx}
  \makeindex
\usepackage[
  symbols,                %list of symbols
%  nonumberlist,       %do not show page numbers
  seeautonumberlist,
  hyperfirst=false,
  toc,                         %show listings as entries in table of contents
  section=chapter, %use section level for toc entries
  counter=section] %countered by section
{glossaries}

  \altnewglossary{categories}{cat}{Categories}
  \glsenablehyper
  \makeglossaries

%%  Usage of glossaryentry
%\newglossaryentry{<\label>}  {name=<\what occurs in the glossary>, description=<\>,  text=<\what occurs in the context>, sort=<\How this term by sorted>, type=<\>}
%%
%\usepackage[refpage,intoc]{nomencl}
%  \def\nomname{Notations}
%  \setlength{\nomlabelwidth}{.20\hsize}
%  \setlength{\nomitemsep}{-\parsep}
%    \makenomenclature

\usepackage{xifthen}% provides \isempty test
\newcommand{\termin}[2][]{%
  \ifthenelse{\isempty{#1}}%
    {\textbf{{#2}}\index{#2}}% if #1 is empty
    {\textbf{{#1}}\index{#2}}% if #1 is not empty
}
%
%                                    Define Colors
%
\usepackage{color}
  \newcommand{\red}{\color{red}}
  \newcommand{\blue}{\color{blue}}

%%%%%%%%%%%%%%%%%%%%%%%%%%%%%%%%%%
%
%                        Define Theorem Environments
%
%%%%%%%%%%%%%%%%%%%%%%%%%%%%%%%%%%
\usepackage{amsmath,amsthm} % the Standard AMS Package
\usepackage{mathtools} %Many tools

%\renewcommand{\proofname}{\textbf{Solution}}

\newtheoremstyle{question}{1.5ex plus 1ex minus .2ex}{1.5ex plus 1ex minus .2ex}{\large\itshape}{}{\bfseries}{}{1em}{}
\theoremstyle{question}
\newtheorem{qst}{Question}[section]
\renewcommand{\theqst}{\arabic{qst}}
\newtheorem{subqst}{Question}[qst]
\newtheorem*{qst*}{Question}

\theoremstyle{theorem}
\newtheorem{thm}{Theorem}[section]
\newtheorem{lem}[thm]{Lemma}
\newtheorem{prop}[thm]{Proposition}
\newtheorem{cor}[thm]{Corollary}

\theoremstyle{definition}
\newtheorem{defn}[thm]{Definition}
\newtheorem{exam}[thm]{Example}
\newtheorem{ex}{}[chapter]
\newtheorem*{axiom}{Axiom}

\theoremstyle{remark}
\newtheorem*{rem}{Remark}
\newtheorem*{rec}{Recall}
%%%%%%%%%%%%%%%%%%%%%%%%%%%%%%%%%%
%
%                              Chapters and Sections
%
%%%%%%%%%%%%%%%%%%%%%%%%%%%%%%%%%%
\CTEXsetup[name={{$\S$},}]{section}
\CTEXsetup[name={{\Frak Cap}.,},aftername={\quad},format={\centering}]{chapter}
\CTEXoptions[contentsname={\textbf{Contents}}]
%\CTEXsetup[name={$\mathscr{P}$,}]{part}

%\newcommand{\Subsection}[1]{\subsection*{#1}\addcontentsline{toc}{subsection}{#1}}
\setcounter{secnumdepth}{1}% Depth of Sections
\setcounter{tocdepth}{1}% Depth of Contents
\usepackage{minitoc}

%%%%%%%%%%%%%%%%%%%%%%%%%%%%%%%%%%
%
%                                Letters and Notations
%
%%%%%%%%%%%%%%%%%%%%%%%%%%%%%%%%%%
%%%%%%%%%%%%%%%%%%%%%%%%%%%%%%%%%%
%
%               A Common Setting File Made by Gau Syu
%                              GauSyu@Gmail.com
%
%###########################################
% @ Letters and Notations
%     @@ Letters
%     @@ Math Characters
%     @@ Arrows
%     @@ Functions
%     @@ Terminology
%     @@ Map Descriptions
%     @@ Typical Commutative Diagrams
%     @@ Arrows in Diagram
%     @@ Words
%%%%%%%%%%%%%%%%%%%%%%%%%%%%%%%%%%
\usepackage{amssymb,amsfonts,bbm,mathrsfs,upgreek}
%
%                                           Letters
%
\def\Aa{{\cal A}}
\def\Bb{{\cal B}}
\def\Cc{{\cal C}}
\def\Dd{{\cal D}}
\def\Ee{{\cal E}}
\def\Ff{{\cal F}}
\def\Gg{{\cal G}}
\def\Hh{{\cal H}}
\def\Ii{{\cal I}}
\def\Jj{{\cal J}}
\def\Kk{{\cal K}}
\def\Ll{{\cal L}}
\def\Mm{{\cal M}}
\def\Nn{{\cal N}}
\def\Oo{{\cal O}}
\def\Pp{{\cal P}}
\def\Qq{{\cal Q}}
\def\Rr{{\cal R}}
\def\Ss{{\cal S}}
\def\Tt{{\cal T}}
\def\Uu{{\cal U}}
\def\Vv{{\cal V}}
\def\Ww{{\cal W}}
\def\Xx{{\cal X}}
\def\Yy{{\cal Y}}
\def\Zz{{\cal Z}}

\def\AA{{\mathbb A}}
\def\BB{{\mathbb B}}
\def\CC{{\mathbb C}}
\def\DD{{\mathbb D}}
\def\EE{{\mathbb E}}
\def\FF{{\mathbb F}}
\def\GG{{\mathbb G}}
\def\HH{{\mathbb H}}
\def\II{{\mathbb I}}
\def\JJ{{\mathbb J}}
\def\KK{{\mathbb K}}
\def\LL{{\mathbb L}}
\def\MM{{\mathbb M}}
\def\NN{{\mathbb N}}
\def\OO{{\mathbb O}}
\def\PP{{\mathbb P}}
\def\QQ{{\mathbb Q}}
\def\RR{{\mathbb R}}
\def\SS{{\mathbb S}}
\def\TT{{\mathbb T}}
\def\UU{{\mathbb U}}
\def\VV{{\mathbb V}}
\def\WW{{\mathbb W}}
\def\XX{{\mathbb X}}
\def\YY{{\mathbb Y}}
\def\ZZ{{\mathbb Z}}

\def\Aaa{\mathscr{A}}
\def\Bbb{\mathscr{B}}
\def\Ccc{\mathscr{C}}
\def\Ddd{\mathscr{D}}
\def\Eee{\mathscr{E}}
\def\Fff{\mathscr{F}}
\def\Ggg{\mathscr{G}}
\def\Hhh{\mathscr{H}}
\def\Iii{\mathscr{I}}
\def\Jjj{\mathscr{J}}
\def\Kkk{\mathscr{K}}
\def\Lll{\mathscr{L}}
\def\Mmm{\mathscr{M}}
\def\Nnn{\mathscr{N}}
\def\Ooo{\mathscr{O}}
\def\Ppp{\mathscr{P}}
\def\Qqq{\mathscr{Q}}
\def\Rrr{\mathscr{R}}
\def\Sss{\mathscr{S}}
\def\Ttt{\mathscr{T}}
\def\Uuu{\mathscr{U}}
\def\Vvv{\mathscr{V}}
\def\Www{\mathscr{W}}
\def\Xxx{\mathscr{X}}
\def\Yyy{\mathscr{Y}}
\def\Zzz{\mathscr{Z}}

\def\aa{\mathfrak{a}}
\def\bb{\mathfrak{b}}
\def\cc{\mathfrak{c}}
\def\dd{\mathfrak{d}}
\def\ee{\mathfrak{e}}
\def\ff{\mathfrak{f}}
\def\gg{\mathfrak{g}}  % Knuth uses $\gg$ for ``>>''.
\def\hh{\mathfrak{h}}
\def\ii{\mathfrak{i}}
\def\jj{\mathfrak{j}}
\def\kk{\mathfrak{k}}
\def\ll{\mathfrak{l}}  % Knuth uses $\ll$ for ``<<''.
\def\mm{\mathfrak{m}}
\def\nn{\mathfrak{n}}
\def\oo{\mathfrak{o}}
\def\pp{\mathfrak{p}}
\def\qq{\mathfrak{q}}
\def\rr{\mathfrak{r}}
\def\ss{\mathfrak{s}}
\def\tt{\mathfrak{t}}
\def\uu{\mathfrak{u}}
\def\vv{\mathfrak{v}}
\def\ww{\mathfrak{w}}
\def\xx{\mathfrak{x}}
\def\yy{\mathfrak{y}}
\def\zz{\mathfrak{z}}

\def\BBb{\mathfrak{B}}
\def\CCc{\mathfrak{C}}
\def\EEe{\mathfrak{E}}
\def\FFf{\mathfrak{F}}
\def\GGg{\mathfrak{G}}
\def\IIi{\mathfrak{I}}
\def\PPp{\mathfrak{P}}
\def\SSs{\mathfrak{S}}
\def\UUu{\mathfrak{U}}

\def\aaa{{a}}
\def\bbb{{b}}
\def\ccc{{c}}
\def\ddd{{d}}

%
%                                  Math Characters
%
\def\<{\langle}
\def\>{\rangle}
\def\anti{\mathpzc{S}}
\def\ctimes{\textrm{\c{$\otimes$}}}
\def\sminus{\smallsetminus}
\def\Wedge{\mbox{$\bigwedge$}}
\def\lrtimes{\Join}

\def\Hodge{\widetilde{\Delta}}

%
%                                         Arrows
%
\def\acts{\curvearrowright}
\def\epi{\twoheadrightarrow}
\def\from{\leftarrow}
\def\isom{\overset{\sim}{\to}}
\def\longto{\longrightarrow}
\def\mono{\rightarrowtail}
\def\onto{\twoheadrightarrow}
\def\injection{\hookrightarrow}
\def\then{\Rightarrow}
\def\Then{\Longrightarrow}
\def\To{\longto}
\def\Ot{\longleftarrow}
\def\tofrom{\leftrightarrow}
\def\tto{\rightrightarrows}
\def\down{\downarrow}

%
%                                          Functions
%
\newcommand{\norm}[1]{\lVert #1\rVert} % Norm
\newcommand{\dual}[1]{{#1}^{\wedge}}% Dual
\newcommand{\codual}[1]{{#1}^{\vee}}% Codual
\newcommand{\pfrac}[2]{\frac{\partial{#1}}{\partial{#2}}}

\newcommand{\markar}[1]{\stackrel{{#1}}{\longrightarrow}}
\newcommand{\markal}[1]{\stackrel{{#1}}{\longleftarrow}}
\newcommand{\defen}{\stackrel{\text{def}}{\iff}}
\newcommand{\local}[2]{\left.{#1}\right|_{#2}}%Local #1 at #2

%
%                                        Combinations
%
\newcommand{\exseq}[5]{{#1} \xrightarrow{{#2}} {#3} \xrightarrow{{#4}} {#5}}
\newcommand{\longexseq}[5]{{#1} \markar{{#2}} {#3} \markar{{#4}} {#5}}
\newcommand{\shortexseq}[5]{1\longrightarrow{#1}\xrightarrow{#2}{#3}\xrightarrow{#4}{#5}\longrightarrow{1}}
\newcommand{\myldto}[1]{$\mathop{\rightsquigarrow}\limits_{\mathclap{\text{\scriptsize #1}}}$}
\newcommand{\mysim}[1]{\mathop{\sim}\limits_{\mathclap{#1}}}
\newcommand{\defeq}{\stackrel{\text{def}}{=}}
\newcommand{\iso}[1]{\stackrel{#1}{\cong}}
%
%                                    Map Descriptions
%
\newcommand{\mapdes}[4]
  {
    \begin{align*}
       #1 & \longrightarrow #2 \\
       #3 & \longmapsto #4
    \end{align*}
  }
\newcommand{\longmapdes}[5]
  {
    \begin{align*}
      #1\colon  #2 & \longrightarrow  #3 \\
            #4 & \longmapsto  #5
    \end{align*}
  }
\newcommand{\isodes}[4]
  {
    \begin{align*}
      #1 & \cong  #2 \\
      #3 & \leftrightarrow  #4
    \end{align*}
  }
%
%                        Typical Commutative Diagrams
%
\newcommand{\initial}[6]
{
\begin{displaymath}
   \xymatrix{
     {#1} \ar[r]^{#2} \ar[dr]_{#4} & {#3} \ar@{-->}[d]^{#6} \\
     & {#5}
   }
\end{displaymath}
}
\newcommand{\terminal}[6]
{
\begin{displaymath}
   \xymatrix{
     {#5} \ar[dr]^{#4} \ar@{-->}[d]_{#6} & \\
     {#3} \ar[r]_{#2} & {#1}
   }
\end{displaymath}
}
\newcommand{\functor}[8]
{
\begin{displaymath}
   \xymatrix{
     {#1}\ar[d]_{#2}\ar[r]^-{#4} & {#5}\ar[d]^{#8} \\
     {#3}\ar[r]^-{#6} & {#7}
   }
\end{displaymath}
}

\newcommand{\Functor}[9]
{
\begin{displaymath}
   \xymatrix{
     {#1}\ar[rr]^{#2} & & {#3}\\
     {#4}\ar[dd]^{#5}="a" \ar@{|->}[rr]  & & {#7}\ar[dd]_{#8}="b" \ar@{|->} "a";"b"\\
     & & \\
     {#6}\ar@{|->}[rr] & & {#9}
   }
\end{displaymath}
}
%
%                                    Arrows in Diagram
%
\newdir{ (}{{}*!/-5pt/@^{(}}
\newdir{ >}{{}*!/-5pt/@{>}}
%
%                                              Words
%
\DeclareMathOperator{\ab}{ab}%abelian group
\DeclareMathOperator{\Abtf}{\mathbf{Ab}_{\mathrm{tf}}}%Abelian group with torsion-free
\DeclareMathOperator{\Ad}{Ad}
\DeclareMathOperator{\Add}{\mathbf{Add}}
\DeclareMathOperator{\ad}{ad}
\DeclareMathOperator{\adj}{adj}
\DeclareMathOperator{\Alt}{Alt}
\DeclareMathOperator{\an}{an}%Berkovich analytic space
\DeclareMathOperator{\ann}{ann}
\DeclareMathOperator{\Ann}{Ann}
\DeclareMathOperator{\Aut}{Aut}
\DeclareMathOperator{\Bil}{Bil}%Bilinear
\DeclareMathOperator{\ch}{ch}
\DeclareMathOperator{\Char}{char}
\DeclareMathOperator{\cls}{cls}
\DeclareMathOperator{\coim}{coim}
\DeclareMathOperator{\coker}{coker}
\DeclareMathOperator{\di}{d}
\DeclareMathOperator{\diag}{diag}
\DeclareMathOperator*{\dirlim}{\underrightarrow{\lim}}%direct limit
\DeclareMathOperator{\D}{D}
\DeclareMathOperator{\Der}{Der}%Derivations
\DeclareMathOperator{\End}{End}
\DeclareMathOperator{\ev}{ev}
\DeclareMathOperator{\et}{{\acute{e}t}}%\'{E}tale
\DeclareMathOperator{\Ext}{Ext}
\DeclareMathOperator{\filt}{\mathfrak{f}}%filtration  %\ff
\DeclareMathOperator{\Fct}{\mathbf{Fct}}%Functor
\DeclareMathOperator{\Fix}{Fix}%Fixed points
\DeclareMathOperator{\Forget}{Forget}
\DeclareMathOperator{\Free}{Free}
\DeclareMathOperator{\Fun}{\mathbf{Fun}}
\DeclareMathOperator{\gr}{gr}
\DeclareMathOperator{\Gal}{Gal}
\DeclareMathOperator{\Gr}{Gr}
\DeclareMathOperator{\Hess}{Hess}%Hessian
\DeclareMathOperator{\Hom}{Hom}%Hom bifunctor
\DeclareMathOperator{\Nat}{Nat}%Nat bifunctor
\DeclareMathOperator{\im}{im}
\DeclareMathOperator*{\invlim}{\underleftarrow{\lim}}%inverse limit
%\DeclareMathOperator*{\testsum}{L.M.}%inverse limit
\DeclareMathOperator{\id}{id}%Identity
\DeclareMathOperator{\Id}{\mathbf{I}}%Identity
\DeclareMathOperator{\Inn}{Inn}
\DeclareMathOperator{\Int}{Int}
\DeclareMathOperator{\IR}{IR}
\DeclareMathOperator{\Isw}{Isw}
\DeclareMathOperator{\Is}{\mathfrak{Is}}%Isolator
\DeclareMathOperator{\Lie}{\mathfrak{L}}%lie algebra
\DeclareMathOperator{\Mor}{Mor}
\DeclareMathOperator{\ML}{\mathbf{M.L.}}
\DeclareMathOperator{\nat}{nat}
\DeclareMathOperator{\N}{\mathbb{N}}
\DeclareMathOperator{\ob}{ob}%Object
\DeclareMathOperator{\obj}{\mathcal{T}}
\DeclareMathOperator{\ord}{ord}
\DeclareMathOperator{\op}{op}%opposite ring
\DeclareMathOperator{\Op}{\mathcal{O}\mathfrak{p}\mathrm{ext}}%the sets of congruence classes of extensions
\DeclareMathOperator{\Pic}{Pic}%Picard group
\DeclareMathOperator{\prim}{prim}
\DeclareMathOperator{\Proj}{Proj}
\DeclareMathOperator{\Quot}{Qout}%Quotient functor
\DeclareMathOperator{\rad}{rad}
\DeclareMathOperator{\Rad}{Rad}
\DeclareMathOperator{\Rt}{Rt}
\DeclareMathOperator{\rank}{rank}
\DeclareMathOperator{\sgn}{sgn}
\DeclareMathOperator{\Span}{span} % Annoyingly, \span is already a command in TeX, and redefining it leads to other problems.
\DeclareMathOperator{\Spec}{Spec}
\DeclareMathOperator{\Sw}{Sw} % Swan conductor
\DeclareMathOperator{\Split}{\mathbf{Split}}%Karoubi envelope
\DeclareMathOperator{\supp}{supp}
\DeclareMathOperator{\Tor}{Tor}
\DeclareMathOperator{\tor}{tor}
\DeclareMathOperator{\Tr}{Tr}

\DeclareMathOperator{\GL}{GL}
\DeclareMathOperator{\PGL}{PGL}
\DeclareMathOperator{\PSL}{PSL}
\DeclareMathOperator{\SL}{SL}
\DeclareMathOperator{\SO}{SO}
\DeclareMathOperator{\GO}{O}
\DeclareMathOperator{\SP}{Sp}
\DeclareMathOperator{\Spin}{Spin}
\DeclareMathOperator{\SU}{SU}
\DeclareMathOperator{\GU}{U}
\DeclareMathOperator{\Pt}{Pt}


%%%%%%%%%%%
%         url
%%%%%%%%%%%
\def\nlab{\href{http://ncatlab.org}{{$n$Lab}} }
\def\hrefacc{\href{http://katmat.math.uni-bremen.de/acc/}{\emph{The Joy of Cats}} }

\def\gl{\gg\ll}
\def\sl{\mathfrak{sl}}
\def\so{\ss\oo}
\def\sp{\ss\pp}
\def\su{\ss\uu}

\def\st{\textrm{ s.t. }}
\def\RP{\mathbb{R}\mathbf{P}}
\def\Real{\mathbb{R}}

\DeclareMathOperator{\Br}{Br}
\renewcommand{\mod}{\mathop{\mathrm{mod}}}

%
%                                          Notations
%

\loadglsentries{Glossaries/TableOfCats}
\loadglsentries{Glossaries/TableOfSymbols}
%%%%%%%%%%%%%%%%%%%%%%%%%%%%%%%%%%
%
%                                 PDF File Information
%
%%%%%%%%%%%%%%%%%%%%%%%%%%%%%%%%%%
\hypersetup{
             pdftitle={Seminar notes on Algebra},
             pdfauthor={Gao, Xu},
             pdfsubject={algebra},
             pdfkeywords={seminar, notes},
             pdfproducer={XeLaTeX},
    colorlinks=true,
    citecolor=black,
    filecolor=black,
    linkcolor=black,
    urlcolor=black
}
%%%%%%%%%%%%%%%%%%%%%%%%%%%%%%%%%%
%
%                                           Title
%
%%%%%%%%%%%%%%%%%%%%%%%%%%%%%%%%%%
\title{{\giant Seminar notes on}\\ \Giant\textsc{Algebra}}
\author{Gao, Xu}
\date{}

\begin{document}
\frontmatter
\maketitle
\dominitoc
\chapter*{\giant\Edward Preface}
\addstarredchapter{Preface}

This is a note for a reading group organized by me. The main topic of this note is category theory.

In 2012, I start to read Serge Lang's \emph{Algebra}, i.e. \cite{lang2002algebra}, with Zhu, Yiyi and Zhang, Hanbin. Dissatisfied with element-based proof, I turn to initiate a project aim to give category-style proof. implicated by this conceit of mine, the seminar finally stumped after one year, when we finished the seven chapter of Lang's book. After that, I began to organize our notes and found that many category facts are not so obvious and that there are many ways to uniform notions from and beyond category theory. Fascinated by $n$Lab and its philosophy, I dropped myself into an endless waste of time.

The first thing I did is to recheck details in category theory. But how to organize them had really got me there. The final decision is to follow Borceux's \emph{Handbook of Categorical Algebra}, i.e. \cite{borceux}, since my purpose is to organize my proofs so that i can refer them easily. What's more, there are many classical examples in Borceux's book, which may gives me great help. That's how \emph{BMO} born.

But this plan turns out to be a folly. For one thing, even I haven written all the proofs by myself, what I have done is nothing but rewriting Borceux's book. On the other hand, since I try to cram everything I found in $n$Lab under the topics into the corresponding sections, I actually spent much more time on what I should and seriously implicated my major study project. Once I realize this, I suspended \emph{BMO} and then quickly Borceux's book (only first volume, of course). Then, I was busy to apply for PhD programs and hardly bashed into a wall.

Now I have no time and will to continue this note. But a little modification may be good. That is the \emph{neo-BMO}.

Thanks to all.

\begin{flushright}
  \emph{Gao, Xu}
\end{flushright}

\tableofcontents
\mainmatter
\pagestyle{plain}
\setcounter{minitocdepth}{2}
\part{Mathematical Structures}
  The notion of ``Mathematical structure'', as an explicit mathematical concept, first appears in N.Bourbaki's \emph{Elements of Mathematics}, Volume 1 \emph{Theory of Sets}, Chapter 4 ``Structures''. By using axiomatic set theory, Bourbaki suggested a general definition of ``mathematical structure'' and ``isomorphism'', although this theory was rarely used even in his own book.

  Category Theory, as newborn fundamental theory, provides a more suitable language to describe mathematical structures and influence modern mathematics and other subjects profoundly. Hence more and more modern algebra textbooks involve an introduction to category theory.

  In this part, we introduce some knowledge of category theory we will mention in the rest of this book. Most of them can be found in every textbook of category theory.

  The standard reference is MacLane's famous \emph{Categories for the Working Mathematician} \cite{lane1998categories}.
  In addition, F.W.Lawvere and S.H.Schanuel's \emph{Conceptual Mathematics} \cite{lawvere1997conceptual}, S.Awodey's \emph{Category Theory} \cite{awodey2010category} are straightaway textbooks.
  F.Borceux \emph{Handbook of categorical algebra} \cite{borceux} is an informative Encyclopedia.

  Although unnecessary in principle, having basic algebraic training is helpful to understand this part. One can refer
  R.Godement's \emph{Cours d'alg{\`e}bre} \cite{godement1963cours}, N.Jacobson's \emph{Basic Algebra} \cite{jacobson1980basic}, M.Artin's \emph{Algebra} \cite{artin2011algebra} and even any undergraduate textbook of algebra.

  List of chapters:

  The Language of Categories

  Set Theory

  Theory of Limits

  Special Objects and Morphisms

  Universal Structures

  Internal Category Theory

  \chapter{The Language of Categories}
  In this chapter, we start with the basic vocabulary of categories, functors, natural transformations, monomorphisms, epimorphisms, isomorphisms. The analogies between monomorphisms and epimorphisms, covariant and contravariant functors, lead to the famous duality principle which is, with the Yoneda lemma, one of the key results of the first chapter.
\minitoc
\newpage
\section{Introduction}
\subsection{Why category?}
  Many similar phenomena and constructions with similar properties occur in completely different mathematical fields.
  To describe precisely such phenomena and investigate such constructions simultaneously, the language of categories emerges.

  For past years, categorists have developed a symbolism that allows one quickly to visualize quite complicated facts by means of diagrams.

  Nowadays, category theory become a powerful language provides suitable vehicles that allow one to transport problems from one area of mathematics to another area, where solutions are sometimes easier. Therefore, the language of categories become more and more popular in modern mathematics and other fields like logic, computer science, linguistics and philosophy.

  More history comment can be found in many textbooks about category theory.
%Category Theory is a way of treating metaphor rigorously - in great complex nests, perhaps, but still rigorously.  And of having rigor without losing all meaning.

%These are the points on usefulness of category theory that Graham Hutton once mentioned in a course on category theory:
%Building bridges—exploring relationships between various mathematical objects, e.g., Products and Function
%Unifying ideas - abstracting from unnecessary details to give general definitions and results, e.g., Functors
%High level language - focusing on how things behave rather than what their implementation details are e.g. specification vs implementation
%Type safety - using types to ensure that things are combined only in sensible ways e.g. (f: A -> B  g: B -> C) => (g o f: A -> C)
%Equational proofs—performing proofs in a purely equational style of reasoning
\subsection{What is a category?}
  There are several ways to define what is a category; in the usual foundations of mathematics, these two definitions are equivalent. We now provide a popular one.
  \begin{defn}\label{def:category}
  A \termin{category} $\Cc$ consists of the following data:
  \begin{itemize}
    \item a collection $\ob\Cc$ of \termin[objects]{object (category theory)}.
    \item a collection $\hom\Cc$ of \termin[morphisms]{morphism (category theory)} (or \termin[arrows]{arrow (category theory)}, \termin[maps]{map (category theory)}) between objects.

             Each morphism $f$ has a unique \termin[source]{source (category theory)} object $A$ and \termin[target]{target (category theory)} object $B$.

             We write $f\colon A\To B$, and say ``$f$ is a morphism from $A$ to $B$'', ``$A$ is the \termin[domain]{domain (category theory)} of $f$'' and ``$B$ is the \termin[codomain]{codomain (category theory)} of $f$''.

             We write $\Hom(A, B)$\glsadd{hom} (or $\Hom_{\Cc}(A, B)$ when there may be confusion about to which category $\Hom(A, B)$ refers) to denote the collection of all morphisms from $A$ to $B$. (Some authors write $\Mor(A, B)$ or simply $\Cc(A,B)$ instead.)
    \item for every three objects $A,B$ and $C$, a binary operation
             \begin{equation*}
               \Hom(A, B) \times \Hom(B, C) \To \Hom(A, C)
             \end{equation*}
             called \termin{composition of morphisms}.

             The composition of $f\colon A \To B$ and $g\colon B \To C$ is written as $g\circ f$ or simply $gf$. (Some authors use ``diagrammatic order'', writing $f;g$ or $fg$.)
  \end{itemize}
  subject to the following axioms:
  \begin{description}
    \item[associativity] if $f\colon A \To B, g\colon B \To C$ and $h\colon C \To D$ then
                                 \begin{equation*}
                                   h\circ(g\circ f) = (h\circ g)\circ f
                                 \end{equation*}
    \item[identity] for every object $A$, there exists a morphism $1_A\colon A \To A$ (some times write $\id_A$) called the \termin[identity morphism]{identity morphism} for $A$, such that for every morphism $f\colon A \To B$, we have $1_B \circ f = f = f \circ 1_A$.\glsadd{identityM}
  \end{description}
  From these axioms, one can prove that there is exactly one identity morphism for every object. Some authors use a slight variation of the definition in which each object is identified with the corresponding identity morphism.
  \end{defn}
  \begin{rem}
     To emphasize the category $\Cc$, one often say an object (resp. morphism, arrow, map) in $\Cc$ or a $\Cc-$objects (resp. $\Cc-$morphism, $\Cc-$arrow, $\Cc-$map).
  \end{rem}
\begin{rem}
  Category theory provide a framework to discuss so-called ``categorical properties''. Informally, a \termin{categorical property} is a statement about objects and arrows in a category. More technically, a category provide a two-typed first order language with objects and morphisms as distinct types, together with the relations of an object being the source or target of a morphism and a symbol for composing two morphisms, thus a category property is a statement in such a language.
\end{rem}

  Let us now define a "homomorphism of categories".
  \begin{defn}
    A \termin{functor} $F$ from a category $\Aa$ to a category $\Bb$ consists of the following data:
    \begin{itemize}
      \item a mapping
                 \begin{equation*}
                   \ob\Aa\To\ob\Bb
                 \end{equation*}
                 between the collections of objects of $\Aa$ and $\Bb$; the image of $A\in\ob\Aa$ is written $F(A)$ or just $F A$;
      \item for every pair of objects $A, A'$ of $\Aa$, a mapping
                 \begin{equation*}
                   \Hom_{\Aa}(A,A')\To\Hom_{\Bb}(F(A),F(A'))
                 \end{equation*}
                 the image of $f\in\Hom_{\Aa}(A,A')$ is written $F(f)$ or just $F f$.
    \end{itemize}
    subject to the following axioms:
    \begin{itemize}
      \item for every pair of morphisms $f\in\Hom_{\Aa}(A,A'), g\in\Hom_{\Aa}(A',A'')$,
                 \begin{equation*}
                   F(g\circ f) = F(g)\circ F(f)
                 \end{equation*}
      \item for every object $A\in\ob\Aa$,
                 \begin{equation*}
                   F(1_A) = 1_{F(A)}
                 \end{equation*}
    \end{itemize}
  \end{defn}

  Given two functors $F\colon\Aa\to\Bb$ and $G\colon\Bb\to\Cc$, a pointwise composition immediately produces a new functor $G\circ F\colon\Aa\to\Cc$. This composition law is obviously associative.

  On the other hand, every category $\Cc$ has an identity functor, which can be simply obtained by choosing every mapping in the above definition to be the identity. It is an identity for the previous composition law.

  To summarize, the relationship between categories and functors is just like the one between objects and morphisms in a category. Thus a careless argument could lead to the conclusion that categories and functors constitute a new category, which is doubtful.

\subsection{Foundations and size issues}
    You may find that we use the undefined word ``collection'' instead of ``set'' or ``class''. The chosen of the word depends on the logical foundations we chose: when we use ZFC with one universe, then we choose the word ``set'' and all classical mathematical objects we considered such as sets, groups etc is small respect to the universe; when we use class theory like NBG, then we choose the word ``class'' and all classical mathematical objects are sets. More details can be found in the beginning of \cite{borceux} or section 1.8 of \cite{awodey2010category}.

    However, the foundation of category theory is independent of the axioms of set theory. In fact, category theory can be used to provide the foundations for mathematics as an alternative to set theory. Therefore we can rewrite the sentences of our axioms about what is a category in fully formal logic, which allow us to use the undefined word ``collection''. More details can be found in \nlab or \cite{lane1998categories}.

    Nevertheless, without set theory, the size-discussions such as ``a collection of all XXX'' do not make sense. Indeed, one of the lessons from the Russell's paradox is that the unrestricted usage of quantifiers is very dangerous.

    After choose an axiom system of set theory, the size issues make sense. In usual class theory such as NBG, a class is \emph{small} means it is a set, while in ZFC with one universe, a set is \emph{small} means it is an element of the universe.

    A category $\Cc$ is called \termin[small]{small category} if both $\ob\Cc$ and $\hom\Cc$ are small,
    and \termin[proper large]{proper large category} otherwise.
    A category is called \termin[locally small]{locally small category} if for each pair of objects $A$ and $B$, $\Hom(A, B)$ is small. Many important categories in mathematics (such as the category of sets), although not small, are at least locally small.
    For this reason, people tend to use the word ``category'' instead of ``locally small category''.
    Following this convention, we will simply put a locally small category called ``category''.
    This does not cause ambiguity, since we will still use the word ``\termin{large category}'' if needed.

    Under the above agreements, one can safely claim that all small categories and functors constitute a category, which is usually denoted by $\Cat$.
    On the other hand, the size of the ``category'' of all (locally small) categories and functors are larger than every large category, thus it does not exist under the usual set-theoretic axioms.

    Of course, one can also choose an axiom system of set theory which promise numerous sizes rather than just two: ``small'' and ``large''. In this case, all smaller categories and functors constitute a larger category. For instance, all locally small categories and functors form a ``very large'' (namely, larger than proper large categories) category, called $\CAT$: see \hrefacc i.e.\cite{acc}.

    More details about the foundations and size issues can be found in \nlab and related publications.

\newpage\section{Examples}
  The discussion of small categories and functors provide the first example of category in this book, i.e. $\Cat$. We should introduce some other examples in this section.

  In fact, traditional mathematics has provided a number of examples by different ways.
  \begin{exam}
    Many traditional mathematical structures are obtained by attaching some structures on sets. They provided a lot of obvious examples of categories.
    \begin{itemize}
      \item Sets
      \footnote{By this word we mean the small ones the set theory provided, e.g. small sets if we use ZFC with one universe. In such case, to avoid ambiguity, we use ``large set'' to call the general sets, which may be not small. A usual class theory already provides a word ``class'', thus one can also simply assume we have chose a usual class theory as our foundational set theory.}
       and functions: $\Set$.
      \item Groups and group homomorphisms: $\Grp$.
      \item Rings and ring homomorphisms: $\Ring$.
      \item Real vector spaces and linear mappings: $\Vect_{\RR}$.
      \item Right $R-$modules and module homomorphisms: $\Mod_R$.
      \item Topological spaces and continuous mappings: $\Top$.
      \item Uniform spaces and uniformly continuous functions: $\Uni$.
      \item Differentiable manifolds and smooth mappings: $\Diff$.
      \item Metric spaces and metric mappings: $\Met$.
      \item Real Banach spaces and bounded linear mappings: $\Banb$.
      \item Real Banach spaces and linear contractions: $\Ban$.
    \end{itemize}
    All of these categories encapsulate one ``kind of mathematical structure''. These are often called ``concrete'' categories (we will introduce a technical definition that these examples all satisfy later).
  \end{exam}
  \begin{exam}\label{exam:category2}
    Some mathematical devices can also be viewed as categories.
    \begin{itemize}
      \item $\Mat$: The set of natural numbers $\N$ can be viewed as a category as following: choose as objects the natural numbers and as arrows from $n$ to $m$ the matrices with $n$ rows and $m$ columns; the composition is the usual product of matrices.
      \item Every set $S$ can be viewed as a category whose objects are the elements of $S$ and the only morphisms are identities.

                 In general, a category whose only morphisms are the identities is called a \termin{discrete category}.
      \item A poset $(S,<)$ can be viewed as a category whose objects are the elements of $S$ and the set $\Hom(x,y)$ of morphisms is a singleton when $x < y$ and is empty otherwise.
                The possibility of defining a (unique) composition law is just the transitivity axiom of the partial order; the existence of identities is just the reflexivity axiom.
      \item A monoid $(M,\cdot)$ can be seen as a category $\Mm$ with a single object $\ast$ and $\Hom(\ast,\ast)=M$ as a set of morphisms; the composition law is just the multiplication of the monoid.
    \end{itemize}
  \end{exam}

  Do not worry if some of these examples are unfamiliar to you. Later on, we will take a closer look at some of them. In addition, there are also many examples emerge from outside of mathematics, such as logic, computer science, linguistics and philosophy. Those examples can be found in \cite{awodey2010category} and related publications.

\subsection{Basic examples via a given category}
From a given category $\Cc$, there are various ways to construct new categories. Here are some basic constructions.
\begin{exam}
  Any category $\Cc$ can itself be considered as a new category in a different way: the objects are the same as those in the original category but the arrows are those of the original category reversed. This is called the \termin[dual]{dual category} or \termin{opposite category} and is denoted by $\Cc^{\op}$.\glsadd{opposite}
\end{exam}
Obviously, the double dual category of a category is itself. The concept of dual category implies an important principle, the \emph{duality principle}. We will see soon it in this chapter.
\begin{exam}[\termin{slice category}]
  Let us fix an object $I\in\ob\Cc$. The category $\Cc/I$ of ``arrows over $I$'' is defined by the following.\glsadd{Slice}
  \begin{itemize}
    \item Objects: the arrows of $\Cc$ with codomain $I$.
    \item Morphisms from an object $(f\colon A\to I)$ to another $(g\colon B\to I)$:
               the morphisms $h\colon A\to B$ in $\Cc$ satisfying the ``commutative triangles over $I$''
               \begin{displaymath}
                 \xymatrix@R=0.5cm{
                    A\ar[rr]^{h}\ar[dr]_{f} && B\ar[dl]^{g} \\
                    &I&                }
               \end{displaymath}
               i.e.  $g\circ h = f$.
    \item   The composition law is that induced by the composition of $\Cc$.
  \end{itemize}
\end{exam}
  \begin{rem}
    Notice that in the case of $\Set$, a function $f\colon A\to I$ can be identified with the $I-$indexed family of sets $\{f^{-1}(i)\}_{i\in I}$ so that the previous category is just that of $I-$indexed families of sets and $I-$indexed families of functions.
  \end{rem}

\begin{exam}[\termin{coslice category}]
  Again fixing an object $I\in\ob\Cc$, the category $I/\Cc$ of ``arrows under $I$'' is defined by the following.\glsadd{Coslice}
  \begin{itemize}
    \item Objects: the arrows of $\Cc$ with domain $I$.
    \item Morphisms from an object $(f\colon I\to A)$ to another $(g\colon I\to B)$:
               the morphisms $h\colon A\to B$ in $\Cc$ satisfying the ``commutative triangles under $I$''
               \begin{displaymath}
                 \xymatrix@R=0.5cm{
                    &I\ar[dl]_{f}\ar[dr]^{g}& \\
                    A\ar[rr]_{h} && B   }
               \end{displaymath}
               i.e.  $h \circ f = g$.
    \item   The composition law is that induced by the composition of $\Cc$.
  \end{itemize}
\end{exam}

\begin{exam}
  Now we consider all the arrows of $\Cc$. The category $\Cc^{\to}$  of all arrows is defined by the following. \glsadd{Arr}
  \begin{itemize}
    \item Objects: the arrows of $\Cc$.
    \item A morphism from an object $(f\colon A \to B)$ to another $(g\colon C \to D)$ is a pair $(h\colon A\to C, k\colon B\to D)$ of morphisms of $\Cc$ satisfying a ``commutative square''
               \begin{displaymath}
                 \xymatrix{
                    A\ar[d]_{h}\ar[r]^{f} & B\ar[d]^{k} \\
                    C\ar[r]_{g}&D                }
               \end{displaymath}
    \item   The composition law is induced pointwise by the composition in $\Cc$.
  \end{itemize}
\end{exam}

\subsection{The simplest examples}
We now introduce some simple examples which can be inspired by the axioms of categories immediately.
\begin{exam}
  The category $\mathbf{0}$ has no objects or arrows.

  The category $\one$ has one object and its identity arrow. It looks like
               \begin{displaymath}\glsadd{terminalCate}
                 \xymatrix@R=0.5cm{
                    \bullet              }
               \end{displaymath}

  The category $\mathbf{2}$ has two objects, their required identity arrows, and exactly one arrow between the objects. It looks like
               \begin{displaymath}
                 \xymatrix@R=0.5cm{
                    \bullet\ar[r] & \bullet             }
               \end{displaymath}

  The category $\mathbf{3}$ has three objects, their required identity arrows, exactly one arrow from the first to the second object, exactly one arrow from the second to the third object, and exactly one arrow from the first to the third object (which is therefore the composite of the other two). It looks like
               \begin{displaymath}
                 \xymatrix@R=0.5cm{
                    \bullet\ar[r]\ar[dr] & \bullet\ar[d] \\
                    & \bullet            }
               \end{displaymath}
\end{exam}

One can see these categories look like quivers. Indeed, a category can be viewed a quiver with extra structure. More details can be found in \nlab.

The notation $\one,\mathbf{2},\mathbf{3}$ comes from the fact that these categories are in fact the ordinals regarded as posets and thus categories. We will discuss this in the later part.

\newpage\section{Monic-, epi- and isomorphisms}
The basic notions about functions are injectivity, surjectivity and bijectivity. To generalize them to the context of category theory, one need characterize them without using elements. Early category theorists believed that the cancellation properties provide the correct generalization, thus they define the following notions.
  \begin{defn}
    A morphism $f$ is called \termin{monic}, or a \termin{monomorphism}, if it is left cancellable, that means
    for any morphisms
    \begin{displaymath}
    \xymatrix@1{\cdot\ar@<0.5ex>[r]^{\alpha}\ar@<-0.5ex>[r]_{\beta} &\cdot\ar[r]^{f} &\cdot}
    \end{displaymath}
    $f\alpha=f\beta$ implies $\alpha=\beta$.

    Dually, $f$ is called \termin{epi}, or an \termin{epimorphism}, if it is right cancellable, that means for any morphisms
    \begin{displaymath}
    \xymatrix@1{\cdot\ar[r]^{f} &\cdot\ar@<0.5ex>[r]^{\alpha}\ar@<-0.5ex>[r]_{\beta} &\cdot}
    \end{displaymath}
    $\alpha f=\beta f$ implies $\alpha=\beta$.
  \end{defn}

  The followings are the basic propositions about monomorphisms and epimorphisms.
  \begin{prop}
    every identity morphism is both monic and epi. The composite of two monomorphisms (resp. epimorphisms) is also a monomorphism (resp. epimorphism).
  \end{prop}
  \begin{prop}[Triangle lemma]\label{prop:triangle lemma}
    If the composite $g\circ f$ of two morphisms is monic, then so is $f$; if the composite is epi, then so is $g$.
  \end{prop}

  By the definition, a morphism is monic means it is left cancellable, but this does not implies the existence of its left inverse. For epimorphisms, the argument is similar.
  \begin{defn}
    Consider two morphisms $f\colon A \to B$ and $g\colon B \to A$ in a category. When $g \circ f = 1_A$, $f$ is called a \termin[section]{section (category theory)} of $g$, $g$ is called a \termin[retraction]{retraction (category theory)} of $f$ and $A$ is called a \termin[retract]{retract (category theory)} of $B$.
  \end{defn}
  \begin{prop}
    In a category, every section is a monomorphism and every retraction is an epimorphism.
  \end{prop}
  \begin{defn}
    A morphism $f\colon A\to B$ is called a \termin{split monomorphism}, if it has a retraction.
    Dually, $f$  is called a \termin{split epimorphism}, if it has a section.
    If $f$ is both split monic and split epi, then we say it is an \termin{isomorphism} and $A$ is \termin{isomorphic} to $B$, denoted by $A\approx B$.  \glsadd{isomorphic}
  \end{defn}
  \begin{rem}
    It is clear that any split monomorphism must be monic and any split epimorphism is epi, hence any isomorphism is both monic and epi. However, the converses are not true in general case.

    A morphism which is both monic and epi is traditionally called a \termin{bimorphism}, although this is a bad name cause confusions when we go to higher category theory. A category in which every bimorphism is an isomorphism is called \termin[balanced]{balanced category}.
  \end{rem}

\subsection{Examples}
  \begin{exam}
    In $\Set$, $\Grp$, $\Mod_R$ or $\Top$, monomorphisms (resp. epimorphisms) are precisely those morphisms that have injective (resp. surjective) underlying functions. The verifications will be given in corresponding chapters.
  \end{exam}
  \begin{exam}\label{exam:nonsurjective epi}
    In $\Ring$, monomorphisms are precisely those morphisms that have injective underlying functions. However, epimorphisms are not necessarily surjective. For instance, the inclusion map $\ZZ\hookrightarrow\QQ$ is a non-surjective epimorphism.
    To see this, note that any ring homomorphism on $\QQ$ is determined entirely by its action on $\ZZ$. A similar argument shows that the canonical ring homomorphism from any commutative ring $R$ to any one of its localizations is an epimorphism.
  \end{exam}
  \begin{exam}\label{exam:noninjective mono}
    In the category $\DivAb$ of divisible abelian groups and group homomorphisms between them, there are monomorphisms that are not injective:
    consider, for example, the quotient map $q\colon\QQ\to\QQ/\ZZ$ from additive group of rational numbers. This is obviously not an injective map. Nevertheless, it is a monomorphism in this category.
    Indeed, choose $G$ a divisible abelian group and $f,g\colon G \to\QQ$ two group homomorphisms such that $q\circ f = q\circ g$ Putting $h = f - g$ we have $q\circ h = 0$ and the thesis becomes $h=0$. Given an element $x\in G$, $h(x)$ is an integer since $q \circ h = 0$. If $h(x)\neq0$, then one can easily find a contradiction.
  \end{exam}
  \begin{exam}
    In the category of connected pointed topological spaces and pointed continue maps, every coverings map are monomorphisms although they are usually not injective. This is just the unique lifting property of covering maps, one can find it in a textbook about algebraic topology, for example, \cite{AllenHatcher}.
  \end{exam}
  \begin{exam}
    In $\Set$, $\Grp$, $\Mod_R$, isomorphisms are precisely those morphisms that have bijective underlying functions.
    In $\Top$, isomorphisms are exactly the homeomorphisms. Unfortunately, a continuous bijection is in general not a homeomorphism. For instance, the map from the half-open interval $[0,1)$ to the unit circle $S^1$ (thought as a subspace of the complex plane) which sends $x$ to $e^{2\pi x i}$ is continuous and bijective but not a homeomorphism since the inverse map is not continuous at $1$.
  \end{exam}

  The counter-examples above shows that the notions of monomorphism and epimorphism do not meet the original requirements, thus category theorists develop some variations to fix this. One can find them in \nlab or \hrefacc.

\newpage\section{Natural transformations}
  Just as the study of groups is not complete without a study of homomorphisms, so the study of categories is not complete without the study of functors.
  However, the study of functors is itself not complete without the study of the morphisms between them: the natural transformations.

  \begin{defn}
    Consider two functors $F,G$ from a category $\Aa$ to $\Bb$. A \termin{natural transformation} $\alpha\colon F\then G$ from $F$ to $G$ is a famliy of morphisms\glsadd{naturalTrans}
    \begin{equation*}
    (\alpha_A\colon F(A)\To G(A))_{A\in\ob\Aa}
    \end{equation*}
    of $\Bb$ indexed by the objects in $\Aa$ satisfying the following commutative diagrams for every morphism $f\colon A \to A'$ in $\Aa$
    \begin{displaymath}
      \xymatrix{
         F(A)\ar[r]^{\alpha_A}\ar[d]_{F(f)}&G(A)\ar[d]^{G(f)}\\
         F(A')\ar[r]^{\alpha_{A'}}&G(A')
      }
    \end{displaymath}
    i.e. $\alpha_{A'}\circ F(f) = G(f) \circ \alpha_{A}$.
  \end{defn}

  Let $F,G,H$ be functors from $\Aa$ to $\Bb$ and $\alpha\colon F\then G, \beta\colon G\then H$ be natural transformations. Then the formula
  \begin{equation*}
    (\beta\circ\alpha)_A = \beta_A\circ\alpha_A
  \end{equation*}
  defines a new natural transformation $\beta\circ\alpha\colon F\then H$.

  This composition law is clearly associate and possesses identity for each functor.
  Thus, a careless argument would deduce the existence of a category whose objects are the functors from $\Aa$ to $\Bb$ and whose morphisms are the natural transformations between them. Such a category is called the \termin{functor category} from $\Aa$ to $\Bb$ and usually denoted by $[\Aa,\Bb]$ or $\Bb^{\Aa}$. \glsadd{Fun}
  \begin{rem}
    We say this argument is careless since there is a size issue:

    If $\Aa$ and $\Bb$ are \emph{small}, then $[\Aa,\Bb]$ is also \emph{small}.

    If $\Aa$ is \emph{small} and $\Bb$ is \emph{locally small}, then $[\Aa,\Bb]$ is still \emph{locally small}.

    Even if $\Aa$ and $\Bb$ are \emph{locally small}, if $\Aa$ is not \emph{small}, then $[\Aa,\Bb]$ will usually not be \emph{locally small}.

    As a partial converse to the above, if $\Aa$ and $[\Aa,\Set]$ are \emph{locally small}, then $\Aa$ must be \emph{essentially small}: see \href{http://tac.mta.ca/tac/volumes/1995/n9/1-09abs.html}{\emph{Freyd \& Street (1995)}}.
  \end{rem}

  In the above discussion, we have used a first composition law for natural transformations.
  In fact, there exists another possible type of composition for natural transformations.

  \begin{prop}
    Consider the following situation:
      \begin{displaymath}
        \xymatrix{
           \Aa\rtwocell^{F}_{G}{\alpha} &\Bb\rtwocell^{F'}_{G'}{\beta} & \Cc
        }
      \end{displaymath}
    Where $\Aa,\Bb,\Cc$ are categories, $F,G,F',G'$ are functors and $\alpha,\beta$ are natural transformations.

    First, we have the composite functors $F'F$ and $G'G$ and a commutative square for every object $A\in\ob\Aa$:
     \begin{displaymath}
        \xymatrix{
           F'F(A)\ar[r]^{F'(\alpha_A)}\ar[d]_{\beta_{F(A)}} & F'G(A)\ar[d]^{\beta_{G(A)}}\\
           G'F(A)\ar[r]^{G'(\alpha_A)} & G'G(A)
        }
    \end{displaymath}

    Now define $(\beta\ast\alpha)_A$ to be the diagonal of this square, i.e.
    \begin{equation*}
      (\beta\ast\alpha)_A = \beta_{G(A)}\circ F'(\alpha_A) = G'(\alpha_A)\circ\beta_{F(A)}
    \end{equation*}
    Then $\beta\ast\alpha$ is also a natural transformation, called the \termin{Godement product} of $\alpha$ and $\beta$. \glsadd{Godpord}
  \end{prop}

  Use the naturalities and functorialities, one can easily check this proposition and also the following.

  \begin{prop}[Interchange law]
  Consider this situation
      \begin{displaymath}
        \xymatrix{
          \Aa \ruppertwocell^{}_{}{\alpha} \rlowertwocell^{}_{}{\beta} \ar[r]
          & \Bb\ruppertwocell^{}_{}{\alpha'} \rlowertwocell^{}_{}{\beta'} \ar[r]
          & \Cc
        }
      \end{displaymath}
    Where $\Aa,\Bb,\Cc$ are categories and $\alpha,\beta, \alpha', \beta'$ are natural transformations. Then the following equality holds:
    \begin{equation*}
      (\beta'\circ\alpha')\ast(\beta\circ\alpha) = (\beta'\ast\beta)\circ(\alpha'\ast\alpha)
    \end{equation*}
  \end{prop}

  For the sake of brevity and with the notations of the previous propositions, we shall often write $\beta\ast F$ instead of $\beta\ast1_{F}$ or $G\ast\alpha$ instead of $1_{G}\ast\alpha$.

\subsection{Remarks on naturality}
  You may have seen the word ``\emph{natural}'' on different occasions. But what does this word mean? Intuitively, it makes reference to a description which is independent of any choices.

  Category theory offer a formal definition.

  Recall such a word usually occurs in a satiation where something be transformed into another. To say this process is \emph{natural}, in the sense of category theory, means it can be realized by a \emph{natural transformation}.

  For instance, the term ``naturally isomorphic'' can be formalized by
  \begin{defn}
    Let $\alpha$ be a natural transformation between two functors $F$ and $G$ from the category $\Aa$ to $\Bb$.
    If, for every object $A$ in $\Aa$, the morphism $\alpha_A$ is an isomorphism in $\Bb$, then $\alpha$ is said to be a
    \termin{natural isomorphism}, and $F$ and $G$ are said to be \termin{naturally isomorphic}, denoted by $F\cong G$.
  \end{defn}

  \begin{exam}[Opposite group]
    Statements such as
    \begin{quote}
      ``Every group is \emph{naturally isomorphic} to its opposite group''
    \end{quote}
    abound in modern mathematics.

    What the above statement really means is:
    \begin{quote}
      ``The identity functor $\Id \colon \Grp \To \Grp$ is \emph{naturally isomorphic} to the opposite functor $\op \colon \Grp \To \Grp$.''
    \end{quote}

    Such a translation also automatically provide a proof to the original statement.
  \end{exam}

  \begin{exam}[Double dual]
    Let $k$ be a field, then for every vector space $V$ over $k$ we have a ``natural'' injective linear map $V \To V^{\ast\ast}$ from $V$ into its double dual. These maps are ``natural'' in the following sense: the double dual operation is a functor, and the maps are the components of a natural transformation from the identity functor to the double dual functor.
  \end{exam}

  However, ``unnatural'' isomorphisms also abound in traditional mathematics.
  \begin{exam}[Dual of finite-dimensional vector spaces]
  Origin:
  \href{http://en.wikipedia.org/wiki/Natural_transformation#Example:_dual_of_a_finite-dimensional_vector_space}{\emph{Wikipedia}}
  Revised by:
  \href{http://mathoverflow.net/a/139398/43771}{\emph{MathOverflow}}

   Every finite-dimensional vector space is isomorphic to its dual space, but this isomorphism is not natural.

   One reason, which is given by \emph{Wikipedia}, is that this isomorphism relies on an arbitrary choice of isomorphism. However, this is a completely different matter and has nothing to do with naturality: the linear dual is a contravariant functor while the identity functor on $\FinVect_{k}$ is covariant, thus there is no possibility to compare them via a natural transformation.

   A more acceptable reason comes from a poset in \emph{MathOverflow}. Dan Petersen pointed that if we just consider the category of finite-dimensional vector spaces and linear isomorphisms, temporarily denote it by $\Cc$. Then there are two obvious functors $\Cc\to\Cc^{\op}$: the linear dual, and the natural isomorphism $\Cc\to\Cc^{\op}$ maps each linear isomorphism to its inverse. These functors are \textbf{unnaturally isomorphic}.

   However, take as objects finite-dimensional vector spaces with a nondegenerate bilinear form and maps linear transforms that respect the bilinear form. Then the resulting category has a natural isomorphism from the linear dual to the identity.
  \end{exam}

  To formalize the ideal that some isomorphisms are not natural, one can introduce the notion of \termin{infranatural transformation}, which is just a family of morphisms indexed by the objects in the source category. Thus an \termin{unnatural isomorphism} is just an infranatural isomorphism which is not natural.

  \begin{exam}
    Quote from \href{http://mathoverflow.net/a/139392/43771}{\emph{MathOverflow}}

    Take $\Cc$ to be the category with one object and two morphisms. Then the identity functor is \textbf{unnaturally isomorphic} to the functor that sends both morphisms to the identity map.
  \end{exam}

  In practice, a particular map between individual objects is said to be a \textbf{natural isomorphism}, meaning implicitly that it is actually defined on the entire category, and defines a natural transformation of functors, otherwise, an \textbf{unnatural isomorphism}.
  \begin{rem}
    Some authors distinguish notations, using $\cong$ for natural isomorphisms and $\approx$ for isomorphisms may not be natural, reserving $=$ for equalities.
  \end{rem}

  The examples in \emph{Wikipedia} may not fit, since what they compared are in fact functors with different domains.
   Some right counterexamples can be found in this poset in \href{http://mathoverflow.net/questions/139388/example-of-an-unnatural-isomorphism}{\emph{MathOverflow}}.

\newpage\section{Contravariant functors}
  Sometimes, we will consider a mapping between categories which reverse the arrows, for instance, the inverse image mapping of functions. Such kind of mappings are essentially functors from the dual of the ordinary category, while people used to image them as ``functors'' from the ordinary one for brevity.
  For this reason, categorists introduce the concept of contravariant functors.
  \begin{defn}
    Let $\Aa, \Bb$ be two categories, a \termin{contravariant functor} from $\Aa$ to $\Bb$ is a functor from $\Aa^{\op}$ to $\Bb$.
  \end{defn}
  Ordinary functors are also called \termin{covariant functor} in order to distinguish them from \emph{contravariant} ones.
  \begin{exam}
    A contravariant functor from a category $\Cc$ to $\Set$ is traditionally called a \termin{presheaf} on $\Cc$. The category of presheaves on $\Cc$ is denoted by $\PSh(\Cc)$. More generally, it is frequently to call a contravariant functor from $\Cc$ to $\Dd$ a $\Dd-$valued presheaf on $\Cc$.
  \end{exam}
  \begin{rem}
    For the reason of size issue, people often require the domain of a presheaf to be small.
  \end{rem}
  One define the natural transformations between contravariant functors similarly as the covariant case.
  \begin{defn}
    Consider two contravariant functors $F,G$ from a category $\Aa$ to a category $\Bb$. A \termin{natural transformation} $\alpha\colon F\then G$ from $F$ to $G$ is a famliy of morphisms
    \begin{equation*}
    (\alpha_A\colon F(A)\To G(A))_{A\in\ob\Aa}
    \end{equation*}
    of $\Bb$ indexed by the objects in $\Aa$ satisfying the following commutative diagrams for every morphism $f\colon A \to A'$ in $\Aa$
    \begin{displaymath}
      \xymatrix{
         F(A)\ar[r]^{\alpha_A}&G(A)\\
         F(A')\ar[r]^{\alpha_{A'}}\ar[u]^{F(f)}&G(A')\ar[u]_{G(f)}
      }
    \end{displaymath}
    i.e. $\alpha_{A}\circ F(f) = G(f) \circ \alpha_{A'}$.
  \end{defn}

  All results about functors can be transported to the contravariant case. One can easily check that, or just apply the duality principle to obtain this transposition.

  The similar idea that take both the dual of source and target categories provides the following notion.
  \begin{defn}
    Every functor $F\colon\Aa\To\Bb$ induces the \termin{opposite functor} $F^{\op}\colon \Aa^{\op} \To \Bb^{\op}$ maps objects and morphisms identically to $F$.  \glsadd{oppositeF}
  \end{defn}
  Although $F^{\op}$ works as $F$, it can be distinguished from $F$ since $\Aa^{\op}$ does not coincide as $\Aa$ as categories and similarly for $\Bb^{\op}$. The motivation to define opposite functor is similar as opposite ring, it inverse the order of composition law.

\subsection{Examples of functors}
  Here is some simple examples of functors and contravariant functors.
  \begin{exam}
    Every category $\Cc$ has an identity functor $\Id_{\Cc}$ as its identity under the composition of functors. \glsadd{identityF}
  \end{exam}
  \begin{exam}
    For any ``concrete'' category, for instance $\Grp$, there is a functor from it to $\Set$, called the \termin{forgetful functor}: it maps a group $G$ to the underlying set $G$ and a homomorphism $f$ to the corresponding function $f$. We will introduce the technical definition of concrete category and the forgetful functor later.
  \end{exam}
  \begin{exam} \glsadd{powerF} \glsadd{copowerF}
    The \emph{power-set functor} $\Pp\colon\Set\to\Set$ from the category of sets to itself is obtained by mapping a set $S$ to its power set $\Pp(S)$ and a function $f\colon A\to B$ to the ``direct image mapping'' from $\Pp(A)$ to $\Pp(B)$.

    Its duality, the \emph{contravariant power-set functor} $\Qq$ maps a set $S$ to its power set $\Pp(S)$ and a function $f\colon A\to B$ to the ``inverse image mapping'':
    \mapdes{\Pp(B)}{\Pp(A)}{U}{f^{-1}(U)}
  \end{exam}
  \begin{exam}\glsadd{constantF}
    The functor $\Delta_B\colon\Aa \To \Bb$ which maps every object of $\Aa$ to a fixed object $B$ in $\Bb$ and every morphism in $\Aa$ to the identity morphism on $B$ is called a \termin{constant functor} or \termin{selection functor} to $B$.
  \end{exam}
  \begin{exam}\glsadd{diagonalF}
    The \termin{diagonal functor} $\Delta$ is defined as the functor from $\Bb$ to the functor category $[\Aa,\Bb]$ which sends each object $B$ in $\Bb$ to the constant functor to $B$.
  \end{exam}
  \begin{exam}\label{exam:hom bifunctor}
    $\Hom(-,-)$ can itself be viewed as two functors:

    Fix an object $A$ in $\Cc$, then $X\mapsto\Hom(A,X)$ defined a functor from $\Cc$ to $\Set$ which maps a morphism $f\colon X\to Y$ to the the mapping
    \longmapdes{f_{\ast}}{\Hom(A,X)}{\Hom(A,Y)}{\phi}{f\circ\phi}
    \glsadd{pushforward}

    Fix an object $B$ in $\Cc$, then $X\mapsto\Hom(X,B)$ defined a contravariant functor from $\Cc$ to $\Set$ which maps a morphism $f\colon X\to Y$ to the the mapping
    \longmapdes{f^{\ast}}{\Hom(Y,B)}{\Hom(X,B)}{\phi}{\phi\circ f}
    \glsadd{pullback}

    Moreover, one can prove that, for every morphisms $A\to B$ and $C\to D$ in $\Cc$, the following diagram is commutative:
    \begin{displaymath}
      \xymatrix{
         \Hom(A,C)\ar[r]&\Hom(A,D)\\
         \Hom(B,C)\ar[r]\ar[u]&\Hom(B,D)\ar[u]
      }
    \end{displaymath}
  \end{exam}

\newpage\section{Full and faithful functors}
  One may find that a ``concrete'' category seems can be included in $\Set$ via the forgetful functor. But such a ``inclusion'' is not like coincide those in usual sense because it is not usually injective. For instance, there are in general many different group structures on a same set. In fact, such a functor is what we call ``faithful functor''.
  \begin{defn}
  A Functor $F\colon \Aa\to\Bb$ is said to be
  \begin{enumerate}
%    \setlength{\itemindent}{2ex}
    \item \termin[faithful]{faithful functor} (resp. \termin[full]{full functor}, resp. \termin[fully faithful]{fully faithful functor}) if for any $X,Y\in\ob\Aa$, the map $\Hom_{\Aa}(X,Y)\to\Hom_{\Bb}(F(X),F(Y))$ is injective (resp. surjective, resp. bijective).
    \item \termin[essentially surjective]{essentially surjective functor} if for each $B\in\ob\Bb$, there exists $A\in\ob\Aa$ and an isomorphism $F(A)\approx B$.
  \end{enumerate}
  \end{defn}
  \begin{rem}
  A faithful functor need not to be injective on objects or morphisms. That is, two objects $X$ and $X'$ may map to the same object in $\Bb$ (which is why the range of a fully faithful functor is not necessarily equivalent to $\Aa$),
  and two morphisms $f \colon X\To Y$ and $f' \colon X'\To Y'$ (with different domains and codomains) may map to the same morphism in $\Bb$.

  Likewise, a full functor need not be surjective on objects or morphisms. There may be objects in $\Bb$ not of the form $F(A)$ for some $A$ in $\Aa$. Morphisms between such objects clearly cannot come from morphisms in $\Aa$.
  \end{rem}

  However, we have
  \begin{prop}
    A functor is injective on morphisms if and only if it is faithful and injective on objects.
    Such a functor is called an \termin[embedding]{embedding (category theory)}.
  \end{prop}

  The basic facts about the above notions are the following
  \begin{prop}\label{prop:tri-(full,faithful)}
    Let $F\colon\Aa\to\Bb$ and $G\colon\Bb\to\Cc$ be functors.
    \begin{enumerate}
      \item If $F$ and $G$ are both isomorphisms (resp. embeddings, faithful, or full), then so is $G\circ F$.
      \item If $G\circ F$ is an embedding (resp. faithful), then so is $F$.
      \item If $F$ is essentially surjective and $G\circ F$ is full, then $G$ is full.
    \end{enumerate}
  \end{prop}
  \begin{proof}
    \emph{1} is obvious. Apply the triangle lemma to hom-sets, we get \emph{2}.

    Under the condition of \emph{3}, each object $B$ in $\Bb$ is isomorphic to some $F(A)$ for $A\in\ob\Aa$. Thus a morphism $h\colon G(B)\to G(B')$ in $\Cc$ is isomorphic to a morphism $h'\colon GF(A)\approx G(B)\to G(B')\approx GF(A')$. By the fullness of $G\circ F$, $h'=GF(f)$ for some morphism $f\colon A\to A'$ in $\Aa$, thus we get a morphism $g\colon B\to B'$ such that $G(g)=h$.
  \end{proof}

  Full functors and faithful functors have good properties.
  \begin{prop}
    Let $F\colon\Aa\to\Bb$ be a faithful functor, then
    \begin{enumerate}
      \item it \termin[reflects monomorphisms]{reflect (category theory)}. That is, for every morphism $f$ in $\Aa$, $F(f)$ is monic implies $f$ is monic.
      \item it \textbf{reflects epimorphism}. That is, for every morphism $f$ in $\Aa$, $F(f)$ is epi implies $f$ is epi.
      \item if it is also full, then it \textbf{reflects isomorphisms}.
    \end{enumerate}
  \end{prop}
  \begin{proof}
    Let $F(f)$ be monic, then for every morphisms $g,h$ such that $f\circ g=f\circ h$, we have $F(f)\circ F(g) = F(f)\circ F(h)$, thus $F(g)=F(h)$. Since $F$ is faithful, $g=h$. The proof of \emph{2} is similar.

    Let $F(f)$ be an isomorphism with inverse $g$. Since $F$ is full, $g=F(h)$ for some $\Aa-$morphism $h$. Then $h$ is the inverse of $f$ since $F$ is faithful.
  \end{proof}

  \begin{cor}
    A fully faithful functor $F\colon\Aa\to\Bb$ is necessarily injective on objects up to isomorphism. That is, every objects $A,A'$ in $\Aa$ which are mapped to isomorphic objects in $\Bb$ must be isomorphic.
  \end{cor}

  One can easily check that every functor $F\colon\Aa\to\Bb$ must \termin[preserve isomorphisms]{preserve (category)}, that is, for every isomorphism $f$ in $\Aa$, $F(f)$ is an isomorphism.
  However, the conditions to preserve monomorphisms and epimorphisms are much stronger. In fact, even fully faithful functors can not promise this.
  But if we assume $F$ to be both fully faithful and essentially surjective, one can verify it do preserve monomorphisms and epimorphisms.
  \begin{defn}
    A functor is said to be a \termin{weak equivalence}, if it is fully faithful and essentially surjective.
  \end{defn}

  In fact, weak equivalence preserves and reflects more than just those properties mentioned above, by suitable assumptions on foundations, it keeps every interesting categorical properties, thus plays an important role in modern mathematics.

\subsection{Equivalence of categories}
  To characterize the concept that two categories share the same properties, the most natural way is consider the isomorphisms in $\Cat$ (or $\CAT$, more generally). Thus, we define
  \begin{defn}
    Two categories $\Aa$ and $\Bb$ are said to be \termin[isomorphic]{isomorphic categories}, if there exists two functors $F\colon\Aa\to\Bb$ and $G\colon\Bb\to\Aa$ which are mutually inverse to each other.
  \end{defn}
  One can easily check that the inverse of an isomorphism $F$ must be unique, thus usually denoted by $F^{-1}$.

  \begin{prop}
    A functor is an isomorphism if and only if it is full, faithful, and bijective on objects.
  \end{prop}

  \begin{exam}
  The basic fact in the representation theory of finite groups is that the category of linear representations of a finite group is isomorphic to the category of left modules over the corresponding group algebra.
  \end{exam}

  However, such example is rare since the condition to be isomorphic is too strong to be satisfied in practice.
  A more pragmatic notion is the equivalence of categories.
  \begin{defn}
    Two categories $\Aa$ and $\Bb$ are said to be \termin{equivalent}, if there exists a pair of functors $F\colon\Aa\to\Bb$ and $G\colon\Bb\to\Aa$ such that there are natural isomorphisms $F\circ G\cong \Id_{\Bb}$ and $\Id_{\Aa}\cong G\circ F$. In this case, we say $F$ is an \termin{equivalence} from $\Aa$ to $\Bb$ with a \termin{weak inverse} $G$.
  \end{defn}
  \begin{rem}
    Knowledge of an equivalence is usually not enough to reconstruct its weak inverse and the natural isomorphisms: there may be many choices. (see examples below)
  \end{rem}
  \begin{rem}
    There is no standard notation of equivalentness, $\Aa\equiv\Bb$ and $\Aa\simeq\Bb$\glsadd{equivalent} are frequently used.
  \end{rem}

  The most obvious relationship between equivalence and weak equivalence is the following.
  \begin{prop}
    An equivalence must be a weak equivalence.
  \end{prop}
  \begin{proof}
    Let $F\colon\Aa\to\Bb$ be an equivalence with a weak inverse $G$ and two natural isomorphism $\Id_{\Aa}\iso{\alpha} G\circ F, F\circ G \iso{\beta} \Id_{\Bb}$.
    The essential surjectivity follows from $\beta$, full faithfulness follows from $\alpha$.
  \end{proof}

  Before going forward, we prove some basic properties of weak equivalences.
  \begin{prop}
    If $F\colon\Aa\to\Bb$ and $G\colon\Bb\to\Cc$ are weak equivalences, then so is $G\circ F$.
  \end{prop}
  \begin{proof}
    The fullness and faithfulness follows from Proposition \ref{prop:tri-(full,faithful)}. The essential surjectivity is easy to check.
  \end{proof}

  Assuming the axiom of choice true, we have
  \begin{prop}
    $F\colon\Aa\to\Bb$ is an equivalence if and only if it is a weak equivalence.
  \end{prop}
  \begin{proof}
    Let $F\colon\Aa\to\Bb$ be a weak equivalence, then we constitute a weak inverse $G\colon\Bb\to\Aa$ as following.

    For each $\Bb-$object $B$, CHOICE an $\Aa-$object $A$ such that $F(A)\approx B$, set $G(B)=A$.
    For each $\Bb-$morphism $f\colon B\to B'$, set $G(f)$ to be the inverse image of following composition
    \begin{equation*}
      FG(B)\approx B\markar{f} B'\approx FG(B')
    \end{equation*}
    in $\Aa$, the existence and uniqueness of such an inverse image comes from the fact that $F$ is fully faithful.

    $F\circ G\cong \Id_{\Bb}$ comes from the constitution above. To see $\Id_{\Aa}\cong G\circ F$, just notice that $F$ reflects isomorphisms.
  \end{proof}

  However, if we do not assume the axiom of choice, then there is a related concept weaker than equivalence.
  \begin{defn}
    Two categories $\Aa$ and $\Bb$ are said to be \termin{weak equivalent}, if there exist a category $\Cc$ and two weak equivalence $F\colon\Cc\to\Aa$ and $G\colon\Cc\to\Bb$.
  \end{defn}

\subsection{Examples}
\begin{exam}
  The basic fact in the linear algebra is that $\Mat$ is equivalent to the category $\FinVect$ of finite-dimensional vector spaces and linear mappings. However, they are not isomorphic.
\end{exam}
\begin{exam}
  Let $\Cc$ a category with two object $A,B$, and four morphisms: two are the identities $1_A, 1_B$, two are isomorphisms $f\colon A\to B$, $g\colon B\to A$.
  Then $\Cc$ is equivalent to $\one$ through the functor which maps every objects in $\Cc$ to $\bullet$, and every morphisms in $\Cc$ to the identity.
  However, there are two weak inverse: one maps $\bullet$ to $A$ and the other to $B$.
\end{exam}
\begin{exam}
  Consider a category $\Cc$ with one object $X$, and two morphisms $1_X, f\colon X\to X$, where $f \circ f = 1_X$.
  Of course, $\Cc$ is equivalent to itself and $1_{\Cc}$ is an equivalence. However, there are two natural isomorphisms: one induced from $1_X$ and the other from $f$. This example shows that even the weak inverse is unique, the choices of natural isomorphisms may not.
\end{exam}

\newpage\section{Subcategories}
  Like subsets, subgroups and so on, we now introduce the notion of subcategories in the similar way.
  \begin{defn}
    A \termin{subcategory} of $\Cc$ is a category whose objects and morphisms form subcollections of $\Cc$'s respectively.
  \end{defn}
    The definition of subcategory implicit a functor which works as an inclusion on the collections of objects and morphisms, named the \termin{inclusion functor}.

  There are two different notions formalize the idea that a subcategory which is big enough to reveal the entire category.
  \begin{defn}
    If the inclusion functor is full, then we say the subcategory is \termin[full]{full subcategory}.
    If the inclusion functor is surjective on objects, then we say the subcategory is \termin[wide]{wide subcategory} (or \termin[lluf]{lluf category}).
  \end{defn}

  It is obvious that the inclusion functor is necessarily an embedding.
  Conversely,
  just as subsets of a set $S$ can be identified with isomorphism classes of injective functions into $S$, subcategories of a category $\Cc$ can be identified with isomorphism classes of monic functors into $\Cc$.
  A functor is easily verified to be monic (in $\Cat$, or $\CAT$ more generally) if and only if it is an embedding.

  Consequently, inclusions are (up to isomorphism) precisely the embeddings.
  \begin{prop}
    A functor $F\colon \Aa \to \Bb$ is a (full) embedding if and only if it factors through a (full) subcategory $\Cc$ of $\Bb$ by an isomorphism $G\colon \Aa \to \Cc$ and an inclusion $E\colon \Cc \to \Bb$. That is $F=E\circ G$.
  \end{prop}
  \begin{proof}
    Set $\Cc$ to be the image of $\Aa$ in $\Bb$.
  \end{proof}
  Moreover, inclusions are (up to (weak) equivalence) precisely the faithful functors.
  \begin{prop}
    A functor $F\colon \Aa \to \Bb$ is faithful if and only if it factors through an inclusion $E_1\colon\Aa\to\Cc$, a weak equivalence $G\colon\Cc\to\Dd$, and an inclusion $E_2\colon\Dd\to\Bb$. That is $F=E_2\circ G\circ E_1$.
  \end{prop}
  \begin{proof}
    Set $\Dd$ to be the full subcategory of $\Bb$ whose objects are same as the image of $\Aa$.
    Set $\Cc$ to be the category whose objects are same as $\Aa$, while morphisms as $\Dd$.
  \end{proof}

  Categories can be classified into different equivalence classes, to characterize this, we have
  \begin{defn}
    A category is said to be \termin[skeletal]{skeletal category} if its objects that are isomorphic are necessarily equal.
    Traditionally, a \termin[skeleton]{skeleton (category theory)} of a category $\Cc$ is defined to be a skeletal subcategory of $\Cc$ whose inclusion functor exhibits it as equivalent to $\Cc$.
  \end{defn}
  \begin{rem}
    However, in the absence of the axiom of choice, it is more appropriate to define a skeleton of $\Cc$ to be any skeletal category which is weakly equivalent to $\Cc$.
  \end{rem}
  \begin{defn}
    A category is said to be \termin[essentially small]{essentially small category} if it is equivalent to a small category. Assuming the axiom of choice, this is the same as saying that it has a small skeleton.
  \end{defn}
  \begin{prop}
    Any two skeletons of a category are isomorphic. Conversely, two categories are equivalent if and only if they have isomorphic skeletons.
  \end{prop}
  \begin{rem}
    In the absence of the axiom of choice, the term ``equivalent'' must be replaced by ``weak equivalent''.
  \end{rem}

\subsection{Terminological remark}
    In many fields of mathematics, objects satisfying some ``universal property'' are not unique on the nose, but only \emph{unique up to unique isomorphism}. It can be tempting to suppose that in a skeletal category, where any two isomorphic objects are equal, such objects will in fact be unique on the nose. However, under the most appropriate definition of ``unique'' this is \textbf{not} true (in general), because of the presence of automorphisms.

    More explicitly, consider the notion \emph{cartesian product} (see Definition \ref{def:product}) in a category as an example. Although we colloquially speak of  ``a product'' of objects $A$ and $B$ as being the object $A\times B$, strictly speaking a product is a triple which consists of the object $A\times B$ together with the projections $A\times B\to A$ and $A\times B\to B$ which exhibit its universal property.
    Thus, even if the category in question is skeletal, so that there can be only one object $A\times B$ that is a product of $A$ and $B$, in general this object can still ``be the product of $A$ and $B$'' in many different ways (in the sense that the projection maps are different): those different ways are then related by an automorphism of the object.

    Finally, it is true in a few cases, though, that skeletality implies uniqueness on the nose. For instance, a \emph{terminal object} (see Definition \ref{def:universal-object}) can have no nonidentity automorphisms, so in a skeletal category, a terminal object (if one exists) really is unique on the nose.

\newpage\section{Comma categories}
  We indicate now a quite general process for constructing new categories from given ones. This type of construction will be used very often in this book.

  \begin{defn}
    Suppose that $\Aa$, $\Bb$, and $\Cc$ are categories, and $S$ and $T$ (for source and target) are functors
          \begin{displaymath}
            \xymatrix{
               \Aa\ar[r]^{S} & \Cc & \Bb\ar[l]_{T}                }
          \end{displaymath}
    We can form the \termin{comma category} $(S\down T)$ as follows:  \glsadd{commma}
    \begin{itemize}
      \item The objects are all triples $(A,f,B)$ with $A$ an object in $\Aa$, $B$ an object in $\Bb$, and $f\colon S(A)\To T(B)$ a morphism in $\Cc$.
      \item The morphisms from $(A,f,B)$ to $(A',f',B')$ are all pairs $(g,h)$
                 where $g\colon A\To A'$ and $h\colon B\To B'$ are morphisms in $\Aa$ and $\Bb$ respectively, such that the following diagram commutes:
                 \begin{displaymath}
                   \xymatrix{
                       S(A)\ar[d]_{S(g)}\ar[r]^{f} & T(B)\ar[d]^{T(h)}  \\
                       S(A')\ar[r]^{f'} & T(B')           }
                 \end{displaymath}
      \item Morphisms are composed by taking $(g,h)\circ(g',h')$ to be $(g\circ g',h\circ h')$, whenever the latter expression is defined.
      \item The identity morphism on an object $(A,f,B)$ is $(1_{A},1_{B})$.
    \end{itemize}
  \end{defn}
  \begin{rem}
    Some people prefer the notation $(S/T)$ rather than $(S\down T)$.
  \end{rem}

  The following proposition provide the ``universal property'' of the comma category.
  \begin{prop}\label{prop:comma-uni}
    For each comma category there are two canonical \termin[forgetful functors]{forgetful functor} from it.
    \begin{itemize}
      \item \termin{domain functor}, $U\colon(S\down T)\To\Aa$, which maps:
      \begin{itemize}
        \item objects: $(A,f,B)\mapsto A$;
        \item morphisms: $(g,h)\mapsto g$;
      \end{itemize}
      \item \termin{codomain functor}, $V\colon(S\down T)\To\Bb$, which maps:
      \begin{itemize}
        \item objects: $(A,f,B)\mapsto B$;
        \item morphisms: $(g,h)\mapsto h$;
      \end{itemize}
    \end{itemize}
    Meanwhile, there exists a natural transformation $\alpha\colon S\circ U \then T\circ V$.
                 \begin{displaymath}
                   \xymatrix{
                       (S\down T)\ar[r]^-{V}\ar[d]_-{U}
                       &\Bb\ar[d]^{T}\\
                       \Aa\ar[r]_{S} \ar@{}[ur]^{\alpha}|-{\SelectTips{eu}{}\object@{=>}}
                       &\Cc %\ultwocell\omit
                               }
                 \end{displaymath}

    Moreover, comma category is the universal one respect to the above property.
    That is, if there exist another category $\Dd$ together with two functor $U'\colon\Dd\to\Aa$ and $V'\colon\Dd\to\Bb$ such that there exists a natural transformation $\alpha'\colon S\circ U' \then T\circ V'$.
                 \begin{displaymath}
                   \xymatrix{
                       \Dd\ar[r]^-{V'}\ar[d]_-{U'}
                       &\Bb\ar[d]^{T}\\
                       \Aa\ar[r]_{S} \ar@{}[ur]^{\alpha'}|-{\SelectTips{eu}{}\object@{=>}}
                       &\Cc %\ultwocell\omit
                               }
                 \end{displaymath}
    Then there exist a unique functor $W\colon\Dd\to(S\down T)$ such that
    \begin{equation*}
      U\circ W = U'\qquad V\circ W = V'\qquad \alpha\ast W = \alpha'.
    \end{equation*}
  \end{prop}
  \begin{proof}
    The property follows just from the definition of the comma category, where the natural transformation $\alpha$ is defined componentwise by
    \begin{equation*}
      \alpha_{(A,f,B)}=f
    \end{equation*}

    If there exist another quadruple $(\Dd,U',V',\alpha')$ satisfies the property, then we can define a functor $W\colon\Dd\to (S\down T)$ by
    \begin{align*}
      W(D) &= (U'(D),\alpha'_D ,V'(D)) \\
      W(f) &= (U'(f), V'(f))
    \end{align*}
    It is easy to check that
    \begin{equation*}
      U\circ W = U'\qquad V\circ W = V'\qquad \alpha\ast W = \alpha'.
    \end{equation*}

    Conversely, the above equations enforce any functor satisfies them must be equal to $W$.
  \end{proof}

\subsection{Examples}
  \begin{exam}%[Slice category]
    $(\Id_{\Cc} \down \Delta_I)$, also denoted by $(\Cc \down I)$ is called the \termin{slice category} over $I$ or \emph{the category of objects over $I$}.
  \end{exam}

  \begin{exam}%[Coslice category]
    $(\Delta_I \down \Id_{\Cc})$, also denoted by $(I \down \Cc)$ is called the \termin{coslice category} under $I$ or \emph{the category of objects under $I$}.
  \end{exam}

  \begin{exam}%[Arrow category]
    $(\Id_{\Cc}\down\Id_{\Cc})$ is the \termin{arrow category} $\Cc^{\to}$.
  \end{exam}

  \begin{exam}\glsadd{T-arrow}\glsadd{S-arrow}
    In the case of the slice or coslice category, the identity functor may be replaced with some other functor $F$; this yields a family of categories particularly useful in the study of adjoint functors. Let $s,t$ be given object in $\Cc$.
    An object of $(s\down F)$ is called a \emph{morphism from $s$ to $F$} or a \termin{$F-$structured arrow} with domain $s$ in.
    An object of $(F\down t)$ is called a \emph{morphism from $F$ to $t$} or a \termin{$F-$costructured arrow} with codomain $t$ in.
  \end{exam}

  \begin{exam}\label{exam:Elts}
    Let $F\colon\Cc\to\Set$ be a functor, $1\colon\one\to\Set$ be the functor maps the object of $\one$ to a singleton. The the comma category $(1\down F)$ is called the \termin{category of elements of $F$}, denoted by $\Elts(F)$. It can be explicitly described in the following way.
  \begin{itemize}
    \item Objects: the pairs $(X,x)$, where $X\in\ob\Cc$ and $x\in F(X)$.
    \item Morphisms $f\colon(A,a)\To(B,b)$:
               the morphisms $f\colon A\to B$ in $\Cc$ such that $F(f)(a)=b$.
    \item The composition law is that induced by the composition of $\Cc$.
  \end{itemize}
  The codomain functor of $\Elts(F)$ is also called the \termin{forgetful functor}.
  \end{exam}

  \begin{exam}
    Let $\Delta_{\Aa}$ denote the \termin{constant functor}\glsadd{ConstantF} from $\Aa$ to $\one$ and $\Delta_{\Bb}$ likewise. Then the comma category $(\Delta_{\Aa}\down\Delta_{\Bb})$ is just the \termin[product]{product of categories} $\Aa\times\Bb$ of $\Aa$ and $\Bb$, which can be described as below.
  \begin{itemize}
    \item Objects:
               the pairs of objects $(A, B)$, where $A$ is an object of $\Aa$ and $B$ of $\Bb$;
    \item Morphisms $f\colon(A,B)\To(A',B')$:
               the pairs of arrows $(a,b)$, where $a\colon A\to A'$ is an arrow of $\Aa$ and $b\colon B\to B'$ is an arrow of $\Bb$;
    \item The composition law is the component-wise composition from the contributing categories:
                                    \begin{equation*}
                                      (a', b') \circ (a, b) = (a' \circ a, b' \circ b);
                                    \end{equation*}
  \end{itemize}
  \end{exam}
  The product $\Aa\times\Bb$ comes with two ``\termin[projection]{projection (functor)}'' functors
  \begin{equation*}
    p_{\Aa}\colon\Aa\times\Bb\To\Aa\qquad p_{\Bb}\colon\Aa\times\Bb\To\Bb
  \end{equation*}
  which are defined by the formulas
  \begin{align*}
    p_{\Aa}(A,B)=A,&\quad p_{\Bb}(A,B) = B,\\
    p_{\Aa}(a,b) =a,&\quad p_{\Bb}(a,b) = b.
  \end{align*}

  These data satisfy the following ``universal property''.
  \begin{prop}
    Consider two categories $\Aa$ and $\Bb$. For every category $\Cc$ and every pair of functors $F\colon\Dd\to\Aa, G\colon\Dd\to\Bb$, there exists a unique functor $H\colon\Dd\to\Aa\times\Bb$ such that $p_{\Aa}\circ H=F, p_{\Bb}\circ H=G$.
  \end{prop}
  \begin{proof}
    This follows straightly from Proposition \ref{prop:comma-uni}.
  \end{proof}
  A point of terminology: a functor defined on the product of two categories is generally called a \termin{bifunctor} (a functor of two ``variables''). In practice, something is said to be \textbf{natural} or \textbf{functorial} in $X_1,X_2,\cdots$ implicits it is actually a functor of variables $X_1,X_2,\cdots$.
  \begin{exam}
    $\Hom_{\Cc}(-,-)$ is a bifunctor from $\Cc^{\op}\times\Cc$ to $\Set$.
  \end{exam}


\newpage\section{The duality principle}
  One may have noticed that every result proved for covariant functors has its counterpart for contravariant functors and every result proved for monomorphisms has its counterpart for epimorphisms.
  These facts are just special instances of a very general principle.

  In the remark under Definition \ref{def:category}, we have shown that a category provides a two-sorted first order language and categorical properties are statements in this language.

  Once we have a categorical property $\sigma$, then the dual $\sigma^{\op}$ can be obtained by reversing arrows and compositions. That is
  \begin{enumerate}
    \item   Interchange each occurrence of ``source'' in $\sigma$ with ``target''.
    \item   Interchange the order of composing morphisms.
  \end{enumerate}

  Then one can easily find that a property $\sigma$ in $\Cc$ is logical equivalent to the property $\sigma^{\op}$ in $\Cc^{\op}$.

  Consequently, we have
  \begin{thm}[The duality principle for categories]
  $ $
  \begin{center}
    Whenever a property $\Pp$ holds for all categories,

    then the property $\Pp^{\op}$ holds for all categories.
  \end{center}
  \end{thm}

  \begin{exam}
    A morphism is monic if and only if the reverse morphism in the opposite category is epi.
  \end{exam}

\newpage\section{Yoneda lemma and representable functors}
  In this section, we will prove an important theorem. Before doing this, we give the useful concept of representable functors.

  Representability is one of the most fundamental concepts of category theory, with close ties to the notion of adjoint functor and to the Yoneda lemma. It is the crucial concept underlying the idea of universal property. The concept permeates much of algebraic geometry and algebraic topology.
  \begin{defn}
    For a functor $F\colon\Cc^{\op}\to\Set$ (also called a \emph{presheaf} on $\Cc$ ), a \termin{representative} of $F$ is a specified natural isomorphism
    \begin{equation*}
      \Phi\colon\Hom_{\Cc}(-,X)\To F
    \end{equation*}
    where the object $X$ in $\Cc$ is called a \termin{representing object} (or \termin{universal object}) of $F$.

    If such a representative exists, then we say the functor $F$ is \termin[representable]{representable functor} and is \textbf{represented} by $X$.

    Similarly, a covariant functor $F\colon\Cc\to\Set$ is said to be \textbf{representable}, if it is representable when view it as a presheaf on $\Cc^{\op}$.
  \end{defn}

  Given a category $\Cc$, there exists a functor:
  \begin{equation*}\glsadd{Yoneda}
    \Upsilon\colon\Cc\To\PSh(\Cc)
  \end{equation*}
  sends any object $X\in\ob\Cc$ to the presheaf $\Hom_{\Cc}(-,X)$.

  The Yoneda lemma asserts that the set of morphisms from the presheaf represented by $X$ into any other presheaf $F$ is in natural bijection with the set $F(X)$ that this presheaf assigns to $X$.

  Formally:
  \begin{thm}[Yoneda lemma]
    There is a canonical isomorphism
    \begin{equation*}
      \Hom_{\PSh(\Cc)}(\Upsilon(X),F)\cong F(X)
    \end{equation*}
    natural in both $X$ and $F$.
  \end{thm}
  \begin{rem}
    In some literature it is customary to denote the presheaf represented by $X$ as $h_X$. In that case the above is often written
    \begin{equation*}\glsadd{Nat}
      \Nat(h_X,F)\cong F(X)
    \end{equation*}
    to emphasize that the morphisms of presheaves are natural transformations of the corresponding functors.
  \end{rem}

  \begin{proof}
    The crucial point is that any natural transformation
    \begin{equation*}
      \alpha\colon\Hom_{\Cc}(−,X)\then F
    \end{equation*}
    is entirely fixed by the value $\alpha_X(1_X)\in F(X)$ of its component
    \begin{equation*}
    \alpha_X\colon\Hom(X,X)\to F(X)
    \end{equation*}
    on the identity morphism $1_X$. And every such value extends to a natural transformation $\alpha$.

    To see this, we fix a value $\alpha_X(1_X)\in F(X)$ and consider an arbitrary object $A$ in $\Cc$. If $\Hom(A,X)=\varnothing$, then the component $\alpha_A$ must be the trivial function from empty set. If there exists a morphism $f\colon A\to X$, then by the naturality condition, the following commutative square has already been determined.
          \begin{displaymath}
            \xymatrix{
               \Hom(X,X)\ar[r]^-{\alpha_X}\ar[d]_{f^{\ast}}& F(X)\ar[d]^{F(f)} \\
               \Hom(A,X)\ar[r]^-{\alpha_A}& F(A)               }
          \end{displaymath}
    Consequently, all components of $\alpha$ have been determined.

    Conversely, given a value $a=\alpha_X(1_X)\in F(X)$, define $\alpha$ by components as following:
    \begin{equation*}
      \alpha_A(f):=F(f)(a)\quad\forall A\in\ob\Cc,\forall f\in\Hom(A,X)
    \end{equation*}
    It is easy to check this is a natural transformation.

    The naturalities on $X$ and $F$ is easy to check.
  \end{proof}

\subsection{Corollaries}
  The Yoneda lemma has the following direct consequences. Like the Yoneda lemma, they are as easily established as they are useful and important.
  \begin{cor}
    The functor $\Upsilon$ is a full embedding.
  \end{cor}
  This $\Upsilon$ is customary called the \termin{Yoneda embedding}
  \begin{proof}
    For any $A,B\in\ob\Cc$, by the Yoneda lemma, we have
    \begin{equation*}
      \Hom_{\PSh(\Cc)}(\Upsilon(A),\Upsilon(B))\cong (\Upsilon(B))(A) = \Hom_{\Cc}(A,B)
    \end{equation*}
    Thus $\Upsilon$ is fully faithful. The injectivity on objects is obvious.
  \end{proof}

  \begin{cor}\label{coro:Yoneda2}
    For any $A,B\in\ob\Cc$, we have
    \begin{equation*}
      \Upsilon(A)\cong\Upsilon(B)\iff A\cong B
    \end{equation*}
  \end{cor}
  \begin{proof}
    Since $\Upsilon$ is fully faithful, thus reflects isomorphisms.
  \end{proof}

  \begin{cor}
    Let $F$ be a presheaf on $\Cc$, then a presentation of $F$ is uniquely determined by the universal object $X$ together with an element $u\in F(X)$. Such a pair $(X,u)$ satisfies the following universal property:
    \begin{quote}
      For every pair $(A,a)$, where $A\in\ob\Cc$ and $a\in F(A)$, there is a unique morphism $f\colon A\to X$ such that $F(f)(u)=a$.
    \end{quote}
  \end{cor}
  \begin{proof}
    Notice that, a presentation $\Phi$ of $F$ is nothing but an element in the set $\Hom_{\PSh(\Cc)}(\Upsilon(X),F)$, thus by Yoneda lemma it can be corresponded to an element of $F(X)$ via $\Upsilon$, say $u\in F(X)$. Then $\Phi$ is uniquely determined by $X$ and $u$.

    On the other hand, for each object $A$ in $\Cc$, $\Phi_{A}$ gives a 1-1 corresponding between $\Hom_{\Cc}(A,X)$ and $F(A)$, thus the universal property follows.
  \end{proof}
  \begin{rem}
    Someone may doubt the existence of the morphism in this universal property. It is possible that there is not morphisms from $A$ to $X$. But in this case, Yoneda lemma ensure that $F(A)$ is empty, thus there is no pair $(A,a)$ at all.
  \end{rem}
  \begin{rem}
    The \nlab gives another description of this corollary: the presentation of $F$ is the \emph{terminal object} in the comma category $(\Upsilon\down\Delta_{F})$.
  \end{rem}

\newpage\section{Exercises}
\begin{ex}
  Let $f\colon X\to Y$ be a morphism in $\Cc$. Show that
  \begin{enumerate}
    \item $f$ is monic if and only if the induced function
               \begin{equation*}
                 f_{\ast}\colon\Hom(C,X)\to\Hom(C,Y)
               \end{equation*}
               is injective for every object $C$.
    \item $f$ is epi if and only if the induced function
               \begin{equation*}
                 f^{\ast}\colon\Hom(Y,C)\to\Hom(X,C)
               \end{equation*}
               is injective for every object $C$.
    \item $f$ is split epi if and only if the induced function
               \begin{equation*}
                 f_{\ast}\colon\Hom(C,X)\to\Hom(C,Y)
               \end{equation*}
               is surjective for every object $C$.
    \item $f$ is split monic if and only if the induced function
               \begin{equation*}
                 f^{\ast}\colon\Hom(Y,C)\to\Hom(X,C)
               \end{equation*}
               is surjective for every object $C$.
  \end{enumerate}
\end{ex}
\begin{ex}
  Let $S,T\colon\Aa\to\Bb$ be two functors. Shows that there exists a 1-1 corresponding between the set of natural transformations from $S$ to $T$ and the set of sections of both domain functor and codomain functor.
\end{ex}
\begin{ex}\label{prop:power law for functor}
  For any categories $\Aa,\Bb,\Cc$,
  \begin{enumerate}
    \item $[\Aa,\Bb]^{\op}\simeq[\Aa^{\op},\Bb^{\op}]$
    \item $(\Aa\times\Bb)^{\Cc}\simeq\Aa^{\Cc}\times\Bb^{\Cc}$
    \item $\Cc^{\Aa\times\Bb}\simeq(\Cc^{\Aa})^{\Bb}\simeq(\Cc^{\Bb})^{\Aa}$
  \end{enumerate}
\end{ex}
\begin{rec}
  $\Bb^{\Aa}$ is just another notation of $[\Aa,\Bb]$.
\end{rec}
\begin{ex}
  Show that the functor category $[\Aa,\Bb]$ is functorial in both $\Aa$ and $\Bb$. That means, $[-,-]\colon\Cat^{\op}\times\Cat\to\Cat$ is a bifunctor.
\end{ex}
\begin{ex}
  Use the duality principle to define the notion of covariant representable functor explicitly.
\end{ex}
\begin{ex}
  Show that a covariant representable functor preserves monomorphisms. On the other hand, a contravariant representable functor maps an epimorphism to a monomorphism.
\end{ex}
\begin{ex}
  Prove the covariant Yoneda lemma: There is a canonical isomorphism
    \begin{equation*}
      \Nat(\Hom(X,-),F)\cong F(X)
    \end{equation*}
    natural in both $X$ and $F$.
\end{ex}
\begin{ex}
  Consider the category $\PSh(\Cc)$ of presheaves on a small category $\Cc$, show that a morphism in $\PSh(\Cc)$ is monic if and only if its every component is monic.
  However, if we replace $\PSh(\Cc)$ by an arbitrary functor category $[\Cc,\Dd]$, then the previous statement is no longer valid. Show this by a counterexample.
\end{ex}

  \chapter{Theory of Limits}
  In this chapter, we develop a general theory about limits in categories. Cartesian product, quotient, kernel, union, intersection, and etc. are just particular cases of this theory.
\minitoc
\newpage
\section{Diagrams and cones}
  In category theory, people often draw commutative diagrams by draw some objects connected by arrows labelled by morphisms.
  For instance, every kind of limit require such diagrams to define their universal property.
  Therefore, to give a general definition of limits, we need to formalize the concept of diagram first.
  \begin{defn}
    Let $\Ii$ and $\Cc$ be categories. A \termin{diagram of shape $\Ii$} or a \termin{$\Ii-$diagram} in $\Cc$ is a functor $D\colon \Ii\to \Cc$.
  \end{defn}
  \begin{rem}
    The category $\Ii$ is called the \termin{index category} of the diagram $D$. If it is small, then we say the diagram is small; if it is finite (that means it has only finitely many objects and arrows), then we say the diagram is finite.
    For an object $i$ in the index category, we often write $D_i$ instead of $D(i)$.
  \end{rem}

\begin{defn}
  A \termin{cone} over a diagram $D$ is an object in the comma category $(\Delta\downarrow D)$, where $\Delta\colon\Cc\to[\Ii,\Cc]$ is the diagonal functor. In other words, a cone over $D$ is a natural transformation from a constant functor $\Delta_C$ to $D$.

  More specifically, a cone over $D$ consists of an object $C$ in $\Cc$ and a family of arrows in $\Cc$,
  \begin{equation*}
    (c_i\colon C\longrightarrow D_i)_{i\in\ob\Ii}
  \end{equation*}
  such that for each arrow $\alpha\colon i\to j$ in $\Ii$, the following triangle commutes.
  \begin{displaymath}
      \xymatrix{
         &C\ar[ld]_{c_i}\ar[rd]^{c_j}&\\
         D_i\ar[rr]^{D_{\alpha}}&&D_j
      }
  \end{displaymath}

  A morphism of cones
  \begin{equation*}
    \theta\colon (C,c_i)\longrightarrow (C',c'_i)
  \end{equation*}
  is an arrow $\theta$ in $\Cc$, making each triangle
  \begin{displaymath}
      \xymatrix{
         C\ar[rd]_{c_i}\ar[rr]^{\theta}&&C'\ar[ld]^{c'_i}\\
         &D_i&
      }
  \end{displaymath}
  commute.
\end{defn}
\begin{rem}
  We also denote the category of cones over $D$ by $\Cone(D)$.
\end{rem}

%  Dually, a co-cone is a cone in the dual category. That is
\begin{defn}
  A \termin{co-cone} under a diagram $D$ is an object in the comma category $(D\downarrow\Delta)$. In other words, a co-cone under $D$ is a natural transformation from $D$ to a constant functor $\Delta_C$.

  More specifically, a co-cone under $D$ consists of an object $C$ in $\Cc$ and a family of arrows in $\Cc$,
  \begin{equation*}
    (c_i\colon D_i\longrightarrow C)_{i\in\ob\Ii}
  \end{equation*}
  such that for each arrow $\alpha\colon i\to j$ in $\Ii$, the following triangle commutes.
  \begin{displaymath}
      \xymatrix{
         D_i\ar[rr]^{D_{\alpha}}\ar[rd]_{c_i}&&D_j\ar[ld]^{c_j} \\
         &C&
      }
  \end{displaymath}

  A morphism of co-cones
  \begin{equation*}
    \theta\colon (C,c_i)\longrightarrow (C',c'_i)
  \end{equation*}
  is an arrow $\theta$ in $\Cc$, making each triangle
  \begin{displaymath}
      \xymatrix{
         &D_i\ar[ld]_{c_i}\ar[rd]^{c'_i}& \\
         C\ar[rr]^{\theta}&&C'
      }
  \end{displaymath}
  commute.
\end{defn}
\begin{rem}
  We also denote the category of co-cones under $D$ by $\Cocone(D)$.
\end{rem}

\begin{rem}
  In the definition above, we mention the functor category $[\Ii,\Cc]$, which is not locally small in general and thus not a category in usual sense. To fix this, one approach is to consider small diagrams only, or at most essentially small diagrams. This works well in practice.

  A more immediate approach is to notice that in the definition, we only need the category $\Cone(D)$ and $\Cocone(D)$ to be locally small, which is always true even $\Ii$ is not essentially small.
\end{rem}

Commutative triangles, squares and, more generally, diagrams abound in category theory. We now give them a formal definition.

\begin{defn}
  Let $f\colon A\to B$ be a morphism. If there exists an object $C$ equipped with two morphisms $g\colon A\to C$ and $h\colon C\to B$ such that $f = h\circ g$. Then we say ``$f$ \termin{factors through} $C$ by $g$ and $h$'', or simply ``$f$ factors through $C$ (resp. $g$ or $h$)''. The triple $(C,g,h)$ (or one of them) is called a \termin{factorization} of $f$.
\end{defn}

\begin{defn}
  A diagram $D$ is said to be \termin[commutative]{commutative diagram}, if it factors through a poset. Informally, a diagram is commutative means its composition is path-independent.
\end{defn}

\subsection{Exercises}
  \begin{ex}
    Let $\Ii$ be one of the following categories, describe the diagrams of shape $\Ii$ and their cones and co-cones in terms of objects and morphisms.
    \begin{itemize}
      \item $\one$, which has only one object and one morphism.
      \item a \termin{discrete category}, whose only morphisms are the identities.
      \item two parallel arrows
                     \begin{displaymath}
                           \xymatrix{
                              1\ar@<0.5ex>[r]^{\alpha}\ar@<-0.5ex>[r]_{\beta} &2
                           }
                     \end{displaymath}
      \item two convergence arrows (called a \termin{cospan})
                \begin{displaymath}
                  \xymatrix{
                     \cdot\ar[r]&\cdot&\cdot\ar[l]
                  }
                \end{displaymath}
      \item two divergence arrows (called a \termin{span})
                \begin{displaymath}
                  \xymatrix{
                     \cdot&\cdot\ar[l]\ar[r]&\cdot
                  }
                \end{displaymath}
      \item a \termin{directed partially ordered set} (DPOS), which means a poset satisfies that for each two elements $i,j$, there exist another element $k\in I$ such that $i\leqslant k,j\leqslant k$.
    \end{itemize}

    For example, when $\Ii=\mathbf{0}$, the empty category, there is only one diagram of shape $\mathbf{0}$: the empty one. A cone over or a co-cone under the empty diagram is essentially just an object.
  \end{ex}

\newpage\section{Initial and terminal objects}
  \begin{defn}\label{def:universal-object}
    Let $\Cc$ be a category. An \termin{initial object} of $\Cc$ is an object $0$ in $\Cc$ such that for every object $X$ in $\Cc$, there exists precisely one morphism $0\to X$.
    Dually, an object $1$ is called a \termin{terminal object}, if for every object $X$ in $\Cc$, there exists a unique morphism $X\to 1$.
  \end{defn}
  \begin{rem}
    It is easy to see that the initial object and terminal object are unique up to unique isomorphism.
  \end{rem}

  \begin{exam}
    In the category $\Set$, the empty set is the initial object and a singleton is a terminal object. The same holds in the category $\Top$. Similar, the empty category $\mathbf{0}$ is the initial object in $\Cat$ and $\one$ is the terminal object.
  \end{exam}
  \begin{rem}
    Therefore $\one$ is sometimes called the \termin{terminal category}.
  \end{rem}
  \begin{exam}
    In the categories of groups, abelian groups, vector spaces, Banach spaces, and so on, $\{0\}$ is both the initial and the terminal object.
  \end{exam}

  Not every category has terminal objects, for example:
  \begin{exam}
    The category of infinite groups do not have a terminal object: given any infinite group $G$ there are infinitely many morphisms $\ZZ\to G$, so $G$ cannot be terminal.
  \end{exam}

  \begin{defn}
    If a category $\Cc$ has a terminal object $1$, then a \termin{global element} of another object $X$ is a morphism $1\to X$.
  \end{defn}
  \begin{exam}
    In $\Set$, global elements are just elements.
  \end{exam}
  \begin{exam}
    In $\Cat$, global elements are objects.
  \end{exam}
  \begin{exam}
    In a slice category $\Cc/I$, a global element of the object $A\to I$ is just a section of it in $\Cc$.
  \end{exam}


\newpage\section{Limits and colimits}
\begin{defn}
  A \termin{limit} of a diagram $D\colon \Ii\to\Cc$ is a terminal object in the comma category $\Cone(D)$.
  A \termin{colimit} is a initial object in $\Cocone(D)$.
  % In particular, a finite limit is a limit of a diagram on a finite index category $\Ii$.
\end{defn}
\begin{rem}
  Since the limit of a diagram $D$ is unique up to unique isomorphism, one can just write $\invlim D$ for the limit without confusion.

  However, it is usually helpful to indicate how the functor is evaluated on objects, in which case the limit is written in the form $\invlim\limits_i D_i$.\glsadd{limit}

  Since a cone is not just an object (sometimes, we colloquially call it a \textbf{limit}) but together with a family of morphisms (called \termin[projections]{projection (category theory)}), thus the full formula should be like this
  \begin{equation*}
    \left(p_j\colon\invlim_{i\in\ob\Ii}D_i \To D_j\right)_{j\in\ob\Ii}
  \end{equation*}

  Like limit, a colimit of $D$ can be dented by $\dirlim D$ for brevity, or $\dirlim\limits_i D_i$ if it is needed to indicate how the functor is evaluated on objects, or the full formula             \glsadd{colimit}
  \begin{equation*}
    \left(q_j\colon\dirlim_{i\in\ob\Ii}D_i \Ot D_j\right)_{j\in\ob\Ii}
  \end{equation*}
  consisting of an object (called \textbf{colimit}) and a family of morphisms (called \termin[injections]{injection (category theory)}).
\end{rem}
\begin{rem}
  In some schools of mathematics, limits are called \termin[projective limits]{projective limit}, while colimits are called \termin[inductive limits]{inductive limit}. Also seen are (respectively) \termin[inverse limits]{inverse limit} and \termin[direct limits]{direct limit}. Both these systems of terminology are alternatives to using ``co-'' when distinguishing limits and colimits.
\end{rem}
\begin{rem}
  Often, the general theory of limits (but not colimits!) works better if the functor $D$ is taken to be contravariant.
  Thus, some authors define a \termin{projective system} (or, \termin{inverse system}) as a contravariant diagram $\Ii^{\op}\to\Cc$ and an \termin{inductive system} (or, \termin{direct system}) as a covariant diagram $\Ii\to\Cc$.
\end{rem}
\begin{rem}
  Sometimes the term ``limit'' or ``colimit'' just refers to the vertices object in a limit cone or colimit co-cone. To emphasize this, we may call them ``\termin{limit object}'' and ``\termin{colimit object}'' if needed.
\end{rem}

  Here are some typical examples.
\begin{exam}
  A limit of the empty diagram is just a \termin{terminal object}, while a colimit is just an \termin{initial object}.
\end{exam}
\begin{exam}
  A limit of the identity functor is just an \termin{initial object}, while a colimit is just a \termin{terminal object}.
\end{exam}
\begin{exam}
  Take $\Ii=\{1,2\}$ the discrete category with two objects and no nonidentity arrows. A diagram $D\colon\Ii\to\Cc$ hence is a pair of objects $D_1, D_2$ in $\Cc$. A cone over $D$ is an object $C$ equipped with arrows
  \begin{displaymath}
      \xymatrix{
         D_1&C\ar[l]_-{c_1}\ar[r]^-{c_2}&D_2
      }
  \end{displaymath}
  A co-cone under $D$ is an object $C$ equipped with arrows
  \begin{displaymath}
      \xymatrix{
         D_1\ar[r]^-{c_1}&C&D_2\ar[l]_-{c_2}
      }
  \end{displaymath}
  A limit of $D$ is just a \termin[product]{product (category theory)} of $D_1$ and $D_2$ in $\Cc$.
  A colimit of $D$ is just a \termin[coproduct]{coproduct (category theory)} of $D_1$ and $D_2$ in $\Cc$.
\end{exam}

\begin{exam}\label{exam:equalizer}
  Take $\Ii$ to be the following category:
  \begin{displaymath}
      \xymatrix{
         1\ar@<0.5ex>[r]^{\alpha}\ar@<-0.5ex>[r]_{\beta} &2
      }
  \end{displaymath}
  Hence a diagram $D$ of shape $\Ii$ looks like
  \begin{displaymath}
      \xymatrix{
         D_1\ar@<0.5ex>[r]^{D_{\alpha}}\ar@<-0.5ex>[r]_{D_{\beta}} &D_2
      }
  \end{displaymath}
  A limit of $D$ is an \termin{equalizer} of $D_{\alpha},D_{\beta}$, a colimit is a \termin{coequalizer}.
\end{exam}

\begin{exam}
%  Let $(I,\leqslant)$ be a \termin{filtered partially ordered set} (FPOS), which means a poset satisfies that for each two elements $i,j$, there exist another element $k$ such that $k\leqslant i,k\leqslant j$. As in Example \ref{exam:category2}, treat $I$ as a category.
  Let $(I,\leqslant)$ be a \termin{directed set} (dpos), which means a poset satisfies that for each two elements $i,j$, there exist another element $k\in I$ such that $i\leqslant k,j\leqslant k$. As in Example \ref{exam:category2}, treat $I$ as a category.

  Let $\Cc$ be a category, a projective system (inverse system) then is a set of objects $\{A_i\}_{i\in I}$ such that for each $i\leqslant j$, there exists a morphism $A_{i\leqslant j}\colon A_j\to A_i$ such that for any $i\leqslant j\leqslant k$, $A_{i\leqslant j}A_{j\leqslant k}=A_{i\leqslant k}$.
  A inductive system (direct system) is a set of objects $\{B_i\}_{i\in I}$ such that for each $i\leqslant j$, there exists a morphism $B_{i\leqslant j}\colon B_i\to B_j$ such that for any $i\leqslant j\leqslant k$, $B_{j\leqslant k}B_{i\leqslant j}=B_{i\leqslant k}$.
\end{exam}

A limit of a diagram sometimes works like a monomorphism, although an individual arrow in it may not be monic.
\begin{prop}\label{prop:globalmonic}
  Let $(L,\{p_i\}_{i\in\ob\Ii})$ be a limit. Then two morphisms $f,g\colon B\to A$ are equal as long as $p_if=p_ig$ for every $i\in\ob\Ii$.
\end{prop}

A colimit of a diagram sometimes works like an epimorphism, although an individual arrow in it may not be epi.
\begin{prop}\label{prop:globalepi}
  Let $(C,\{q_i\}_{i\in\ob\Ii})$ be a colimit. Then two morphisms $f,g\colon A\to B$ are equal as long as $fq_i=gq_i$ for every $i\in\ob\Ii$.
\end{prop}

%Colimits are linked to limits via the following proposition.
\begin{prop}\label{prop:limit-colimit}
  Let $F\colon\Ii^{\op}\to\Cc$ and $G\colon\Jj\to\Cc$ be two diagrams and $X, Y$ be two objects. If the limits and colimits involved exist, then we have
  \begin{align*}
    \Hom(X,\invlim_{\Ii^{\op}} F_j) &\approx \invlim_{\Ii^{\op}} \Hom(X,F_j) \\
    \Hom(\dirlim_{\Jj} G_i, Y) &\approx \invlim_{\Jj^{\op}} \Hom(G_i,Y)
  \end{align*}
\end{prop}
  \begin{proof}
    For any connection morphism $\phi^i_j\colon F_i\To F_j$ in the diagram $F$, the corresponding connection map in diagram $(\Hom(X,F_i))$ is
    \mapdes{\Hom(X,F_i)}{\Hom(X,F_j)}{f}{\phi^i_j\circ f}

    For any connection morphism $\psi^i_j\colon G_i\To G_j$ in the diagram $G$, the corresponding connection map in diagram $(\Hom(G_i,Y))$ is
    \mapdes{\Hom(G_j,Y)}{\Hom(G_i,Y)}{f}{f\circ\psi^i_j}

    Use these correspondings, the statements are easy to verify.
  \end{proof}
\subsection{Exercises}
  \begin{ex}
    Shows that a colimit of a diagram $D$ is just the opposite of a limit of its opposite $D^{\op}$. And use this fact to give another proof of the second statement in Proposition \ref{prop:limit-colimit} under the assumption that the first one it true.
  \end{ex}
  \begin{ex}
    Describe the limits in $\Set$ explicitly.
  \end{ex}
  \begin{ex}\label{def:limit as rep functor}
    In \cite{Cat&Shf}, a projective limit of a projective system $D\colon\Ii^{\op}\to\Cc$ is defined as a representative of the representable presheaf on $\Cc$ which maps every object $X$ in $\Cc$ to the limit of projective system $\Hom(X,D)$ in $\Set$.
    (Of course, if the presheaf is not representable or even not exists, then the projective limits do not exist.)

    Similarly, an inductive limit of an inductive system $D'\colon\Ii\to\Cc$ is defined as a representative of the representatble functor which maps every object $X$ in $\Cc$ to the limit of projective system $\Hom(D',X)$ in $\Set$.

    Shows that these definitions are coincide with our definitions given in this section. A benefit of such a system of notations is that it allows us to talk about some properties of limits without assumption their existence.
  \end{ex}

  \begin{ex}
    Let $D\colon\Ii\to[\Jj,\Cc]$ be an inductive system, then, by \ref{prop:power law for functor}, it can be also viewed as a functor from $\Jj$ to $[\Ii,\Cc]$.
    Shows that
    \begin{equation*}
      (\dirlim_{\Ii}D)(j)\cong\dirlim_{\Ii}(D(j))\quad\forall j\in\ob\Jj
    \end{equation*}
    i.e. the functor $\dirlim_{\Ii}D$ is naturally isomorphic to the functor which maps every object $j$ in $\Jj$ to the inductive limit of the functor $D(j)$.

    Prove the similar result for projective limit.
  \end{ex}

  \begin{ex}
    Let $\Ii$ be a \termin{connected category} (that is a nonempty category in which every two objects are connected by a zigzag of morphisms) and $X$ be an object in a category $\Cc$. Shows that $X\approx\invlim\Delta_X$ and $\dirlim\Delta_X\approx X$.
    [Hint: one can direct check them. Or first prove them in $\Set$, then use \ref{def:limit as rep functor}.]
  \end{ex}

\newpage\section{Products and coproducts}
\begin{defn}\label{def:product}
  Consider a diagram of shape $\Ii$, where $\Ii$ is a discrete category, it looks like a family of objects without arrows between them. A limit of such a diagram is called a \termin{product} of these objects. Dually, a colimit of such a diagram is called a \termin{coproduct} of these objects.

  More explicitly, a product of a family of objects $\{A_i\}_{i\in I}$ is an object $P$ equipped with a family of ``projections'' $\{p_i\colon P\to A_i\}_{i\in I}$ satisfying the following universal property:
  \begin{quote}
    For any object $T$ and morphisms $\{t_i\colon T\to A_i\}_{i\in I}$, there exist a unique morphism $T\to P$ making the triangles commutative:
    \begin{displaymath}
      \xymatrix{
         T\ar@{-->}^{\exists!}[r]\ar[dr]_{t_i}&P\ar[d]^{p_i}\\
         &A_i
      }
    \end{displaymath}
  \end{quote}
\end{defn}
\begin{rem}
  Since the product (resp. coproduct) is unique up to unique isomorphism, it make sense to write ``the'' product (resp. coproduct) of a family of objects $\{A_i\}_{i\in I}$ and we often denote it by $\prod_{i\in I} A_i$ (resp. $\coprod_{i\in I} A_i$). \glsadd{prod}\glsadd{coprod}
\end{rem}
\begin{rem}
  The product of two objects $A$ and $B$ is traditionally denoted by $A\times B$. A morphism from a cone $(C,f,g)$ to the product $A\times B$ is the unique factorization of $f$ and $g$ through $A\times B$. It is usually denoted by $\<f,g\>$.
  \glsadd{mor-prod}
\end{rem}
\begin{exam}
  Let $f\colon A\to B$ be a morphism, then $(A,1_A,f)$ is also a cone over the diagram of two discrete objects $A, B$. The unique cone-morphism $\<1_A,f\>$ is called the \termin[graph]{graph of morphism} of $f$ and denoted by $\Gamma_f$.
  \glsadd{graph}
\end{exam}
\begin{rem}
  It is a common mistake to think that the projections of a product are epimorphisms. This is not true, not even in the category $\Set$.

  Another common mistake is to think that once the object $\prod_{i\in I} A_i$ in a product has been fixed, the corresponding projections are necessarily unique. It is not at all.
\end{rem}

  Since the nullary discrete category is the empty category, the \termin{nullary product} is just the terminal object. Similar, the \termin{unary product} of any object is itself.

  Since a binary discrete category is itself symmetric and can be viewed as a full subcategory of a trinary discrete category, the \termin{binary product} is a commutative associate binary operation on a category.

  More general, we have
  \begin{prop}\label{prop:assofprod}
    Consider a set $I$ and a partition $I = \bigcup_{k\in K}J_k$ of this set into disjoint subsets. Consider a family $\{C_i\}_{i\in I}$ of objects in a category $\Cc$. When all the products involved exist, the following isomorphism holds:
    \begin{equation*}
      \prod_{i\in I}C_i \approx \prod_{k\in K}(\prod_{j\in J_k}C_j)
    \end{equation*}
  \end{prop}
  \begin{rem}
    It should be noticed that the existence of the product of a family of objects does not imply the existence of the product of a subfamily of those objects. For example, consider the full subcategory $\Set_n$ of $\Set$ whose objects are the sets with fewer than $n$ elements. It is easy to prove that products in $\Set_n$, when they exist, are just cartesian products of sets. Therefore the product of two sets with $i$ and $j$ elements exists in $\Set_n$ precisely when $ij<n$. But the product of an arbitrary family containing the empty set always exists: it is just the empty set.
  \end{rem}

  By duality, we have similar facts for \termin{nullary coproduct}, \termin{unary coproduct}, \termin{binary coproduct} and, more general,
  \begin{prop}\label{prop:assofcoprod}
    Consider a set $I$ and a partition $I = \bigcup_{k\in K}J_k$ of this set into disjoint subsets. Consider a family $\{C_i\}_{i\in I}$ of objects in a category $\Cc$. When all the products involved exist, the following isomorphism holds:
    \begin{equation*}
      \coprod_{i\in I}C_i \approx \coprod_{k\in K}(\coprod_{j\in J_k}C_j)
    \end{equation*}
  \end{prop}

  \begin{exam}
    In $\Set$, products are just the cartesian products, while coproducts are the disjoint unions.
    \begin{equation*}
      \coprod_{i\in I}X_i = \{(x_i,i)\mid x_i\in X_i, i\in I\}
    \end{equation*}
  \end{exam}
  \begin{exam}
    In $\Cat$, products are the products of categories, while coproducts are the category whose objects and hom-sets are disjiont unions.
  \end{exam}


\newpage\section{Equalizers and coequalizers}
  \begin{defn}
    Let $f,g\colon A\to B$ be two parallel morphisms, which can be view as a diagram. an \termin{equalizer} is a limit of the diagram.     Dually, a \termin{coequalizer} is a colimit.
  \end{defn}
  \begin{rem}
    Since the equalizer (resp. coequalizer) is unique up to unique isomorphism, it make sense to write ``the'' equalizer (resp. coequalizer) of two parallel morphisms $f,g\colon A\to B$ and we often denote it by $\ker(f,g)$ (resp. $\coker(f,g)$).The notation follows from the alternative name: the \termin{difference kernel} (resp. \termin{difference cokernel}).\glsadd{equalizer}\glsadd{coequalizer}
  \end{rem}

  \begin{prop}
    An equalizer (resp. coequalizer) of two morphisms must be monic (resp. epi).
  \end{prop}
  \begin{proof}
    Let $(K,k)$ be an equalizer of $f,g\colon A\to B$. For any parallel morphisms $u,v\colon T\to K$ such that $k\circ u=k\circ v$, the composition $k\circ u$ itself is a cone over $f,g\colon A\to B$. Thus it must factor through $K$ by a unique morphism $T\to K$, which force $u=v$.

    Similar argument shows that a coequalizer must be epi.
  \end{proof}

  \begin{prop}
    An epi equalizer must be an isomorphism. A monic coequalizer must be an isomorphism.
  \end{prop}
  \begin{proof}
    Let $(K,k)$ be an epi equalizer of $f,g\colon A\to B$, then $f\circ k=g\circ k$ implies $f=g$. Thus $(A,1_{A})$ is also an equalizer, which force $k$ to be an isomorphism. The proof for monic coequalizer is similar.
  \end{proof}

  \begin{exam}
    In $\Set$, the equalizer of two functions $f,g\colon X\to Y$ is
    \begin{equation*}
      K=\{x\in X\mid f(x)=g(x)\}
    \end{equation*}
    and the coequalizer is $C=Y/K$, i.e. the quotient set of $Y$ with the equivalence relation generated by $f(x)\sim g(x)$ for all $x\in X$.
  \end{exam}

  \begin{exam}
    Coequalisers can be large: There are exactly two functors from the category $\one$ having one object and one identity arrow, to the category $\mathbf{2}$ with two objects and exactly one non-identity arrow going between them. The coequaliser of these two functors is the monoid of natural numbers under addition, considered as a one-object category. In particular, this shows that while every coequalising arrow is epic, it is not necessarily surjective.
  \end{exam}

\subsection{Exercises}
  \begin{ex}
    Describe a coproduct of a family of objects $\{A_i\}_{i\in I}$ by a universal property.
  \end{ex}
  \begin{ex}
    Describe an equalizer and a coequalizer of two parallel morphisms by universal properties.
  \end{ex}

\newpage\section{Pullbacks and pushouts}
  \begin{defn}
    Let $\Ii$ be
    \begin{displaymath}
      \xymatrix{
         \cdot&\cdot\ar[l]\ar[r]&\cdot
      }
    \end{displaymath}
    then a projective system of shape $\Ii$ is of the form
    \begin{displaymath}
      \xymatrix{
         A\ar[r]^{f}&C&B\ar[l]_{g}
      }
    \end{displaymath}
    Thus a limit of it can be view as a commutative square:
    \begin{displaymath}
      \xymatrix{
         P\ar[r]^{f'}\ar[d]_{g'}&B\ar[d]^{g}\\
         A\ar[r]^{f}&C
      }
    \end{displaymath}
    we call it a \termin{pullback square} or \termin{cartesian diagram}, and say $g'$ is the \termin{pullback} of $g$ through $f$, $f'$ is the \termin{pullback} of $f$ through $g$.
    We also call this limit a \termin{fibre product} of $A$ and $B$ over $C$, and denote it by $A\times_CB$.\glsadd{fibreprod}

    A inductive system of shape $\Ii$ is of the form
    \begin{displaymath}
      \xymatrix{
         B&C\ar[l]_{g}\ar[r]^{f}&A
      }
    \end{displaymath}
    Thus a colimit of it can be view as a commutative square in $\Cc$:
    \begin{displaymath}
      \xymatrix{
         C\ar[r]^{f}\ar[d]_{g}&A\ar[d]^{g'}\\
         B\ar[r]^{f'}&P
      }
    \end{displaymath}
    we call it a \termin{pushout square} or \termin{cocartesian diagram}, and say $g'$ is the \termin{pushout} of $g$ through $f$, $f'$ is the \termin{pushout} of $f$ through $g$. We also call this limit the \termin{fibre sum} or \termin{amalgamed sum} of $A$ and $B$ under $C$, and denoted by $A\amalg_CB$.\glsadd{amalgam}
  \end{defn}

  \begin{prop}
    Monomorphisms are \termin{stable} under pullback, that means the pullback of a monomorphism is also a monomorphism. Moreover, so are isomorphisms.
  \end{prop}
  \begin{proof}
    Assume $g\colon B\to C$ is monic and $g'\colon P\to A$ is its pullback through $f\colon A\to C$. Let $u,v\colon T\to P$ be two parallel morphisms such that $g'\circ u=g'\circ v$, then
    \begin{equation*}
      g\circ f'\circ u = f\circ g'\circ u = f\circ g'\circ v = g\circ f'\circ v
    \end{equation*}
    Thus $f'\circ u=f'\circ v$. Therefore $T$ equipped with compositions $g'\circ u, g\circ f'\circ u$ and $f'\circ u$ is a cone over the diagram, which force $u=v$.

    Similar, assume $g$ is an isomorphism and $g'$ is its pullback through $f$. Then $A$ equipped with $1_A$ and $g^{-1}\circ f$ is a cone over the diagram, which force $A$ itself be a fibre product as $P$, thus $g'$ is an isomorphism.
  \end{proof}
  \begin{rem}
    Epimorphisms may not be stable under pullback.
  \end{rem}


  By the duality principle, it is easy to get corresponding properties for pushouts from the knowledge of pullbacks.

\subsection{Kernel pairs}
  \begin{defn}
    A \termin{kernel pair} of a morphism in a category is the fiber product of the morphism with itself.
  \end{defn}
  \begin{prop}
    Let $(P,\alpha,\beta)$ be a kernel pair of a morphism $f\colon A\to B$, then $\alpha,\beta$ are split epimorphisms.
  \end{prop}
  \begin{proof}
    Since $(A,1_A,1_A)$ is a cone over the diagram, there exists a unique morphism $t\colon A\to P$ such that $1_A=\alpha\circ t=\beta\circ t$. Thus $\alpha,\beta$ are retractions of $t$ and must be split epimorphisms.
  \end{proof}
  \begin{prop}\label{prop:mono and kernel pair}
    Consider a morphism $f\colon A\to B$, then followings are equivalent:
    \begin{enumerate}
      \item $f$ is a monomorphism;
      \item the kernel pair of $f$ exists and is given by $(A, 1_A, 1_A)$;
      \item the kernel pair $(P, \alpha, \beta)$ of $f$ exists and is such that $\alpha=\beta$.
    \end{enumerate}
  \end{prop}

  \begin{prop}\label{prop:effective1}
    In a category, if a coequalizer has a kernel pair, then it is a coequalizer of its kernel pair.
  \end{prop}
  \begin{proof}
    Consider the following diagram
    \begin{displaymath}
      \xymatrix{
         &X\ar@<-0.5ex>[d]_{x}\ar@<0.5ex>[d]^{y}\ar[dl]_{z}&\\
         P\ar@<0.5ex>[r]^{\alpha}\ar@<-0.5ex>[r]_{\beta}&A\ar[r]^{f}\ar[d]_{g}&B\ar[dl]^{h}\\
         &C&
      }
    \end{displaymath}
    where, $f=\coker(x,y)$ and $\alpha,\beta$ is the kernel pair of $f$. Hence there exists a unique factorization $z$ such that $\alpha\circ z=x, \beta\circ z=y$.
    We need to show that $f=\coker(\alpha,\beta)$.

    To do this, consider an arbitrary morphism $g$ such that $g\circ\alpha=g\circ\beta$. Then $g\circ x=g\circ\alpha\circ z=g\circ\beta\circ z=g\circ y$. Thus we get a unique factorization $h$ through $f=\coker(x,y)$.
  \end{proof}
  Conversely, we have
  \begin{prop}\label{prop:effective2}
    In a category, if a kernel pair has a coequalizer, then it is a kernel pair of its coequalizer.
  \end{prop}
  \begin{proof}
    Consider the following diagram
    \begin{displaymath}
      \xymatrix{
         &X\ar@<-0.5ex>[d]_{x}\ar@<0.5ex>[d]^{y}\ar[dl]_{z}&\\
         P\ar@<0.5ex>[r]^{\alpha}\ar@<-0.5ex>[r]_{\beta}&A\ar[r]^{f}\ar[d]_{g}&B\ar[dl]^{h}\\
         &C&
      }
    \end{displaymath}
    where, $\alpha,\beta$ is the kernel pair of $g$ and $f=\coker(\alpha,\beta)$. Since $g\circ\alpha=g\circ\beta$, we get a unique factorization $h$ through $f=\coker(\alpha,\beta)$.
    We need to show that $\alpha,\beta$ is the kernel pair of $f$.

    To do this, consider two parallel morphisms $x,y$ such that $f\circ x=f\circ y$. Then $g\circ x=h\circ f\circ x=h\circ f\circ y=g\circ y$. Thus there exists a unique factorization $z$ such that $\alpha\circ z=x, \beta\circ z=y$.
  \end{proof}

\subsection{Diagram chase}
  When duel with universal properties, people often check equations along the commutative diagram, such a method is called \termin{diagram chase}.

  The following proportion is a good example.
  \begin{prop}[Associativity]\label{prop:assofpullback}
    Consider a commutative diagram as below:
    \begin{displaymath}
      \xymatrix{
         \cdot\ar[r]\ar[d]&\cdot\ar[r]\ar[d]&\cdot\ar[d]\\
         \cdot\ar[r]&\cdot\ar[r]&\cdot
      }
    \end{displaymath}
    \begin{enumerate}
      \item  If the two small squares are cartesian, then so is the outer rectangle.
      \item  If the right square and the outer rectangle are cartesian, then so is the left square.
    \end{enumerate}
  \end{prop}
  \begin{proof}
    For \emph{1}, since we want to prove the outer rectangle is cartesian, we draw a commutative diagram as below:
    \begin{displaymath}
      \xymatrix{
         \cdot\ar@/_/[ddr]_{u}\ar@/^/[rrrd]^{v}\ar@{-->}[dr]|{w}&&&\\
         &\cdot\ar[r]_{a}\ar[d]^{c}&\cdot\ar[r]_{b}\ar[d]^{d}&\cdot\ar[d]^{e}\\
         &\cdot\ar[r]_{f}&\cdot\ar[r]_{g}&\cdot
      }
    \end{displaymath}

    More explicitly, let $u,v$ be two morphisms from a same domain and is such that $g\circ f\circ u=e\circ v$. Then we need to show that there exists a unique morphism $w$ such that $c\circ w=u, b\circ a\circ w=v$.

    To prove the existence, we chase the diagram as following:

    1, Since $g\circ f\circ u=e\circ v$ and the right square is cartesian, there exists a unique morphism $x$ such that $d\circ x=f\circ u, b\circ x=v$.

    2, Since $d\circ x=f\circ u$ and the left square is cartesian, there exists a unique morphism $w$ such that $c\circ w=u, a\circ w=x$.

    3, Since $c\circ w=u, a\circ w=x, b\circ x=v$, we have $c\circ w=u, b\circ a\circ w=v$.

    To prove the uniqueness, we assume that there exists a morphism $w'$ such that $c\circ w'=u, b\circ a\circ w'=v$ and prove that $a\circ w'=x$. Then the uniqueness follows from the condition that the left square is cartesian.
    To do this, we chase the diagram as following:

    1, Since $d\circ a=f\circ c$ and $c\circ w'=u$, we have $d\circ a\circ w'=f\circ u$.

    2, Since $d\circ a\circ w'=f\circ u, b\circ a\circ w'=v$ and the right square is cartesian, $a\circ w'=x$.

    For \emph{2}, since we want to prove the left square is cartesian, we draw a commutative diagram as below:
    \begin{displaymath}
      \xymatrix{
         \cdot\ar@/_/[ddr]_{u}\ar@/^/[rrd]^{v}\ar@{-->}[dr]|{w}&&&\\
         &\cdot\ar[r]_{a}\ar[d]^{c}&\cdot\ar[r]_{b}\ar[d]^{d}&\cdot\ar[d]^{e}\\
         &\cdot\ar[r]_{f}&\cdot\ar[r]_{g}&\cdot
      }
    \end{displaymath}

    More explicitly, let $u,v$ be two morphisms from a same domain and is such that $f\circ u=d\circ v$. Then we need to show that there exists a unique morphism $w$ such that $c\circ w=u, a\circ w=v$.

    To prove the existence, we chase the diagram as following:

    1, Since $e\circ b=g\circ d$ and $f\circ u=d\circ v$, we have $g\circ f\circ u=e\circ b\circ v$.

    2, Since $g\circ f\circ u=e\circ b\circ v$ and the outer rectangle is cartesian, there exists a unique morphism $w$ such that $c\circ w=u, b\circ a\circ w=b\circ v$.

    3, Since $d\circ a=f\circ c$ and $c\circ w=u$, we have $d\circ a\circ w=f\circ u$.

    4, Since $d\circ a\circ w=f\circ u, b\circ a\circ w=b\circ v$ and the right square is cartesian, $a\circ w=v$.

    To prove the uniqueness, we assume that there exists a morphism $w'$ such that $c\circ w'=u, a\circ w'=v$ and prove that $b\circ a\circ w'=b\circ v$. Then the uniqueness follows from the condition that the outer rectangle is cartesian. However, this is obvious.
  \end{proof}
  \begin{cor}
    The pullback of a commutative triangle (if it exists) is a commutative triangle.
    \begin{displaymath}
      \xymatrix@R=0.3cm{
         \cdot\ar@{-->}[rr]\ar[dd]\ar[dr]&&\cdot\ar[dd]\ar[dr]&\\
         &\cdot\ar@{-->}[rr]\ar[dl]&&\cdot\ar[dl]\\
         \cdot\ar@{-->}[rr]&&\cdot
      }
    \end{displaymath}
  \end{cor}

\subsection{Examples}
  \begin{exam}\label{exam:fibreprodofset}
    In the category $\Set$, the pullback of pair $\xymatrix@1{A\ar[r]^{f}&C&B\ar[l]_{g}}$ is given by
    \begin{gather*}
      P=\{(a,b)\mid a\in A, b\in B, f(a)=f(b)\},\\
      g'(a,b)=a,f'(a,b)=b
    \end{gather*}
  \end{exam}
  \begin{exam}
    Under the conditions of \ref{exam:fibreprodofset}, when $B$ is a subset of $C$ and $g$ is the canonical inclusion, then $P$ is isomorphic to the fibre $f^{-1}(B)$, i.e. the inverse image of $B$ along $f$.
  \end{exam}
  \begin{rem}
    In general, when $g$ is monic, we can write $P$ as $f^{-1}(B)$ and call it ``\termin[the inverse image of $g$ along $f$]{inverse image}''.
  \end{rem}
  \begin{exam}\label{exam:intersection=pullback}
    Under the conditions of \ref{exam:fibreprodofset}, if both $A$ and $B$ are subsets of $C$ with $f,g$ the canonical inclusions, then $P$ is isomorphic to the intersection $A\cap B$.
  \end{exam}
  \begin{rem}
    In general, when $f,g$ are monic, we can write $P$ as $A\cap B$ and call it ``\termin[the intersection of $f$ and $g$]{intersection}''.
  \end{rem}
  \begin{exam}
    In the category $\Set$, the kernel pair of a function $f$ is just the equivalence relation on $A$ determined by $f$.
  \end{exam}
\subsection{Exercises}
  \begin{ex}
    Describe pullbacks and pushouts by universal properties.
  \end{ex}
  \begin{ex}
    Describe the corresponding concepts and propositions in this section for pushouts. Then prove these propositions.
  \end{ex}
  \begin{ex}\label{prop:pullback=eq+prod}
    Let $\xymatrix@1{A\ar[r]^{f}&C&B\ar[l]_{g}}$ be a pair of morphisms, and $(A\times B, p_A, p_B)$ be a product of $A$ and $B$, $e\colon E\to A\times B$ is an equalizer of $f\circ p_A$ and $g\circ p_B$. Then the following diagram is cartesian:
    \begin{displaymath}
      \xymatrix{
         E\ar[r]^{p_B\circ e}\ar[d]_{p_A\circ e}&B\ar[d]^{g}\\
         A\ar[r]^{f}&C
      }
    \end{displaymath}
    Conversely, if there exists a morphism $e\colon E\to A\times B$ making the above square cartesian, then $(E,e)$ is an equalizer of $f\circ p_A$ and $g\circ p_B$.
  \end{ex}

\newpage\section{Limits and colimits in $\Set$}
\begin{prop}
  The initial object in $\Set$ is the empty set $\varnothing$, while a terminal object is a singleton $\Pt$.
\end{prop}
\begin{prop}
  The product of a set of sets $\{X_i\}_{i\in I}$ is the usual cartesian product of them: $\prod_{i\in I}X_i =\{(x_i)_{i\in I}\mid\forall i\in I, x_i\in X_i\}$.
  The coproduct of them is the disjoint union $\bigsqcup_{i\in I}X_i =\{(i,x)\mid i\in I, x\in X_i\}$.
\end{prop}
\begin{prop}
  the product of all sets is the empty set $\varnothing$. Indeed, any product of a set of sets  which involves $\varnothing$ must be $\varnothing$ itself. However, the coproduct of all sets do not exists.
\end{prop}
\begin{prop}
  The equalizer of two parallel functions $f_0,f_1\colon S\to T$ is the subset of $S$: $\{s\in S\mid f_0(s)=f_1(s)\}$.
  The coequalizer of them is the quotient set of $T$ by the equivalence relation $R\subset T\times T$ generated by $\{(f_0(s),f_1(s))\mid s\in S\}$. More explicitly, for any $x,y\in T$, $XRy$ means there exists a sequence of elements $s_1,\cdots,s_n\in S$ and a sequence of binary digits $\epsilon_1,\cdots,\epsilon_n$ such that
    \begin{align*}
      f_{\epsilon_1}(s_1) & = x\\
      &\cdots\\
      f_{1-\epsilon_i}(s_i) & = f_{\epsilon_{i+1}}(s_{i+1})\\
      &\cdots\\
      f_{1-\epsilon_n}(s_n) & = y
    \end{align*}
\end{prop}
\begin{prop}
  The fibre product of $f\colon A\to C$ and $g\colon B\to C$ is the subset of the cartesian product:
  $A\times_CB=\{(a,b)\mid f(a)=g(b)\}\subset A\times B$.
\end{prop}
\begin{prop}
  The amalgamed sum of $f\colon C\to A$ and $g\colon C\to B$ is the quotient of the disjoint union:
  $A\amalg_CB=(A\sqcup B)/R$ by the equivalence relation $R$ generated by $\{(f(s),g(s))\mid s\in S\}$
\end{prop}

\begin{prop}
  The limit of $D\colon\Ii^{\op}\to\Set$ is
  the set of natural transformations from the diagram constant on the singleton to $D$, or equivalently,
  the set of global elements of $D$.
\end{prop}

  To give an explicit description, let $D\colon\Ii^{\op}\to\Set$ be a small diagram, it is not difficult to see that a cone over the diagram $D$ is the same thing as a cone over the following diagram: (For brevity, we abbreviate ``target of $f$'' and ``source of $f$'' just as $t.f$, $s.f$.)
  \begin{displaymath}
    \xymatrix{\prod\limits_{i\in\ob\Ii} D_i\ar@<0.5ex>[r]^-{\alpha}\ar@<-0.5ex>[r]_-{\beta}&\prod\limits_{f\in\hom\Ii}D_{s.f}}
  \end{displaymath}
  Where $\alpha,\beta$ is given by
  \begin{align*}
    \alpha(x_i)_{i\in\ob\Ii} &= (x_{s.f})_{f\in\hom\Ii}  \\
    \beta(x_i)_{i\in\ob\Ii}  &= (D_f(x_{t.f}))_{f\in\hom\Ii}
  \end{align*}

  Therefore, the limit $L$ of $D$ can be given by the equalizer
  \begin{displaymath}
    \xymatrix{L\ar@{ >->}[r]&\prod\limits_{i\in\ob\Ii} D_i\ar@<0.5ex>[r]^-{\alpha}\ar@<-0.5ex>[r]_-{\beta}&\prod\limits_{f\in\hom\Ii}D_{s.f}}
  \end{displaymath}

  More explicitly,
  \begin{equation*}
    L=\{(x_i)_{i\in\ob\Ii}\mid x_i\in D_i; D_f(x_i)=x_j, \forall f\colon j\to i\}
  \end{equation*}

  Dually,
  let $D\colon\Ii\to\Set$ be a small diagram,
  the colimit $C$ of $D$ can be given by the coequalizer
  \begin{displaymath}
    \xymatrix{
    \bigsqcup\limits_{f\in\hom\Ii}D_{s.f}\ar@<0.5ex>[r]^-{\alpha}\ar@<-0.5ex>[r]_-{\beta}
    &\bigsqcup\limits_{i\in\ob\Ii} D_i\ar@{->>}[r]
    &C
    }
  \end{displaymath}
  Where $\alpha,\beta$ is given by
  \begin{align*}
    \alpha(x_{s.f}\in D_{s.f})  &= x_{s.f}\in D_{s.f}  \\
    \beta(x_{s.f}\in D_{s.f})   &= D_f(x_{s.f})\in D_{t.f}
  \end{align*}

  The colimit of $D$ is given by a quotient set
  \begin{equation*}
    \dirlim D = (\bigsqcup_{i\in\ob\Ii} D_i)/\sim
  \end{equation*}
  where the equivalence relation $\sim$ is that which is generated by
  \begin{equation*}
    \{(x\in D_i)\sim(x'\in D_{i'})\mid \exists(f\colon i\to i')\st(D_f(x)=x')\}
  \end{equation*}


\newpage\section{Complete categories}
  In a category $\Cc$, if every diagram of shape $\Ii$ admits a limit (resp. colimit), then we say it \textbf{has} all $\Ii-$limits (resp. $\Ii-$colimits). For instance, the category $\Set$ has all pullbacks, pushouts, kernel pairs, cokernel pairs, equalizers, coequalizers and has products and coproducts of a set of objects. In later body, one can see the existence of previous limits (resp. colimits) ensure $\Set$ has all small limits (resp. colimits).

  Such a category is undoubtedly very nice, thus one may ask when a category has all limits or colimits.

  \begin{defn}
    A category is said to be \termin[complete]{complete category}, if every small diagram in it has a limit. Similar, a \termin{finitely complete category} is such a category, in which every finite diagram has a limit.
    Dually, we have concepts of \termin{cocomplete category} and \termin{finitely cocomplete category}.
  \end{defn}

    Large (not small) limits may exists. For example, the product of all sets in $\Set$ is the empty set.

    However, this is an exception. Indeed, a category has all limits must be \termin[thin]{thin category}. That is, its every hom-set is a singleton.

  We do not prove this. Instead, we give a similar proposition.
  \begin{prop}\label{prop:complete small category = proset}
    A complete small category must be thin.
  \end{prop}
  \begin{proof}
    Let $\Cc$ be a complete small category. For any pair of objects $A,B$ in $\Cc$ and any $f,g\in\Hom(A,B)$. We need to show $f=g$.

    If not, then consider the product of $\kappa$ copies of $B$, say $B^{\kappa}$. Where $\kappa$ is the cardinal of the set of all arrows of $\Cc$. Then, use $f$ and $g$, one can already construct $2^{\kappa}$ distinct cones over the copies of $B$, each of them admits a factorization $A\to B^{\kappa}$. From this, it is clear that $\Hom(A,B^{\kappa})$ has the cardinal no less than $2^{\kappa}$, which contracts with the assumption of $\kappa$.
  \end{proof}

  Based on the same reasoning, a category has all limits with index category larger than itself must be thin. Particularly, a finite complete finite category must be a finite preordered set.

  Our next goal is to get the condition of completeness for non-small categories.

  Consider our most familiar category $\Set$. Let $D\colon\Ii^{\op}\to\Set$ be a small diagram, it is not difficult to see that a cone over the diagram $D$ is the same thing as a cone over the following diagram: (For brevity, we abbreviate ``target of $f$'' and ``source of $f$'' just as $t.f$, $s.f$.)
  \begin{displaymath}
    \xymatrix{\prod\limits_{i\in\ob\Ii} D_i\ar@<0.5ex>[r]^-{\alpha}\ar@<-0.5ex>[r]_-{\beta}&\prod\limits_{f\in\hom\Ii}D_{s.f}}
  \end{displaymath}
  Where $\alpha,\beta$ is given by
  \begin{align*}
    \alpha(x_i)_{i\in\ob\Ii} &= (x_{s.f})_{f\in\hom\Ii}  \\
    \beta(x_i)_{i\in\ob\Ii}  &= (D_f(x_{t.f}))_{f\in\hom\Ii}
  \end{align*}

  Therefore, the limit $L$ of $D$ can be given by the equalizer
  \begin{displaymath}
    \xymatrix{L\ar@{ >->}[r]&\prod\limits_{i\in\ob\Ii} D_i\ar@<0.5ex>[r]^-{\alpha}\ar@<-0.5ex>[r]_-{\beta}&\prod\limits_{f\in\hom\Ii}D_{s.f}}
  \end{displaymath}

  More explicitly,
  \begin{equation*}
    L=\{(x_i)_{i\in\ob\Ii}\mid x_i\in D_i; D_f(x_i)=x_j, \forall f\colon j\to i\}
  \end{equation*}

  This construction can be generalized to an arbitrary category. Thus we get the following theorem.
  \begin{thm}\label{thm:complete_category}
    A category $\Cc$ is complete if and only if each set of objects has a product and each pair of parallel arrows has an equalizer.
  \end{thm}
  \begin{proof}
    It suffices to show the ``if''.

    Let $D\colon\Ii^{\op}\to\Cc$ be a small diagram. We construct the products
    \begin{equation*}
      \prod_{i\in\ob\Ii}D_i\qquad\&\qquad\prod_{f\in\hom\Ii}D_{s.f}
    \end{equation*}
    with $\{p'_i\}_{i\in\ob\Ii}$ and $\{p''_f\}_{f\in\hom\Ii}$ be their respective projections.

    Let $\alpha$ be the unique factorization such that $p''_f\circ\alpha = p'_{s.f}$, $\beta$ be the unique factorization such that $p''_f\circ\beta = D_f\circ p'_{t.f}$, and $(L,l)$ be the equalizer of pair $(\alpha,\beta)$. Denote each $p'_i\circ l$ by $p_i$, now we prove that $(L,(p_i)_{i\in\ob\Ii})$ is a limit of $D$.
    \begin{displaymath}
    \xymatrix{
    C\ar@<0.5ex>[r]^{q}\ar@<-0.5ex>[r]_{\bar{q}}\ar@/_1.5pc/[drr]_-{q_i}\ar@{-->}@/^2pc/[rr]^-{q'}&
    L\ar[r]^-{l}\ar[dr]_-{p_i}&
    \prod\limits_{i\in\ob\Ii} D_i\ar@<0.5ex>[r]^-{\alpha}\ar@<-0.5ex>[r]_-{\beta}\ar[d]^-{p'_i}\ar[dr]_{p'_j}&
    \prod\limits_{f\in\hom\Ii}D_{s.f}\ar[d]^-{p''_f} \\
    &
    &D_i\ar[r]_{D_f}
    &D_j
    }
    \end{displaymath}

    First, it is a cone over $D$. Indeed, for any arrow $f\colon j\to i$, we have
    \begin{align*}
      D_f\circ p_i & = D_f\circ p'_i\circ l \\
                           & = p''_f\circ\beta\circ l \\
                           & = p''_f\circ\alpha\circ l \\
                           & = p'_j\circ l = p_j
    \end{align*}

    Moreover, let $(C,(q_i)_{i\in\ob\Ii})$ be another cone over $D$, we need to show that there exists a unique $\Cone(D)-$morphism $q$, that is a unique factorization $q$ such that $p_i\circ q = q_i$.

   Since $(C,(q_i)_{i\in\ob\Ii})$ is also a cone over the family of objects $\{D_i\}_{i\in\ob\Ii}$. There exists a unique factorization $q'$ such that $p'_i\circ q' = q_i$.

   For each arrow $f\colon j\to i$, we have
    \begin{align*}
      p''_f\circ\alpha\circ q' & = p'_j\circ q' \\
                           & = q_j \\
                           & = D_f\circ q_i \\
                           & = D_f\circ p'_i\circ q'\\
                           & = p''_f\circ\beta\circ q'
    \end{align*}
    From which $\alpha\circ q'=\beta\circ q'$. This implies the existence of the factorization $q$ such that $l\circ q=q'$. Then
    \begin{equation*}
      p_i\circ q = p'_i\circ l\circ q = p'_i\circ q' = q_i
    \end{equation*}
    So this $q$ is our require factorization.

    The uniqueness is easy to see by the universal property of equalizer. Or, consider another factorization $\bar{q}$ such that $p_i\circ \bar{q} = q_i$. Since $l$ must be monic, it suffices to show $l\circ q=l\circ\bar{q}$. Indeed, for any $i\in\ob\Ii$, we have
    \begin{align*}
      p'_i\circ l\circ q & = p'_i\circ q' \\
                           & = q_i \\
                           & = p_i\circ \bar{q} \\
                           & = p'_i\circ l \circ \bar{q}
    \end{align*}
    Then, by Proposition \ref{prop:globalmonic}, $l\circ q=l\circ\bar{q}$.
  \end{proof}

  \begin{prop}
    For a category $\Cc$, the followings are equivalent.
    \begin{enumerate}
      \item $\Cc$ is finitely complete;
      \item $\Cc$ has terminal objects, binary products and equalizers;
      \item $\Cc$ has terminal objects and pullbacks.
    \end{enumerate}
  \end{prop}
  \begin{proof}
    \emph{1} $\then$ \emph{2} and \emph{1} $\then$ \emph{3} are obvious. \emph{2} $\then$ \emph{3} comes from Exercise \ref{prop:pullback=eq+prod}.
    Or, by the associativity of products (see \ref{prop:assofprod}) and induction, \emph{2} implies the existence of products of finite and non-empty family of objects. Notice that, in the proof of Theorem \ref{thm:complete_category}, when the diagram $D$ is finite, then so are the corresponding products, therefore \emph{2} $\then$ \emph{1}.

    To show \emph{3} $\then$ \emph{2}, we assume \emph{3}. It is obvious that a fibre product over the terminal object is just a product. To show the existence of equalizers, consider a pair of morphisms $\xymatrix@1{f,g\colon A\ar@<0.5ex>[r]\ar@<-0.5ex>[r]&B}$.
    For any triple $(P,k,l)$ making the following square commutative
    \begin{displaymath}
    \xymatrix{
      P\ar[r]^{l}\ar[d]_{k}&A\ar[d]_-{\Gamma_g}\\
      A\ar[r]^-{\Gamma_f}&A\times B
    }
    \end{displaymath}
    we have
    \begin{gather*}
      k = p_A\circ\Gamma_f\circ k = p_A\circ\Gamma_g\circ l = l \\
      f\circ k = p_B\circ\Gamma_f\circ k = p_B\circ\Gamma_g\circ l = g\circ l
    \end{gather*}
    This implies that $k=l$ and $(P,k)$ is a cone over the pair $(\Gamma_f,\Gamma_g)$.
    Therefore the pullback of $(\Gamma_f,\Gamma_g)$ is precisely the equalizer of $(f,g)$.
  \end{proof}

  \begin{defn}
    A category $\Ii$ is said to be finitely generated if
    \begin{itemize}
      \item $\Ii$ has finitely many objects;
      \item there are finitely many arrows $f_1,\cdots,f_n$ such that each arrow of $\Ii$ is the composite of finitely many of these $f_i$.
    \end{itemize}
 \end{defn}

  \begin{prop}\label{prop:finitely generated limits}
    Let $D$ be a diagram in a finitely complete category. If the index category of $D$ is finitely generated, then $\invlim D$ exists.
  \end{prop}
  \begin{proof}
    Notice that a cone over $D$ is uniquely determined by these $D_{f_i}$, thus we can replace the second product in the proof of \ref{thm:complete_category} by $\prod\limits_{i=1}^n D_{s.f_i}$, and the statement follows.
  \end{proof}

\subsection{Exercises}
  \begin{ex}
    Describe and verify the dual notations and propositions in this section for colimit.
  \end{ex}
  \begin{ex}
    Let $\Cc$ be a complete category and $\Ii$ a small category. Then taking projective limit objects is a functor from $[\Ii^{\op},\Cc]$ to $\Cc$, while taking inductive limit objects is a functor from $[\Ii,\Cc]$ to $\Cc$.
  \end{ex}
  \begin{ex}
    Product preserving monomorphisms. That is the product of a family of monomorphisms is also a monomorphism.
  \end{ex}
  \begin{ex}
    Let $\Cc$ be a complete category and $f\colon A\to B$ a morphism in $\Cc$. Then taking pullback along $f$ is a functor from $\Cc/B$ to $\Cc/A$ and taking pushout is a functor form $A/\Cc$ to $B/\Cc$.
  \end{ex}


\newpage\section{Persevering limits}
  In this section and the followings, we will consider how limits or colimits behave under functors. Thus to view them as natural transformations may be more suitable.
  \begin{defn}
    A functor $F \colon \Aa \to \Bb$ is said to \termin{preserve limits} of shape $\Ii^{\op}$ if, whenever $\mu\colon \Delta_L\then D$ is a limit of a diagram $D\colon\Ii^{\op}\to\Aa$, the cone $F\ast \mu$ is then a limit of the diagram $F\circ D\colon\Ii^{\op}\to\Bb$. Briefly
    \begin{equation*}
      F(\invlim D) \approx \invlim F\circ D
    \end{equation*}

    A functor that preserves all small limits is simply said to \textbf{preserve limits} and is called a \termin{continuous functor}.
  \end{defn}
  \begin{rem}
    From the definition, on can see that a functor preserving limits of shape $\Ii^{\op}$ also preserving the existence of limits of $\Ii^{\op}-$diagrams. Indeed, $F(\invlim D)$ gives a limit of $F\circ D$.
  \end{rem}
    As an immediate consequence of Theorem \ref{thm:complete_category}, we get
  \begin{prop}
    Let $\Aa$ be a (finitely) complete category and $\Bb$ an arbitrary category A functor $F\colon\Aa\to\Bb$ preserves (finite) limits precisely when it preserves (finite) products and equalizers.
  \end{prop}
    By Proposition \ref{prop:mono and kernel pair}, we have
  \begin{prop}
    A functor which preserves pullbacks also preserves monomorphisms.
  \end{prop}

  The following proposition provides examples of continuous functors.
  \begin{prop}\label{prop:representable functor preserves limits}
    A covariant representable functor is continuous. Moreover, it preserves all existing limits including large ones.
  \end{prop}
  \begin{proof}
    It suffices to show that for any $C\in\ob\Cc$ and $D$ a diagram in $\Cc$, we have
    \begin{equation*}
      \Hom(C,\invlim D) \approx \invlim \Hom(C,D)
    \end{equation*}
    Which is precisely Proposition \ref{prop:limit-colimit}.
  \end{proof}
  Similarly, we have
  \begin{prop}
    A representable presheaf transforms colimits into limits and epimorphisms to monomorphisms.
  \end{prop}

  A related concept of preserving limits is reflecting limits.
  \begin{defn}
    A functor $F \colon \Aa \to \Bb$ is said to \termin{reflect limits} of shape $\Ii^{\op}$ when, for each cone $\mu\colon \Delta_L\then D$ over a diagram $D\colon\Ii^{\op}\to\Aa$, it is a limit of $D$ if its image $F\ast\mu$ is a limit of the diagram $F\circ D\colon\Ii^{\op}\to\Bb$.
    A functor that reflects all small limits is simply said to \textbf{reflect limits}.
  \end{defn}
  \begin{rem}
    A limits reflecting functor may \textbf{not} ensure the existence of limits. It is possible that the diagram $F\circ D$ has limits in $\Bb$ but none of them has a preimage in $\Aa$.
  \end{rem}

  \begin{prop}
    Let $F\colon\Aa\to\Bb$ be a continuous functor. If $\Aa$ is complete and $F$ reflects isomorphisms, then $F$ also reflects limits.
  \end{prop}
  \begin{proof}
    Let $\mu\colon \Delta_C\then D$ be a cone over a small diagram $D\colon\Ii^{\op}\to\Aa$ such that $F\ast\mu$ is a limit of $F\circ D$.

    Since $\Aa$ is complete, there exists a limit $\nu\colon \Delta_L\then D$ of $D$ and a unique factorization $f\colon C\to L$ of $\mu$ by $\nu$.

    The continuous functor $F$ transforms it to a limit $F\ast\nu$ of $F\circ D$ and $f$ to an isomorphism between these two limits of $F\circ D$.

    Since $F$ reflects isomorphisms, $f$ is also an isomorphism and thus $\mu$ is a limit of $D$.
  \end{proof}

  \begin{prop}
    Let $\Aa,\Bb$ be finitely complete categories and $F\colon\Aa\to\Bb$ a functor which preserves (resp. reflects) finite limits. Then $F$ preserves (resp. reflects) finitely generated limits.
  \end{prop}
  \begin{proof}
    A finitely generated limit can be expressed via equalizers and finite products (see \ref{prop:finitely generated limits}), from which the result follows.
  \end{proof}

  \begin{prop}
    A fully faithful functor reflects limits.
  \end{prop}
  \begin{proof}
    Notice that a fully faithful functor reflects terminal objects, thus the result follows.
  \end{proof}

\subsection{Exercises}
  \begin{ex}
    Describe the corresponding concepts (``preserve colimits'', ``cocontinuous functor'', ``reflect colimit'') and propositions for colimits and then prove them.
  \end{ex}
  \begin{ex}
    Consider a category $\Cc$ having products and a diagram $D$ with small index category. Construct $\alpha$ and $\beta$ as in the proof of Theorem \ref{thm:complete_category}. Prove that $\ker(\alpha,\beta)$ exists if and only if $\invlim D$ exists.
  \end{ex}
  \begin{ex}
    Consider a category $\Aa$ having products and a functor $F\colon\Aa\To\Bb$ preserving products and equalizers. Show that $F$ preserves limits.
  \end{ex}


\newpage\section{Absolute colimits}
  In the previous section we were concerned with a functor preserving all limits. Now we shall have a look at those limits preserved by all functors. In fact we shall develop the theory in the case of colimits since this is the case most commonly referred to in the examples.
  \begin{defn}
    A particular colimit diagram in a category $\Cc$ is called an \termin{absolute colimit} if it is preserved by every functor with domain $\Cc$.
  \end{defn}
  In general a colimit is absolute because the colimit is a colimit for purely ``diagrammatic'' reasons. For instance, split epimorphisms is absolute in this sense, because of the equation describe the section of a split epimorphism is preserved under any functor. While, an general epimorphism may be not absolute.
  \begin{exam}
    Initial objects are never absolute. Indeed, if $0$ is an initial object, then it is never preserved by the covariant representable functor $\hom(0,-)\colon\Cc\to\Set$.
  \end{exam}
  \begin{exam}
    Similarly, coproducts are never absolute.
  \end{exam}
  \begin{exam}
    The trivial example for absolute coequalizers is that the coequalizer of a pair of same epimorphism is absolute. Indeed, it is just the identity of codomain.
  \end{exam}
  The most common example is a split coequalizer.
  \begin{defn}
    A \termin{split coequalizer} $(C,e)$ of a pair $(f,g)$ is a co-cone under this pair such that the morphism $(f,e)\colon g\to e$ has a section in the arrow category.
  \end{defn}
  \begin{prop}\label{prop:split coeq is abs}
    A split coequalizer is an absolute coequalizer.
  \end{prop}
  \begin{proof}
    Let $(r,s)\colon e\to g$ be a section of $(f,e)$ in the arrow category. Then we have
    \begin{equation*}
      f\circ r=1_B\quad e\circ s=1_C\quad s\circ e=g\circ r
    \end{equation*}
    \begin{displaymath}
    \xymatrix{
      A\ar@<0.5ex>[r]^{f}\ar@<-0.5ex>[r]_{g}
      &B\ar[r]^{e}\ar@/_1.5pc/[l]_{r}\ar[dr]_-{p}
      &C\ar@/_1.5pc/[l]_{s}\ar@<0.5ex>[d]^-{q}\ar@<-0.5ex>[d]_-{\bar{q}}\\
      &&D
    }
    \end{displaymath}

    By the definition, $(C,e)$ is a co-cone under the pair $(f,g)$. Now we consider another co-cone $(D,p)$. Define $q=p\circ s$, then we have
    \begin{align*}
      q\circ e & = p\circ s\circ e \\
       & = p\circ g\circ r \\
       & = p\circ f\circ r \\
       & = p
    \end{align*}
    and if there exists another $\bar{q}$ such that $\bar{q}\circ e = p$, then
    \begin{align*}
      \bar{q} & = \bar{q}\circ e\circ s \\
       & = p\circ s\\
       & = q
    \end{align*}

    So, $(C,e)$ is a coequalizer of $(f,g)$. Moreover, since the equalities of the statement are preserved by any functor, the same conclusion applies to the image of above diagram under any functor.
  \end{proof}\label{prop:abs coeq}
  We now give a general characterization of absolute coequalizers, the previous example are just special cases of it.
  \begin{prop}
    Let $(C,e)$ be a co-cone under a pair $f_0,f_1\colon A\to B$. $(C,e)$ is an absolute coequalizer of $(f_0,f_1)$ if and only if there exist a section $s$ of $e$, a sequence of morphisms $r_1,\cdots,r_n\colon B\to A$ and a sequence of binary digits $\epsilon_1,\cdots,\epsilon_n$ such that
    \begin{align*}
      e\circ s & = 1_C \\
      f_{\epsilon_1}\circ r_1 & = 1_B\\
      &\cdots\\
      f_{1-\epsilon_i}\circ r_i & = f_{\epsilon_{i+1}}\circ r_{i+1}\\
      &\cdots\\
      f_{1-\epsilon_n}\circ r_n & = s\circ e
    \end{align*}
  \end{prop}
  \begin{proof}
    To show the ``only if'', just notice that, an absolute coequalizer must be preserved, in particular, by the hom-functor $\Hom(C,-)$ and $\Hom(B,-)$.

    The previous one shows the function $e_{\ast}\colon\Hom(C,B)\to\Hom(C,C)$ is split epi, thus the existence of the section $s$ follows. The latter one shows that $1_B$ and $s\circ e$ are in the same equivalence class in $\Hom(B,B)$ under the equivalence relation generated by the image of $(f_0,f_1)\colon A\to B\times B$, thus there exists a sequence connected them.

    Conversely, it is easy to check that if such section $s$ and sequence $r_1,\cdots,r_n$ exist, the given co-cone must be a coequalizer, for essentially the same reasoning in the proof of Proposition \ref{prop:split coeq is abs}.
  \end{proof}

  More generally, we offer some pleasant characterizations for absolute colimits.
  \begin{thm}
     For a co-cone $\mu\colon D\then\Delta_C$ under a diagram $D\colon \Ii\to \Cc$, the followings are equivalent:
     \begin{enumerate}
       \item $\mu$ is an absolute colimit.
       \item $\mu$ is a colimit which is preserved by the Yoneda embedding $\Upsilon$.
       \item $\mu$ is a colimit which is preserved by the covariant representable functors $\Hom(D_i,-)\colon \Cc\to \Set$ (for all $i\in\ob\Ii$) and $\Hom(C,-)\colon \Cc\to \Set$.
       \item There exists $i_0\in\ob\Ii$ and $d_0\colon C\to D_{i_0}$ such that
       \begin{enumerate}
         \item For every $i\in\ob\Ii$, $d_0 \circ \mu_i$ and $1_{D_i}$ are in the same connected component of the comma category $(D_i \down D)$.
         \item $\mu_{i_0} \circ d_0 = 1_{C}$.
       \end{enumerate}
     \end{enumerate}
  \end{thm}
  \begin{proof}
    \emph{1 $\then$ 2 $\then$ 3} are obvious.

    To see \emph{3 $\then$ 4}, notice that in $\Set$, a colimit is a quotient set of the disjoint union of objects in the diagram. Apply this fact to the colimit $\Hom(C,-)\ast\mu$, we see that $1_C\in\Hom(C,C)$ must have a preimage in some $\Hom(C,D_{i_0})$. That means $\mu_{i_0}$ have sections $d_0\colon C\to D_{i_0}$. Which is \emph{b)}.

    Next we check \emph{a)} for such $i_0$ and $d_0$. To do this, we consider the colimit $\Hom(D_i,-)\ast\mu$. We have
    \begin{align*}
      \mu_{i_0}\circ d_0\circ \mu_i & = \mu_i \\
      1_{D_i}\circ \mu_i & = \mu_i
    \end{align*}
    Thus $d_0 \circ \mu_i$ and $1_{D_i}$ have the same image in the colimit $\Hom(D_i,C)$, which is a quotient set of the disjoint union $\bigsqcup\limits_{j\in\ob\Ii}\Hom(D_i,D_j)$. Notice that the quotient is under the equivalence relation generated by the morphisms in the diagram $D$. Then the similar reasoning as in the proof of Proposition \ref{prop:abs coeq} shows that there exists a sequence of morphisms in the diagram $D$ such that $d_0 \circ \mu_i$ and $1_{D_i}$ are connected by them. Which is \emph{a)}.

    Finally, we prove \emph{4 $\then$ 1}. Let $\mu\colon D\then\Delta_C$ be a co-cone satisfies the conditions in \emph{4} and $\nu\colon D\then\Delta_T$ be another co-cone. We need to show that there exists a unique factorization $f\colon C\to T$ of $\nu$ by $\mu$ and make sure such a proof is purely ``diagrammatic''.

    Let $f=\nu_{i_0}\circ d_0$. Then $f\circ \mu_i = \nu_{i_0} \circ (d_0 \circ \mu_i)$. Since $d_0 \circ \mu_i$ and $1_{D_i}$ are connected in $(D_i \down D)$, we have
    \begin{align*}
      f\circ \mu_i & = \nu_{i_0} \circ (d_0 \circ \mu_i) \\
      & \cdots \\
      & = \nu_i \circ 1_{D_i} = \nu_{i}
    \end{align*}
    Which shows that $f$ is the required factorization.

    To show the uniqueness, consider another factorization $\bar{f}$. Then
    \begin{align*}
      \bar{f} & = \bar{f} \circ \mu_{i_0} \circ d_0 \\
      & = \nu_{i_0} \circ d_0 = f
    \end{align*}

    Therefore, $\mu$ is a colimit of $D$. Moreover, the equations in this proof are obvious preserved by any functor. Thus $\mu$ is absolute.
  \end{proof}
\subsection{Exercises}
\begin{exam}
  Let $(C,e)$ be the coequalizer of its kernel pair $(f,g)$. Shows that if $e$ is split epi, then the coequalizer is split and thus absolute.
\end{exam}


\newpage\section{Final functor}
A final functor $F \colon\Ii\to\Jj$ is which that if we can restrict diagrams on $\Jj$ to diagrams on $\Ii$ along $F$ without changing their colimit.

\begin{defn}
  A functor $F\colon\Ii\to\Jj$ is said to be \termin[cofinal]{cofinal functor} if for every object $j \in \Jj$ the comma category $(j\down F)$ is connected.
  A functor is said to be \termin[initial]{initial functor} (or \termin[co-cofinal]{co-cofinal functor}) if its opposite is final.
\end{defn}
\begin{rem}
  Beware that these properties are pretty much unrelated to that of a functor being an initial object or terminal object in the functor category $[\Ii,\Jj]$.
\end{rem}
\begin{rem}
  The prefix ``co-'' in the term ``cofinal'' does not mean it is the dual of ``final''. Indeed, ``\termin{final functor}'' is just another name of ``cofinal functor''.
\end{rem}

\begin{lem}\label{lem:restrict diagram}
  Let $F\colon\Ii\to\Jj$ be a functor and $D\colon\Jj\to\Cc$ be a diagram. Then any co-cone $\alpha$ under $D$ can be restricted to be a co-cone $\alpha\ast F$ under $D\circ F$. If $F$ is cofinal, then any co-cone $\alpha$ under $D\circ F$ can be uniquely extended to a co-cone $\beta$ under $D$ such that $\beta\ast F=\alpha$.
\end{lem}
\begin{proof}
  The first statement is obvious, we now prove the second.

  For any $j\in\ob\Jj$, since $(j\down F)$ is nonempty, then there exists an $\Jj-$arrow $j\to F(i)$ for some $i\in\Ii$. Let $\beta_j$ be the composite $\alpha_i\circ D(j\to F(i))$. Then such a morphism $\beta_j$ is independent for the choice of $i$. Indeed, for any $\Ii-$arrow $i\to i'$ inducing a morphism in $(j\down F)$, it is easy to see that $\alpha_i\circ D(j\to F(i)) = \alpha_{i'}\circ D(j\to F(i'))$. Since $(j\down F)$ is connected, the same thing hold for all objects in it and the result composite is then independent for choice of $i$.
  By similar reasoning, it is easy to see that $\beta$ form a co-cone under $D$.

  Since $\beta_{F(i)} = \alpha_i\circ D(1_{F(i)}) = \alpha_i$, we get $\beta\ast F=\alpha$.

  If there exists another co-cone $\gamma$ under $D$ such that $\gamma\ast F=\alpha$, then we have $\gamma_{F(i)}=\alpha_i$ for all $i\in\ob\Ii$. For any $j\in\ob\Jj$, since $(j\down F)$ is nonempty, then there exists an $\Jj-$arrow $j\to F(i)$ for some $i\in\Ii$. Then we have $\gamma_j = \gamma_{F(i)}\circ D(j\to F(i)) = \alpha_i\circ D(j\to F(i)) =\beta_j$.
\end{proof}
\begin{rem}
  Similar result holds for cones.
\end{rem}

\begin{thm}\label{thm:cofinal}
Let $F\colon\Ii\to\Jj$ be a functor, then the following conditions are equivalent.
  \begin{enumerate}
    \item $F$ is cofinal.
    \item For all diagrams $D\colon\Jj\to\Set$ the natural function between colimits
          \begin{equation*}
              \dirlim D \circ F \to \dirlim D
          \end{equation*}
          is bijective.
    \item For all categories $\Cc$ and all diagrams $D\colon\Jj\to\Cc$ the natural morphism between colimits
          \begin{equation*}
              \dirlim D \circ F \to \dirlim D
          \end{equation*}
          is an isomorphism.
    \item For all diagrams $D\colon\Jj^{op}\to\Set$ the natural function between limits
          \begin{equation*}
              \invlim D \to \invlim D \circ F^{\op}
          \end{equation*}
          is bijective.
    \item For all categories $\Cc$ and all diagrams $D\colon\Jj^{\op}\to\Cc$ the natural morphism
          \begin{equation*}
              \invlim D \to \invlim D \circ F^{\op}
          \end{equation*}
          is an isomorphism.
    \item For all $j \in \Jj$
          \begin{equation*}
              \dirlim_{i\in\ob\Ii} \Hom_{\Jj}(j,F(i))\approx\Pt
          \end{equation*}
          where $\Pt$ denote a singleton.
\end{enumerate}
\end{thm}
\begin{rem}
  The isomorphisms in these condition should be understood as follows: whenever one of the both sides exists, then so does the other side and the morphism thus exists and must be an isomorphism. Or see \ref{def:limit as rep functor}.
\end{rem}



\begin{proof}
  \emph{3} $\then$ \emph{2}, \emph{5} $\then$ \emph{4}, \emph{2} $\Leftrightarrow$ \emph{4} and \emph{3} $\Leftrightarrow$ \emph{5} are obvious.

%  By \ref{def:limit as rep functor}, we can regard these limits and colimits as presheaves and functors to avoid to discuss the existence of them.

  \emph{1} $\then$ \emph{3}. Assume $\dirlim D\circ F = \nu\colon D\circ F\then \Delta_C$, by Lemma \ref{lem:restrict diagram}, we can uniquely extended it to a co-cone $\mu\colon D\then\Delta_C$. We now show that it is a colimit of $D$.

  For any co-cone $\tau\colon D\then\Delta_T$ under $D$, $\tau\ast F$ is a co-cone under $D\circ F$,  then we get a unique factorization $\phi\colon C\to T$ of $\tau\ast F$ by $\nu=\mu\ast F$. That is $\phi\circ\mu\ast F=\tau\ast F$. By  Lemma \ref{lem:restrict diagram}, it can be uniquely extended to a co-cone under $D$, which is $\phi\circ\mu = \tau$.

  If there exists another factorization $\psi\colon C\to T$ of $\tau$ by $\mu$. Then it is also a factorization of $\tau\ast F$ by $\nu=\mu\ast F$ and thus $\psi=\phi$.

  Conversely, assume $\dirlim D = \mu\colon D\then \Delta_C$. Any co-cone under $D\circ F$ can be uniquely extended to a co-cone under $D$, then the unique factorization follows and $\mu\ast F$ is a colimit of $D\circ F$.

  \emph{2} $\then$ \emph{6}. For an arbitrary $j_0\in\ob\Jj$. Let $D$ be the functor $j\mapsto\Hom_{\Jj}(j_0,j)$, then
  $\dirlim D\circ F = \dirlim D$. We only need to check that $\dirlim D$ is a singleton. Indeed, for any $f\in\Hom_{\Jj}(j_0,j)$, we have $f\circ 1_{j_0} = f$. Consider the functions
  \begin{equation*}
    \Hom(j_0,j_0)\markar{f_{\ast}}\Hom(j_0,j)\to\dirlim D
  \end{equation*}
  One can see that the image of $f$ in $\dirlim D$ should be the image of $1_{j_0}$ and thus $\dirlim D$ is a singleton.

  \emph{6} $\Leftrightarrow$ \emph{1}. Since $\dirlim_{i\in\ob\Ii} \Hom_{\Jj}(j,F(i))$ is a quotient set of the disjoint union of those $\Hom_{\Jj}(j,F(i))$ and two objects $j\to F(i)$ and $j\to F(i')$ are in the same equivalent class if they are connected by a zigzag of morphisms in $(j\down F)$, the cardinal of $\dirlim_{i\in\ob\Ii} \Hom_{\Jj}(j,F(i))$ is the same as the cardinal of the set of connected components in $(j\down F)$.
\end{proof}

\begin{lem}\label{lem:connected category}
  A category $\Ii$ is connected if and only if $\dirlim_{\Ii}\Delta_{\Pt}=\Pt$.
\end{lem}
\begin{proof}
  Since $\dirlim_{\Ii}\Delta_{\Pt}$ is a quotient set of the disjoint union of a family of $\Pt_i$ indexed by $\Ii$ and two objects $x_i\in\Pt_i$ and $x_{i'}\in\Pt_{i'}$ are in the same equivalent class if they are connected by a zigzag of functions between those singletons, the cardinal of $\dirlim_{\Ii}\Delta_{\Pt}$ is the same as the cardinal of the set of connected components in $\Ii$.
\end{proof}

\begin{prop}
If $F\colon\Ii\to\Jj$ is cofinal then $\Ii$ is connected if and only if $\Jj$ is.
\end{prop}
\begin{proof}
  Let $\Delta^{\Ii}_{\Pt}\colon\Ii\to\Set$ and $\Delta^{\Jj}_{\Pt}\colon\Jj\to\Set$ be the constant functors. Then $\Delta^{\Ii}_{\Pt}=\Delta^{\Jj}_{\Pt}\circ F$ and thus $\dirlim\Delta^{\Ii}_{\Pt}=\dirlim\Delta^{\Jj}_{\Pt}$.
\end{proof}

\begin{prop}
Let $F\colon\Ii\to\Jj$ and $G\colon\Jj\to\Kk$ be two functors.
\begin{enumerate}
  \item If $F$ and $G$ are cofinal, then so is their composite $G \circ F$.
  \item If $F$ and the composite $G \circ F$ are cofinal, then so is $G$.
  \item If $G$ is a fully faithful functor and the composite $G \circ F$ is cofinal, then both functors separately are cofinal.
\end{enumerate}
\end{prop}
\begin{proof}
  Let $D\colon\Kk\to\Set$ be a functor, then consider the natural functions:
  \begin{equation*}
    \dirlim D\circ G\circ F \To \dirlim D\circ G \To \dirlim D
  \end{equation*}

  Then \emph{1}, \emph{2} are obvious. For \emph{3}, since $G$ is fully faithful, we have $(j\down F)\simeq(G(j)\down G\circ F)$. The latter comma category is connected since $G \circ F$ is cofinal, thus $F$ is cofinal and the result follows from \emph{2}.
\end{proof}

\begin{exam}
If $\Ii$ has a terminal object then the functor $F\colon\one \to \Ii$ that picks that terminal object is cofinal: for every $i \in \ob\Ii$ the comma category $(i\down F)$ is equivalent to $\one$. The converse is also true: if a functor $\one\to\Ii$ is cofinal, then its image is a terminal object.

In this case the statement about preservation of colimits states that the colimit over a category with a terminal object is the value of the diagram at that object. Which is also readily checked directly.
\end{exam}

\begin{defn}
  A subcategory $\Ii$ of $\Jj$ is said to be \termin[cofinal]{cofinal subcategory} if the inclusion functor is cofinal.
\end{defn}
A useful sufficient condition is
\begin{prop}\label{prop:cofinal subcategory}
  A full subcategory $\Ii$ of $\Jj$ is cofinal if for any $j\in\Jj$, there exists an arrow $j\to i$ such that $i\in\Ii$.
\end{prop}
\begin{cor}
  If $\Ii$ has a terminal object $1$, then the $\{1\}$ itself is a cofinal subcategory of $\Ii$.
\end{cor}
\begin{cor}
  If $\Ii$ has a terminal object $1$, then for any $\Ii-$diagram $D$, $\mu\colon D\then \Delta_{D_1}$ is a colimit of $D$.
  Similarly, If $\Ii$ has a initial object $0$, then for any $Ii-$diagram $D$, $\mu\colon\Delta_{D_0}\then D$ is a limit of $D$.
\end{cor}

\begin{defn}
  A category $\Ii$ is said to be \termin[cofinally small]{cofinally small category} if there exists a small category $\Ii_0$ and a cofinal functor $\Ii_0\to\Ii$.
\end{defn}

\begin{defn}
  Two diagrams $D,D'$ in a category $\Cc$ are said to be \termin[cofinal]{cofinal diagram} if they have equivalent colimits.
\end{defn}
\begin{rem}
  It is often said that two diagrams are cofinal even when neither has a colimit, if they acquire a common colimit on passing to a suitable completion of $\Cc$. This can probably be phrased internally to $\Cc$, at the cost of intuition.
\end{rem}

\subsection{Exercises}
  \begin{ex}
    Let $D\colon\Ii^{\op}\to\Cc$ be a diagram. The category $\Cone(D)$ has a functor to $\Cc$, mapping each cone to its vertex. Prove that $D$ has a limit if and only if this functor has a colimit.
  \end{ex}

\newpage\section{Commutativity of limits and colimits}
  In this section, we consider the following kind of diagram
  \begin{equation*}
    D\colon\Ii\times\Jj\To\Cc
  \end{equation*}

  If we fix an $i\in\ob\Ii$, the bifunctor $D$ induces the following diagram:
  \begin{equation*}
    D(i,-)\colon\Jj\To\Cc
  \end{equation*}

  Then we can consider its limits and colimits.
%  After that, we vary the index $i$ and consider the limits and colimits.

  A morphism $f\colon i\to i'$ in $\Ii$ induces a natural transformation
  \begin{equation*}
    D(f,-)\colon D(i,-)\then D(i',-)
  \end{equation*}
  and actually defines a functor
  \longmapdes{D(f,-)_{\ast}}{\Cone(D(i,-))}{\Cone(D(i',-))}{\mu}{D(f,-)\circ\mu}
%  \longmapdes{D(f,-)^{\ast}}{\Cocone(D(i',-))}{\Cocone(D(i,-))}{\nu}{\nu\circ D(f,-)}

  If the limits of $D(i,-)$ and $D(i',-)$ exist, then $D(f,-)_{\ast}$ maps the limit of $D(i,-)$ to a cone over $D(i',-)$, which is equipped with a unique morphism to the limit of $D(i',-)$. Therefore the functor induces a unique factorization:
  \begin{displaymath}
    \xymatrix{
      {\invlim_{\Jj} D(i,-)}\ar@{=>}[d] \ar@{-->}[r]^{L_f} & {\invlim_{\Jj} D(i',-)}\ar@{=>}[d] \\
      D(i,-)\ar@{=>}[r] & D(i',-)
    }
  \end{displaymath}
  Notice that since cones are actually natural transformations from constant functors, thus the top row is just a morphism in $\Cc$.

  If all the functor $D(i,-)$ have limits, then taking limit is a functor $L\colon\Ii\to\Cc$ which maps object $i$ to $\invlim_{\Jj} D(i,-)$ and arrow $f$ to $L_f$.

  Now, $L$ itself is a diagram of shape $\Ii$, thus we can consider its limit and colimit, which are denoted by
  \begin{equation*}
    \invlim_{\Ii}\invlim_{\Jj} D \qquad\text{and}\qquad \dirlim_{\Ii}\invlim_{\Jj} D
  \end{equation*}

  On the other hand, an analogous reasoning shows that if all the functor $D(i,-)$ have colimits, then taking colimit is a functor $C\colon\Ii\to\Cc$. Thus we can consider its limit and colimit, which are denoted by
  \begin{equation*}
    \invlim_{\Ii}\dirlim_{\Jj} D \qquad\text{and}\qquad \dirlim_{\Ii}\dirlim_{\Jj} D
  \end{equation*}

  Similarly, we can begin with fixing a $j\in\Jj$ and do the same things. Then we have the following mixed limits and colimits
  \begin{equation*}
    \invlim_{\Jj}\invlim_{\Ii} D \qquad \dirlim_{\Jj}\invlim_{\Ii} D \qquad \invlim_{\Jj}\dirlim_{\Ii} D \qquad \dirlim_{\Jj}\dirlim_{\Ii} D
  \end{equation*}

  Here the question follows: what is the relationship of these limits and colimits listed above ?

\subsection{Interchange property}
  \begin{prop}\label{prop:interchange property of limits}
    Consider a complete category $\Cc$ and two small categories $\Ii,\Jj$. Given a functor $D\colon\Ii\times\Jj\to\Cc$ and using the previous notations, the following interchange property holds:
    \begin{equation*}
      \invlim_{\Ii}\invlim_{\Jj} D \cong \invlim_{\Jj}\invlim_{\Ii} D
    \end{equation*}
  \end{prop}
  \begin{proof}
    We construct some canonical morphisms and show that they are isomorphisms. Then the require property follows.

    Notice that an arrow $(i,j)\to(i',j')$ in $\Ii\times\Jj$ consists of an arrow $i\to i'$ in $\Ii$ and an arrow $j\to j'$ in $\Jj$, that the limit $\invlim_{\Ii}\invlim_{\Jj} D$ is a cone over those $\invlim_{\Jj} D(i,-)$ while each of them is a cone over those $D(i,j)$ with $i$ fixed and that the similar fact holds for $\invlim_{\Jj}\invlim_{\Ii} D$. We have the following commutative diagram
  \begin{displaymath}
    \xymatrix{
      &&{\invlim_{\Jj}\invlim_{\Ii} D}\ar[d]\ar[dr]\ar@{-->}[dl]_{\nu}
      &\\
      &{\invlim_{\Ii\times\Jj} D}\ar[dr]\ar@{-->}[d]\ar@{-->}[r]\ar@{-->}@/_1.5pc/[dl]_{\mu'}\ar@{-->}@/^1.5pc/[ur]^{\nu'}
      &{\invlim_{\Ii} D(-,j)}\ar[d]\ar[r]
      &{\invlim_{\Ii} D(-,j')}\ar[d]
      \\{\invlim_{\Ii}\invlim_{\Jj} D}\ar[r]\ar[dr]\ar@{-->}[ur]^{\mu}
      &{\invlim_{\Jj} D(i,-)}\ar[r]\ar[d]
      &{D(i,j)}\ar[r]\ar[d]\ar[dr]
      &{D(i,j')}\ar[d]
      \\&{\invlim_{\Jj} D(i',-)}\ar[r]
      &{D(i',j)}\ar[r]
      &{D(i',j')}
    }
  \end{displaymath}

  Then we can see that both $\invlim_{\Ii}\invlim_{\Jj} D$ and $\invlim_{\Jj}\invlim_{\Ii} D$ are cones over $D$ thus there exist unique factorizations $\mu$ and $\nu$ respectively. Those are our require canonical morphisms.

  On the other hand, for any fixed $i$, the limit $\invlim_{\Ii\times\Jj} D$ is a cone over $D(i,-)$, thus the unique factorization exists. Moreover, this factorization commute with the morphisms between those $D(i,-)$ and make $\invlim_{\Ii\times\Jj} D$ to be a cone over $L$, which ensure that there exists a unique factorization $\mu'$. Similarly, a unique factorization $\mu'$.

  It is easy to check that $\mu'$ and $\nu'$ are the inverse of $\mu$ and $\nu$ respectively, thus $\mu$ and $\nu$ are isomorphisms.

  To see the naturality, consider another diagram $D'$ of shape $\Ii\times\Jj$ and a natural transformation $\alpha$ between $D$ and $D'$. Then we have a similar commutative diagram for $D'$ as above one and the lower right squares are connected by $\alpha$. Then use the universal properties of limits, it is easy to see that $\alpha$ provides a corresponding of those diagrams and the naturality follows.
  \end{proof}
  \begin{exam}
    \begin{equation*}
      \prod_{i}(\prod_{j} D_{i,j}) \approx \prod_{j}(\prod_{i} D_{i,j})
    \end{equation*}
  \end{exam}
  \begin{exam}
    \begin{equation*}
      \ker(\prod_i f_i,\prod_i g_i) \approx \prod_i \ker(f_i,g_i)
    \end{equation*}
  \end{exam}

\subsection{Filtered colimits commute with finite limits}
  Now we consider the mixed interchange property for
  \begin{equation*}
    \dirlim_{\Ii}\invlim_{\Jj} D \qquad\text{and}\qquad \invlim_{\Jj}\dirlim_{\Ii} D
  \end{equation*}

  However, it is easy to see that the mixed interchange property does not hold in general. For instance, consider the following mixed interchange property
  \begin{equation*}
    (A\times B)\sqcup(C\times D)\approx(A\sqcup B)\times(C\sqcup D)
  \end{equation*}
  It is easy to see that it is false just by a cardinality argument.

  Now there is a very important case in which the mixed interchange property holds in $\Set$: this is the case where $\Ii$ is filtered and $\Jj$ is finite.
  \begin{defn}
    A \termin{filtered category} is a category $\Cc$ in which every finite diagram has a cocone.
  \end{defn}
  This can be rephrased in more elementary terms by saying that
  \begin{prop}\label{prop:filtered category}
    A category $\Cc$ is filtered if and only if
    \begin{enumerate}
      \item $\Cc$ is not empty.
      \item For any two objects $A,B\in\ob\Cc$, there exists an object $C\in\ob\Cc$ and morphisms $A\to C$ and $B\to C$.
      \item For any two parallel morphisms $f,g\colon A\to B$ in $\Cc$, there exists a morphism $h\colon B\to C$ such that $h\circ f=h\circ g$.
    \end{enumerate}
  \end{prop}
  \begin{proof}
    Just as all finite colimits can be constructed from initial objects, binary coproducts, and coequalizers, so a cocone on any finite diagram can be constructed from these three.

    In deed, whenever there is two objects in the diagram, we can apply \emph{b)} to find an object under them, if there is an arrow between them, we can then apply \emph{c)} to treat the triangle commutative. After this surgery, we get a smaller diagram such that a co-cone under the original diagram is a co-cone under it. Since we consider finite diagrams, after finitely many steps, we get an object, which is a require co-cone.
  \end{proof}

  \begin{exam}
    A directed poset (or filtered poset, it depends on how you view the order as arrows) is a filtered category.
  \end{exam}
  \begin{exam}
    Filtered categories are not necessarily small, or even not essentially small. For example, the category of all ordinals is filtered and not essentially small.
  \end{exam}
  \begin{exam}
    Every category with a terminal object is filtered.
  \end{exam}
  \begin{exam}
    Every category which has finite colimits is filtered.
  \end{exam}
  \begin{exam}
    A product of filtered categories is filtered.
  \end{exam}

  Recall that the colimit of a diagram $D$ in the category $\Set$ is given by a quotient set
  \begin{equation*}
    \dirlim D = (\bigsqcup_{i\in\ob\Ii} D_i)/\sim
  \end{equation*}
  where the equivalence relation $\sim$ is that which is generated by
  \begin{equation*}
    \{(x\in D_i)\sim(x'\in D_{i'})\mid \exists(f\colon i\to i')\st(D_f(x)=x')\}
  \end{equation*}

  If the index category $\Ii$ is filtered, then the equivalence relation $\sim$ can be write down explicitly
  \begin{equation*}
    (x\in D_i)\sim(x'\in D_{i'})\text{ if }\exists(f\colon i\to i'',g\colon i'\to i'')\st(D_f(x)=D_g(x'))
  \end{equation*}

  This gives a very esay description of colimits in $\Set$ and allowed us to write down the mixed interchange property in $\Set$ explicitly.

  \begin{thm}\label{thm:mixed interchange property}
    Consider a filtered category $\Ii$ and a finite category $\Jj$. Given a functor $D\colon\Ii\times\Jj\to\Set$, the following mixed interchange property holds:
    \begin{equation*}
      \dirlim_{\Ii}\invlim_{\Jj} D \cong \invlim_{\Jj}\dirlim_{\Ii} D
    \end{equation*}
  \end{thm}
  \begin{proof}
    First of all, we construct the canonical morphism between them. Consider the following commutative diagram
    \begin{displaymath}
      \xymatrix{
      &{\invlim_{\Jj} D(i,-)}\ar[r]\ar[d]\ar[dl]
      &{D(i,j)}\ar[r]\ar[d]
      &{D(i,j')}\ar[d]
      \\{\dirlim_{\Ii}\invlim_{\Jj} D}\ar@{-->}[ddrr]_{\lambda}
      &{\invlim_{\Jj} D(i',-)}\ar[r]\ar[l]
      &{D(i',j)}\ar[r]\ar[d]
      &{D(i',j')}\ar[d]
      \\&&{\dirlim_{\Ii} D(-,j)}\ar[r]
      &{\dirlim_{\Ii} D(-,j')}
      \\&&{\invlim_{\Jj}\dirlim_{\Ii} D}\ar[u]\ar[ur]
      &
      }
    \end{displaymath}

    It is easy to see that each $\invlim_{\Jj} D(i,-)$ is a cone over those $\dirlim_{\Ii} D(-,j)$ and thus there exists a unique factorization from it to $\invlim_{\Jj}\dirlim_{\Ii} D$. Consider these factorizations for all $\invlim_{\Jj} D(i,-)$, it is easy to check that $\invlim_{\Jj}\dirlim_{\Ii} D$ is a co-cone under them. Thus the unique factorization $\lambda$ exists.

    Moreover, this diagrammatic construction ensure that the canonical morphism is natural.

    But, as we have seen in the beginning of this subsection, the canonical morphism is not an isomorphism in general. So we need to write down it explicitly in our cases.

    Recall that, in $\Set$, the limit $\invlim_{\Jj} D(i,-)$ is given by
    \begin{equation*}
      L_i=\{(x_j)_{j\in\ob\Jj}\mid x_j\in D(i,j); D(i,f)(x_j)=x_{j'}, \forall f\colon j\to j'\}
    \end{equation*}
    Thus
    \begin{equation*}
      \dirlim_{\Ii}\invlim_{\Jj} D = (\bigsqcup_{i\in\ob\Ii} L_i)/\sim
    \end{equation*}
    Where the equivalence relation $\sim$ is given by
    \begin{align*}
      &((x_j)_{j\in\ob\Jj}\in L_i) \sim ((y_j)_{j\in\ob\Jj}\in L_{i'})\quad \text{if}\\
      & \exists(f\colon i\to i'',g\colon i'\to i'')\st(\forall j\in\ob\Jj)(D(f,j)(x_j)=D(g,j)(y_j))
    \end{align*}

    Similarly,
    \begin{equation*}
      \invlim_{\Jj}\dirlim_{\Ii} D = \{([x_j])_{j\in\ob\Jj}\mid x_j\in C_j; C_f([x_j])=[x_{j'}], \forall f\colon j\to j'\}
    \end{equation*}
    Where
    \begin{equation*}
      C_j = (\bigsqcup_{i\in\ob\Ii} D(i,j)/\sim_j\qquad C_f = [D(i,f)]
    \end{equation*}
    and the equivalence relation $\sim_j$ is given by
    \begin{align*}
      &(x\in D(i,j))\sim_j(y\in D(i',j))\quad \text{if}\\
      &\exists(f\colon i\to i'',g\colon i'\to i'')\st(D(f,j)(x)=D(g,j)(y))
    \end{align*}

    Use these explicit descriptions above, it is not difficult to verify that the canonical morphism $\lambda$ is given by
    \begin{equation*}
      \lambda([(x_j)_{j\in\ob\Jj}]) = ([x_j])_{j\in\ob\Jj}
    \end{equation*}

    Now we prove it is bijective.

    Let us prove first that $\lambda$ is injective. Consider $(x_j)_{j\in\ob\Jj}\in L_i$ and $(y_j)_{j\in\ob\Jj}\in L_{i'}$ such that $[x_j]=[y_j]$ for every $j\in\ob\Jj$. We need to show that $[(x_j)_{j\in\ob\Jj}]=[(y_j)_{j\in\ob\Jj}]$.

    For each $j\in\ob\Jj$, $[x_j]=[y_j]$ means there exists arrows $f_j\colon i\to i_j$ and $g_j\colon i'\to i_j$ such that $D(f_j,j)(x_j)=D(g_j,j)(y_j)$. For these $f_j$ and $g_j$, since $\Ii$ is filtered, we can find two composite morphisms $f\colon i\to i''$ and $g\colon i'\to i''$ such that $D(f,j)(x_j)=D(g,j)(y_j)$ for all $j\in\ob\Jj$. Which is that $[(x_j)_{j\in\ob\Jj}]=[(y_j)_{j\in\ob\Jj}]$.

    Let us now prove that $\lambda$ is surjective. Consider $([x_j])_{j\in\ob\Jj}$ in the right set, we need to show that there exist $[(y_j)_{j\in\ob\Jj}]$ in the left set such that $x_j\sim_jy_j$ for every $j\in\ob\Jj$.

    For those $x_j\in D(i_j,j)$, since $\Ii$ is filtered, we can find morphisms $f_j\colon i_j\to i$ with common target. Then we have $D(f_j,j)(x_j)\sim_jx_j$ for every $j\in\ob\Jj$. But this does not mean that $(D(f_j,j)(x_j))_{j\in\ob\Jj}$ lies in $L_i$.

    %Recall that $(y_j)_{j\in\ob\Jj}$ lies in some $L_{i'}$ requires that for all $g\colon j\to j'$, $D(i',g)(x_j)=x_{j'}$.
    Since $([x_j])_{j\in\ob\Jj}\in\invlim_{\Jj}\dirlim_{\Ii} D$, we have $C_d([x_j])=[x_{j'}]$ for every $d\colon j\to j'$. Which means $D(i_j,d)(x_j)\sim_{j'}x_{j'}$ and thus $D(f_j,d)(x_j)\sim_{j'}D(f_{j'},j)(x_{j'})$. Which means there exists arrows $g_d,h_d\colon i\to i_d$ such that
    \begin{equation*}
      D(g_d\circ f_j,d)(x_j) = D(h_d\circ f_{j'},j)(x_{j'})
    \end{equation*}

    Consider the diagram constituted of all the morphisms $g_d,h_d$ in $\Ii$, we then find a single morphism $k\colon i\to i'$ such that
    \begin{equation*}
      D(k\circ f_j,d)(x_j) = D(k\circ f_{j'},j)(x_{j'})
    \end{equation*}
    for all arrows $d$. Therefore the family $(D(k\circ f_j,j)(x_j))_{j\in\ob\Jj}$ lies in $L_i$ and its equivalence class is still mapped by $\lambda$ to $([x_j])_{j\in\ob\Jj}$.
  \end{proof}
  \begin{rem}
    This proof is due to \cite{borceux}. I recognize that it is very technical (but not difficult), but I don't have a soft one.
  \end{rem}

  The mixed interchange property in $\Set$ is the crucial property of filtered colimits.
  \begin{prop}\label{prop:filtered category 2}
    Let $\Ii$ be a small category. Then the following are equivalent:
    \begin{enumerate}
      \item $\Ii$ is filtered;
      \item colimits under $\Ii$ commute with finite limits in $\Set$.
      \item the diagonal functor $\Delta\colon\Ii\to[\Jj,\Ii]$ is cofinal for every finite category $\Jj$.
    \end{enumerate}
  \end{prop}
  \begin{proof}
    \emph{a)} $\then$ \emph{b)} is just Theorem \ref{thm:mixed interchange property}.
    To show \emph{b)} $\then$ \emph{a)}, we check the conditions in Proposition \ref{prop:filtered category}.

    Since the finite limit $\invlim\Pt$ is not always a singleton while a colimit of empty diagram in $\Set$ is always a singleton, $\Ii$ must be nonempty.

    For any $1,2\in\ob\Ii$, consider the discrete diagram $\{\Hom(j,i)\}_{j=1,2;i\in\ob\Ii}$. Then we have
    \begin{equation*}
      \dirlim_{\Ii}(\Hom(1,i)\times\Hom(2,i)) \approx (\dirlim_{\Ii}\Hom(1,i))\times(\dirlim_{\Ii}\Hom(2,i))
    \end{equation*}
    The components of right hand are singletons, thus so is the left hand. Hence there exists some $i\in\ob\Ii$ such that $\Hom(1,i)\times\Hom(2,i)$ is nonempty.

    For any arrows $f,g\colon1\to2$ in $\Ii$, consider the diagram consisting of objects $\{\Hom(j,i)\}_{j=1,2;i\in\ob\Ii}$ and functions $f_i,g_i\colon\Hom(2,i)\to\Hom(1,i)$ for every $i\in\ob\Ii$. Then we have
    \begin{equation*}
      \dirlim_{\Ii}\ker(f_i,g_i) \approx \ker(\dirlim_{\Ii}f_i,\dirlim_{\Ii}g_i)
    \end{equation*}
    The right hand is a singleton, thus there exist some $i$ such that $\ker(f_i,g_i)$ is nonempty.

    \begin{displaymath}
      \xymatrix@R=0.5cm{
        &\Delta_i%\ar[dr]
        &\\D
        \save [u].[d].[dr].[ur]*[F--]\frm{}
        \restore
        \ar@{}[ur]^{}|-{\SelectTips{eu}{}\object@{=>}}
        \ar@{}[dr]^{}|-{\SelectTips{eu}{}\object@{=>}}
        &\ar@{}[r]^{}|-{\SelectTips{eu}{}\object@{=>}}
        &\Delta_{i''}
        \\&\Delta_{i'}%*++=''b''%\ar[ur]
        &
        }
    \end{displaymath}

    \emph{c)} $\then$ \emph{a)} is obvious. For \emph{a)} $\then$ \emph{c)}, let $\Jj$ be a finite category, for any $D\in\ob[\Jj,\Ii]$, there exists a co-cone $D\then \Delta_i$, thus $(D\down\Delta)$ is nonempty.
    To see it is connected, consider two co-cone $D\then \Delta_i$ and $D\then \Delta_{i'}$, then the commutative diagrams in them form a new diagram $D^{+}\colon\Jj^{+}\to\Ii$, thus there exists a co-cone $D^{+}\then \Delta_{i''}$. But there exists a inclusion $\Jj\to\Jj^{+}$, then we get a co-cone $D\then D^{+}\then \Delta_{i''}$, which connects $D\then \Delta_i$ and $D\then \Delta_{i'}$.
  \end{proof}

  \begin{exam}
    In $\Set$, consider a set $X$ and the diagram $\Ii$ constituted of the finite subsets of $X$ and the canonical inclusions between them. This diagram is filtered and the filtered colimit of it is obviously $X$.
  \end{exam}
  \begin{exam}
    A poset is a filtered category. Particularly, a series indexed by natural numbers is a filtered diagram, thus its colimit can be interchanged with finite limits.
  \end{exam}
  \begin{defn}
    A category is said to be \termin[cofiltered]{cofiltered category} if its opposite is filtered.
  \end{defn}


\subsection{Sifted colimits commute with finite products}
  \begin{defn}
    A category $\Ii$ is called \termin[sifted]{sifted category} if colimits of diagrams of shape $\Ii$ (called \termin{sifted colimits}) commute with finite products in $\Set$.
  \end{defn}
  \begin{exam}
    Every filtered category is sifted.
  \end{exam}

  Since any $n-$ary product ($n\geqslant2$) can be constructed from binary products, we have
  \begin{prop}
    A category $\Ii$ is sifted if and only if colimits of $\Ii-$diagrams commute with nullary products and binary products.
  \end{prop}

  Like filtered category, we have the following proposition.
  \begin{prop}
    A small category $\Ii$ is sifted if and only if it is nonempty and the diagonal functor $\Delta\colon\Ii\to\Ii\times\Ii$ is cofinal.
  \end{prop}
  \begin{proof}
    For any two objects $i,i'$ in $\Ii$, the diagonal functor $\Delta\colon\Ii\to\Ii\times\Ii$ is cofinal implies that there exists a cospan $i\to i''\from i'$. Therefore the condition
    \begin{quote}
    ``$\Ii$ is nonempty and the diagonal functor $\Delta$ is cofinal''
    \end{quote}
    is equivalent to the condition
    \begin{quote}
    ``$\Ii$ is connected and the diagonal functor $\Delta$ is cofinal''.
    \end{quote}

    1, $\Ii$ is connected if and only if colimits of $\Ii-$diagrams commute with nullary products. Indeed, a nullary products is just the terminal object $\Pt$, thus the commutativity means $\dirlim_{\Ii}\Pt\approx\Pt$, which is equivalent to say that $\Ii$ is connected by Lemma \ref{lem:connected category}.

    2, The diagonal functor $\Delta$ is cofinal if and only if colimits of $\Ii-$diagrams commute with binary products. Indeed, If $D,D'$ are two $\Ii-$diagrams, then we have the following commutative diagram
    \begin{displaymath}
      \xymatrix{
        \dirlim (D_i\times D'_i) \ar[r]\ar@{=}[d]& (\dirlim D) \times (\dirlim D')\\
        \dirlim (D\times D')\circ\Delta \ar[r]& \dirlim D\times D'\ar[u]_-{\approx}
        }
    \end{displaymath}

    Then the upper morphism is an isomorphism if and only if so is the bottom, which is equivalent to say that the diagonal functor $\Delta$ is cofinal by Theorem \ref{thm:cofinal}.
  \end{proof}
  \begin{cor}
    Every category having finite coproducts is sifted.
  \end{cor}
  \begin{proof}
    Since a category having finite coproducts is nonempty (it has an initial object) and each category of cospans has an initial object (the coproduct) thus connected.
  \end{proof}

\subsection{Coproducts commute with connected limits}
  \begin{prop}
    Let $\Ii$ be a discrete small category, $\Jj$ a connected category and $D\colon\Ii\times\Jj\to\Set$ a functor. Then the canonical function
    \begin{equation*}
      \coprod_{\Ii}\invlim_{\Jj} D(i,j) \To \invlim_{\Jj} \coprod_{\Ii} D(i,j)
    \end{equation*}
    is bijective.
  \end{prop}
  \begin{proof}
    We have
    \begin{align*}
      &\invlim_{\Jj} D(i,j) = \{(x_j)_{j\in\ob\Jj}\mid x_j\in D(i,j); D(i,f)(x_j)=x_{j'}, \forall f\colon j\to j'\} \\
      &\invlim_{\Jj} \coprod_{\Ii} D(i,j) = \{(x_j)_{j\in\ob\Jj}\mid x_j\in \coprod_{\Ii}D(i,j);(\ast)\}
    \end{align*}
    where the condition $(\ast)$ is
    \begin{quote}
      for every $f\colon j\to j'$, we have $D(i,f)(x_j)=x_{j'}$ for some $i$.
    \end{quote}

    When $\Jj$ is connected, this condition require all the components of an element in $\invlim_{\Jj} \coprod_{\Ii} D(i,j)$ must be in the same $\invlim_{\Jj} D(i,j)$ for some $i$, which implies the canonical function is bijective.
  \end{proof}

\subsection{Universality of colimits}
  Let $\Cc$ be a category having pullbacks and colimits of shape $\Ii$.

  Let $B$ be a colimit object of some $\Ii-$diagram $D$, then foe any $\Cc-$morphism $f\colon A\to B$, consider the following pullback diagram
    \begin{displaymath}
      \xymatrix{
        f^{\ast}D\ar@{}[r]^{}|-{\SelectTips{eu}{}\object@{=>}}
        \ar@{}[d]^{}|-{\SelectTips{eu}{}\object@{=>}}
        &D\ar@{}[d]^{}|-{\SelectTips{eu}{}\object@{=>}}
        \\
        A\ar[r]^{f}&B
        }
    \end{displaymath}
  Since taking pullback along $f$ is a functor from $\Cc/B$ to $\Cc/A$, the left is then a co-cone under $f^{\ast}D$.

  Use this notation, we define
  \begin{defn}
    A colimit is said to be \termin{universal} if it is preserved under pullback. That means the pullback of it along any morphism is again a colimit.
  \end{defn}
  \begin{rem}
    We say that colimits of shape $\Ii$ are \termin{stable by base change} or \termin{stable under pullback} if for every $\Ii-$diagram $D$, its colimit is universal.
  \end{rem}

  \begin{thm}
    In $\Set$, small colimits are universal.
  \end{thm}
  \begin{proof}
    Since any small colimit can be constructed from coproducts and coequalizers, it suffices to prove the result separately for them.

    Let $\{B_i\}_{i\in I}$ be a set of sets, their coproduct is just the disjoint union $B=\bigsqcup_I B_i$. Consider an arbitrary function $f\colon A\to B$ and the cartesian diagram
    \begin{displaymath}
      \xymatrix{
        A_i\ar[d]\ar[r]&B_i\ar[d]
        \\
        A\ar[r]^{f}&B
        }
    \end{displaymath}
    Then, by simple compute, we get
    \begin{equation*}
      A_i = \{(a,f(a))\mid a\in A, f(a)\in B_i\} \approx \{a\in A\mid f(a)\in B_i\}
    \end{equation*}
    And thus $A\approx\bigsqcup_I A_i$.

    Let $f,g\colon S\to T$ be two parallel functions and $(Q,q)$ is their coequalizer. Consider an arbitrary function $h\colon P\to Q$ and the pullback diagram along $h$
    \begin{displaymath}
      \xymatrix{
        S'\ar[r]^{h''}\ar@<-0.5ex>[d]_{f'}\ar@<0.5ex>[d]^{g'}
        &S\ar@<-0.5ex>[d]_{f}\ar@<0.5ex>[d]^{g}
        \\
        T'\ar[d]_{p}\ar[r]^{h'}&T\ar[d]^{q}
        \\
        P\ar[r]^{h}&Q
        }
    \end{displaymath}
    Then $p\circ f' = p\circ g'$ and $f',g'$ are pullbacks of $f,g$ along $h'$ respectively (by Proposition \ref{prop:assofpullback}).

    Since $q$ is surjective, so is $p$, thus we can regard $P$ as a quotient set of $T'$ by the equivalence relation $R$:
    \begin{equation*}
      xR y \iff p(x)=p(y)
    \end{equation*}
    Then we need to show that $R$ is generated by $R_0=\{(f'(s'),g'(s'))\mid s'\in S'\}$.

    Since $p\circ f' = p\circ g'$, $R$ contains $R_0$.
    Conversely, consider two elements $x,y\in T'$ such that $p(x)=p(y)$. Then
    \begin{equation*}
      q\circ h' (x) = h\circ p (x) = h\circ p (y) = q\circ h'(y)
    \end{equation*}

    Since $(Q,q)=\coker(f,g)$, $(h' (x),h' (y))$ is contained in the equivalence relation generated by $\{(f(s),g(s))\mid s\in S\}$.
    Thus $R$ is contained in the equivalence relation generated by its inverse image in along $h'$. Denote this inverse image by $R_1$, then we only need to show $R_1$ is contained in $R_0$.

    Indeed, for any $(x,y)\in R_1$, there exists a $s\in S$ such that
    \begin{equation*}
      f(s)=h' (x),g(s)=h' (y)
    \end{equation*}
    But the upper two squares are cartesian, so there exists a $s'\in S'$ such that
    \begin{equation*}
      f'(s')=x,g'(s')=y,h''(s')=s
    \end{equation*}
    Thus $(x,y)\in R_0$.
  \end{proof}
  The universality of colimits is a very peculiar property which is much less common than to be filtered.

\subsection{Exercises}
  \begin{ex}
    Prove the interchange property for colimits.
  \end{ex}
  \begin{ex}
    Construct the colimit of a filtered diagram in $\Set$ explicitly.
  \end{ex}


\newpage\section{Limits and colimits in functor categories}
  Limits in functor categories are computed pointwise.
  \begin{prop}
    Consider categories $\Ii^{\op},\Aa,\Bb$, with $\Ii^{\op}$ and $\Aa$ small. Let $D\colon\Ii^{\op}\to[\Aa,\Bb]$ be a functor. If for every $\Aa-$object $A$ the functor $D(-)(A)\colon\Ii^{\op}\to\Bb$ has a limit, then $D$ has a limit as well and this limit is computed pointwise.
  \end{prop}
  \begin{proof}
    Recall that $[\Ii^{\op},[\Aa,\Bb]]\simeq[\Aa,[\Ii^{\op},\Bb]]$ (see \ref{prop:power law for functor}), this is the crucial fact we used in the proof.

    For any $A\in\ob\Aa$, let $\mu_A\colon\Delta_{L_A}\then D(-)(A)$ be the limit of $D(-)(A)$. For any morphism $f\colon A\to A'$, since $D$ is a functor from $\Ii$ to $[\Aa,\Bb]$, it induces a natural transformation $\phi_f\colon D(-)(A)\then D(-)(A')$. It is easy to see that $\phi_f\circ\mu_A$ is also a cone over $D(-)(A')$, thus there exists a unique morphism $L_f\colon A\to A'$ making the following diagram commute
    \begin{displaymath}
      \xymatrix{
        L_A\ar[d]_{L_f}\ar@{}[r]^-{\mu_A}|-{\SelectTips{eu}{}\object@{=>}}
        &D(-)(A)\ar@{}[d]^{\phi_f}|-{\SelectTips{eu}{}\object@{=>}}\\
        L_{A'}\ar@{}[r]^-{\mu_{A'}}|-{\SelectTips{eu}{}\object@{=>}}
        &D(-)(A')
        }
    \end{displaymath}

    Therefore the above data define a functor $L\colon\Aa\to\Bb$ mapping object $A$ to $L_A$ and morphism $f$ to $L_f$. This functor can also be view as a functor from $\Aa$ to $[\Ii^{\op},\Bb]$ mapping $A$ to $\Delta_{L_A}$. Then the above data define a natural transformation from such functor to $D$ viewed as a functor from $\Aa$ to $[\Ii^{\op},\Bb]$.
    Then, by interchange variables, we get a cone $\mu\colon\Delta_L\then D$.

    To show $\mu$ is a limit of $D$, we consider another cone $\tau\colon\Delta_T\then D$. By interchange variables, we get a natural transformation and apply to $f\colon A\to A'$, we get the following commutative diagram
    \begin{displaymath}
      \xymatrix{
        T_A\ar[d]_{T_f}\ar@{}[r]^-{\tau_A}|-{\SelectTips{eu}{}\object@{=>}}
        &D(-)(A)\ar@{}[d]^{\phi_f}|-{\SelectTips{eu}{}\object@{=>}}\\
        T_{A'}\ar@{}[r]^-{\tau_{A'}}|-{\SelectTips{eu}{}\object@{=>}}
        &D(-)(A')
        }
    \end{displaymath}

    For each object $A$, $\tau_A$ is a cone over $D(-)(A)$, thus there exists a unique factorization $\alpha_A\colon T_A\to L_A$ of $\tau_A$ by $\mu_A$. Then it is not difficult to check the commutativity of the following diagram
    \begin{displaymath}
      \xymatrix{
        T_A\ar[d]_{T_f}\ar[r]^{\alpha_A}
        &L_A\ar[d]^{L_f}\\
        T_{A'}\ar[r]^{\alpha_{A'}}
        &L_{A'}
        }
    \end{displaymath}

    Thus $\alpha$ is a natural transformation from $T$ to $L$ and then the unique factorization between $\Delta_T$ and $\Delta_L$ follows.
  \end{proof}

  As an immediate corollary we get
  \begin{thm}\label{thm:limits in Fun}
    If $\Aa$ is small and $\Bb$ is complete, then $[\Aa,\Bb]$ is complete and its limits are computed pointwise. Similarly, if $\Bb$ is cocomplete, then $[\Aa,\Bb]$ is cocomplete and its colimits are computed pointwise,
  \end{thm}
  \begin{cor}
    Consider a small category $\Cc$. Then we have:
    \begin{enumerate}
      \item $\PSh(\Cc)$ is complete and cocomplete.
      \item In $\PSh(\Cc)$, filtered colimits commute with finite limits.
      \item In $\PSh(\Cc)$, sifted colimits commute with finite products
      \item In $\PSh(\Cc)$, coproducts commute with connected limits
      \item In $\PSh(\Cc)$, colimits are universal.
    \end{enumerate}
  \end{cor}
  \begin{proof}
    By Theorem \ref{thm:limits in Fun} and results in the previous section.
  \end{proof}

  \begin{cor}
    If $\Cc$ is a small category. Then the Yoneda embedding $\Upsilon\colon\Cc\To\PSh(\Cc)$ preserves limits.
  \end{cor}
  \begin{rem}
    The Yoneda embedding does not in general preserve colimits.
  \end{rem}

  \begin{thm}
    Let $\Cc$ be a small category and $D\colon\Cc^{\op}\to\Set$ a presheaf. In $\PSh(\Cc)$, $D$ can be presented as the colimit of a diagram just constituted of representable functors and representable natural transformations.
  \end{thm}
  \begin{proof}
    Consider the composite functor
    \begin{equation*}
      \Elts(D)^{\op}\markar{\phi_D}\Cc\markar{\Upsilon}\PSh(\Cc)
    \end{equation*}
    where $\Elts(D)$ is the category of elements of $D$ defined in Example \ref{exam:Elts} and $\phi_D$ is the opposite of the forgetful functor.
    The its diagram is constituted of representable functors and representable natural transformations. We claim that $D$ is the colimit of it.

    The crucial fact we used in the proof is that for any presheaf $F$, the natural transformation $\alpha\colon\Hom(-,X)\then F$ is uniquely determined by the object $F(X)$ and the element $\alpha_X(1_X)$.

    Use this fact, it is easy to see that $D$ is a co-cone under $\Upsilon\circ\phi_D$. Moreover, Let $\Phi\colon\Upsilon\circ\phi_D \then \Delta_T$ be an arbitrary co-cone, $\Phi_{(D(X),x)}\colon\Hom(,X)\then T$ is uniquely determined by $T(X)$ and $\Phi_{(D(X),x)}(1_X)$. Then we get a natural transformation $\alpha\colon D\to T$ whose component is given by
    \longmapdes{\alpha_X}{D(X)}{T(X)}{x}{\Phi_{(D(X),x)}(1_X)}

    It is not difficult to check it is the required unique factorization.
  \end{proof}

  \begin{exam}
    The category $[\Aa.\Bb]$ can be complete even when $\Bb$ is not. An obvious example is obtained by taking $\Aa$ and $\Bb$ to be empty: $\Bb$ is not complete or cocomplete, since it does not have a terminal or an initial object. But $[\Aa,\Bb]$ is the category with just one single object (the empty functor) and the identity on it; that category is obviously both complete and cocomplete. And since $\Aa$ doesn't have any object, limits in $[\Aa,\Bb]$ are still pointwise! %See exercise 2.17.10 for a non-pointwise limit.
  \end{exam}

\subsection{Exercises}
  \begin{ex}
    If $\Aa$ is small and $\Bb$ is complete, then a natural transformation $\alpha$ in $[\Aa,\Bb]$ is monic if and only if its every component is monic.
  \end{ex}
  \begin{ex}
    Consider the category $\mathbf{2}$ with two objects $0,1$ and one single non-identity arrow $0\to1$. Find a category $\Cc$ and two functors from $\mathbf{2}$ to $\Cc$ such that their product exists but is not pointwise. [Hint: find a poset.]
  \end{ex}


\newpage\section{Limits and colimits in comma categories}
\begin{prop}
  Consider two complete categories $\Aa,\Bb$ and two continuous functors $F\colon\Aa\to\Cc,G\colon\Bb\to\Cc$, then the comma category $(F\down G)$ is complete and the domain functor $U\colon(F\down G)\to\Aa$ and codomain functor $V\colon(F\down G)\to\Bb$ are continuous.
\end{prop}
\begin{proof}
  Let $D\colon\Ii^{\op}\to(F\down G)$ be a small diagram. We construct its limit by the limits of $U\circ D$ and $V\circ D$, thus the statement follows.

  Assume $\mu\colon\Delta_A\to U\circ D$ and $\nu\colon\Delta_B\to V\circ D$ are limits of $U\circ D$ and $V\circ D$ respectively. Since $F,G$ are continuous, $F\ast\mu$ and $G\ast\nu$ are limits of $F\circ U\circ D$ and $G\circ V\circ D$ respectively. On the other hand, there is a natural transformation $\alpha\colon F\circ U\then G\circ V$, so we get the factorization $h\colon F(A)\to G(B)$ making the following diagram commute:
    \begin{displaymath}
      \xymatrix{
        F(A)\ar[d]_{h}\ar@{}[r]^-{F\ast\mu}|-{\SelectTips{eu}{}\object@{=>}}
        &F\circ U\circ D\ar@{}[d]^-{\alpha\ast D}|-{\SelectTips{eu}{}\object@{=>}}\\
        G(B)\ar@{}[r]^-{G\ast\nu}|-{\SelectTips{eu}{}\object@{=>}}
        &G\circ V\circ D
        }
    \end{displaymath}
    By the universal property of comma category, the both vertical data form a functor from $\Ii^{\op}$ to $(F\down G)$ and the horizontal data form a natural transformation $\eta\colon\Delta_{(A,h,B)}\then D$. It is now straightforward to check that $\eta$ is the limit of $D$.
\end{proof}

\begin{cor}
  Let $\Cc$ be a small category and $F\colon\Cc\to\Set$ a continuous functor, then $\Elts(F)$ is complete and the forgetful functor $\Elts(F)\to\Cc$ is continuous.
\end{cor}

  Let's consider slider categories. It should be noticed that the constant functor $\Delta_I$ does not, in general, preserve limits or colimits. Indeed in $\one$ one has $\ast\times\ast=\ast$ and $\ast\amalg\ast=\ast$, but generally $I \times I \neq I$ and $l \amalg I \neq I$. Nevertheless we have the following result.
\begin{prop}
  Consider a category $\Cc$ and a fixed object $I\in\Cc$.
  \begin{enumerate}
    \item If $\Cc$ is complete, $\Cc/I$ is complete.
    \item If $\Cc$ is cocomplete, $\Cc/I$ is cocomplete.
  \end{enumerate}
\end{prop}
The similar result hold for coslide categories.

Notice that the identity functor $\Id_{\Cc}$ is continuous, these results are just special cases of the following one, which we leave as exercises.
\subsection{Exercises}
  \begin{ex}
    Let $\Aa,\Bb$ be two complete (resp. cocomplete) categories, $F\colon\Aa\to\Cc$ a continuous (resp. cocontinuous)  functor and $G\colon\Bb\to\Cc$ an arbitrary functor. Show that the comma category $(F \down G)$ is complete (resp. cocomplete). [Hint: use Proposition \ref{prop:globalmonic} and \ref{prop:globalepi}]
  \end{ex}
  \begin{ex}
    Show that the domain functor $\Cc/I\to\Cc$ and codomain functor $I/\Cc\to\Cc$ reflects limits and colimits. So the limits and colimits in slide and coslide categories are computed as limits in the original categories.
  \end{ex}

  \chapter{Special Objects and Morphisms}

\minitoc
\newpage
\section{Subobjects and quotient objects}
%subobject and quotient, well-power cat, intersection and union
Let $\Cc$ be a category. Naively, a \termin{subobject} of an object $X$ should be the isomorphism class of monomorphisms to $X$.

To formalize this ideal, first notice that monomorphisms to $X$ actually form a full subcategory $\Cc_X$ of the slice category $\Cc/X$. Given any two objects  $(A,f), (B,g)$ in $\Cc_X$, since $g$ is monic, there can not be more than one $\Cc-$morphisms making the following diagram commutes:
      \begin{displaymath}
        \xymatrix@R=0.5cm{
          A\ar[dr]_f\ar[rr]&&B\ar[dl]^g\\
          &X&
         }
      \end{displaymath}
Thus $\Cc_X$ is thin. Moreover, by the triangle lemma \ref{prop:triangle lemma}, the morphisms in $\Cc_X$ must be monic when viewed as $\Cc-$morphisms.

Now, we already have the category $\Cc_X$ of monomorphisms to $X$. So, instead of define what are subobjects, we can just define the category of subobjects of $X$, denoted by $\Sub_{\Cc}(X)$, as the skeleton of $\Cc_X$. When the category $\Cc$ is understood, we can omit $\Cc$ from the symbol.

Dually, a \termin{quotient object} of an object $X$ is just a subobject of $X$ in the opposite category $\Cc^{\op}$.

\begin{rem}
  Notice that the notation of subobject and quotient object may not be suitable abstract of  sub- and quotient in usual sense.
\end{rem}
\begin{exam}
  In $\Ring$, $\QQ$ is a quotient of $\ZZ$ because the natural inclusion map $\ZZ\hookrightarrow\QQ$ is epi (see Example \ref{exam:nonsurjective epi}). Similarly, In the category of monoids, $\ZZ$ is a quotient of $\NN$.
\end{exam}
\begin{exam}
  In $\Top$, the subobjects of an object are not just the subspaces. Indeed, if $B$ has a finer topology than $A$, then $B$ is also a subobject of $A$. The same story happened in quotient objects. In fact, for every topological space, the identity map from itself to the trivial topological space on the same underlying set is an epimorphism.
\end{exam}

The problem is that the notions of monomorphism and epimorphism are a bit too general to capture the properties we want. So we should begin by defining some ``stronger'' notions.

\subsection{Intersections and unions}
\begin{defn}
  If for any object $x$ of $\Cc$, $\Sub(X)$ is small, the we say $\Cc$ is \termin[well-powered]{well-powered category}.
\end{defn}
Since in a well-powered category, $\Sub(X)$ is actually a poset, we can consider the usual order operations on it. But as we will show in later chapter, the theory of orders is just a special case of category theory, and the usual order operations can be defined in a general thin category.
\begin{defn}
  Consider an object $X$ of $\Cc$. The \termin[intersection]{intersection of subobjects} of a family of subobjects is their infimum in $\Sub(X)$, the \termin[union]{union of subobjects} of a family of subobjects is their supremum.
\end{defn}
\begin{prop}\label{prop:compute intersection and union}
  Consider an object $X$ of a category $\Cc$ and suppose $\Sub(X)$ is a set. The following conditions are equivalent:
\begin{enumerate}
  \item  the intersection of every family of subobjects of $X$ exists;
  \item  the union of every family of subobjects of $X$ exists.
\end{enumerate}
\end{prop}
\begin{proof}
  In a proset, we have
  \begin{align*}
    \sup_{i\in I}s_i &= \inf\{s\mid\forall i\in I, s_i \leqslant s\}\\
    \inf_{i\in I}s_i &= \sup\{s\mid\forall i\in I, s \leqslant s_i\}
  \end{align*}
  And the results follow.
\end{proof}
\begin{rem}
  There is a size issue. If we do not assume that $\Sub(X)$ is a set, then $\{S\mid\forall i\in I, S_i \leqslant S\}$ or $\{S\mid\forall i\in I, S \leqslant S_i\}$ may not form a set even if $I$ is a set.
\end{rem}

\begin{prop}\label{prop:intersection=pullback}
  The intersection of two subobjects of the same objects is given by their pullback. (see the remark of Example \ref{exam:intersection=pullback})
\end{prop}
\begin{prop}
  In a complete category, the intersection of a family of subobjects of a fixed object always exists.
\end{prop}
\begin{proof}
  For a non-empty family $(s_i\colon S_i\mono X)_{i\in I}$ of subobjects of a fixed object $X$, their generalized fibre product that is the limit of the diagram form by those morphisms $s_i$ is the intersection. For an empty intersection i.e. the terminal object, it is the identity $1_X$.
\end{proof}
\begin{cor}\label{prop:complete well-powered}
  In a complete and well-powered category, the intersection and the union of every family of subobjects of a fixed object always exist.
\end{cor}
\begin{rem}
  At this stage one should avoid a classical mistake. Computing the union of two subobjects is by no means a problem dual to that of computing their intersection. Dualizing Proposition \ref{prop:intersection=pullback} tells us something about the poset of quotients of $X$, not about unions in $\Sub(X)$. 
  
  In the same way let us observe that in \ref{prop:complete well-powered} the existence of unions is by no means related to any assumption on colimits: it relies on the formal formulas used in \ref{prop:compute intersection and union}. In particular a finite version of \ref{prop:complete well-powered} does not hold: a finitely complete and well-powered category certainly admits finite intersections of subobjects (see \ref{prop:intersection=pullback}), but not in general finite unions of subobjects. Finite unions have been constructed in \ref{prop:complete well-powered} using possibly infinite intersections. For a counterexample, just consider a semilattice which is not a lattice.
\end{rem}

\begin{prop}
  In a category with (finite) coproducts and strong-epi-mono factorizations, the union of a (finite) family of subobjects always exists.
\end{prop}

\section{Factorization system}

\newpage\section{Strong monomorphisms and epimorphisms}
%regular, extremal, strong
A monomorphism is \emph{regular} if it behaves like an embedding.
A \emph{regular} epimorphism is a morphism $A\to B$ in a category which behaves in the way that a covering is expected to behave, in the sense that ``$B$ is the union of the parts of $A$, identified with each other in some specified way''.
\begin{defn}
  A \termin{regular monomorphism} is an morphism which is an equaliser of some pair of parallel morphisms. A \termin{regular epimorphism} is a coequaliser of some pair of parallel morphisms.
\end{defn}
\begin{exam}
  The kernel of a group/ring/module/etc. homomorphism is a regular monomorphism: indeed, $\ker f$ is the equaliser of $f$ and the zero morphism.
\end{exam}
\begin{exam}
  In the category of topological spaces, the inclusion of a subspace $B\hookrightarrow A$ is a regular monomorphism: it is the equaliser of its characteristic map $A\to 2$ (where $2$ is given the indiscrete topology) and a constant map. Conversely, every regular monomorphism is (isomorphic to) the inclusion of a subspace.
\end{exam}

\begin{defn}
  An \termin{effective monomorphism} is a morphism which is the equalizer of its own cokernel pair. An \termin{effective epimorphism} is a morphism which is the coequalizer of its own kernel pair. (ref. \ref{prop:effective1}, \ref{prop:effective2})
\end{defn}

\begin{prop}
  In a finitely complete and cocomplete category, every regular monomorphism/epimorphoism is effective.
\end{prop}
\begin{proof}
  See Proposition \ref{prop:effective1}.
\end{proof}

\begin{defn}
  An \termin{extremal monomorphism} is a monomorphism $f\colon A\to B$  that cannot be factored through a nontrivial quotient object of $A$. In other word, if $f=g\circ e$ with $e$ an epimorphism, then $e$ is an isomorphism.
  An \termin{extremal epimorphism} is an epimorphism $f\colon A\to B$ that cannot be factored through a nontrivial subobject of $B$.  In other word, if $f=m\circ g$ with $m$ a monomorphism, then $m$ is an isomorphism.
\end{defn}
\begin{rem}
  Notice that whenever we have a factorization $f=g\circ e$ of a monomorphism $f$, then $e$ must be also monic. Thus if $e$ is epi, then it is already both monic and epi. Thus the notions of extremal monomorphism and extremal epimorphism do not make sense in a balanced category.
\end{rem}

\begin{prop}
  In a category,
  \begin{enumerate}
    \item every regular epimorphism is extremal,
    \item if a composite $f\circ g$ is an extremal epimorphism, $f$ itself is an extremal epimorphism,
    \item a morphism which is both a monomorphism and an extremal epimorphism is an isomorphism.
  \end{enumerate}
\end{prop}


\section{Generators}

\section{Projective and injective objects}

  \chapter{Universal Structures}
The universal constructions in category theory include
\begin{itemize}
  \item representable functors
  \item adjoint functors
  \item limits/colimits
  \item ends/coends
  \item Kan extensions
  \item dependent sums/dependent products
\end{itemize}

Each of these may be defined by requiring it to satisfy a universal property. A universal property is a property of some construction which boils down to (is manifestly equivalent to) the property that an associated object is a universal initial object of some (auxiliary) category.

In good cases, every single one of these is a special case of every other, so somehow one single concept here comes to us with many different faces.

\minitoc
\newpage
\section{Adjoint functors}
The key ideal of adjoint functor is that they are dual to each other.
  \begin{defn}
    Two functor $L\colon\Aa\to\Bb$ and $R\colon\Bb\to\Aa$ are said to be \termin[adjoint]{adjoint functor}, or more explicitly $L$ is the \termin{left adjoint} of $R$ and $R$ is the \termin{right adjoint} of $L$, if there exists a binary natural isomorphism
    \begin{equation*}
      \Hom_{\Bb}(L(-),-)\cong\Hom_{\Aa}(-,R(-))
    \end{equation*}
    This isomorphism is called the \termin{adjunction isomorphism} and the image of an element under this isomorphism is called its \termin{adjunct}.
  \end{defn}

  \begin{defn}
    An \termin{adjunction} $L \dashv R$\glsadd{adjunction} is a pair of two functors $L\colon\Aa\to\Bb$ and $R\colon\Bb\to\Aa$ equipped with natural transformations $\eta\colon\Id_{\Aa} \to R \circ L$ (called \termin{unit}) and $\epsilon\colon L \circ R \to \Id_{\Bb}$ (called \termin{counit}) satisfying the \termin{triangle identities}
    \begin{displaymath}
      \xymatrix@R=0.5cm{
        &L\circ R\circ L\ar[dr]^-{\epsilon\ast L}&&&R\circ L\circ R\ar[dr]^-{R\ast\epsilon}&\\
        L\ar[ur]^-{L\ast\eta}\ar[rr]^{\Id_L}&&L&R\ar[ur]^-{\eta\ast R}\ar[rr]^{\Id_R}&&R
        }
    \end{displaymath}
  \end{defn}
  \begin{rem}
    As diagrams, the triangle identities are
    \begin{displaymath}
      \xymatrix{
        \Aa\ar[r]^{L}\rruppertwocell<12>^{\Id_{\Aa}}_{}{\eta}&\Bb\ar[r]^{R}\rrlowertwocell<-12>^{}_{\Id_{\Bb}}{\epsilon}&\Aa\ar[r]^{L}&\Bb\ar@{}[r]|{=}&\Aa\ar[r]^{L}&\Bb
        }
    \end{displaymath}
    and
    \begin{displaymath}
      \xymatrix{
        \Bb\ar[r]^{R}\rrlowertwocell<-12>^{}_{\Id_{\Bb}}{\epsilon}&\Aa\ar[r]^{L}\rruppertwocell<12>^{\Id_{\Aa}}_{}{\eta}&\Bb\ar[r]^{R}&\Aa\ar@{}[r]|{=}&\Bb\ar[r]^{R}&\Aa
        }
    \end{displaymath}
  \end{rem}
  \begin{prop}
    Two functors $L\colon\Aa\to\Bb$ and $R\colon\Bb\to\Aa$ are adjoint if and only if they form an adjunction $L \dashv R$. The left or right adjoint of any functor, if it exists, is unique up to unique isomorphism.
  \end{prop}
  \begin{proof}
    If $L$ and $R$ are adjoint, then for any $\Aa-$object $A$, we have
    \begin{equation*}
      \Hom_{\Bb}(L(A),L(A))\approx\Hom_{\Aa}(A,R\circ L(A))
    \end{equation*}

    Thus, by taking adjuncts of identities, we get a natural transformation $\eta\colon\Id_{\Aa} \to R \circ L$. Similarly, we get the other one $\epsilon\colon L \circ R \to \Id_{\Bb}$.

    On the other hand, by the naturality of  adjunction isomorphism, we get the following commutative diagram
    \begin{displaymath}
      \xymatrix{
        \Hom_{\Bb}(L(A),L(A))&\Hom_{\Aa}(A,R\circ L(A))\ar@{}[l]|-{\approx}\\
        \Hom_{\Bb}(L\circ R\circ L(A),L(A))\ar[u]_{\circ(L\ast\eta)_{L(A)}}
        &\Hom_{\Aa}(R\circ L(A),R\circ L(A))\ar[u]_{\circ\eta_A}\ar@{}[l]|-{\approx}
        }
    \end{displaymath}
    Thus we get
    \begin{equation*}
      (\epsilon\ast L)_{L(A)}\circ(L\ast\eta)_{L(A)} = 1_{L(A)}
    \end{equation*}
    Therefore
    $(\epsilon\ast L)\circ(L\ast\eta) = \Id_L$.
    Similarly, $(R\ast \epsilon)\circ(\eta\ast R) = \Id_R$.

    Conversely, if $L$ and $R$ form an adjunction, let $\eta\colon\Id_{\Aa} \to R \circ L$ and $\epsilon\colon L \circ R \to \Id_{\Bb}$ be the corresponding natural transformations. Then we get the following functions:
    \longmapdes{l_{A,B}}{\Hom_{\Bb}(L(A),B)}{\Hom_{\Aa}(A,R(B))}{f}{R(f)\circ\eta_A}
    \longmapdes{r_{A,B}}{\Hom_{\Aa}(A,R(B))}{\Hom_{\Bb}(L(A),B)}{f}{\epsilon_B\circ L(f)}

    It is not difficult to verify that they form two mutually inverse natural transformations.

    Assume $L,L'$ are both left adjoint of $R$, then
    \begin{equation*}
      \Hom_{\Bb}(L(-),-)\cong\Hom_{\Aa}(-,R(-))\cong\Hom_{\Bb}(L'(-),-)
    \end{equation*}
    Thus, by Yoneda Lemma, $L\cong L'$. Similarly, the right adjoint is unique up to unique isomorphism.
  \end{proof}

\subsection{Adjoint functors and representable functors}
  Let $F\colon\Aa\to\Bb$ be a functor between essentially small categories, then it induces a ``pullback'' functor
  \begin{equation*}
    F^{\ast}\colon\PSh(\Bb)\To\PSh(\Aa)
  \end{equation*}
  By restriction along the Yoneda embedding $\Upsilon\colon\Bb\to\PSh(\Bb)$ this yields the functor
  \begin{equation*}
    \overline{F}\colon\Bb\markar{\Upsilon}\PSh(\Bb)\markar{F^{\ast}}\PSh(\Aa)
  \end{equation*}
  \begin{defn}
    If for some $B\in\ob\Bb$, $\overline{L}(B)$ is representable and is represented by $A$ with a universal element $f$, then we say $(A,f)$ is a \termin{coreflection} of $B$ along $F$.
  \end{defn}
  In plain word, $(A,f)$ is a coreflection of $B$ along $F$ means
  \begin{equation*}
    \phi\colon\Hom_{\Aa}(-,A)\To\Hom_{\Bb}(F(-),B)
  \end{equation*}
  is an isomorphism and $\phi_A(1_A)=f$.
  \begin{prop}
    $F$ has a right adjoint if and only if every $B\in\ob\Bb$ has a coreflection along $F$.
  \end{prop}
  \begin{proof}
    If $F$ has a right adjoint $R$, then for any $B\in\ob\Bb$, $(R(B),\epsilon_B)$ is a coreflection of it along $F$.

    If every $B\in\ob\Bb$ has a coreflection along $F$, then it is easy to see that the representative is functorial on $B$, thus we get a functor $R\colon\Bb\to\Aa$ such that $\overline{F}\cong\Upsilon\circ R$. Then $R$ is the right adjoint of $F$.
  \end{proof}

  Similarly, we have the notion of reflection. In plain word, $(A,f)$ is a \termin{reflection} of $B$ along $F$ means
  \begin{equation*}
    \phi\colon\Hom_{\Aa}(A,-)\To\Hom_{\Bb}(B,F(-))
  \end{equation*}
  is an isomorphism and $\phi_A(1_A)=f$. Thus $F$ has a left adjoint precisely if every $B\in\ob\Bb$ has a reflection along $F$.

\subsection{Adjoint functors and universal arrows}
  \begin{defn}
    Given a functor $R\colon\Bb\to\Aa$, and an object $A$ in $\Aa$, a \termin{universal arrow} from $A$ to $R$ is an initial object of the comma category $(A\down R)$. That is, it consists of an object $L(A)\in\Bb$ and a morphism $i_A : A\to R\circ L(A)$ (called the \termin{unit}) such that for any $B\in\Bb$, any morphism $f \colon A\to R(B)$ factors through the unit $i_A$ as
    \begin{displaymath}
      \xymatrix@R=0.5cm{
        &A\ar[dl]_{i_A}\ar[dr]^{f}&\\
        R\circ L(A)\ar[rr]^{R(\tilde{f})}&&R(B)
        }
    \end{displaymath}
    for a unique $\tilde{f}\colon L(A)\to B$.%, the adjunct of $f$.
  \end{defn}
  \begin{prop}
    $R$ has a left adjoint if and only if every $A\in\ob\Aa$ has a universal arrow to $R$.
  \end{prop}
  \begin{proof}
    If $R$ has a left adjoint $L$. For any $A\in\ob\Aa$ and $B\in\ob\Bb$,
    let $\tilde{f} \colon L(A) \to B$ be the image of $f \colon A \to R(B)$ under the bijection
    \begin{equation*}
      \Hom_{\Aa}(A,R(B))\approx\Hom_{\Bb}(L(A),B)
    \end{equation*}
    and consider the naturality square
    \begin{displaymath}
      \xymatrix{
        \Hom_{\Bb}(L(A),L(A))\ar@{}[r]|-{\approx}\ar[d]^{\tilde{f}\circ}
        &\Hom_{\Aa}(A,R\circ L(A))\ar[d]^{R(\tilde{f})\circ}
        \\
        \Hom_{\Bb}(L(A),B)\ar@{}[r]|-{\approx}
        &\Hom_{\Aa}(A,R(B))
        }
    \end{displaymath}

    Then the unit $i_A\colon A\to R\circ L(A)$ is the image of $1_{L(A)}$ under the hom-isomorphism. Chase this identity through the commuting square, we obtain
    \begin{displaymath}
      \xymatrix{
        (1_{L(A)})\ar@{|->}[r]\ar@{|->}[d]^{\tilde{f}\circ}
        &(i_A)\ar@{|->}[d]^{R(\tilde{f})\circ}
        \\
        (\tilde{f})\ar@{|->}[r]
        &(f)
        }
    \end{displaymath}

    Conversely, if every $A\in\ob\Aa$ has a universal arrow $(L(A),i_A)$ to $R$. Then we have a bijection
    \begin{equation*}
      \Hom_{\Aa}(A,R(B))\cong\Hom_{\Bb}(L(A),B)
    \end{equation*}
    which it is easy to see is natural in $B$. In this case there is a unique way to make $L$ into a functor so that this isomorphism is natural in $A$ as well.
  \end{proof}

\subsection{Adjoint functors and cographs}
  Recall that a graph of a function $f\colon X\to Y$ is the subset
  \begin{equation*}\glsadd{Graph}
    \Graph(f) = \{(x,f(x))\mid x\in X\} \subset  X\times Y
  \end{equation*}
  determined by the pullback square
    \begin{displaymath}
      \xymatrix{
        \Graph(f)\ar[d]\ar[r]&Y\ar[d]^{1_Y}\\
        X\ar[r]^{f}&Y
        }
    \end{displaymath}

  Imitate this, let $F\colon\Aa\to\Bb$ be a functor, then the comma category $(F^{\op}\down\Id_{\Bb})$ can be view as the \termin[graph]{graph of a functor} of $F$. (Here we take the opposite functor for technical reason.)

  On the other hand, the graph of a function can also be obtained by this square:
    \begin{displaymath}
      \xymatrix{
        \Graph(f)\ar@{ (->}[d]\ar[r]&1\ar[d]\\
        X\times Y\ar[r]^-{\chi_f}&\{0,1\}
        }
    \end{displaymath}
    where $\chi_f$ is the characteristic function
    \begin{equation*}
      \chi_f(x,y)=
      \begin{cases}
      1&\text{if }f(x)=y\\
      0&\text{otherwise}
      \end{cases}
    \end{equation*}

    Imitate this, a graph of a functor can also be view as the category of elements of the characteristic functor $\chi_F$
    \begin{equation*}
      \chi_F\colon\Aa^{\op}\times\Bb\markar{F^{\op}\times\Id_{\Bb}}\Bb^{\op}\times\Bb\markar{\Hom_{\Bb}(-,-)}\Set
    \end{equation*}

    Indeed, an object in $\Elts(\chi_F)$ is a pair $(h,\Hom_{\Bb}(F^{\op}(A),B))$ such that $h\in\Hom_{\Bb}(F^{\op}(A),B)$. It can be viewed as an object $(A,h,B)$ in $(F^{\op}\down\Id_{\Bb})$ and vice versa.

    A morphism from $(h,\Hom_{\Bb}(F^{\op}(A),B))$ to $(h',\Hom_{\Bb}(F^{\op}(A'),B'))$ in $\Elts(\chi_F)$ is a morphism $\phi\colon(A,B)\to(A',B')$ in $\Aa^{\op}\times\Bb$, which consists of an $\Aa-$morphism $f\colon A\to A'$ and a $\Bb-$morphism $g\colon B\to B'$, such that it induces a hom-function mapping $h$ to $h'$. In other word, $g\circ h\circ F^{\op}(f^{\op}) = h'$. Thus this morphism can be also viewed as a morphism from $(A,h,B)$ to $(A',h',B')$.

  Let's consider the notion of \termin{cograph of a functor}, which should be dual to that of graph of a functor.

  In the terminology system of Bill Lawvere, a cograph of a function $f\colon X\to Y$ is the quotient set determined by the pushout square
    \begin{displaymath}
      \xymatrix{
        X\ar[r]^{f}\ar[d]_{1_X}&Y\ar[d]\\
        X\ar[r]&\Cograph(f)
        }
    \end{displaymath}
    However, this definition is lose of fine. So Jacob Lurie give another definition in high-dimensional category theory context, see \cite{lurieHTT}. Here we give the $0-$dimensional case.

    Let $f\colon X\to Y$ be a function, then it induces a functor $\bar{f}$ from the category $\mathbf{2}$ to $\Set$ mapping the nonidentity arrow to $f$. Then the cograph\glsadd{Cograph} of $f$ is the category of elements of $\bar{f}$, as described there: the objects of $X\star^fY$ are the disjoint union of $X$ and $Y$: $\ob(X\star^fY)=X\sqcup Y$ and the nontrivial morphisms are of the form $x\to y$ whenever $x\in X, y\in Y$ and $f(x)=y$. Such a construction is obviously finer than the original one.

    Imitate this, we have
    \begin{defn}
      Let $F\colon\Aa\to\Bb$ be a functor, the \termin[cograph]{cograph of a functor} of $F$ is the category $\Aa\star^F\Bb$ with $\ob(\Aa\star^F\Bb)=\ob\Aa\sqcup\ob\Bb$ and with
      \begin{equation*}
        \Hom_{\Aa\star^F\Bb}(A,B)=
        \begin{cases}
          \Hom_{\Aa}(A,B)&\text{if }A,B\in\ob\Aa\\
          \Hom_{\Bb}(A,B)&\text{if }A,B\in\ob\Bb\\
          \Hom_{\Bb}(F(A),B)&\text{if }A\in\ob\Aa,B\in\ob\Bb\\
          \varnothing&\text{if }A\in\ob\Bb,B\in\ob\Aa
        \end{cases}
      \end{equation*}
      with composition defined as induced from $\Aa$, from $\Bb$, and from the action of $F$.
    \end{defn}
    \begin{prop}
      Two functors $L\colon\Aa\to\Bb$ and $R\colon\Bb\to\Aa$ are adjoint if and only if $\Aa\star^L\Bb$ and $(\Bb^{\op}\star^{R^{\op}}\Aa^{\op})^{\op}$ are isomorphic under $\Aa$ and $\Bb$.
      \begin{displaymath}
        \xymatrix{
          \Aa\ar@{ (->}[d]\ar@{ (->}[dr]&\Bb\ar@{ (->}[dl]\ar@{ (->}[d]\\
          \Aa\star^L\Bb\ar@{}[r]|-{\approx}&(\Bb^{\op}\star^{R^{\op}}\Aa^{\op})^{\op}
         }
      \end{displaymath}
    \end{prop}
    \begin{proof}
      The canonical functors from $\Aa$ and $\Bb$ to the cographs are the inclusions, thus an isomorphism $\phi$ from $\Aa\star^L\Bb$ to $(\Bb^{\op}\star^{R^{\op}}\Aa^{\op})^{\op}$ under them must work as the identity functor of $\Aa$ and $\Bb$ when restrict on $\Aa$ and $\Bb$ respectively.

      Therefore, $\Aa\star^L\Bb$ and $(\Bb^{\op}\star^{R^{\op}}\Aa^{\op})^{\op}$ are isomorphic if and only if
      \begin{equation*}
        \Hom_{\Bb}(F(A),B) \cong \Hom_{\Bb}(A,R(B))
      \end{equation*}
      naturally in $A\in\ob\Aa$ and $B\in\ob\Bb$. That is, $L$ and $R$ are adjoint.
    \end{proof}



\subsection{Exercises}
  \begin{ex}
    Consider the category $\Cc$ with a single object $\ast$ and just two arrows: the identity $1$ on $\ast$ and a morphism $f$ such that $f \circ f = 1$. Show that $f$ determines a natural transformation $\phi\colon\Id\to\Id$. The identity functor $\Id$ is left adjoint to itself and the unit $\eta\colon\Id\to\Id\circ\Id$ and counit $\epsilon\colon\Id\circ\Id\to\Id$ can both be chosen to be the identity; but they can also both be chosen to be the transformation $\phi$. However, the identity and $\phi$ do not exhibit the adjunction.
  \end{ex}
  \begin{ex}
    Consider the category $\mathbf{Idem}$ whose objects are the pairs $(X,v)$ of a set $X$ provided with an idempotent endomorphism $v\colon X\to X, v\circ v=v$. A morphism $(X, v) \to (Y, w)$ is just a function $f\colon X \to Y$ satisfying $w\circ f = f\circ v$.
    There is a canonical full embedding of the category $\Set$ in $\mathbf{Idem}$:
    \longmapdes{\imath}{\Set}{\mathbf{Idem}}{X}{(X,1_X)}
    Consider now the functor determined by
    \longmapdes{\jmath}{\mathbf{Idem}}{\Set}{(X,v)}{\{x\in X\mid v(x)=x\}}
    Prove that $\jmath$ is both left and right adjoint to $\imath$.
  \end{ex}


\newpage\section{Properties of adjoint functors}
  \begin{prop}
    Consider the following situation:
      \begin{displaymath}
        \xymatrix{
          \Aa\ar@<-0.5ex>[r]_{F}&\Bb\ar@<-0.5ex>[l]_{G}\ar@<-0.5ex>[r]_{H}&\Cc\ar@<-0.5ex>[l]_{K}
         }
      \end{displaymath}
  where $F,G,H,K$ are functors, with $G$ left adjoint to $F$ and $K$ left adjoint to $H$. In this case $G\circ K$ is left adjoint to $H\circ F$.
  \end{prop}

  \begin{prop}\label{prop:adjoint functor preserves limits}
    Let $L\colon\Aa\to\Bb$ and $R\colon\Bb\to\Aa$ be a pair of adjoint functors. Then $L$ preserves colimits in $\Aa$, $R$ preserves limits in $\Bb$.
  \end{prop}
  \begin{proof}
    Let $D\colon\Ii^{\op}\to\Bb$ be a diagram whose limit $\invlim D$ exists. Then we have a sequence of natural isomorphisms, natural in $A\in\ob\Aa$:
    \begin{align*}
      \Hom_{\Aa}(A,R(\invlim D)) & \cong \Hom_{\Bb}(L(A),\invlim D)  \\
       & \cong \invlim\Hom_{\Bb}(L(A),D) \\
       & \cong \invlim\Hom_{\Aa}(A,R(D)) \\
       & \cong \Hom_{\Aa}(A,\invlim R(D))
    \end{align*}
    where we used the adjunction isomorphism and the fact that any hom-functor preserves limits (see Proposition \ref{prop:representable functor preserves limits}). Because this is natural in $A$ the Yoneda lemma implies that we have an isomorphism
    \begin{equation*}
      R\invlim D\approx\invlim R(D)
    \end{equation*}
    The argument that shows the preservation of colimits by $L$ is analogous.
  \end{proof}

Let $\Ii$ be a small category and $F$ a $\Ii-$diagram in $\Cc$, we have a presheaf
\begin{equation*}
  \Cc\markar{\Delta}[\Ii,\Cc]\markar{\Hom(-,F)}\Set
\end{equation*}
It is not difficult to verify that it is represented by $C$ if and only if $C$ is a limit object of $F$.

Similarly, $L$ is a colimit object of $F$ precisely if it is a representing object of the functor
\begin{equation*}
  \Cc\markar{\Delta}[\Ii,\Cc]\markar{\Hom(F,-)}\Set
\end{equation*}

Therefore, if $\Cc$ has all $\Ii-$limits and colimits, we have:
\begin{prop}\label{prop:limits as adjoint functors}
  Let the category $\Ii$ be small and $\Cc$ has all $\Ii-$limits and colimits. The left adjoint of the diagonal functor $\Delta\colon\Cc\to[\Ii,\Cc]$ sends every diagram to its colimit. The right adjoint of $\Delta$ sends every diagram to its limit.
  \begin{equation*}
    \dirlim\dashv\Delta\dashv\invlim
  \end{equation*}
\end{prop}

  \begin{prop}
    Consider a functor $L\colon\Aa\to\Bb$ with a right adjoint $R\colon\Bb\to\Aa$. If $\Ii$ is any small category, then $R_{\ast}\colon[\Ii,\Bb]\to[\Ii,\Aa]$ is itself right adjoint to $L_{\ast}\colon[\Ii,\Aa]\to[\Ii,\Bb]$.
  \end{prop}
  \begin{proof}
    Let $\eta\colon\Id_{\Aa} \to R \circ L$ and $\epsilon\colon L \circ R \to \Id_{\Bb}$ be the natural transformations describing the adjunction $L\dashv R$. Then for any functor $F\colon\Ii\to\Aa$ and $G\colon\Ii\to\Bb$, we have corresponding natural transformations
    \begin{align*}
      \eta\ast G&\colon G\then L_{\ast}R_{\ast}G  \\
      \epsilon\ast F&\colon R_{\ast}L_{\ast}F\then F
    \end{align*}
    This yields two natural transformations
    \begin{align*}
      \bar{\eta}&\colon \Id_{[\Ii,\Bb]}\then L_{\ast}\circ R_{\ast}  \\
      \bar{\epsilon}&\colon R_{\ast}\circ L_{\ast}\then \Id_{[\Ii,\Aa]}
    \end{align*}
    It is easy to verify that they satisfies the triangle identities.
  \end{proof}

\subsection{Fully faithful adjoint functors}

  \begin{prop}
    Let $L\colon\Aa\to\Bb$ and $R\colon\Bb\to\Aa$ be a pair of adjoint functors with unit $\eta\colon\Id_{\Aa} \to R \circ L$ and counit $\epsilon\colon L \circ R \to \Id_{\Bb}$. Then
    \begin{enumerate}
      \item $R$ is faithful if and only if the counit $\epsilon$ is pointwise epi, i.e. its component over every object is an epimorphism.
      \item $L$ is faithful if and only if the unit $\eta$ is pointwise monoic.
      \item $R$ is full if and only if the counit $\epsilon$ is pointwise split monoic.
      \item $L$ is full if and only if the unit $\eta$ is pointwise split epi.
      \item $R$ is full and faithful if and only if the counit $\epsilon$ is a natural isomorphism.
      \item $L$ is full and faithful if and only if the unit $\eta$ is a natural isomorphism.
      \item The following are equivalent:
      \begin{enumerate}
        \item $L$ and $R$ are both full and faithful;
        \item $L$ is an equivalence with weak inverse $R$;
        \item $R$ is an equivalence.
      \end{enumerate}
    \end{enumerate}
  \end{prop}
  \begin{proof}
    For the characterization of faithful $R$ by epi counit components, just notice that $\epsilon_X\colon L\circ R(X)\to X$ being an epimorphism is equivalent to the induced function
    \begin{equation*}
      \Hom(X,B)\to\Hom(L\circ R(X),B)
    \end{equation*}
    being an injection for all objects $B$.
    By adjointness, we have an isomorphism
    \begin{equation*}
      \Hom(L\circ R(X),B)\approx\Hom(R(X),R(B))
    \end{equation*}
    Then, by the formula for adjuncts and the triangle identity, the composite of them
    \begin{equation*}
      R_{X,B}\colon\Hom(X,B)\to\Hom(L\circ R(X),B)\approx\Hom(R(X),R(B))
    \end{equation*}
    is computed by
    \begin{align*}
      (X\markar{f}B)&\mapsto(L\circ R(X)\To X\markar{f} B)\\
       &\mapsto(R\circ L\circ R(X)\To R(X)\markar{R(f)}R(B))\\
       &\mapsto(R(X)\To R\circ L\circ R(X)\To R(X)\markar{R(f)}R(B))\\
       &=(R(X)\markar{R(f)}R(B))
    \end{align*}
    Therefore $R_{X,B}$ is just the induced hom-function of the functor $R$.

    The characterization of full $R$ is just the same reasoning applied to the fact that $\epsilon_X\colon L\circ R(X)\to X$ is a split monomorphism iff for all objects $B$ the induced function
    \begin{equation*}
      \Hom(X,B)\to\Hom(L\circ R(X),B)
    \end{equation*}
    is a surjection.

    The other statements can be proved analogously or just obtained as corollaries.
  \end{proof}
  \begin{prop}\label{prop:two side adjoint}
    Consider a functor $F\colon\Aa\to\Bb$ with both a left adjoint functor $G$ and a right adjoint functor $H$. If one of those adjoint functors is fully faithful, so is the other adjoint functor.
  \end{prop}
  \begin{proof}
    Let $\eta\colon\Id_{\Bb}\then F\circ G$ and $\epsilon\colon G\circ F\then\Id_{\Aa}$ be the unit and counit of the adjunction $G\dashv F$, while $\alpha\colon\Id_{\Aa}\then H\circ F$ and $\beta\colon F\circ H\then\Id_{\Bb}$ be unit and counit of $F\dashv H$.

    Assume $G$ is fully faithful, then $\eta$ is an isomorphism, let $\mu$ be its inverse. To show $H$ is fully faithful, it suffices to show that $\beta$ is an isomorphism.

    To do this, we consider the following composite
    \begin{equation*}
      \Id_{\Bb}\markar{\eta} F\circ G\markar{F\ast\alpha\ast G} F\circ H\circ F\circ G \markar{(F\circ H)\ast\mu} F\circ H
    \end{equation*}
    We claim that it is the inverse of $\beta$.

    First, we calculate
    \begin{equation*}
      \beta \circ \left((F\circ H)\ast\mu\right) \circ (F\ast\alpha\ast G) \circ \eta
    \end{equation*}
    That is the composite of the natural transformations
    \begin{displaymath}
      \xymatrix{
        \Bb\ar[r]^{G}\ar@/^5pc/[rrrr] ^{\Id_{\Bb}}="a"
        \rrlowertwocell<-12>_{\Id_{\Bb}}^{}{\mu}
        &\Aa\ar[r]^{F}\ar@/^2.5pc/[rr] ^{\Id_{\Aa}}="b"\ar@{}[rr]^{}="c"
        &\Bb\ar[r]^{H}\rrlowertwocell<-12>_{\Id_{\Bb}}^{}{\beta}
        &\Aa\ar[r]^{F}
        &\Bb
        \ar@{}^{\eta}|-{\SelectTips{eu}{}\object@{=>}} "a";"b"
        \ar@{}^{\alpha}|-{\SelectTips{eu}{}\object@{=>}} "b";"c"
        }
    \end{displaymath}
    By the triangle identity of $F\dashv H$, the above composite becomes
    \begin{displaymath}
      \xymatrix{
        \Bb\ar[r]^{G}\rruppertwocell^{\Id_{\Bb}}_{}{\eta}\rrlowertwocell^{}_{\Id_{\Bb}}{\mu}
        &\Aa\ar[r]^{F}
        &\Bb
        }
    \end{displaymath}
    Thus
    \begin{equation*}
      \beta \circ \left((F\circ H)\ast\mu\right) \circ (F\ast\alpha\ast G) \circ \eta = \Id
    \end{equation*}

    Now we calculate
    \begin{equation*}
      \left((F\circ H)\ast\mu\right) \circ (F\ast\alpha\ast G) \circ \eta \circ \beta
    \end{equation*}
    That is the composite of the natural transformations
    \begin{displaymath}
      \xymatrix{
        \Bb\ar[r]^{H}\rrlowertwocell<-12>_{\Id_{\Bb}}^{}{\beta}
        &\Aa\ar[r]^{F}
        &\Bb\ar[r]^{G}\ar@/^5pc/[rrrr] ^{\Id_{\Bb}}="a"
        \rrlowertwocell<-12>_{\Id_{\Bb}}^{}{\mu}
        &\Aa\ar[r]^{F}\ar@/^2.5pc/[rr] ^{\Id_{\Aa}}="b"\ar@{}[rr]^{}="c"
        &\Bb\ar[r]^{H}
        &\Aa\ar[r]^{F}
        &\Bb
        \ar@{}^{\eta}|-{\SelectTips{eu}{}\object@{=>}} "a";"b"
        \ar@{}^{\alpha}|-{\SelectTips{eu}{}\object@{=>}} "b";"c"
        }
    \end{displaymath}
    Notice that $\mu\ast F = F\ast \epsilon$ (Since $(F\ast \epsilon)\circ(\eta\ast F)=\Id_F$), thus the above composite becomes
    \begin{displaymath}
      \xymatrix{
        \Bb\ar[r]^{H}\ar@/_5pc/[rrrr] _{\Id_{\Bb}}="1"
        &\Aa\ar[r]^{F}\ar@/_2.5pc/[rr] _{\Id_{\Aa}}="2"\ar@{}[rr]^{}="3"
        &\Bb\ar[r]^{G}\ar@/^5pc/[rrrr] ^{\Id_{\Bb}}="a"
        &\Aa\ar[r]^{F}\ar@/^2.5pc/[rr] ^{\Id_{\Aa}}="b"\ar@{}[rr]^{}="c"
        &\Bb\ar[r]^{H}
        &\Aa\ar[r]^{F}
        &\Bb
        \ar@{}^{\eta}|-{\SelectTips{eu}{}\object@{=>}} "a";"b"
        \ar@{}^{\alpha}|-{\SelectTips{eu}{}\object@{=>}} "b";"c"
        \ar@{}^{\epsilon}|-{\SelectTips{eu}{}\object@{=>}} "3";"2"
        \ar@{}^{\beta}|-{\SelectTips{eu}{}\object@{=>}} "2";"1"
        }
    \end{displaymath}
    By the triangle identities of $F\dashv H$ and $G\dashv F$, the above composite is just the identity transformation $\Id_{F\circ H}$.
  \end{proof}
  \begin{defn}
    A pair of adjoint functors that is also an equivalence of categories is called an \termin{adjoint equivalence}.
  \end{defn}
  \begin{cor}
    Let $F\colon\Aa\to\Bb$ be an equivalence. If $\Aa$ is (finitely) complete, then so is $\Bb$.
  \end{cor}


\subsection{The adjoint functor theorem}
  Adjoint functor theorems are theorems stating that under certain conditions a functor that preserves limits is a right adjoint, and that a functor that preserves colimits is a left adjoint. It is one of the most important result in this chapter and elementary category theory.

  The basic idea of an adjoint functor theorem is that if we could assume that a category $\Bb$ had all limits over small and large diagrams, then for $R\colon\Bb\to\Aa$ a functor that preserves all these limits we might define its left adjoint $L$ by taking $L(A)$ to be the limit over the comma category $(A\down R)$ of the projection functor $(A\down R)\to \Bb$,
  \begin{equation*}
    L(A)=\invlim_{A\to R(B)} B
  \end{equation*}

  Because $R$ preserves limits, we would have an isomorphism
  \begin{equation*}
    R\circ L(A)=\invlim_{A\to R(B)} R(B)
  \end{equation*}
  and hence an obvious morphism of cone tips $A\To R\circ L(A)$. Moreover, thoese morphisms form a natural transformation $\eta\colon\Id_{\Aa} \to R \circ L$.

  Because with this definition there would be for every $B$ an obvious morphism (the component morphism over $B$ of the limiting cone)
  \begin{equation*}
    L\circ R(B)=\invlim_{R(B)\to R(B')} B' \To B
  \end{equation*}
  Moreover, thoese morphisms form a natural transformation $\epsilon\colon L \circ R \to \Id_{\Bb}$.

  It is easy to check that $\eta$ and $\epsilon$ would be the unit and counit of the adjunction $L\dashv R$.

  The problem with this would-be argument is that in general the comma category $(A\down R)$ may not be small. But one can generally not expect a category to have all large limits: even if every $(A\down R)$ is considered small, a classical theorem of Freyd says that any complete small category is a preordered set (see Proposition \ref{prop:complete small category = proset}). Thus, the argument we gave above is necessarily an adjoint functor theorem for preordered sets:
  \begin{thm}
    If $R\colon\Bb\to\Aa$ is any functor between preordered sets such that $\Bb$ has, and $R$ preserves, all small products, then $R$ has a left adjoint.
  \end{thm}
  To obtain adjoint functor theorems for categories that are not preordered sets, one must therefore impose various additional ``size conditions'' on the category $\Bb$ and/or the functor $R$.

  \begin{defn}
    A functor $F\colon \Aa\to \Bb$ satisfies the \termin{solution set condition} if for every $\Bb-$object $B$ there exists a set $S_B$ of $\Aa-$objects such that any morphism $h\colon B \to F(A)$ can be factored as
    \begin{equation*}
      B\markar{f}F(A')\markar{F(t)}F(A)
    \end{equation*}
    for some $t\colon A'\to A$ and some $A'\in S_B$.
  \end{defn}
  \begin{rem}
    The fact that $B$ admits a reflection $(A,f)$ along $F$ implies that $S_B = \{A\}$ can be chosen as solution set. So the solution set condition is a much weaker requirement than the existence of a reflection.
  \end{rem}

  \begin{thm}[General adjoint functor theorem]
    If a continuous functor $R\colon\Bb\to\Aa$ satisfies the solution set condition and $\Bb$ is complete, then $R$ has a left adjoint.
  \end{thm}
  \begin{proof}
    Consider the full subcategory $\Ss_A$ of $(A\down R)$ whose objects are the pair $(B,f)$ with $B\in S_A$. Since this category is small, it suffices to show that the inclusion functor $\Ss_A\hookrightarrow(A\down R)$ is co-cofinal.

    Indeed, the solution set condition can be reformulated as the fact that, given $(B,f)\in\ob(A \down R)$, there exist $(B',f')\in\ob\Ss_A$ and a morphism $t\colon(B', f') \to (B, f)$, this is precisely the dual of requirement in Proposition \ref{prop:cofinal subcategory}.

    Since $R$ is continuous, $(A\down R)$ is complete, and the universal arrow from $A$ to $R$ is just a limit over $\Ss_A$ thus exists.
  \end{proof}

  \begin{thm}[Special adjoint functor theorem]
    A continuous functor $R\colon\Bb\to\Aa$ has a left adjoint, if its domain $\Bb$ satisfies the following conditions:
    \begin{enumerate}
      \item $\Bb$ is complete;
      \item $\Bb$ is well-powered;
      \item $\Bb$ has a cogenerating set.
    \end{enumerate}
  \end{thm}
  \begin{proof}
    We assume that $\Bb$ is complete, well-powered, has a cogenerating set $\{B_i\}_{i\in I}$.
    As before, for each object $A$ of $\Aa$, the comma category $(A\down R)$ is complete. Moreover, it is easy to check that it is well-powered, and that the set of all objects of the form $A\to R(B_i)$ is a cogenerating set for $(A\down R)$.

    It then remains to prove that any complete, well-powered category $\Cc$ with a cogenerating set $\{K_s\}_{s\in S}$ has an initial object. The initial object $0$ is constructed as the minimal subobject of $\prod_sK_s$. Then, given $f,g\colon0\to X$, the equalizer $\ker(f,g)$ is isomorphic to $0$ because $0$ is minimal, and so $f=g$: there is at most one arrow $0\to X$ for each $X$.

    On the other hand, for each $X$ the canonical map
    \begin{equation*}
      i\colon X\To\prod_{s\in S}K_s^{\Hom(X,K_s)}
    \end{equation*}
    is monic since the $K_s$ cogenerate.
    The following pullback of it,
      \begin{displaymath}
        \xymatrix{
          K\ar[r]\ar[d]&X\ar[d]^{i}\\
          \prod_sK_s\ar[r]&\prod_{s}K_s^{\Hom(X,K_s)}
         }
      \end{displaymath}
      gives a subobject $K$ of $\prod_sK_s$ that maps to $X$. Since $0$ is minimal, there exists a morphism $0\to K\to X$, and we conclude $0$ is initial.
  \end{proof}
  \begin{cor}
    A continuous functor $R\colon\Bb\to\Aa$ has a left adjoint, if its domain $\Bb$ satisfies the following conditions:
    \begin{enumerate}
      \item $\Bb$ is complete;
      \item $\Bb$ is well-powered;
      \item $\Bb$ has a cogenerator.
    \end{enumerate}
  \end{cor}
  \begin{cor}
    A continuous functor $R\colon\Bb\to\Aa$ has a left adjoint, if its domain $\Bb$ satisfies the following conditions:
    \begin{enumerate}
      \item $\Bb$ is complete;
      \item $\Bb$ has a strong generating family;
      \item $\Bb$ has a cogenerating set.
    \end{enumerate}
  \end{cor}
  \begin{cor}
    A continuous functor $R\colon\Bb\to\Aa$ has a left adjoint, if its domain $\Bb$ satisfies the following conditions:
    \begin{enumerate}
      \item $\Bb$ is complete;
      \item $\Bb$ has a strong generator;
      \item $\Bb$ has a cogenerator.
    \end{enumerate}
  \end{cor}



\newpage\section{Reflective subcategories}
  \begin{defn}
    A full subcategory is \termin[reflective]{reflective subcategory} if the inclusion functor has a left adjoint.
    The left adjoint is sometimes called the \termin{reflector}, and a functor which is a reflector is called a \termin{reflection}.
  \end{defn}
  \begin{rem}
    Of course, there are dual notions of \termin{coreflective subcategory}, \termin{coreflector}, and \termin{coreflection}.
  \end{rem}
  \begin{rem}
    Whenever $\Aa$ is a full subcategory of $\Bb$, we can say that objects of $\Aa$ are objects of $\Bb$ with some extra property. But if $\Aa$ is reflective in $\Bb$, then we can turn this around and (by thinking of the left adjoint as a forgetful functor) think of objects of $\Bb$ as objects of $\Aa$ with (if we're lucky) some extra structure or (in any case) some extra stuff.

    This can always be made to work by brute force, but sometimes there is something insightful about it. For example, a metric space is a complete metric space equipped with a dense subset. Or, a possibly nonunital ring is a unital ring equipped with a unital homomorphism to the ring of integers.
  \end{rem}

  \begin{prop}
    Let $\Aa$ be a reflective subcategory of $\Bb$. Then a morphism $f$ is monic in $\Aa$ if and only if it is monic in $\Bb$.
  \end{prop}
  \begin{proof}
    The ``if'' is obvious. To show the ``only if'', consider a monomorphism $f\colon A\to B$ in $\Aa$. Let $\imath\colon\Aa\hookrightarrow\Bb$ be the inclusion, $r$ be its left adjoint and $\epsilon\colon r\circ\imath\to\Id_{\Aa}$ be the counit.
    Then for any object $T$ with two morphisms $x,y\colon T\to\imath(A)$ in $\Bb$, we have
    \begin{equation*}
      \Hom_{\Bb}(T,\imath(B)) \approx \Hom_{\Aa}(r(T),B)
    \end{equation*}
    Then it maps $\imath(f)\circ x$ and $\imath(f)\circ y$ to $\epsilon_B\circ r(\imath(f)\circ x)$ and $\epsilon_B\circ r(\imath(f)\circ y)$. But
    \begin{equation*}
      \epsilon_B\circ (r\circ\imath)(f) = f\circ\epsilon_A
    \end{equation*}
    Therefore, if $\imath(f)\circ x = \imath(f)\circ y$ then $f\circ\epsilon_A \circ r(x)=  f\circ\epsilon_A\circ r(y)$, thus
    \begin{equation*}
      \epsilon_A \circ r(x)= \epsilon_A\circ r(y)
    \end{equation*}

    On the other hand, we have
    \begin{equation*}
      \Hom_{\Bb}(T,\imath(A)) \approx \Hom_{\Aa}(r(T),A)
    \end{equation*}
    which maps $x$ and $y$ to $\epsilon_A\circ r(x)$ and $\epsilon_A\circ r(y)$ respectively.
    Therefore, $x=y$ and thus $\imath(f)$ is monic.
  \end{proof}


  \begin{prop}\label{prop:reflection of limits}
    A reflective subcategory $\Aa$ of a (finitely) complete category $\Bb$ is (finitely) complete.
  \end{prop}
  \begin{proof}
    Let $\imath\colon\Aa\hookrightarrow\Bb$ be the inclusion functor with left adjoint $r$. For a (finitely) small diagram $D\colon\Ii^{\op}\to\Aa$, $\imath\circ D$ is a diagram in $\Bb$. Since $\Bb$ is (finitely) complete, we have a limit of $\imath\circ D$, say $\mu\colon\Delta_L\then\imath\circ D$.

    Consider the cone $r\ast\mu\colon\Delta_{r(L)}\then F\circ\imath\circ D$. Since $\imath$ is fully faithful, the counit $\epsilon$ is an isomorphism, then $r\circ\imath\circ D\cong D$, thus $r\ast\mu$ is a cone over $D$. Then $\imath\ast r\ast\mu$ is a cone over $\imath\circ D$. Thus there exists a unique factorization $f\colon r(L)\to L$ of $\imath\ast r\ast\mu$ by $\mu$. On the other hand, by the adjunct of $1_{r(L)}$ gives a morphism $\eta_L\colon L\to\imath\circ r(L)$ in $\Bb$.
      \begin{displaymath}
        \xymatrix{
          \imath\circ r(L)\ar@<-0.5ex>[r]_-{f}\ar@{}[dr]_{(\imath\circ r)\ast\mu}|-{\SelectTips{eu}{}\object@{=>}}
          &L\ar@<-0.5ex>[l]_-{\eta_L}\ar@{}[d]^-{\mu}|-{\SelectTips{eu}{}\object@{=>}}\\
          &\imath\circ D
         }
      \end{displaymath}

      By the naturality of the unit $\eta$, the above diagram commutes. Then $f\circ\eta_L = 1_L$.
      Since $\imath$ is fully faithful, we have a unique $g\colon r(L)\to r(L)$ such that $\imath(g)=\eta_L\circ f$.
      We have
      \begin{equation*}
        \imath(g)\circ\eta_L = \eta_L\circ f\circ\eta_L = \eta_L = \imath(1_{r(L)})\circ\eta_L
      \end{equation*}

      Since taking adjunct $g \mapsto \imath(g)\circ\eta_L$ is unique, $g=1_{r(L)}$ and thus $f$ is an isomorphism. Therefore $(\imath\circ r)\ast\mu$ is a limit of $\imath\circ D$ and then $r\ast\mu$ is a limit of $D$.
  \end{proof}
  By duality, a coreflective subcategory of a cocomplete category is itself cocomplete. This obvious remark stands here just to emphasize the fact that our next result is by no means dual to the previous one.
  \begin{prop}\label{prop:reflection of colimits}
    A reflective subcategory $\Aa$ of a (finitely) cocomplete category $\Bb$ is (finitely) cocomplete.
  \end{prop}
  \begin{proof}
    Let $\imath\colon\Aa\hookrightarrow\Bb$ be the inclusion functor with left adjoint $r$. For a (finitely) small diagram $D\colon\Ii\to\Aa$, $\imath\circ D$ is a diagram in $\Bb$. Since $\Bb$ is (finitely) cocomplete, we have a colimit of $\imath\circ D$, say $\mu\colon\imath\circ D\then\Delta_C$. Since left adjoint preserves colimits, $r\ast\mu\colon r\circ\imath\circ D\then\Delta_{r(C)}$ is a colimit of $r\circ\imath\circ D$. But since $\imath$ is fully faithful, the counit $\epsilon$ is an isomorphism, then $r\circ\imath\circ D\cong D$. Hence $F\ast\mu$ is a colimit of $D$.
  \end{proof}

  \begin{defn}
    A \termin{localization} of a category $\Bb$ with finite limits is a reflective subcategory $\Aa$ of $\Bb$ whose reflection preserves finite limits.
  \end{defn}
  \begin{defn}
    An \termin{essential localization} of a category $\Bb$ is a reflective subcategory $\Aa$ of $\Bb$ whose rejection itself admits a left adjoint.
  \end{defn}

  A functor with a left adjoint preserves all limits (Proposition \ref{prop:adjoint functor preserves limits}). So when $\Bb$ has finite limits, every essential localization of $\Bb$ is certainly a localization.
  Moreover if $\imath\colon\Aa\to\Bb$ is the canonical inclusion and $L\dashv r\dashv\imath$ are the reflection and its left adjoint, the functor $L$ is again full and faithful (by Proposition \ref{prop:two side adjoint}).
  Therefore the full subcategory $L(\Aa)\subset\Bb$ is, up to an equivalence, a coreflective subcategory of $\Bb$. It should be noticed that $L$ has in general no reason at all to coincide with the canonical inclusion $\imath$.

  \begin{prop}
    Consider a category $\Bb$ with finite limits and filtered colimits, and a localization $\Aa$ of $\Bb$. If in $\Bb$, finite limits commute with filtered colimits, the same property holds in $\Aa$.
  \end{prop}
  \begin{proof}
    Let $\imath\Aa\to\Bb$ be the canonical inclusion and $r$ be its left adjoint. Consider a diagram
    \begin{equation*}
      D\colon\Ii\times\Jj\To\Aa
    \end{equation*}
    with $\Ii$ finite and $\Jj$ filtered.

    The proof of Proposition \ref{prop:reflection of limits} shows that
    \begin{equation*}
      \invlim_{\Jj}D \approx r(\invlim_{\Jj}\imath\circ D)
    \end{equation*}

    Then, by proof of Proposition \ref{prop:reflection of colimits},
    \begin{equation*}
      \dirlim_{\Ii}r(\invlim_{\Jj}\imath\circ D) \approx r(\dirlim_{\Ii}\imath\circ r(\invlim_{\Jj}\imath\circ D))
    \end{equation*}

    Notice that $\imath,r$ preserve limits and $r$ preserves colimits, we have
    \begin{equation*}
       r(\dirlim_{\Ii}\imath\circ r(\invlim_{\Jj}\imath\circ D)) \approx r\circ\imath\circ r (\dirlim_{\Ii}\invlim_{\Jj}\imath\circ D)
    \end{equation*}

    Since finite limits commute with filtered colimits in $\Bb$, we have
    \begin{equation*}
       \dirlim_{\Ii}\invlim_{\Jj}\imath\circ D \approx \invlim_{\Jj}\dirlim_{\Ii}\imath\circ D
    \end{equation*}

    Based on the above, we have
    \begin{align*}
      \dirlim_{\Ii}\invlim_{\Jj}D &\approx r\circ\imath\circ r (\invlim_{\Jj}\dirlim_{\Ii}\imath\circ D) \\
                                                  &\approx r\circ\imath (\invlim_{\Jj}\dirlim_{\Ii}r\circ\imath\circ D)  \\
                                                  &\approx \invlim_{\Jj}\dirlim_{\Ii}D
    \end{align*}

    Where the last isomorphism is based on the constructions in proof of Proposition \ref{prop:reflection of limits} and \ref{prop:reflection of colimits}.
  \end{proof}

\subsection{Epireflective subcategories}
  \begin{defn}
    Consider a category $\Bb$ and a reflective subcategory $\Aa$ of $\Bb$. Its reflection $r$ is said to be an \termin{epireflection} if the unit $\eta\colon\Id_{\Bb}\to\imath\circ r$ is pointwise epi.
  \end{defn}

  \begin{prop}
    Consider a category $\Bb$ in which every morphism can be factored as an epimorhism followed by a strong monomorphism. For a reflective subcategory $r\dashv\imath\colon\Aa\hookrightarrow\Bb$, the following conditions are equivalent:
    \begin{enumerate}
      \item the reflection is an epireflection;
      \item given a strong monomorphism $u\colon B\to\imath(A)$ in $\Bb$, then the object $B$ belongs to $\Aa$,
    \end{enumerate}
  \end{prop}

\newpage
\section{Kan extensions}
  The \termin{Kan extension} of a functor $F\colon\Aa\to\Bb$ with respect to a functor $p\colon\Aa\to\Aa'$ is, if it exists, a kind of best approximation to the problem of finding a functor $\Aa'\to\Bb$ such that
      \begin{displaymath}
        \xymatrix{
          \Aa\ar[r]^{F}\ar[d]_{p}&\Bb\\
          \Aa'\ar[ur]&
         }
      \end{displaymath}
  hence to extending the domain of $F$ through $p$ from $\Aa$ to $\Aa'$.

  Similarly, a \termin{Kan lift} is the best approximation to lifting a functor $F\colon\Aa\to\Bb$ through a functor $q\colon\Bb'\to\Bb$ to a functor $\hat{F}$:
      \begin{displaymath}
        \xymatrix{
          &\Bb'\ar[d]^{q}\\
          \Aa\ar[ur]^{\hat{F}}\ar[r]^{F}&\Bb
         }
      \end{displaymath}

  Kan extensions are ubiquitous.

  \begin{defn}
    Consider two functors $F\colon\Aa\to\Bb$ and $p\colon\Aa\to\Aa'$. The \termin{left Kan extension} of $F$ along $p$, if it exists, is a pair $(G,\alpha)$ where
    \begin{itemize}
      \item $G\colon\Aa'\to\Bb$ is a functor;
      \item $\alpha\colon F\then G\circ p$ is a natural transformation
    \end{itemize}
    satisfying the following universal property:
    if $(H,\beta)$ is another pair with
    \begin{itemize}
      \item $H\colon\Aa'\to\Bb$ is a functor;
      \item $\beta\colon F\then H\circ p$ is a natural transformation
    \end{itemize}
    there exists a unique natural transformation $\gamma\colon G\then H$ satisfying the equality $(\gamma\ast p)\circ\alpha=\beta$.
      \begin{displaymath}
        \xymatrix{
          \Aa\ar[rr]^{F}\rrlowertwocell<-12>~{\dir{}}~'{\dir{}}^{}_{}{\alpha}\ar[dr]_{p}&&\Bb\\
          &\Aa'\ar[ur]^{G}\urlowertwocell^{}_{H}{\gamma}&
         }
      \end{displaymath}

    We shall use the notation $\Lan_pF$ to denote the \termin{left Kan extension} of $F$ along $p$. The notation $\Ran_pF$ is used for the dual notion of \termin{right Kan extension}.\glsadd{Lan}\glsadd{Ran}
  \end{defn}

    Let $p\colon\Aa\to\Aa'$ be a functor between small categories. For $\Bb$ any other category, write
  \begin{equation*}
    p^{\ast}\colon[\Aa',\Bb]\to[\Aa,\Bb]
  \end{equation*}
  for the induced functor which sends a functor $H\colon\Aa'\to\Bb$ to the composite functor $H\circ p\colon\Aa\to\Aa'\to\Bb$.

  Then the left Kan extension of $F$ along $p$ is precisely a corepresentation of the functor $\Hom_{[\Aa,\Bb]}(F,p^{\ast}(-))$, which is a specified natural isomorphism
  \begin{equation*}
    \Hom_{[\Aa,\Bb]}(F,p^{\ast}(-))\cong\Hom_{[\Aa',\Bb]}(\Lan_pF,-)
  \end{equation*}
  with corepresenting object $\Lan_pF$.

  Moreover, if for any functor $F\colon\Aa\to\Bb$, its left Kan extension along $p$ exists, then those corepresenting objects form a functor
  \begin{equation*}
    \Lan_p\colon[\Aa,\Bb]\to[\Aa',\Bb]
  \end{equation*}
  left adjoint to $p^{\ast}$.
  \begin{defn}
    Let $p\colon\Aa\to\Aa'$ be a functor and $p^{\ast}$ its induced pullback functor. The left adjoint of $p^{\ast}$, if it exists, is called the \termin{left Kan extension operation} along $p$ and denoted by $p_{!}$ or $\Lan_p$. The right adjoint, if it exists, is called the \termin{right Kan extension operation} along $p$ and denoted by $p_{\ast}$ or $\Ran_p$.
  \end{defn}

  Use the fact that $[\one,\Aa]\simeq\Aa$ and Proposition \ref{prop:limits as adjoint functors}, we have
  \begin{prop}
    Let $\Aa'$ be the terminal category $\one$, then
    \begin{enumerate}
      \item the left Kan extension operation forms the colimit of a functor;
      \item the right Kan extension operation forms the limit of a functor.
    \end{enumerate}
  \end{prop}

\subsection{Preserving Kan extensions}
  \begin{defn}
    We say that a Kan extension $\Lan_pF$ is \termin[preserved]{preserve Kan extension} by a functor $G$ if the composite $G\circ\Lan_pF$ is a Kan extension of $G\circ F$ along $p$, and moreover the universal natural transformation $G\circ F\then G\circ \Lan_pF\circ p$ is the composite of $G$ with the universal transformation $F\then \Lan_pF\circ p$.
  \end{defn}
  \begin{prop}
    Left adjoint functor preserves left Kan extensions.
  \end{prop}
  \begin{proof}
    Let $L$ be a functor with right adjoint $R$, then we need to show that $L\circ\Lan_pF=\Lan_p(L\circ F)$.
      \begin{displaymath}
        \xymatrix{
          \Aa\ar[d]_p\ar[r]^{F}&\Bb\ar@<0.5ex>[r]^{L}&\Cc\ar@<0.5ex>[l]^{R}\\
          \Aa'\ar[ur]|{\Lan_pF}\ar@/_1.1pc/[urr]|{\Lan_p(L\circ F)}
         }
      \end{displaymath}

      Indeed, for any functor $G\colon\Aa'\to\Cc$, we have
      \begin{align*}
        \Nat(L\circ\Lan_pF,G) &\approx\Nat(\Lan_pF,R\circ G)  \\
         &\approx\Nat(F,R\circ G\circ p) \\
         &\approx\Nat(L\circ F,G\circ p) \\
         &\approx\Nat(\Lan_p(L\circ F),G)
      \end{align*}
      So $L\circ\Lan_pF=\Lan_p(L\circ F)$, by putting successively $H=L\circ\Lan_pF$ and $H=\Lan_p(L\circ F)$.
  \end{proof}

  \begin{prop}
    Consider a functor $p\colon\Aa\to\Aa'$ between small categories. The following conditions are equivalent:
    \begin{enumerate}
      \item $p$ has a right adjoint $q$;
      \item $\Lan_p\Id_{\Aa}$ exists, and for every functor $F\colon\Aa\to\Bb$, the isomorphism $F\circ\Lan_p\Id_{\Aa}\cong\Lan_pF$ holds;
      \item $\Lan_p\Id_{\Aa}$ exists, and the isomorphism $p\circ\Lan_p\Id_{\Aa}\cong\Lan_pp$ holds.
    \end{enumerate}
      \begin{displaymath}
        \xymatrix{
          \Aa\ar[d]_p\ar[r]^{\Id_{\Aa}}&\Aa\ar[r]^{F}&\Bb\\
          \Aa'\ar[ur]|{\Lan_p\Id_{\Aa}}\ar@/_1.1pc/[urr]|{\Lan_pF}
         }
      \end{displaymath}
  \end{prop}
  \begin{proof}
    For \emph{1} $\then$ \emph{2}, let $\eta\colon\Id_{\Aa}\then q\circ p$ and $\epsilon\colon p\circ q\then\Id_{\Aa'}$ be the unit and counit of the adjunction $p\dashv q$, then we get a natural transformation $\alpha=F\ast\eta$. For any functor $G\colon\Aa'\to\Bb$ and natural transformation $\beta\colon F\then G\circ p$, consider the composite
      \begin{displaymath}
        \xymatrix{
          \Aa'\ar[r]^q\rrlowertwocell<-12>^{}_{\Id_{\Aa'}}{\epsilon}&\Aa\ar[r]^p\rruppertwocell<12>^{F}_{}{\beta}&\Aa'\ar[r]^G&\Bb
         }
      \end{displaymath}
    \begin{equation*}
      \gamma=(G\ast\epsilon)\circ(\beta\ast q)\colon F\circ q\then G\circ p\circ q\then G
    \end{equation*}

    Then, by the triangle identity, it is not difficult to see that $(\gamma\ast p)\circ\alpha=\beta$ from the following calculation
      \begin{displaymath}
        \xymatrix{
          \Aa\ar[r]^p\rruppertwocell<12>^{\Id_{\Aa}}_{}{\eta}&\Aa'\ar[r]^q\rrlowertwocell<-12>^{}_{\Id_{\Aa'}}{\epsilon}&\Aa\ar[r]^p\rruppertwocell<12>^{F}_{}{\beta}&\Aa'\ar[r]^G&\Bb
         }
         =
        \xymatrix{
          \Aa\ar[r]^p\rruppertwocell<12>^{F}_{}{\beta}&\Aa'\ar[r]^G&\Bb
         }
      \end{displaymath}

      To see such $\gamma$ is unique, consider another natural transformation $\gamma'$ such that $(\gamma'\ast p)\circ\alpha=\beta$, i.e.
      \begin{displaymath}
        \xymatrix{
          \Aa\ar[r]^p\rruppertwocell<12>^{\Id_{\Aa}}_{}{\eta}&\Aa'\ar[r]^q\rrlowertwocell<-12>^{}_{G}{\gamma'}&\Aa\ar[r]^F&\Bb
         }
         =
        \xymatrix{
          \Aa\ar[r]^p\rruppertwocell<12>^{F}_{}{\beta}&\Aa'\ar[r]^G&\Bb
         }
      \end{displaymath}

      Therefore, we have
      \begin{align*}
        \xymatrix{
          \Aa'\ar[r]^q\rrlowertwocell<-12>^{}_{G}{\gamma'}&\Aa\ar[r]^F&\Bb
         }
        & =
        \xymatrix{
          \Aa'\ar[r]^q\rrlowertwocell<-12>^{}_{\Id_{\Aa'}}{\epsilon}&\Aa\ar[r]^p\rruppertwocell<12>^{\Id_{\Aa}}_{}{\eta}&\Aa'\ar[r]^q\rrlowertwocell<-12>^{}_{G}{\gamma'}&\Aa\ar[r]^F&\Bb
         }
         \\
         & =
        \xymatrix{
          \Aa'\ar[r]^q\rrlowertwocell<-12>^{}_{\Id_{\Aa'}}{\epsilon}&\Aa\ar[r]^p\rruppertwocell<12>^{F}_{}{\beta}&\Aa'\ar[r]^G&\Bb
         }
      \end{align*}

      \begin{align*}
        \gamma' &= \gamma'\circ(F\ast q\ast\epsilon)\circ(\alpha\ast q) \\
                         &= (G\ast\epsilon)\circ(\gamma'\ast p\ast q)\circ(\alpha\ast q)\\
                         &= (G\ast\epsilon)\circ(\beta\ast q) = \gamma
      \end{align*}

      This prove $\Lan_pF=F\circ q$ for every functor $F\colon\Aa\to\Bb$. Specially, $\Lan_p\Id_{\Aa}=q$.

      \emph{2} $\then$ \emph{3} is obvious.

      For \emph{3} $\then$ \emph{1}, we need to show that $\Lan_p\Id_{\Aa}$ is the right adjoint of $p$. The canonical natural transformation $\eta\colon\Id_{\Aa}\then(\Lan_p\Id_{\Aa})\circ p$ should be the unit. It suffices to construct the counit and check the triangle identities.

      Set $q=\Lan_p\Id_{\Aa}$.
      By our assumption, $\alpha=p\ast\eta\colon p\then p\circ q\circ p$ is the canonical natural transformation of $\Lan_pp$.

      Consider the functor $\Id_{\Aa'}\colon\Aa'\to\Aa'$ and the natural transformation $\Id_p\colon p\then p$. Then, by the universal property of Kan extension. there exists a unique natural transformation $\epsilon\colon p\circ q\then\Id_{\Aa'}$ such that $(\epsilon\ast p)\circ\alpha=\Id_p$, which is just the triangle identity
      \begin{displaymath}
        \xymatrix{
          \Aa\ar[r]^p\rruppertwocell<12>^{\Id_{\Aa}}_{}{\eta}&\Aa'\ar[r]^q\rrlowertwocell<-12>^{}_{\Id_{\Aa'}}{\epsilon}&\Aa\ar[r]^p&\Aa'
         }
         =
        \xymatrix{
          \Aa\ar[r]^p&\Aa'
         }
      \end{displaymath}

      For the other triangle identity, consider that the composite $(q\ast\epsilon)\circ(\eta\ast q)$ satisfying
      \begin{displaymath}
        \xymatrix{
          \Aa\ar[r]^p\rruppertwocell<12>^{\Id_{\Aa}}_{}{\eta}&\Aa'\ar[r]^q\rrlowertwocell<-12>^{}_{\Id_{\Aa'}}{\epsilon}
          &\Aa\ar[r]^p\rruppertwocell<12>^{\Id_{\Aa}}_{}{\eta}&\Aa'\ar[r]^q&\Aa
         }
         =
        \xymatrix{
          \Aa\ar[r]^p\rruppertwocell<12>^{\Id_{\Aa}}_{}{\eta}&\Aa'\ar[r]^q&\Aa
         }
      \end{displaymath}

      For functor $q\colon\Aa'\to\Aa$ and natural transformation $\eta\colon\Id_{\Aa}\then q\circ p$, the factorization $\gamma\colon q\then q$ satisfying $(\gamma\ast p)\circ \eta=\eta$ is unique. Therefore, $(q\ast\epsilon)\circ(\eta\ast q)=\Id_q$ and thus $\eta,\epsilon$ are the unit and counit of adjunction $p\dashv q$.
  \end{proof}
  \begin{rem}
    This proposition also shows that adjoint functions are special Kan extensions.
  \end{rem}

\subsection{Pointwise Kan extension}
  \begin{thm}
    Let $F\colon\Aa\to\Bb$ be a functor from a small category $\Aa$ to a cocomplete category $\Bb$. Then for any $p\colon\Aa\to\Aa'$, the left Kan extension of $F$ along $p$ exists and its value on an object $X$ in $\Aa'$ is given by the colimit
    \begin{equation*}
      (\Lan_pF)(X)\cong\dirlim((p\down\Delta_X)\to\Aa\markar{F}\Bb)
    \end{equation*}
    where
    \begin{itemize}
      \item $(p\down\Delta_X)$ is the comma category;
      \item $(p\down\Delta_X)\to\Aa$ is the domain functor.
    \end{itemize}

    Likewise, if $\Bb$ is complete, then for any $p\colon\Aa\to\Aa'$, the right Kan extension of $F$ along $p$ exists and its value on an object $X$ in $\Aa'$ is given by the limit
    \begin{equation*}
      (\Ran_pF)(X)\cong\invlim((\Delta_X\down p)\to\Aa\markar{F}\Bb)
    \end{equation*}
    where
    \begin{itemize}
      \item $(\Delta_X\down p)$ is the comma category;
      \item $(\Delta_X\down p)\to\Aa$ is the codomain functor.
    \end{itemize}
  \end{thm}
  \begin{proof}
    Consider the case of the left Kan extension, the other case works analogously, but dually.

    First notice that the above pointwise definition of values of a functor canonically extends to an actual functor:

    For any morphism $f\colon X_1\to X_2$ in $\Aa'$, we get a functor
    \begin{equation*}
      (p\down\Delta_{X_1})\To(p\down\Delta_{X_2})
    \end{equation*}
    of comma categories, by postcomposition. Then we get a morphism from the diagram
    \begin{equation*}
      (p\down\Delta_{X_1})\to\Aa\markar{F}\Bb
    \end{equation*}
    to
    \begin{equation*}
      (p\down\Delta_{X_2})\to\Aa\markar{F}\Bb
    \end{equation*}
    which induces canonically a corresponding morphism of their colimits.
    \begin{equation*}
       (\Lan_pF)(X_1)\To(\Lan_pF)(X_2)
    \end{equation*}

    Now we consider that, for any $X$ in $\Aa'$, as a cocone of the diagram
    \begin{equation*}
      (p\down\Delta_X)\to\Aa\markar{F}\Bb
    \end{equation*}
    $(\Lan_pF)(X)$ gives a family of commutative diagrams:
      \begin{displaymath}
        \xymatrix@R=0.5cm{
          (p\down\Delta_X)&(p(A_1)\markar{f} X)\ar[rr]^{p(h)}&&(p(A_2)\markar{g} X)\\
          \Bb&F(A_1)\ar[rr]^{F(h)}\ar[dr]_{s_f}&&F(A_2)\ar[dl]^{s_g}\\
          &&(\Lan_pF)(X)&
         }
      \end{displaymath}
      and induces a natural transformation
      \begin{equation*}
        \eta_F\colon F\then p^{\ast}\Lan_pF
      \end{equation*}
      satisfying the required universal property
      by taking
      \begin{equation*}
        \eta_F(A)=s_{1_{p(A)}}\colon F(A)\to(\Lan_pF)(p(A)).
      \end{equation*}

    It is straightforward, if somewhat tedious, to check that these are natural, and that the natural transformation defined this way has the required universal property.
  \end{proof}
  A Kan extension is called pointwise when every $\Lan_pF(X)$ can be computed by the colimit formula in this theorem. It should be noted, though, that pointwise Kan extensions can still exist, and hence the particular requisite (co)limits exist, even if $\Bb$ is not (co)complete.

  Here, we give a formal definition of pointwise Kan extension that doesn't rely on any computational framework
  \begin{defn}
    A right Kan extension is called \termin[pointwise]{pointwise Kan extension} when it is preserved by all representable functors. A left Kan extension is called \termin[pointwise]{pointwise Kan extension} when its opposite is a pointwise right Kan extension.
  \end{defn}

  \begin{thm}\label{thm:pointwise Kan extension}
    A right Kan extension $\Ran_pF$ is pointwise if and only if for any $X$ in $\Aa'$, the limit $\invlim((\Delta_X\down p)\to\Aa\markar{F}\Bb)$ exists.
    Dually, a left Kan extension $\Lan_pF$ is pointwise if and only if for any $X$ in $\Aa'$, the colimit $ \dirlim((p\down\Delta_X)\to\Aa\markar{F}\Bb)$ exists.
  \end{thm}
  \begin{proof}
    Since representable functors preserve limits, the ``if'' is obvious.

    Conversely, assume the right Kan extension $\Ran_pF$ is pointwise. Then for any $X$ in $\Aa'$, we need to show that $(\Ran_pF)(X)$ is the limit of the composite $(\Delta_X\down p)\to\Aa\markar{F}\Bb$.

    Let $H$ denote the composite of the representable functor $\Hom_{\Bb}(B,-)$ with $\Ran_pF$. Then by the covariant Yoneda lemma,
    \begin{equation*}
      \Nat(\Hom_{\Aa'}(X,-),H)\approx H(X)
    \end{equation*}

    By our assumption, $H$ is the right Kan extension of the composite of $\Hom_{\Bb}(B,-)$ and $F$, so we have
    \begin{equation*}
      \Hom_{[\Aa,\Set]}(p^{\ast}(-),\Hom_{\Bb}(B,F(-)))\cong\Hom_{[\Aa',\Set]}(-,H)
    \end{equation*}
    Therefore we get the following bijection.
    \begin{equation*}
      \Hom_{\Bb}(B,(\Ran_pF)(X))\approx\Nat(\Hom_{\Aa'}(X,p(-)),\Hom_{\Bb}(B,F(-)))
    \end{equation*}

    It is not difficult to see that the right side set is equivalent to the set of cones over the composite $(\Delta_X\down p)\to\Aa\markar{F}\Bb$. Thus $(\Ran_pF)(X)$ is its limit.
  \end{proof}

  \begin{prop}
    Let $\Lan_pF$ be a pointwise Kan extension with $p$ fully faithful, then the canonical natural transformation $\eta_F\colon F\then p^{\ast}(\Lan_pF)$ is an isomorphism.
  \end{prop}
  \begin{proof}
    In the proof of Theorem \ref{thm:pointwise Kan extension}, we have constructed the natural transformation
    \begin{equation*}
      \eta_F\colon F\then p^{\ast}(\Lan_pF)
    \end{equation*}
    whose component $\eta_F(A)$ at $A\in\ob\Aa$ is the component $s_{1_{p(A)}}$ of the colimit cocone
    \begin{equation*}
      \dirlim((p\down\Delta_X)\to\Aa\markar{F}\Bb)
    \end{equation*}
    under index $(A,1_{p(A)})$.

    Therefore, to show $\eta_F$ is an isomorphism, it suffices to show that for every $A\in\ob\Aa$, $s_{1_{p(A)}}$ is an isomorphism. To do this, we shall point out that $(A,1_{p(A)})$ is a terminal object in $(p\down\Delta_{p(A)})$, then the colimit is just $F(A)$ and $s_{1_{p(A)}}=1_{F(A)}$.

    Indeed, for any $(A',g)$ in $(p\down\Delta_{p(A)})$, since $p$ is fully faithful, there exists a unique morphism $f\in\Hom(A',A)$ such that $p(f)=g$. Then $f$ is a morphism from $(A',g)$ to $(A,1_{p(A)})$. Conversely, for any morphism $f$ from $(A',g)$ to $(A,1_{p(A)})$, we have $p(f)=g$. Thus $(A,1_{p(A)})$ is the terminal object.
  \end{proof}

\subsection{Exercises}

  \begin{ex}
    Consider two functors $p\colon\Aa\to\Aa'$ and $F\colon\Aa\to\Bb$, with the property
    \begin{equation*}
      \forall A,A\in\ob\Aa, \forall f,f'\colon A\to A',  F(f)=F(f')\Longrightarrow p(f)=p(f')
    \end{equation*}
    Suppose $\Aa,\Aa'$ are small, $\Bb$ is cocomplete and $p$ is full. 
    Show that $\Lan_pF$ exists and the isomorphism $F = \Lan_pF \circ p$ holds.
  \end{ex}
  \begin{ex}
    Consider a functor $p\colon\Aa\to\Aa'$ between small categories. For each object $A$ in $\Aa$, the Kan extension $\Lan_p\Hom_{\Aa}(A,-)$ exists and is given by 
    $\Hom_{\Aa'}(p(A),-)$. [Hint: apply the Yoneda lemma.] The equality 
    \begin{equation*}
      \Lan_p\Hom_{\Aa}(A,-)\circ p = \Hom_{\Aa}(A,-)
    \end{equation*}
    holds precisely when $p$ is fully faithful.
  \end{ex}
  \begin{ex}
    Consider the category $\one$ with a single object $A$ and a single arrow $1_A$, the category $\mathbf{2}$ with two distinct objects $A, B$ and just the identity arrows $1_A,1_B$, and the category $\vec{\mathbf{2}}$ with two distinct objects $A, B$, the identity arrrows $1_A, 1_B$ and an additional arrow $f\colon B \to A$. 
    Define the functors $p\colon \one \to \vec{\mathbf{2}}$ and $F\colon \one \to \mathbf{2}$ by $p(A) = A, F(A) = A$. 
    Observe that the only two functors from $\vec{\mathbf{2}}$ to $\mathbf{2}$ are the constant functors $\Delta_A$ and $\Delta_{A'}$. Check that $A$ a is the left Kan extension $\Lan_pF$. Prove that $(p\down\Delta_A)$ is an empty category and $\Delta_A(B)$ is not isomorphic to the colimit of $(p\down\Delta_X)\to\one\markar{F}\mathbf{2}$. So the Kan extension $\Lan_pF$ is not pointwise.
  \end{ex}

%concrete category


  \chapter{Order Theory}
  In this chapter, we introduce some basic concepts in order theory by treat it as a special case of category theory.
\minitoc
\newpage
\section{Relations and orders}
  \begin{defn}
    Given a family $(S_i)_{i\in I}$ of sets, a \termin{relation} on that family is a subset $R$ of the cartesian product $\prod_{i\in I}S_i$.

    A \termin{unary relation} on $A$ is a relation on the singleton family $(A)$. This is the same as a subset of $A$.

    A \termin{binary relation} on $A$ and $B$ is a relation on the family $(A,B)$, that is a subset of $A\times B$. This is also called a relation from $A$ to $B$.

    A \termin{binary relation} on $A$ is a relation on $(A,A)$, that is a relation from $A$ to itself. This is sometimes called simply a relation on $A$.

    An \termin{$n-$ary relation} on $A$ is a relation on a family of $n$ copies of $A$, that is a subset of $A^n$.

    For a binary relation, one often uses a symbol such as $\sim$ and writes $a\sim b$ instead of $(a,b)\in\sim$. Actually, even when a relation is given by a letter such as $R$, one often sees $aRb$ instead of $(a,b)\in R$, although now that does not look so good.
  \end{defn}

    Binary relations form a $2-$category $\Rel$.

    The objects are sets, the morphisms from $A$ to $B$ are the binary relations from $A$ to $B$, and there is a $2-$morphism from $R$ to $S$ (both from $A$ to $B$) if $R$ implies $S$ (that is, when $(a,b)\in R$, then $(a,b)\in S$).

    The interesting definition is composition
  \begin{defn}
    If $R$ is a relation from $A$ to $B$ and $S$ is a relation from $B$ to $C$, then their \termin{composite relation} $S\circ R$ from $A$ to $C$ is defined as follows:
    \begin{equation*}
      (a,c)\in S\circ R \iff \exists b\in B, (a,b)\in R \wedge (b,c)\in S
    \end{equation*}
  \end{defn}

  The identity morphism is given by \termin{equality} (also called \termin{diagonal}):
  \begin{equation*}\glsadd{equality}
    1_A=\Delta_A:=\{(a,a)\mid a\in A\}
  \end{equation*}

  Moreover, there exists another operation on $\Rel$, the \termin[reverse]{reverse relation}:
  \begin{equation*}
    R^{\op}:=\{(y,x)\mid(x,y)\in R\}
  \end{equation*}

  It is obvious that:
  \begin{equation*}
    (S\circ R)^{\op}=R^{\op}\circ S^{\op}\qquad 1_A^{\op}=1_A
  \end{equation*}

  \begin{defn}
    A binary relation $\sim$ on a set $A$ is \termin[transitive]{transitive relation} if in every chain of three pairwise related elements $x\sim y, y\sim z$, the first and last elements are also related $x\sim z$.
  \end{defn}
  In $\Rel$, a relation $R\colon A\to A$ is transitive if and only if it contains its composite with itself:
  \begin{equation*}
    R^2\subset R
  \end{equation*}
  from which it follows that $R^n\subset R$ for any natural number $n$ except $n=0$.

  \begin{defn}
    A binary relation $\sim$ on a set $A$ is \termin[reflexive]{reflexive relation} if every element of $A$ is related to itself.
  \end{defn}
  In $\Rel$, a relation $R\colon A\to A$ is reflexive if and only if it contains the equality:
  \begin{equation*}
    1_A\subset R
  \end{equation*}

  The antithetical case is
  \begin{defn}
    A binary relation $\sim$ on a set $A$ is \termin[irreflexive]{irreflexive relation} if no element of $A$ is related to itself.
  \end{defn}
  In $\Rel$, a relation $R\colon A\to A$ is irreflexive if and only if it is disjoint from the the equality:
  \begin{equation*}
    1_A\cap R=\varnothing
  \end{equation*}

  Another interesting case is
  \begin{defn}
    A binary relation $\sim$ on a set $A$ is \termin[symmetric]{symmetric relation} if any two elements that are related in one order are also related in the other order.
  \end{defn}
  In $\Rel$, a relation $R\colon A\to A$ is symmetric if and only if it contains its reverse:
  \begin{equation*}
    R^{\op}\subset R
  \end{equation*}

  \begin{defn}
    A binary relation $\sim$ on a set $A$ is \termin[antisymmetric]{antisymmetric relation} if any two elements that are related in both orders are equal.
  \end{defn}
  In $\Rel$, a relation $R\colon A\to A$ is antisymmetric if and only if its intersection with its reverse is contained in the equality:
  \begin{equation*}
    R^{\op}\cap R\subset 1_A
  \end{equation*}

  \begin{defn}
    A binary relation $\sim$ on a set $A$ is \termin[asymmetric]{asymmetric relation} if no two elements are related in both orders.
  \end{defn}
  In $\Rel$, a relation $R\colon A\to A$ is asymmetric if and only if it is disjoint from its reverse:
  \begin{equation*}
    R^{\op}\cap R = \varnothing
  \end{equation*}

  The next important case is
  \begin{defn}
    A binary relation $\sim$ on a set $A$ is \termin[total]{total relation} if any two elements that are related in one order or the other.
  \end{defn}
  In $\Rel$, a relation $R\colon A\to A$ is total if and only if its union with its reverse is the \termin{universal relation}:
  \begin{equation*}
    R^{\op}\cup R = A\times A
  \end{equation*}

  \begin{defn}
    A binary relation $\sim$ on an set $A$ is \termin[left euclidean]{left euclidean relation} if every two elements both related to a third are also related to each other. A relation $\sim$ is \termin[right euclidean]{right euclidean relation} if this works in the other order.
  \end{defn}
  \begin{rem}
    One could also say \termin[euclidean]{euclidean relation} and \termin[coeuclidean]{coeuclidean relation} or \termin[opeuclidean]{opeuclidean relation}.
  \end{rem}
  In $\Rel$, a relation $R\colon A\to A$ is left euclidean if and only if it contains its composite with its reverse:
  \begin{equation*}
    R^{\op}\circ R \subset R
  \end{equation*}
  a relation $R\colon A\to A$ is right euclidean if and only if it contains the composite of its reverse with itself:
  \begin{equation*}
    R\circ R^{\op} \subset R
  \end{equation*}

  An important example of the combination of those concepts is
  \begin{defn}
    An \termin{equivalence relation} on a set $S$ is a binary relation $\sim$ on $S$ that is reflexive, symmetric and transitive.
  \end{defn}
  \begin{prop}
    A binary relation is an equivalence relation if and only if it is reflexive and euclidean in either direction.
  \end{prop}

\subsection{Orders}
    An \termin{order} on a set $S$ is a binary relation that is, at least, transitive.

  Actually, there are several different notions of order that are each useful in their own ways:
  \begin{itemize}
    \item A \termin{preorder} on a set $S$ is a reflexive and transitive relation, generally written $\leqslant$. A \termin{preordered set}, or \termin{proset}, is a set equipped with a preorder.
    \item A \termin{quasiorder} (or \termin{strict preorder}) on a set $S$ is a irreflexive and transitive relation, generally written $<$. A \termin{quasiordered set}, or \termin{quoset}, is a set equipped with a quasiorder.
    \item A \termin{partial order} on a set $S$ is an antisymmetric preorder. A \termin{partial ordered set}, or \termin{poset}, is a set equipped with a partial order.
    \item A \termin{total order} on a set $S$ is a total partial order. A \termin{total ordered set}, or \termin{toset}, is a set equipped with a total order.
  \end{itemize}

  For a preorder $\leqslant$, one can define its strict preorder $<$ by subtracting the diagonal. Conversely, a preorder can be obtained by putting $x\leqslant y$ whenever $x<y$ or $x=y$.

  We have already seen that a poset can be viewed as a category (see Example \ref{exam:category2}). Then all posets form a $2-$category, called $\Pos$.

  A morphism $f\colon A\to B$ between two poset $(A,\leqslant)$ and $(B,\leqslant)$ is a functor by definition. In other word, $f$ is a function preserving the partial order:
  \begin{equation*}
    x\leqslant y\then f(x)\leqslant f(y).
  \end{equation*}
  Such a function is called a \termin{monotone function}.

  It is not difficult to see that for any two poset $A,B$, there exists a natural partial order on the hom-set $\Hom(A,B)$ making it to be a poset and thus proving the $2-$morphisms between monotone functions.

  Recall that a \termin{thin} category is a category whose every hom-set is a singleton. It is not difficult to verify that a \emph{proset}, viewed as a category (follow Example \ref{exam:category2}), is just a thin category.

  Since every poset is also a proset, then $\Pos$ is a sub$-2-$category of the $2-$category of prosets, $\Proset$. But moreover, we have
  \begin{prop}
    Any proset is equivalent to a poset.
  \end{prop}
  This is a special case of the theorem that every category has a skeleton.

\subsection{Negation}
  Negation is the logic duality.

  In the context of relations, the \termin[negation]{negation} $\not\sim$ of a binary relation $\sim$ on $A$ is just the complement in the universal relation $A\times A$. In $\Rel$, we denote the negation of a relation $R\colon A\to A$ by $\neg R$ or $R^c$.

  Now we consider the negation of the conceptions in previous.
  \begin{defn}
    A \termin{comparison} on a set $A$ is a binary relation $\sim$ on $A$ such that in every pair of related elements, any other     element is related to one of the original elements in the same order as the original pair:
    \begin{equation*}
      \forall(x,y,z\in A),x\sim z \then x\sim y\vee y\sim z
    \end{equation*}
  \end{defn}
  Comparison is the negation of transitivity, that means a binary relation $\sim$ on $A$ is a comparison if and only if $\not\sim$ is transitive.

  \begin{defn}
    An \termin{apartness relation} on a set $A$ is a comparison $\sim$ on $A$ that is irreflexive and symmetric.
  \end{defn}
  Apartness relation is the nagation of equivalence relation.

  \begin{defn}
    A binary relation $\sim$ on a set $A$ is \termin[connected]{connected relation} if any two elements that are related in neither order are equal. In other word, $\not\sim$ is antisymmetric.
  \end{defn}
  In $\Rel$, a relation $R\colon A\to A$ is connected if and only if its union with its reverse and the diagonal is the universal relation:
  \begin{equation*}
    R^{\op}\cup R\cup1_A = A\times A
  \end{equation*}

  The negation of total order is
  \begin{defn}
    A \termin{linear order} on a set $S$ is a irreflexive, asymmetric, transitive and connected comparison.	 A \termin{linearly ordered set}, or \termin{loset}, is a set equipped with a linear order.
  \end{defn}

\subsection{Operations on a proset}
  As a special kind of category, we can consider some classical categorical concepts on a proset $A$.
  \begin{itemize}
    \item A \termin{bottom} $\bot$ of $A$ is just the initial object in $A$.
    \item A \termin{top} $\top$ of $A$ is just the terminal object in $A$.
  \end{itemize}
  \begin{rem}
    Since a poset is a skeletal category, the top and bottom in a poset is thus unique.
  \end{rem}
  
  \begin{itemize}
    \item A \termin{lower bound} of a subset $B$ of $A$ is just a cone over elements in $B$.
    \item A \termin{upper bound} of a subset $B$ of $A$ is just a cocone under elements in $B$.
  \end{itemize}

  A proset with a top element and a bottom element is called \termin[bounded]{bounded poset}. This is different from the notion of a subset $B$ of a proset $A$ to be \termin[bounded]{bounded subset}, which means $B$ has both a lower bound and an upper bound.

  More generally, A proset is \termin{bounded above} if it is has a top element and \termin{bounded below} if it has a bottom element.
  
  \begin{itemize}
    \item The \termin{meet} $x\wedge y$ of two elements $x,y\in A$ is just their product in $A$.
    \item The \termin{infimum} $\inf\limits_{i\in I}x_i$ of a family of elements $\{x_i\}_{i\in I}$ of $A$ is just their product in $A$.
  \end{itemize}
  \begin{rem}
    It may be more common to use ``meet'' for a meet of finitely many elements and ``infimum'' for a meet of (possibly) infinitely many elements, but they are the same concept. The meet may also be called the \termin{minimum} if it equals one of the original elements.
  \end{rem}
  \begin{itemize}
    \item The \termin{join} $x\vee y$ of two elements $x,y\in A$ is just their coproduct in $A$.
    \item The \termin{supremum} $\sup\limits_{i\in I}x_i$ of a family of elements $\{x_i\}_{i\in I}$ of $A$ is just their coproduct in $A$.
  \end{itemize}
  \begin{rem}
    It may be more common to use ``join'' for a join of finitely many elements and ``supremum'' for a join of (possibly) infinitely many elements, but they are the same concept. The join may also be called the \termin{maximum} if it equals one of the original elements.
  \end{rem}

  Since for any two elements in a proset, there can not be more than one morphism between them, thus by Theorem \ref{thm:complete_category} and its dual, a proset is complete (resp. cocomplete) as a category if and only if it has all infima (resp. suprema).
  Since we often consider posets, we call a poset satisfying such proerty an \termin{inflattice} (resp. \termin{suplattice}).
  Likewise, a \termin{meet-semilattice} is a poset having all finitely meets, a \termin{join-semilattice} is a poset having all finitely joins.

  \begin{itemize}
    \item A \termin{directed set} is a proset which is a filtered category.
  \end{itemize}
  
\subsection{Exercises}
  \begin{ex}
    Check the ``if and only if'' statements in text and prove the rest propositions.
  \end{ex}
  \begin{ex}
    Given two poset $A,B$, find a natural partial order on the hom-set $\Hom(A,B)$ making it to be a poset.
  \end{ex}
  \begin{ex}
    Point out that the \termin{extensional property} which states that any two elements in a poset having the same lower bounds are equal is actually a special case of the Yoneda Lemma's corollary \ref{coro:Yoneda2}. [Hint: What is $\Upsilon(x)$ for an element $x$ of a poset $A$?]
  \end{ex}


\newpage\section{Subsets of a proset}
  For $(X,\leqslant)$ a proset, a subset $A$ of $X$ is then again a proset whose preorder is the restriction of $\leqslant$ on $A$. In other word, there exists a canonical way to make a subset of a proset be a full subcategory of it. Thus when we say a subset of a proset, we actually mean a full subcategory of it.

  Among subsets of a proset, a special kind is noteworthy, that is chains.
  \begin{defn}
  A \termin{chain} of a proset is a subset that itself is a toset.
  \end{defn}
  This notion plays a critical role in Zorn's lemma and thus the theory of Noetherianess.

  Another important kind of subsets are intervals.

  \begin{defn}
    Given a proset $(X,\leqslant)$ and an elements $x$ of $X$.
    \begin{itemize}
      \item The \termin{upwards unbounded interval} $[x,+\infty)$ is the subset
                 \begin{equation*}
                 \{y\in X\mid x\leqslant y\}
                 \end{equation*}
                 It is also called the \termin{up set} of $x$, and denoted by $x\uparrow$.
      \item The \termin{downwards unbounded interval} $(-\infty,x]$ is the subset
                 \begin{equation*}
                 \{y\in X\mid y\leqslant x\}
                 \end{equation*}
                 It is also called the \termin{down set} of $x$, and denoted by $x\down$.
    \end{itemize}
    Given another element $y$ of $X$, the \termin{bounded interval} $[x,y]$ is the subset
    \begin{equation*}
      \{z\in X\mid x\leqslant z\leqslant y\}
    \end{equation*}

    Moreover, besides the \termin[closed intervals]{closed interval} above, we also have the \termin[open intervals]{open interval}
    \begin{itemize}
      \item $(x,+\infty):=[x,+\infty)\setminus\{x\}=\{y\in X\mid x<y\}$,
      \item $(-\infty,x):=(-\infty,x]\setminus\{x\}=\{y\in X\mid y<x\}$,
      \item $(x,y):=[x,y]\setminus\{x,y\}=\{z\in X\mid x<z<y\}$,
    \end{itemize}
    as well as the \termin[half-open intervals]{half-open interval}
    \begin{itemize}
      \item $[x,y):=[x,y]\setminus\{y\}=\{z\in X\mid x\leqslant z<y\}$,
      \item $(x,y]:=[x,y]\setminus\{x\}=\{z\in X\mid x<z\leqslant y\}$.
    \end{itemize}
  \end{defn}
  It is obvious that, as categories, the up set $x\uparrow$ is isomorphic to the slide category $X/x$, the down set $x\down$ is isomorphic to the coslide category $x/X$.

  \begin{defn}
    In a proset $X$, an \termin{upper set} $U$ is a subset satisfying
    \begin{quote}
      whenever $x\leqslant y$ and $x\in U$, then $y\in U$.
    \end{quote}
    while a \termin{lower set} $L$ is a subset satisfying
    \begin{quote}
      whenever $y\leqslant x$ and $x\in L$, then $y\in L$.
    \end{quote}
    
    Given a subset $A$ of $X$, the upper set \emph{generated} by $A$ is
    \begin{equation*}
      A\uparrow:=\{y\in X\mid \exists x\in A \st x\leqslant y\}
    \end{equation*}
    while the lower set \emph{generated} by $B$ is
    \begin{equation*}
      A\down:=\{y\in X\mid \exists x\in A \st y\leqslant x\}
    \end{equation*}
  \end{defn}
  Obviously, $A\uparrow=\bigcup_{x\in A}x\uparrow$, $A\down=\bigcup_{x\in A}x\down$.

\subsection{Alexandroff topology}
  The Alexandroff topology, is a natural structure of a topological space induced on the underlying set of a proset. Spaces with this topology, called Alexandroff spaces and named after Paul Alexandroff (Pavel Aleksandrov), should not be confused with Alexandrov spaces (which arise in differential geometry and are named after Alexander Alexandrov).
  \begin{defn}
    Given a proset $X$. Declare subset $U$ of $X$ to be an open subset if it is an upper set. This defines a topology on $X$, called the \termin{specialization topology} or \termin{Alexandroff topology}.
  \end{defn}
  \begin{prop}
    A proset $X$ is a poset if and only if its Alexandroff topology is $T_0$.
  \end{prop}


\newpage\section{Lattices}
  Recall that a meet-semilattice is a poset having all finitely meets, a join-semilattice is a poset having all finitely joins.
  It is not difficult to find that to say a poset is a meet-semilattice is the same to say its opposite is a join-semilattice.
  Thus, a meet-semilattice is often called a \termin{semilattice} for brevity.

  A \termin{semilattice homomorphism} $f$ from a semilattice $A$ to a semilattice $B$ is a function from $A$ to $B$ that preserves finitely meets, view $f$ as a functor, this equals to say that $f$ is continuous.

  Semilattices and semilattice homomorphisms form a concrete category $\SemiLat$.

  \begin{defn}
  A poset that is both a meet-semilattice and join-semilattice is called a \termin{lattice}.
  \end{defn}
  A \termin{lattice homomorphism} $f$ from a lattice $A$ to a lattice $B$ is a function from $A$ to $B$ that preserves finitely meets and joins, view $f$ as a functor, this equals to say that $f$ is continuous and cocontinuous.

  Lattices and lattice homomorphisms form a concrete category $\Lat$.

  \begin{defn}
  A poset that is both a inflattice and suplattice is called a \termin{complete lattice}.
  \end{defn}
  \begin{rem}
    By the adjoint functor theorem for posets, having either all infima or all suprema is sufficient for the other. However, a inflattice morphism may preserve only infima, while dually an suplattice morphism may preserve only suprema.
  \end{rem}
  There are concepts of inflattice homomorphisms, suplattice homomorphisms and complete homomorphisms, which can be defined analogously.

  Complete lattices and complete lattice homomorphisms form a concrete category $\CompLat$.



  \chapter{Rings and Modules}
\minitoc
\newpage
\section{Basic definitions}
\subsection{Rings and commutative rings}
  A \termin{ring} is \emph{a monoid internal to abelian groups}. That means, a ring is an abelian group (whose operation is called \emph{addition}) equipped with a monoid structure (whose operation is called \emph{multiplication}).
  \begin{rem}
    Usually, we will use $+$ to denote the addition in a ring and $0$ to denote the \emph{additive zero}. Also, we will use juxtaposition of ring elements to denote the multiplication and $1$ to denote the \emph{multiplicative identity}.
  \end{rem}
  The homomorphisms of rings are the homomorphisms of abelian groups which preserve the monoid structures on them. The category of rings and their homomorphisms is denoted by $\Ring$.
  \begin{rem}
    For some reasons, many authors follow an alternative convention in which a ring is not required to have a multiplicative identity and therefore is merely \emph{a semigroup internal to abelian groups}. Such a structure will be called a \termin{rng} in this book.
  \end{rem}

  A ring is said to be \termin[commutative]{commutative ring} if its monoid structure is abelian. The category of commutative rings and their homomorphisms is denoted by $\CRing$.

  A subset of a ring is said to be a \termin{subring} if the inclusion map is a ring homomorphism.

\subsection{Modules and ideals}
  An abelian group $M$ equipped with a left monoid action of a ring $R$ on $M$ is called a \emph{left module over $R$}, or \emph{left} \termin[$R-$module]{module}.
  \begin{rem}
    Here the ring $R$ is called the \termin{ground ring}, and the action is called \emph{scalar multiplication} or simply \emph{multiplication} and is usually written by juxtaposition.
  \end{rem}
  The homomorphisms of left $R-$modules (usually called \emph{left} \termin[$R-$linear maps]{linear map}) are those homomorphisms of abelian groups preserving the actions of the ground ring $R$. The category of left $R-$modules and their homomorphisms is denoted by $_R\Mod$.

  Likewise, there are \emph{right $R-$modules} and \emph{right $R-$linear maps}, and the category of them is denoted by $\Mod_R$.
  \begin{rem}
    A right $R-$module can be viewed as a left module over the \termin{opposite ring} of $R$. For a given ring $R$, its opposite (denoted by $R^{\op}$) is another ring with the same elements and addition operation, but with the multiplication performed in the reverse order.

    When $R$ is commutative, then its opposite ring is isomorphic to itself. In this case, a left $R-$module is then isomorphic to a right $R-$module and will be simply called an \emph{$R-$module}. The category of $R-$modules and $R-$linear maps is denoted by $R\Mod$.
  \end{rem}

  An \termin[$R-S-$bimodule]{bimodule} is a module that is both a left $R-$module and a right $S-$module such that the two multiplications are compatible. An $R-R-$bimodule is also known as an \emph{$R-$bimodule}. The homomorphisms of bimodules are defined in the natural way. The category of $R-S-$bimodules and their homomorphisms is denoted by $_R\Mod_S$.

  Let $M$ be a left $R-$module and $N$ is a subgroup of $M$. Then $N$ is called a \termin{submodule} (or \emph{$R-$submodule}, to be more explicit) if it is invariant under the action of $R$. For a subset $E$ of $M$, the smallest submodule of $M$ containing $E$ is called the submodule \termin[generated]{generated submodule} by $E$ and is denoted by $RE$. For an element $x$ of $M$, we usually denote $R\{x\}$ simply by $Rx$, such kind of submodules are usually called \termin[principal]{principal submodule}.

  Likewise, one can define the submodule of a right module or bimodule. Note that the notations $RE$ introduced above should be written as $ER$ if we are talking about a right module.

  The ground ring $R$ itself is a left $R-$module under the left multiplication. A submodule of this module is called a \termin{left ideal} of $R$. \termin[Right ideals]{right ideal} and \termin[two-sided ideals]{two-sided ideal} are defined likewise. Note that a two-sided ideal is usually called an \termin{ideal} for short since it is more frequently used.

  For a subset $E$ of $R$, one can also define the left/right/two-sided ideal \termin[generated]{generated ideal} by $E$. We usually denote the two-sided ideal generated by $E$ as $\<E\>$.
  If $R$ is commutative, then both left ideals and right ideals are two-sided ideals, in this case, we also denote the ideal generated by $E$ as $RE$.
  Moreover, the left/right/two-sided ideals generated by one element is called \termin[principal]{principal submodule}.

  Given a ring $R$ and a two-sided ideal $I$ of $R$, the \termin{quotient ring} $R/I$ is the set of cosets of $I$ in $R$ equipped with a ring structure such that the canonical map $R\twoheadrightarrow R/I$ is a ring homomorphism.

\newpage
\section{Category of rings}
  The category of rings and their homomorphisms is denoted by $\Ring$.

\subsection{Completeness}

  If $0=1$ in a ring $R$, then $R$ has only one element, and is called the \termin{zero ring} (denoted by $\mathbf{0}$).
  The zero ring is obviously the terminal object in $\Ring$ but is NOT the initial object (in fact, it is $\mathbb{Z}$). Thus $\Ring$ can NOT be \emph{pre-additive}, a fortiori \emph{abelian}.

  Given a family of rings $\{R_i\}_{i\in I}$, the the cartesian product $\prod_{i\in I}R_i$ of them can be turned into a ring by defining the operations coordinate-wise. The resulting ring is called a \termin[direct product]{direct product of rings} of the rings $R_i$. The direct product of rings are just the \emph{product} in $\Ring$. That means, for every $i$ in $I$ we have a surjective ring homomorphism $p_i\colon \prod_{i\in I}R_i\to R_i$ which projects the product on the $i$th coordinate. The product $\prod_{i\in I}R_i$, together with the \emph{projections} $p_i$, has the following universal property:
  \begin{quote}
    if $T$ is any ring and $f_i\colon T\to R_i$ is a ring homomorphism for every $i$ in $I$, then there exists precisely one ring homomorphism $f\colon T\to \prod_{i\in I}R_i$ such that $p_i\circ f = f_i$ for every $i$ in  $I$.
  \end{quote}
  Or, represented this universal property by $\Hom$,
  \begin{equation*}
    \Hom(T,\prod_{i\in I}R_i) \cong \prod_{i\in I}\Hom(T,R_i).
  \end{equation*}
  \begin{rem}
    Some authors call the direct product of finitely many rings as the direct sum, but this should be avoided since it is NOT the \emph{coproduct} in $\Ring$. Indeed, the inclusion maps $R_i\hookrightarrow R$ are NOT ring homomorphisms since they do not map $1$ to the identity in $R$.
    The true coproducts are something like free products of groups, we will not discuss the details here.
  \end{rem}

  Given two parallel morphisms $f,g\colon R\tto S$, their \termin{equalizer} $\Ker(f,g)$ is the subring of $R$ consists of those elements sharing same images in $S$. That means, this subring $\Ker(f,g)$ together with the inclusion $\imath\colon \Ker(f,g) \hookrightarrow R$, has the following universal property:
  \begin{quote}
    if $T$ is any ring and $h\colon T\to R$ is a ring homomorphism such that $f\circ h = g\circ h$, then there exists precisely one ring homomorphism $u\colon T\to \Ker(f,g)$ such that $\imath\circ u = h$.
  \end{quote}
  Or, represented this universal property by $\Hom$,
  \begin{equation*}
    \Hom(T,\Ker(f,g)) \cong \left\{ h\in\Hom(T,R) \middle|\, f\circ h = g\circ h \right\}.
  \end{equation*}

  The \termin{coequalizer} of $f,g$ is the quotient ring $\Coker(f,g)=S/I$ of $S$, where $I$ is the ideal generated by the image of the abelian group homomorphism $f-g$. That means, this quotient ring $S/I$, together with the canonical map $p\colon S\hookrightarrow S/I$, has the following universal property:
  \begin{quote}
    if $T$ is any ring and $h\colon S\to T$ is a ring homomorphism such that $h\circ f = h\circ g$, then there exists precisely one ring homomorphism $u\colon S/I\to T$ such that $u\circ p = h$.
  \end{quote}
  Or, represented this universal property by $\Hom$,
  \begin{equation*}
    \Hom(\Coker(f,g),T) \cong \left\{ h\in\Hom(S,T) \middle|\, h\circ f = h\circ g \right\}.
  \end{equation*}

  Therefore, $\Ring$ has products and equalizers thus, by the general theorem in category theory, it is \emph{complete}. That means, every categorical limit exists in $\Ring$. Since we can also define coproducts and coequalizers, $\Ring$ is actually also \emph{cocomplete}, which means every categorical colimit exists in $\Ring$.

  As examples, we introduce a kind of important limits we will use later.
    \begin{displaymath}
      \xymatrix{
         R\ar[r]^{f'}\ar[d]_{g'}&B\ar[d]^{g}\\
         A\ar[r]^{f}&C
      }
    \end{displaymath}

  Let's look at the above square, where $A,B,C$ are given objects and $f,g$ are given morphisms. Among all objects $R$ equipped with two morphisms $f',g'$ making the square commutative, there exists a unique terminal one, it is the \termin{fibre product} of $A$ and $B$ over $C$, and is denoted by $A\times_CB$. In layman's word, $A\times_CB$, together with the two morphisms $f',g'$, has the following universal property:
  \begin{quote}
    if $T$ is an object and $f''\colon T\to A$ and $g''\colon T\to B$ are two morphisms such that $f\circ g'' = g\circ f''$, then there exists precisely one morphism $u\colon T\to A\times_CB$ such that $f'\circ u = f''$ and $g'\circ u =g''$.
  \end{quote}
  Or, represented this universal property by $\Hom$,
  \begin{equation*}
    \Hom(T,A\times_CB) \cong \left\{ (f'',g'')\in\Hom(T,B)\times\Hom(T,A) \middle|\, f\circ g'' = g\circ f'' \right\}.
  \end{equation*}

  The fibre product $A\times_CB$ in $\Ring$ can be defined as the subring of $A\times B$ containing those elements $(a,b)$ satisfying $f(a)=g(b)$, $f',g'$ are just the two projections of $A\times B$ restricted in $A\times_CB$.
  \begin{rem}
    The previous square is called a \termin{cartesian diagram} when the triple $(R,f',g')$ is taken to make be the fibre product. In this case, we say $g'$ is the \termin{pullback} of $g$ through $f$ and $f'$ is the \termin{pullback} of $f$ through $g$.
  \end{rem}

  Dually, there are \emph{cofibre products}, \emph{cocartesian diagrams} and \emph{pushouts}. Although they are not easy to describe in detail, their existence in $\Ring$ just follows the cocompleteness.


\subsection{Monomorphisms and kernel pairs}
  The monomorphisms in $\Ring$ are precisely the injective homomorphisms. One may remember that a homomorphism $f$ is injective if and only if its \emph{kernel} $\ker{f}$ is trivial, i.e. equal to $0$.

  However, nonzero kernel is not a ring but an ideal, thus does not exist in $\Ring$. To characterize this property in a category theoretic way, we introduce a special kind of fibre products as the analogy of kernels in $\Ring$

  In a general category, the \termin{kernel pair} of a morphism $f$ is the fibre product of $f$ with itself. In layman's word, the kernel pair $A\times_BA$, together with the two morphisms $g,h\colon A\times_BA\tto A$, has the following universal property:
  \begin{quote}
    if $T$ is an object and $g',h'\colon T\tto B$ is a pair of two parallel morphisms such that $f\circ g' = f\circ h'$, then there exists precisely one morphism $u\colon T\to A\times_BA$ such that $g\circ u = g'$ and $h\circ u = h'$.
  \end{quote}
  \begin{rem}
    It is obvious that the kernel pair morphisms must be \emph{split epi}, which means there exists a morphism $u$ such that $g\circ u = h\circ u = 1_A$
  \end{rem}

  In a category with zero morphisms and kernels, such as the category $\Ab$ of abelian groups, every kernel pair $g,h$ admits a kernel morphism $g-h$ while every kernel morphism $k$ admits a kernel pair $k,0$.

  In $\Ring$, the kernel pair of a homomorphism $f\colon A\to B$ is taken to be
  \begin{equation*}
    A\times_BA := \left\{(a,a')\in A\times A \middle|\, f(a)=f(a')\right\}
  \end{equation*}
  with the two obvious projections $p_1,p_2$. The different of the two projections, i.e. $p_1-p_2$, is not a ring homomorphism. However, the set-theoretic image of this map is nothing but the kernel of $f$.

  We have a general proposition about the relationship between kernel pairs and monomorphisms.
  \begin{prop}\label{prop:kernel pair and monic}
    A morphism $f\colon A\to B$ is a monomorphism if and only if its kernel pair is trivial, i.e. isomorphic to $A$ together with two identity morphisms.
  \end{prop}
  \begin{proof}
    If $f\colon A\to B$ is a monomorphism, then for any parallel morphisms $g,h$, $f\circ g = f\circ h$ implies $g=h$, thus $A$ together with two identity morphisms satisfies the universal property of kernel pair. Conversely, If the kernel pair of $f$ is trivial, then for any parallel morphisms $g,h$ such that $f\circ g = f\circ h$, there exists precisely one morphism $u\colon T\to A\times_BA$ such that $1_A\circ u = g$ and $1_A\circ u = h$, thus $g=h$ and $f$ is a monomorpism.
  \end{proof}

  A mentionable property of kernel pair is:
  \begin{prop}\label{prop:kernel pair}
    If a kernel pair has a coequalizer, then it is a kernel pair of its coequalizer. Conversely, if a coequalizer has a kernel pair, then it is a coequalizer of its kernel pair.
  \end{prop}
  \begin{proof}
    Consider the following diagram
    \begin{displaymath}
      \xymatrix{
         &X\ar@<-0.5ex>[d]_{x}\ar@<0.5ex>[d]^{y}\ar[dl]_{z}&\\
         P\ar@<0.5ex>[r]^{\alpha}\ar@<-0.5ex>[r]_{\beta}&A\ar[r]^{f}\ar[d]_{g}&B\ar[dl]^{h}\\
         &C&
      }
    \end{displaymath}
    where, $\alpha,\beta$ is the kernel pair of $g$ and $f=\Coker(\alpha,\beta)$. Since $g\circ\alpha=g\circ\beta$, we get a unique factor $h$ of $g$ through $f=\Coker(\alpha,\beta)$.
    We need to show that $\alpha,\beta$ is the kernel pair of $f$.

    To do this, consider two parallel morphisms $x,y$ such that $f\circ x=f\circ y$. Then $g\circ x=h\circ f\circ x=h\circ f\circ y=g\circ y$. Thus there exists a unique morphism $z$ such that $\alpha\circ z=x, \beta\circ z=y$.

    Consider the following diagram
    \begin{displaymath}
      \xymatrix{
         &X\ar@<-0.5ex>[d]_{x}\ar@<0.5ex>[d]^{y}\ar[dl]_{z}&\\
         P\ar@<0.5ex>[r]^{\alpha}\ar@<-0.5ex>[r]_{\beta}&A\ar[r]^{f}\ar[d]_{g}&B\ar[dl]^{h}\\
         &C&
      }
    \end{displaymath}
    where, $f=\Coker(x,y)$ and $\alpha,\beta$ is the kernel pair of $f$. Hence there exists a unique morphism $z$ such that $\alpha\circ z=x, \beta\circ z=y$.
    We need to show that $f=\Coker(\alpha,\beta)$.

    To do this, consider an arbitrary morphism $g$ such that $g\circ\alpha=g\circ\beta$. Then $g\circ x=g\circ\alpha\circ z=g\circ\beta\circ z=g\circ y$. Thus we get a unique factor $h$ of $g$ through $f=\Coker(x,y)$.
  \end{proof}

\subsection{Regular images}
  The coequalizer of the kernel pair of a morphism $f$ is called the \termin{regular coimage} of $f$ and is denoted by $\Coim{f}$. In $\Ring$, it is not difficult to see that the regular coimage of a homomorphism $f\colon A\to B$ can be taken to be the quotient ring $A/\ker{f}$.

  Dually, we have \termin{cokernel pairs} for which the dual version of Proposition \ref{prop:kernel pair} and \ref{prop:kernel pair and monic} hold (Note that a cokernel pair is trivial means it is isomorphic to the codomain together with two identity morphisms). The equalizer of the cokernel pair of a morphism $f$ is called the \termin{regular image} of $f$ and is denoted by $\Image{f}$.

  One can see that, in a \emph{pre-abelian category} such as the category $\Ab$ of abelian groups, the regular coimage is just the cokernel of the kernel while the regular image is just the kernel of the cokernel.

  We know that in an abelian category, the regular coimage and regular image coincide and provide an \emph{Epi-mono factorization} through it. That means every morphism in an abelian category can be factored as an epimorphism followed by a monomorphism. Morevoer, such a factorization is unique up to unique isomorphism.

  In general, the regular image and regular coimage of a morpdism admit a ternary factorization from the universal properties of them.
  \begin{thm}[Ternary factorization]\label{thm:ternary factorization}
    If a morphism $f\colon A\to B$ admits a regular image and a regular coimage, then there exists a unique morphism
    \begin{equation*}
      u\colon\Coim{f}\to\Image{f}
    \end{equation*}
    such that the morphism $f$ itself can be factored into the following three morphisms:
    \begin{equation*}
      A \To \Coim{f} \markar{u} \Image{f} \To B.
    \end{equation*}

    Moreover, when $u$ is an isomorphism, the factorization become an Epi-mono factorization which is unique up to unique isomorphism.
  \end{thm}
  \begin{proof}
    By the universal property of $\Coim{f}$, there exists a unique morphism $\Coim{f}\to B$ such that $f$ can be factored as $A\to\Coim{f}\to B$. By the universal property of $\Image{f}$, there exists a unique morphism $\Coim{f}\to \Image{f}$ such that $\Coim{f}\to B$ can be factored as $\Coim{f}\to\Image{f}\to B$. Then this morphism is the desired unique morphism $u$.

    If $u$ is an isomorphism, let's denote the above factorization by $f = i\circ p$. Consider another factorization of $f = m\circ e$ where $e$ is an epimorphism and $m$ is a monomorphism. Then by the universal property of $\Coim{f}$ and $\Image{f}$, there exist a unique morphism $g$ such that $e = g\circ p$ and a unique morphism $h$ such that $m = i\circ h$. Therefore
    \begin{equation*}
      i\circ h\circ g\circ p = m\circ e = f = i\circ p.
    \end{equation*}
    Thus $h\circ g = 1$.
    Moreover,
    \begin{equation*}
      m\circ g\circ h\circ e = i\circ h\circ g\circ h\circ g\circ p = i\circ p = f = m\circ e.
    \end{equation*}
    Thus $g\circ h = 1$. Therefore the factorization $f = m\circ e$ is isomorphic to $f = i\circ p$ through the unique isomorphism $h$.
  \end{proof}
  If this $u$ is an isomorphism, then we say that $f$ is a \termin{strict morphism}.  For example, in an abelian category, every morphism is strict.

  In $\Ring$, the regular image of a homomorphism $f\colon A\to B$ can be taken to be the \emph{joint equalizer} of all pairs of homomorphisms $g,h\colon B\tto C$ satisfying $g\circ f=h\circ f$, i.e. the following subring of $B$:
  \begin{equation*}
    \left\{b\in B \middle|\, \parbox{0.6\textwidth}{for any pair of homomorphisms $g,h\colon B\tto C$ satisfying $g\circ f=h\circ f$, we have $g(b)=h(b)$} \right\}
  \end{equation*}

  However, this is usually not the set-theoretic image of $f$. For instance, the the set-theoretic image of the inclusion $\mathbb{Z}\hookrightarrow\mathbb{Q}$ is $\mathbb{Z}$ while the regular image is the whole $\mathbb{Q}$.

  The problem comes from the trouble that although every surjective ring homomorphisms are epimorphisms in $\Ring$, the converse is NOT true. The inclusion $\mathbb{Z}\hookrightarrow\mathbb{Q}$ is a counterexample. To see this, note that any ring homomorphism on $\mathbb{Q}$ is determined entirely by its action on $\mathbb{Z}$ thus the inclusion is a non-surjective epimorphism.

  As a result, $\mathbb{Z}\hookrightarrow\mathbb{Q}$ fails to be strict. Since it is both a monomorphism and an epimorphism, its kernel pair and cokernel pair are trivial. Then its regular coimage and regular image are isomorphic to the domain and codomain respectively thus not isomorphic to each other.


\subsection{Epimorphisms}
  To characterize the surjective homomorphisms in a categorical way, we will introduce some special kinds of epimorphisms here. We should define them in terms of category theory.

  Except the whole epimorphisms, the most familiar class is the split epimorphisms. Recall that a \termin{split epimorphism} is a morphism $f\colon A\to B$ having a \emph{section}. That means, there exists a morphism $g\colon B\to A$ such that $f\circ g = 1_B$.

  However, being split is too strong for surjective ring homomorphisms, thus in $\Ring$ the class of surjective homomorphisms are between epimorphisms and split epimorphisms. So we need to found some classes of epimorphisms between them.

\subsubsection{Extremal epimorphisms and strong epimorphisms}

  We know that, being an \termin{epi-monomorphism}, which means both an epimorphism and a monomorphism, is not sufficient to implies being an isomorphism. The inclusion $\mathbb{Z}\hookrightarrow\mathbb{Q}$ is such an example in $\Ring$.

  This leads us to define the notion of extremal epimorphisms. An epimorphism $f\colon A\to B$ is said to be an \termin{extremal epimorphism} if it cannot be factored through a nontrivial \emph{subobject} of $B$. That means, if $f=m\circ g$ with $m$ a monomorphism, then $m$ is an isomorphism.

  For extremal epimorphisms, we have the desired property:
  \begin{prop}\label{prop:extremal+mono=iso}
    A morphism which is both a monomorphism and an extremal epimorphism is an isomorphism.
  \end{prop}
  \begin{proof}
    Let $f\colon A\to B$ be both a monomorphism and an extremal epimorphism. Consider the factorization $f = f\circ 1_A$, then since $f$ is also a monomorphism, it is an isomorphism.
  \end{proof}

  Note that in the factorization $f = m\circ g$ where $f$ is an epimorphism and $m$ is a monomorphism, we can always say that $m$ is also an epimorphism and thus an epi-monomorphism. But when we assume $f$ to be extremal, then it become an isomorphism. It seems that making $f$ to be extremal will also make $m$ extremal, and this is true.
  \begin{prop}
    If a composite $g\circ f$ is an extremal epimorphism, then so is $g$.
  \end{prop}
  \begin{proof}
    Assume $g\circ f$ is an extremal epimorphism and $g = m\circ h$ where $m$ is a monomorphism, then $g\circ f = m\circ (h\circ f)$. Thus $m$ is an isomorphism as desired.
  \end{proof}
  However, the composite of two extremal epimorphisms may not be extremal again. The problem comes from that when consider a commutative diagram like below
        \begin{displaymath}
          \xymatrix{
            A\ar[r]^{f}\ar[dr]&B\ar[r]^{g}&C\\
            &D\ar[ur]_{m}&
          }
        \end{displaymath}
  where $f$ and $g$ are extremal epimorphism and $m$ is a monomorphism, we need there exists a morphism $B\to D$ to reduce the case from the factorization of $g\circ f$ to the factorization of $g$.

  This leads us to define the notion of strong epimorphisms. An epimorphism $f\colon A\to B$ is said to be an \termin{strong epimorphism} if it is \emph{left orthogonal} to any monomorphism. That means, in any commutative square
        \begin{displaymath}
          \xymatrix{
            A\ar[r]^{f}\ar[d]&B\ar[d]\\
            C\ar[r]^{m}&D
          }
        \end{displaymath}
  where $m$ is an arbitrary monomorphism, there exists a unique morphism $B\to C$ making both triangles commute:
        \begin{displaymath}
          \xymatrix{
            A\ar[r]^{f}\ar[d]&B\ar[d]\ar@{-->}[dl]\\
            C\ar[r]^{m}&D
          }
        \end{displaymath}

  For strong epimorphisms, composition is not a problem.
  \begin{prop}
    The composite of two strong epimorphisms is also a strong epimorphism. Conversely, if a composite $g\circ f$ is a strong epimorphism, then so is $g$.
  \end{prop}
  \begin{proof}
    Consider the following diagram
        \begin{displaymath}
          \xymatrix{
            A\ar[r]^{f}\ar[d]&B\ar[r]^{g}\ar@{-->}[dl]|-{h}&C\ar[d]\ar@{-->}[dll]|-{u}\\
            D\ar[rr]^{m}&&E
          }
        \end{displaymath}
    where $f,g$ are two strong epimorphisms and $m$ is an arbitrary monomorphism. In square $ABED$, since $f$ is strong epi, there exists a unique morphism $h$ making both triangles commute. In square $BCED$, since $g$ is strong epi, there exists a unique morphism $u$ making  both triangles commute. Then one can verify that $u$ is the desired unique morphism.

    Conversely, assume $g\circ f$ is a strong epimorphism and consider the following diagram
        \begin{displaymath}
          \xymatrix{
            A\ar[r]^{f}\ar@{-->}[dr]&B\ar[d]\ar[r]^{g}&C\ar[d]\ar@{-->}[dl]|-{h}\\
            &D\ar[r]^{m}&E
          }
        \end{displaymath}
    where $m$ is an arbitrary monomorphism. Then in square $ACED$, there exists a unique morphism $h$ making both triangles commute. This $h$ is the desired unique morphism in square $BCED$.
  \end{proof}

  Moreover, we have
  \begin{prop}
    Every strong epimorphism is an extremal epimorphism.
  \end{prop}
  \begin{proof}
    Consider the following commutative diagram
        \begin{displaymath}
          \xymatrix{
            A\ar[r]^{f}\ar[d]&B\ar@{=}[d]\ar@{-->}[dl]\\
            C\ar[r]^{m}&B
          }
        \end{displaymath}
    where $f$ is a strong epimorphism and $m$ is a monomorphism. Then we need to show that $m$ is an isomorphism. This is obvious since it is both a monomorphism and a split epimorphism.
  \end{proof}
  Conversely, the previous discussion shows that an extremal epimorphism may not be strong. However, we have
  \begin{prop}\label{prop:extremal=strong}
    In a category with pullbacks, every extremal epimorphism is strong.
  \end{prop}
  \begin{proof}
    Consider the following commutative diagram
        \begin{displaymath}
          \xymatrix{
            A\ar[r]^{f}\ar[d]_{g}&B\ar[d]_{h}\ar@{-->}[dl]|{u}\\
            C\ar[r]^{m}&D
          }
        \end{displaymath}
    where $f$ is an extremal epimorphism and $m$ is a monomorphism, we need to show that there exists a unique morphism $u\colon B\to C$ such that $u\circ f = g$ and $m\circ u = h$.

    Taking the pullback of $m$ and $h$, then we get the following commutative diagram
        \begin{displaymath}
          \xymatrix{
            A\ar[rr]^{f}\ar[dd]_{g}\ar@{-->}[dr]|{v}&&B\ar[dd]_{h}\ar@/^1pc/@{-->}[ddll]|{u'}\\
            &E\ar@/_/[dl]|{h'}\ar@/^/[ur]|{m'}&\\
            C\ar[rr]^{m}&&D
          }
        \end{displaymath}
    where $v$ is the unique morphism satisfying $h'\circ v=g$ and $m'\circ v=f$. Since $m'$ is a pullback of a monomorphism, it must be a monomorphism again, thus an isomorphism since $f$ is an extremal epimorphism. Then we get a morphism $u:=h'\circ{m'}^{-1}$ and
    \begin{align*}
      u\circ f &= h'\circ{m'}^{-1}\circ f = h'\circ v = g,\\
      m\circ u &= m\circ h'\circ{m'}^{-1} = h.
    \end{align*}
    Thus $u$ is the desired morphism. To see the uniqueness, assume $u'$ is another morphism such that $u'\circ f = g$ and $m\circ u' = h$. Then since $m$ is a monomorphism, $m\circ u' = h = m\circ u$ implies $u'=u$.
  \end{proof}

  Therefore, in a category with pullbacks such as $\Ring$, extremal epimorphisms and strong epimorphisms coincide. One may expect that they also coincide with surjective homomorphisms. Let's leave this thought aside and introduce another family of epimorphisms first.

\subsubsection{Regular epimorphisms}
  Recall that in the ternary factorization of a morphism $f\colon A\to B$ (cf.\ref{thm:ternary factorization}), the regular coimage morphism must be an epimorphism. But not every epimorphism can serves as a regular coimage morphism. For example, one can see that in $\Ring$, the regular coimage morphisms are quotient maps and thus surjective homomorphisms while we already know there exists epimorphisms which are not surjective.

  When a morphism can serve as a regular coimage morphism, it should be called a \termin{regular epimorphism}. However, for some reasons, this name belongs to a bit more general notion, that is a morphism which can serve as a \emph{coequalizer} of some parallel pair of morphisms. While a regular coimage morphism is called an \termin{effective epimorphism}.

  One can see that (by Proportion \ref{prop:kernel pair}), in a category with pullbacks, regular epimorphism are just effective epimorphism.

  Let's consider in such a category, among all regular epimorphisms, there are a special kind of them worth to mention, they are the \emph{universal regular epimorphisms}. A regular epimorphism is said to be \termin[universal]{universal regular epimorphism} if its every pullback is again a regular epimorphism.
  A universal regular epimorphism is also called a \termin{descent morphism}. This name comes from the descent theory in algebraic geometry and has an alternative definition (cf. \cite{Vistoli}).


  It is not difficult to see that universal regular epimorphisms are \emph{stable under pullbacks}. That means every pullback of a universal regular epimorphism is again a universal regular epimorphism. This is because that a pullback of a pullback of a morphism is again a pullback of the original morphism.

  The universal regular epimorphisms are also closed under compositions. That means a composite of two universal regular epimorphisms is again a universal regular epimorphism. To see this, we need a lemma
  \begin{lem}\label{lem:fibre morphism of descent morphism}
    Let $f\colon A\to B$ be a universal regular epimorphism and $g\colon B\to C$ be an arbitrary morphism. Then the \emph{fibre morphism}
    \begin{equation*}
      f\times_Cf\colon A\times_CA\To B\times_CB
    \end{equation*}
    uniquely exists and is an epimorphism.
  \end{lem}
  \begin{proof}
    Taking the kernel pair of $g$ we get the lower right cartesian square. The three other cartesian squares exist and the sides of the upper left square are all epimorphisms since $f$ is a universal regular epimorphism.
    \begin{displaymath}
      \xymatrix{
         A\times_CA\ar[r]\ar[d] & B\times_CA\ar[r]\ar[d] & A\ar@{->>}[d]_{f} \\
         A\times_CB\ar[r]\ar[d] & B\times_CB\ar[r]\ar[d] & B\ar[d]_{g} \\
         A\ar@{->>}[r]^{f} & B\ar[r]^{g} & C
      }
    \end{displaymath}
    Then $f\times_Cf$ is just the composite of two epimorphisms, thus again an epimorphism.
  \end{proof}
  \begin{prop}
    A composite of two universal regular epimorphisms is again a universal regular epimorphism.
  \end{prop}
  \begin{proof}
    Consider the following commutative diagram
    \begin{displaymath}
      \xymatrix{
         &A\times_BA\ar@<0.5ex>[d]^-{}\ar@<-0.5ex>[d]_-{}\ar[dl]_{h}&\\
         A\times_CA\ar@<0.5ex>[r]^-{}\ar@<-0.5ex>[r]_-{}\ar[d]_{f\times_Cf}
         &A\ar[r]^-{g\circ f}\ar[d]_-{f}&C\ar@{=}[d]\\
         B\times_CB\ar@<0.5ex>[r]^-{}\ar@<-0.5ex>[r]_-{}&B\ar[r]^-{g}&C
      }
    \end{displaymath}
    where $f\times_Cf$ is the fibre morphism and $h$ is the unique morphism making the upper left triangles commute. We only need to show that $g\circ f$ is the coequalizer of its kernel pair $A\times_CA\tto A$.

    Let $t\colon A\to T$ be an arbitrary morphism equalizing $A\times_CA\tto A$. We need to show that there exists a unique morphism $u\colon C\to T$ such that $u\circ (g\circ f) = t$.

    By composted by $h$, it equalize $A\times_BA\tto A$. Thus there exists a unique morphism $v\colon B\to T$ such that $v\circ f = t$. Then $v$ equalize the composites $A\times_CA\tto A\markar{f} B$ and thus the composites $A\times_CA\to B\times_CB\tto B$ by the lower left commutative squares. Since the fibre morphism $f\times_Cf$ is an epimorphism, we get that $v$ equalize $B\times_CB\tto B$. Thus there exists a unique morphism $u\colon C\to T$ such that $u\circ g = v$. Then this $u$ is the desired unique morphism.
  \end{proof}

  What's the relationship between regular epimorphisms and strong epimorphisms? It is easy to verify that
  \begin{prop}
    Every regular epimorphism is a strong epimorphism.
  \end{prop}
  \begin{proof}
    Consider the following commutative diagram
        \begin{displaymath}
          \xymatrix{
            A\ar[r]^{f}\ar[d]_{g}&B\ar[d]_{h}\ar@{-->}[dl]|{u}\\
            C\ar[r]^{m}&D
          }
        \end{displaymath}
    where $f$ is a regular epimorphism and $m$ is a monomorphism. $f$ is regular thus a coequalizer of some parallel morphisms, say $x,y$. Then $m\circ g = h\circ f$ equalize $x,y$. Since $m$ is a monomorphism, we get that $g$ equalize $x,y$. Then the desired unique morphism $u\colon B\to C$ comes from the universal property of $f$ as the coequalizer of $x,y$.
  \end{proof}

  Since $\Ring$ is complete, the epimorphisms in $\Ring$ can be classified into five classes:
  \begin{enumerate}
    \item Epimorphisms;
    \item Extremal/strong epimorphisms;
    \item Regular/effective epimorphisms;
    \item Descent morphisms;
    \item Split epimorphisms.
  \end{enumerate}

  We have already know the the class of surjective homomorphisms is between the whole epimorphisms and split epimorphisms. We will prove in the next subsection that the rest three classes are all the same and equivalent to surjective homomorphisms.

\subsection{Regular category and the image factorization}
  The main step in the proof of the our assertion at the end of previous subsection is to show $\Ring$ is a \emph{regular category}.

\subsubsection{Regular category}
  A \termin{regular category} is a \emph{finitely complete} category in which
  \begin{enumerate}
    \item every kernel pair has a coequalizer;
    \item every regular epimorphism is universal.
  \end{enumerate}

  A very important property of regular category is the existence and uniqueness of \emph{regularEpi-mono factorizations}.
  \begin{thm}\label{thm:regular factorization}
    In a regular category, every morphism can be factored as a regular epimorphism followed by a monomorphism. This factorization is unique up to unique isomorphism.
  \end{thm}
  \begin{proof}
    Let $f\colon A\to B$ be a morphism in a regular category. Let $g,h\colon K\tto A$ be its kernel pair and $e\colon A\to\coim{f}$ be its coimage morphism. Then there exists a unique morphism $i\colon\coim{f}\to B$ such that
    \begin{equation*}
      f = (A \markar{e} \coim{f} \markar{i} B).
    \end{equation*}
    Where $e$ is of course a regular epimorphism. We only need to show that $i$ is a monomoprhism and the uniqueness.

    Consider the kernel pair $g',h'\colon K'\tto\coim{f}$ of $i$.
    \begin{displaymath}
      \xymatrix{
         K\ar@<0.5ex>[r]^-{g}\ar@<-0.5ex>[r]_-{h}\ar@{-->}[d]_{e\times_Be}&A\ar[r]^{f}\ar[d]_{e}&B\\
         K'\ar@<0.5ex>[r]^-{g'}\ar@<-0.5ex>[r]_-{h'}&\coim{f}\ar[ru]_{i}&
      }
    \end{displaymath}
    Then by Lemma \ref{lem:fibre morphism of descent morphism}, the fibre morphism $e\times_Be\colon K\to K'$ exists and is an epimorphism. Then
    \begin{equation*}
      g'\circ (e\times_Be) = e\circ g = e\circ h = h'\circ (e\times_Be).
    \end{equation*}
    Thus $g'=h'$ and $i$ is monic.

    To see the uniqueness, assume that there are another factorization
    \begin{equation*}
      f = (A \markar{e'} E \markar{i'} B),
    \end{equation*}
    where $e'$ is a regular epimorphism and $i'$ is a monomorphism. Then
    \begin{equation*}
      i'\circ e'\circ g = f\circ g = f\circ h = i'\circ e'\circ h,
    \end{equation*}
    and thus $e'\circ g = e'\circ h$. By the universal property of coequalizer, there exists a unique morphism $u\colon\coim{f}\to E$ such that $u\circ e = e'$. Since both $e$ and $e'$ are extremal epimorphisms, so is $u$. Note that
    \begin{equation*}
      i'\circ u\circ e = i'\circ e' = i\circ e,
    \end{equation*}
    thus $i'\circ u = i$. Then $u$ is also monic since both $i$ and $i'$ are. Therefore, by Proposition \ref{prop:extremal+mono=iso}, $u$ is an isomorphism as desired.
  \end{proof}

  The axioms of regular category requires every regular morphism being universal. Indeed, we have more:
  \begin{cor}\label{cor:regularCat}
    In a regular category, every extremal epimorphism is a universal regular epimorphism.
  \end{cor}
  \begin{proof}
    Let $f\colon A\to B$ be an extremal epimorphism in a regular category. By Theorem \ref{thm:regular factorization}, it can be written as $f=m\circ e$ where $e\colon A\to\coim{f}$ is the coimage morphism of $f$ and $m$ is a monomorphism. Since $f$ is an extremal epimorphism, $m$ must be an isomorphism. Thus $f$ is a regular epimorphism hence universal.
  \end{proof}

\subsubsection{Image factorizations}
  Know, we come to the main theorem of this subsection
  \begin{thm}\label{thm:RingEpi}
    Let $f$ be a ring homomorphism in $\Ring$, the followings are equivalent:
    \begin{enumerate}
      \item $f$ is surjective;
      \item $f$ is an extremal epimorphism;
      \item $f$ is a strong epimorphism;
      \item $f$ is a regular epimorphism;
      \item $f$ is an effective epimorphism;
      \item $f$ is a universal regular epimorphism.
    \end{enumerate}

    Moreover, $\Ring$ is a regular category and thus every homomorphism can be factored as a regular epimorphism followed by a monomorphism. This factorization is unique up to unique isomorphism..
  \end{thm}
  \begin{proof}
    We only need to show that 1) every regular epimorphism in $\Ring$ is surjective, 2) every surjective homomorphism is a regular epimorphism and 3) every surjective homomorphism is stable under pullback.

    1) is obvious since a coequalizer is the canonical map to a quotient ring by the discussion in previous subsection.

    2) is also not difficult since one can verify that every surjective homomorphism is the regular coimage of itself.

    To see 3), one only need to notice that in a cartesian square
    \begin{displaymath}
      \xymatrix{
         A\times_CB\ar[r]\ar[d]&B\ar[d]\\
         A\ar[r]&C
      }
    \end{displaymath}
    whenever two elements in $A$ and $B$ respectively share the same image in $C$, then they have the same preimage in $A\times_CB$.
  \end{proof}

  From theorem \ref{thm:RingEpi}, one can see that, for any ring homomorphism $f\colon A\to B$, the regular coimage is the smallest subring of $B$ through which $f$ factors. In other words, if we write $f$ as $i\circ p$ where $p\colon A\to\Coim{f}$ is the regular coimage morphism, then $i\colon\Coim{f}\to B$ is a monomorphism having the following universal property:
  \begin{enumerate}
    \item There exists a unique homomorphism $g$ such that $i\circ g = f$.
    \item If $m\colon C\to B$ is a monomorphism and $h\colon A\to C$ is a homomorphism such that $m\circ h = f$, then there exists a unique homomorphism $u\colon \im{f}\to C$ such that $m\circ u = i$.
  \end{enumerate}
  In general, a \emph{subobject} of the \emph{codomain} of a morphism having the above universal property is called the \termin{image} of $f$ and is denoted by $\im{f}$.
  One can verify that in $\Ring$, this concept coincides with the \emph{set-theoretic image}.

  Thus we conclude:
  \begin{cor}[First isomorphism theorem]
    In $\Ring$, for any homomorphism $f$, we have $\im{f}=\Coim{f}$. Moreover, It can be uniquely factored as a regular epimorphism followed by a monomorphism through the image $\im{f}$.
  \end{cor}
  For this reason, the regularEpi-mono factorizations in $\Ring$ are usually cited as \termin[image factorizations]{image factorization}.

\subsection{Characteristic}
  The unique morphism from a ring to the terminal object $\mathbf{0}$ is surjective thus an epimorphism. However, the unique morphism from the initial object $\mathbb{Z}$ to some ring $R$ may not be a monomorphism in general. If it is, we will say $R$ has \termin{characteristic} $0$.

  In general, the morphism $\mathbb{Z}\to R$ can be uniquely factored into a regular epimorphism and a monomorphism, say $\mathbb{Z}\twoheadrightarrow K\hookrightarrow R$.  Note that, the ring $K$ is actually not unique, but unique up to unique isomorphism. As a subring of $R$, it is the smallest one and will be called the \termin{characteristic subring} of $R$, it is the \emph{category theoretical} definition of the characteristic and is denoted by $\Char{R}$.  Moreover, $K$ is isomorphic to a quotient ring of $\mathbb{Z}$, which is $\mathbb{Z}/n\mathbb{Z}$ where $n$ is a natural number. This $n$ is called the \termin{characteristic} of $R$, and denoted by $\Char{R}$. (There is no ambiguities on using same notation to denoted the characteristic subring and the characteristic, just as we use $0$ to denote both the zero ring and zero element.)

  Here we list some basic properties of characteristic:
  \begin{prop}
    Let $f\colon R\to S$ be a ring homomorphism, then the characteristic of $S$ divides the characteristic of $R$. If $f$ is a monomorphism, then those two characteristics are equal.
  \end{prop}
  \begin{proof}
    By the image factorization, the homomorphism $f\colon R\to S$ induces the following commutative diagram
        \begin{displaymath}
          \xymatrix{
            \mathbb{Z}\ar@{->>}[r]\ar@{->>}[d]&\Char{R}\ar@{ (->}[d]\ar@{-->}[dl]|{u}\\
            \Char{S}\ar@{ (->}[r]&S
          }
        \end{displaymath}

    Then since the upper and left morphisms are strong, so there exists a unique morphism $u\colon\Char{R}\to\Char{S}$ making the triangles commute and moreover, it is also a strong epimorphism.

    If $f\colon R\to S$ is a monomorphism, then the right morphism in the above diagram is a composite of two monomorphisms thus also a monomorphism. Then $u$ is a monomorphism and thus an isomorphism.

    We know that a surjective homomorphism of quotients $\mathbb{Z}/n\mathbb{Z}\twoheadrightarrow\mathbb{Z}/m\mathbb{Z}$ of $\mathbb{Z}$ is corresponding to the division $m\mid n$ of integers. From this, we get the desired conclusions.
  \end{proof}
  Therefore, the chains of divisions of integers can be viewed as a sketch of the chains of  morphisms in $\Ring$. Thus, we can classified rings by characteristics and study rings sharing same characteristic while between them there is nothing but divisions of integers.

  \begin{prop}
    Let $R$ be a domain, then its characteristic should be either $0$ or prime.
  \end{prop}
\textbf{【?】}

\newpage
\section{Category of commutative rings}
  The category of commutative rings and their homomorphisms is denoted by $\CRing$. It is a full subcategory of $\Ring$ since the homomorphisms of commutative rings are just the ring homomorphisms.

  Since both $\mathbb{Z}$ and the zero ring $\mathbf{0}$ is commutative, they should be the initial object and terminal object in $\CRing$ respectively. Thus $\CRing$ can NOT be \emph{pre-additive}, a fortiori \emph{abelian}.

  The embedding $\CRing\hookrightarrow\Ring$ preserves \emph{products}, \emph{equalizers} and \emph{coequalizers}. That means those limits and colimits in $\CRing$ are the same as in $\Ring$. In other words, the direct product of a family of commutative rings is commutative, the equalizer and coequalizer of a homomorphism in $\CRing$ is also commutative. However, The embedding does NOT preserve \emph{coproducts}.

  Leaving the trouble of coproducts aside, we have that $\CRing$ is complete and we have kernel pairs and regular coimages as defined in $\Ring$.
  Moreover, the monomorphisms and all kinds of epimorphisms in $\CRing$ coincide in $\Ring$, thus the whole discussion about epimorphisms and image factorizations still hold in $\CRing$.


%\newpage
%\section{Advanced}
%\subsection{Algebras over a ring}
%  NEED RMod being abelian.
%\subsection{Domains and fields}
%  In a given ring, except $0$ and $1$, there are other special elements should be noticed. A \termin{unit} is an element admitting an inverse. A \termin{zero divisor} is an element that divide $0$.













%\part{Group Theory}
%Lattice and monoids

%
%\part{Commutative Objects}
%\chapter{Abelian Groups}
%\chapter{Abelian Category}
%\chapter{Commutative Algebra}
%\chapter{Algebraic Number Theory}
%\chapter{Algebraic Geometry}
%\chapter{Arithmetic Geometry}
%
%\part{Associative Objects}
%\chapter{Groups in Category}
%
%\chapter{Associative Algebra}
%
%\chapter{Field Extension}
%
%
%\part{Galois Objects}
%\chapter{Galois Theory}

%\part{Lie Objects}

%\part{Homology Objects}

%\part{Weird Objects}

%\part{Galois Theory}
%\chapter{Algebraic Extensions}
\section{Finite and Algebraic Extensions}
\begin{exam}[Counterexample]
  Let $\alpha$ be an algebraic element over $K$, $L$ is an extension of $K$, then $[L(\alpha):L]$ integer divide $[K(\alpha):K]$.

  $\alpha=2^{\frac{1}{3}}(-\frac{1}{2}+i\frac{\sqrt 3}{2}), K=Q, L=Q(2^{\frac{1}{3}})$.
\end{exam}

%Algebraic Extensions
%\chapter{Galois Theory}
\cite{morandi1996field}
\section{Galois Extensions}
  \begin{defn}
    Let $K$ be a field and let $G$ be a group of automorphisms of $K$. The subset of $K$ consisting of all elements which is fixed under all $\sigma\in G$ is a field. It is called the \termin{fixed field} of $G$, and denoted by $K^G$.
  \end{defn}

  \begin{defn}
    An algebraic extension $K$ of a field $k$ is called \termin[Galois]{Galois Extension} if it is both normal and separable.
  \end{defn}

  \begin{defn}
    For an extension $K/k$, the group of automorphisms of $K$ over $k$ is denoted by $\Aut(K/k)$. When the extension is Galois, $\Aut(K/k)$ is called the \termin{Galois group} of $K$ over $k$, and is denoted by $\Gal(K/k)$ or $G(K/k)$.
  \end{defn}

  The main theorem in this section is
  \begin{thm}[Galois Connection]
    Let $K$ be a Galois extension of $k$, with Galois group $G$. Denote the set of intermediate field of $K/k$ by $\Int(K/k)$, and the set of subgroups of $G$ by $\Sub(G)$. Then there exists a bijection between them:
    \mapdes{\Sub(G)}{\Int(K/k)}{H}{E=K^H}

    The field $E$ is Galois if and only if $H$ is normal in $G$. If that is the case, then the map $\sigma\mapsto\local{\sigma}{E}$ induces an isomorphism of $G/H$ onto the Galois group of $E$ over $k$.
  \end{thm}

  Lang gives the proofs step by step.

  \begin{thm}
    Let $K$ be a Galois extension of $k$, with Galois group $G$, then $k=K^G$. Denote the set of intermediate field of $K/k$ by $\Int(K/k)$, and the set of subgroups of $G$ by $\Sub(G)$. If $F\in\Int(K/k)$, then $K$ is Galois over $F$. The map
    \mapdes{\Int(K/k)}{\Sub(G)}{F}{G(K/F)}
    is injective.
  \end{thm}

  \begin{defn}
    We say a subgroup $H$ of $G$ \emph{\red belongs} to an intermediate field $F$ if $H=G(K/F)$.
  \end{defn}

  \begin{cor}
    Let $K/k$ be Galois with group $G$. Let $F,F'$ be two intermediate fields, and let $H,H'$ be the subgroups of $G$ belonging to $F,F'$ respectively. Then
    \begin{enumerate}[a)]
      \item  $H\cap H'$ belongs to $FF'$;
      \item  The fixed field of the smallest subgroup of $G$ containing $H,H'$ is $F\cap F'$;
      \item  $F\subset F'$ if and only if $H'\subset H$.
    \end{enumerate}
  \end{cor}
  Such results can be represented by the corresponding between the following two \emph{Hasse diagrams}
                 \begin{displaymath}
                   \xymatrix@!0{
                      & <H\cup H'> &\ar[ddrr]&&& FF' & \\
                      H\ar@{-}[ur]\ar@{-}[dr] && H'\ar@{-}[ul]\ar@{-}[dl] && F\ar@{-}[ur]\ar@{-}[dr] && F'\ar@{-}[ul]\ar@{-}[dl] \\
                      & H\cap H' &\ar[uurr]&&& F\cap F' &            }
                 \end{displaymath}

  \begin{cor}
    Let $E$ be a finite separable extension of a field $k$. Let $K$ be the smallest normal extension of $k$ containing $E$. Then $K$ is finite Galois over $k$, and $\Int(E/k)$ is finite.
  \end{cor}

  \begin{lem}
    Let $E$ be an algebraic separable extension of $k$. Assume that there is an integer $n\geqslant1$ such that every element of $E$ is of degree $\leqslant n$ over $k$. Then $[E:k]\leqslant n$.
  \end{lem}

  Now, we have $K^{G(K/F)}=F$, but how about $G(K/K^H)$.

  \begin{thm}[Artin]
    Let $K$ be a field and let $G$ be a finite group of automorphisms of $K$, of order $n$. Let $k=K^G$ be the fixed field. Then $K$ is a finite Galois extension of $k$, and its Galois group is $G$. We have $[K:k]=n$.
  \end{thm}

  \begin{cor}
    Let $K$ be a finite Galois extension of $k$ and let $G$ be its Galois group. Then every subgroup of $G$ belongs to some $F\in\Int(K/k)$.
  \end{cor}
  \begin{warn}
    This statement is not true when $K$ is an infinite Galois extension of $k$.
  \end{warn}

  \begin{lem}
    Let $K$ be a Galois extension of $k$. Let
    \begin{equation*}
       \lambda\colon K\To \lambda K
    \end{equation*}
    be an isomorphism, then
    \begin{equation*}
      G(\lambda K/\lambda k)^{\lambda}=G(K/k)
    \end{equation*}
    i.e.
    \begin{equation*}
      G(\lambda K/\lambda k)=\lambda G(K/k) \lambda^{-1}
    \end{equation*}
  \end{lem}

  \begin{thm}
    Let $K$ be a Galois extension of $k$ with group $G$. Let $F\in\Int(K/k)$, and let $H=G(K/F)$. Then $F$ is normal over $k$ if and only if $H$ is normal in $G$. If that is the case, then the map $\sigma\mapsto\local{\sigma}{E}$ induces a homomorphism of $G$ onto the Galois group of $F$ over $k$, whose kernel is $H$.
  \end{thm}

  \begin{defn}
    A Galois extension is said to be \termin[abelian]{abelian extension} (resp. \termin[cyclic]{cyclic extension}) if its Galois group is \emph{abelian} (resp. \emph{cyclic}).
  \end{defn}

  \begin{cor}
    Let $K/k$ be abelian (resp. cyclic). If $F\in\Int(K/k)$, then $F/k$ is also abelian (resp. cyclic).
  \end{cor}

  \begin{thm}
    Let $K/k$ be Galois, $F/k$ be an arbitrary extension, then $KF/F, K/(K\cap F)$ are Galois. Moreover, we have an isomorphism
    \isodes{\Gal(KF/F)}{\Gal(K/(K\cap F))}{\sigma}{\local{\sigma}{K}}
                 \begin{displaymath}
                   \xymatrix@1{
                      && KF\ar@{-}[dr] &\\
                      K\ar@{-}[urr] &&& F\ar@{-}[dll] \\
                      & K\cap F\ar@{-}[ul] &&\\
                      & k\ar@{-}[u] &&            }
                 \end{displaymath}
  \end{thm}

  \begin{cor}
    Let $K/k$ be finite Galois, $F/k$ be an arbitrary extension. Then $[KF:F]$ divides $[K:k]$.
  \end{cor}
  \begin{warn}
    The assertion of above corollary is not usually valid if $K/k$ is not Galois.

    Indeed, let
    \begin{equation*}
      \alpha=2^{\frac{1}{3}}\quad\zeta=\frac{-1+\sqrt{-3}}{2}\quad\beta=\alpha\zeta
    \end{equation*}
    and $k=\QQ,K=\QQ(\beta),F=\QQ(\alpha)$. Then
    \begin{equation*}
      [KF:F]=2\quad [K:k]=3
    \end{equation*}
  \end{warn}

  \begin{thm}
    Let $\Gal(K_1/k)=G_1, \Gal(K_2/k)=G_2$, then $K_1K_2/k$ is Galois. Let $G$ be its Galois group, then
    \mapdes{G}{G_1\times G_2}{\sigma}{(\local{\sigma}{K_1},\local{\sigma}{K_2})}
    is injective. If $K_1\cap K_2=k$, then it is an isomorphism.
                 \begin{displaymath}
                   \xymatrix@1{
                      & K_1K_2\ar@{-}[dr] &\\
                      K_1\ar@{-}[ur] && K_2\ar@{-}[dl] \\
                      & K_1\cap K_2\ar@{-}[ul] &\\
                      & k\ar@{-}[u] &           }
                 \end{displaymath}
  \end{thm}

  \begin{cor}
    Let $K_1,\cdots,K_n$ be Galois extensions of $k$ with Galois groups $G_1,\cdots, G_n$. Assume $K_{i+1}\cap(K_1\cdots K_i)=k$, then
    \begin{equation*}
      \Gal(K_1\cdots K_n/k)\cong G_1\times\cdots\times G_n
    \end{equation*}
  \end{cor}

  \begin{cor}
    Let $K$ be finite Galois over $k$ with group $G$, and assume $G=G_1\times\cdots\times G_n$, let $K_i$ be the fixed field of
    \begin{equation*}
      G_1\times\cdots\times 1 \times\cdots\times G_n
    \end{equation*}
    then $K_i$ is Galois over $k$ and $K_{i+1}\cap(K_1\cdots K_i)=k, K=K_1\cdots K_n$.
  \end{cor}

  \begin{thm}
    Assume all fields contained in some common field.
    \begin{enumerate}[(i)]
      \item If $K,L$ are abelian over $k$, then so is $KL$;
      \item If $K/k$ is abelian and $F/k$ is arbitrary, then $KF/F$ is abelian;
      \item IF $K/k$ is abelian and $F\in\Int(K/k)$, then $K/F,F/k$ are abelian.
    \end{enumerate}
  \end{thm}

  \begin{warn}
    The converse of last statement may not be true.
  \end{warn}

  \begin{defn}
    The composite of all abelian extensions of $k$ in $k^{\ac}$ is called the \termin{abelian closure} of $k$ and denoted by $k^{\ab}$.
  \end{defn}

\newpage\section{Examples and Applications}

\newpage\section{Norm and Trace}
  Let $E/k$ be finite, $[E:k]_s=r$, $\alpha\in E$. We define the \termin{norm} and \termin{trace} of $\alpha$ to be
  \begin{align*}
    N_{E/k} &= N^E_k(\alpha) = \prod_{v=1}^{r}\sigma_v\alpha^{[E:k]_i} = \left(\prod_{v=1}^r\sigma_v\alpha\right)^{[E:k]_i} \\
    \Tr_{E/k} &= \Tr^E_k(\alpha) = [E:k]_i\sum_{v=1}^r\sigma_v\alpha
  \end{align*}

  If $E/k$ is separable, then
  \begin{align*}
    N^E_k(\alpha) & =\prod_{\sigma}\sigma\alpha \\
    \Tr^E_k(\alpha) & = \sum_{\sigma}\sigma\alpha
  \end{align*}

  \begin{thm}
    Let $E/k$ be finite, then
    \begin{enumerate}[(i)]
      \item $N^E_k(\alpha)$ is a multiplicative homomorphism of $E^{\ast}$ into $k^{\ast}$ and  $\Tr^E_k(\alpha)$ is an additive homomorphism of $E$ into $k$.
      \item If $E\supset F\supset k$ is a tower, then
                 \begin{equation*}
                   N^E_k=N^F_k\circ N^E_F\quad\text{and}\quad \Tr^E_k=\Tr^F_k\circ\Tr^E_F
                 \end{equation*}
      \item If $E=k(\alpha)$, and $f(X)=\Irr(\alpha,k,X)=X^n+a_{n-1}X^{n-1}+\cdots+a_0$, then
                 \begin{equation*}
                   N^{k(\alpha)}_k(\alpha)=(-1)^na_0\quad\text\quad\Tr^{k(\alpha)}_k(\alpha)=-a_{n-1}
                 \end{equation*}
    \end{enumerate}
  \end{thm}

  \begin{thm}
    Let $E/k$ be finite separable. Then $\Tr^E_k$ is a nonzero functional. The map
    \mapdes{E\times E}{k}{(x,y)}{\Tr(xy)}
    is bilinear, and identifies $E$ with its dual space $\codual{E}$.
  \end{thm}

  \begin{cor}
    Let $\omega_1,\cdots,\omega_n$ be a basis of $E/k$. Then there exists a basis $\omega'_1,\cdots,\omega'_n$ of $E/k$ such that $\Tr(\omega_i\omega'_j)=\delta_{ij}$.
  \end{cor}

  \begin{cor}
    Let $E/k$ be finite separable, and let $\sigma_1,\cdots,\sigma_n$ be the distinct embeddings. Let $w_1,\cdots,w_n$ be elements of $E$. Then the vectors
    \begin{gather*}
      \xi_1=(\sigma_1w_1,\cdots,\sigma_1w_n) \\
      \cdots \\
      \xi_n=(\sigma_nw_1,\cdots,\sigma_nw_n)
    \end{gather*}
    are linearly independent over $E$ if $w_1,\cdots,w_n$ form a basis of $E/k$.
  \end{cor}

  \begin{rem}
    In characteristic $0$, one sees much more trivially that the trace is not identically $0$. Indeed, if $c\in k$ and $c\neq0$, then $\Tr(c)=nc$ where $n=[E;k]$, and $n\neq0$. This argument also holds in characteristic $p$ where $n$ is prime to $p$.
  \end{rem}

  \begin{prop}
    Let $E=k(\alpha)$ be separable. Let
    \begin{equation*}
      f(X)=\Irr(\alpha,k,X)
    \end{equation*}
    and let $f'(X)$ be its derivative. Let
    \begin{equation*}
      \frac{f(X)}{(X-\alpha)}=\beta_0+\beta_1X+\cdots+\beta_{n-1}X^{n-1}
    \end{equation*}
    with $\beta_i\in E$. Then the dual basis of $1,\alpha,\cdots,\alpha^{n-1}$ is
    \begin{equation*}
      \frac{\beta_0}{f'(\alpha)},\cdots,\frac{\beta_{n-1}}{f'(\alpha)}
    \end{equation*}
  \end{prop}

  Define
  \longmapdes{m_{\alpha}}{E}{E}{x}{\alpha x}

  \begin{prop}
    Let $E/k$ be finite and let $\alpha\in E$. Then
    \begin{equation*}
      \det(m_{\alpha})=N_{E/k}(\alpha)\quad\text{and}\quad\Tr(m_{\alpha})=\Tr_{E/k}(\alpha)
    \end{equation*}
  \end{prop}

\newpage\section{Cyclic Extensions}

  \begin{thm}[Hilbert's Theorem 90]
    Let $K/k$ be cyclic of degree $n$ with Galois group $G$. Let $\sigma$ be a generator of $G$. Let $\beta\in K$. The norm $N(\beta)=1$ if and only if there exists an element $\alpha\neq0$ in $K$ such that $\beta=\alpha/\sigma\alpha$.
  \end{thm}

  \begin{thm}
    Let $k$ be a field, $n$ an integer $>0$ prime to the characteristic of $k$, and assume that there is a primitive $n-$th root of unity in $k$.
    \begin{enumerate}[(i)]
      \item Let $K$ be a cyclic extension of degree $n$. Then there exists $\alpha\in K$ such that $K=k(\alpha)$, and $\alpha$ satisfies an equation $X^n-a=0$ for some $a\in k$.
      \item Conversely, let $a\in k$. Let $\alpha$ be a root of $X^n-a$. Then $k(\alpha)$ is cyclic over $k$, of degree $d$, $d\mid n$, and $\alpha^d$ is an element of $k$.
    \end{enumerate}
  \end{thm}

  \begin{thm}[Hilbert's Theorem 90, Additive Form]
    Let $K/k$ be cyclic of degree $n$ with Galois group $G$. Let $\sigma$ be a generator of $G$. Let $\beta\in K$. The trace $\Tr(\beta)=0$ if and only if there exists an element $\alpha\neq0$ in $K$ such that $\beta=\alpha-\sigma\alpha$.
  \end{thm}

  \begin{thm}[Artin-Schreier]
    Let $k$ be a field of characteristic $p$.
    \begin{enumerate}[(i)]
      \item Let $K$ ba a cyclic extension of $k$ of degree $p$. Then there exists $\alpha\in K$ such that $K=k(\alpha)$ and $\alpha$ satisfies an equation $X^p-X-a=0$ with some $a\in k$.
      \item Conversely, given $a\in k$, the polynomial $f(X)=X^p-X-a$ either has one root in $k$, in which case all its roots are in $k$, or it is irreducible. In this latter case, if $\alpha$ is a root then $k(\alpha)$ is cyclic of degree $p$ over $k$.
    \end{enumerate}
  \end{thm}

\newpage\section{Solvable and Radical Extensions}

  \begin{defn}
    A Galois extension is called \termin[solvable]{solvable extension} if its Galois group is solvable. A finite extension $E/k$ is said to be \termin[solvable]{solvable extension} if the smallest Galois extension $K$ of $k$ containing $E$ is solvable.
  \end{defn}
  \begin{rem}
     This is equivalent to saying that there exist a solvable Galois extension $L$ of $k$ containing $E$.
  \end{rem}

  \begin{prop}
    Solvable extension form a distinguished class.
  \end{prop}

  \begin{defn}
    A simple extension $k(\alpha)/k$ is called \termin{solvable by radicals} if it is one of the following type:
    \begin{enumerate}
      \item It is obtained by adjoining a root of unity.
      \item It is obtained by adjoining a root of a polynomial $X^n-a$ with $a\in k$ and $n$ prime to the characteristic.
      \item It is obtained by adjoining a root of an equation $X^p-X-a$ with $a\in k$, if $p$ is the character $>0$.
    \end{enumerate}

    A finite extension $F/k$ is said to be \termin{solvable by radicals} if it is separable and if there exists a finite extension $E/k$ containing $F$, and admitting a tower
    \begin{equation*}
      k=E_0\subset E_1\subset E_2\subset\cdots\subset E_m=E
    \end{equation*}
    such that each step is a simple extension solvable by radicals.
  \end{defn}

  \begin{prop}
    The class of extensions which are solvable by radical is distinguished.
  \end{prop}
  \begin{proof}
    Using lifting.
  \end{proof}

  \begin{thm}
    Let $E/k$ be separable, then $E$ is solvable by radical if and only if $E/k$ is solvable.
  \end{thm}

  \begin{defn}
    A polynomial $f\in k[X]$ is said to be \termin{solvable by radicals} if its splitting field is solvable by radicals.
  \end{defn}

  \begin{defn}
    Let $k$ be a field. The general polynomial of degree $n$ over $k$ is defined to be
    \begin{equation*}
      f(X)=X^n-t_1X^{n-1}+t_2X^{n-2}+\cdots+(-1)^nt_n\in k(t_1,t_2,\cdots,t_n)[X]
    \end{equation*}
    where $t_1,t_2,\cdots,t_n$ are algebraic independent over $k$.
  \end{defn}

  \begin{cor}
    The general polynomial of degree $n$ is solvable by radicals only if $n\leqslant4$.
  \end{cor}

\newpage\section{Abelian Kummer Theory}

  \begin{defn}
    A Galois extension $K/k$ with group $G$ is said to be of \termin{exponent} $m$ if $\sigma^m=1$ for all $\sigma\in G$.
  \end{defn}

    Let $m$ be prime to the characteristic of $k$. We assume $k$ contains a primitive $m-$th root of unity.

    Let $a\in k$, and $\alpha$ be an $m-$th root of $a$. The field $k(\alpha)$ is independent of the choice of $\alpha$ and hence denoted by $k(a^{1/m})$.

    We denote by $k^{\ast m}$ the image of $k^{\ast}$ under the homomorphism $x\mapsto x^m$.

    Let $k^{\ast m}\subset B< k^{\ast}$. We denoted by $k(B^{1/m})$ or $K_B$ the composite of all fields $k(a^{1/m})$ with $a\in B$.

    $K_B/k$ is Galois, let $G$ be its Galois group. Let $\sigma\in G$. Then $\sigma\alpha=\omega_{\sigma}\alpha$ for some $m-$th root of unity $\omega_{\sigma}\in\root_m\subset k^{\ast}$.

    There is a homomorphism:
    \mapdes{G}{\root_m}{\sigma}{\omega_{\sigma}}

    We may write $\omega_{\sigma}=\sigma\alpha/\alpha$, but it is independent of the choice of $\alpha$.

    We denote $\omega_{\sigma}$ by $\<\sigma,\alpha\>$. The map
    \mapdes{G\times B}{\root_m}{(\sigma,a)}{\<\sigma,a\>}
    is bilinear.

  \begin{thm}
    Let $k$ be a field, $m$ an integer $>0$ prime to the characteristic of $k$, and assume that a primitive $m-$root of unity lies in $k$.
    \begin{enumerate}[(i)]
      \item Let $k^{\ast m}\subset B< k^{\ast}$, $K_B=k(B^{1/m})$, then $K_B/k$ is Galois and abelian of exponent $m$.
      \item Consider the bilinear map
                 \mapdes{G\times B}{\root_m}{(\sigma,a)}{\<\sigma,a\>}
                 The kernel of left is $1$, and the kernel of right is $k^{\ast m}$.
      \item $K_B/k$ is finite if and only if $(B:k^{\ast m})$ is finite. If that is the case, then
                 \begin{equation*}
                   B/k^{\ast m}\cong \dual{G}
                 \end{equation*}
                 and
                 \begin{equation*}
                   [K_B:k]=(B:k^{\ast m})
                 \end{equation*}
    \end{enumerate}
  \end{thm}

  \begin{thm}
    There is a $1-1$ corresponding:
    \isodes{\Sub(k^{\ast};k^{\ast m})}{\{\text{abelian extension of $k$ of exponent $m$}\}}{B}{K_B}
  \end{thm}

  \begin{thm}
      Let $k$ be a field with characteristic $p$, we define the operator $\wp$ by
      \begin{equation*}
        \wp(x)=x^p-x
      \end{equation*}
      Then $\wp$ is an additive automorphism of $k$.

      For $\wp(k)\subset B\subset k$, Let $K_B=k(\wp^{-1}B)$.
    \begin{enumerate}[(i)]
      \item There is a $1-1$ corresponding
                 \isodes{\Sub(k:\wp(k))}{\{\text{abelian extension of $k$ of exponent $p$}\}}{B}{K_B}
      \item For $\sigma\in G, a\in B$ and $\alpha\in K_B$ such that $\wp\alpha=a$. Let $\<\sigma,a\>=\sigma\alpha-\alpha$.
                 There is a bilinear map
                 \mapdes{G\times B}{\ZZ/p}{(\sigma,a)}{\<\sigma,a\>}
                 The kernel of left is $1$, and the kernel of right is $\wp(k)$.
      \item $K_B/k$ is finite if and only if $(B:\wp(k))$ is finite. If that is the case, then
                 \begin{equation*}
                   [K_B:k]=(B:\wp(k))
                 \end{equation*}
    \end{enumerate}
  \end{thm}

\newpage\section{The Equation $X^n-a=0$}
  In this section, the roots of unity are not in the ground field.

  \begin{thm}
    Let $k$ be a field and $n$ an integer $\geqslant2$. Let $a\in k,a\neq0$. Assume that for all $p\mid n$, we have $a\in k^{p}$, and if $4\mid n$ then $a\notin-4k^4$. Then $X^n-a$ is irreducible in $k[X]$.
  \end{thm}

  \begin{cor}
    $a\in k^{\ast}$, and $a\notin k^p$ for some $p$. If $p$ is equal to the characteristic of $k$ or $p$ is odd, then for all $r\geqslant1$, the polynomial $X^{p^r}-a$ is irreducible over $k$.
  \end{cor}

  \begin{cor}
    If $k^{\ac}/k$ is finite of degree $>1$, then $k^{\ac}=k(i)$ and its characteristic is $0$.
  \end{cor}
  
  \begin{exam}
    Let $k$ be a field with characteristic not dividing $n$. Let $a\in k^{\ast}$ and $K$ be the splitting field of $X^n-a$. Let $\alpha$ be a root of $X^n-a$ and let $\zeta$ be a primitive $n-$th root of unity. Then
    \begin{equation*}
      K=k(\alpha,\zeta)=k(\alpha,\root_n)
    \end{equation*}
    
    Let $\sigma\in G_{K/k}$. Then there exists some integer $b(\sigma)$ uniquely determined mod $n$, such that
    \begin{equation*}
      \sigma(\alpha)=\alpha\zeta^{b(\sigma)}
    \end{equation*}
    
    There exists an integer $d(\sigma)$ uniquely determined mod $n$, such that
    \begin{equation*}
      \sigma(\zeta)=\zeta^{d(\sigma)}
    \end{equation*}
    
    Let $G(n)$ be the subgroup of $\GL_2(\ZZ/n)$ consisting of all matrices 
    \begin{equation*}
      M=\begin{pmatrix}
          1 & 0 \\
          b & d \\
        \end{pmatrix}
    \end{equation*}
    where $b\in\ZZ/n,d\in(\ZZ/n)^{\ast}$.
    
    Observe that $|G(n)|=n\varphi(n)$, and we obtain an injective map:
    \mapdes{G_{K/k}}{G(n)}{\sigma}{M(\sigma)}
  \end{exam}
  
  \begin{thm}
    Let $k$ be a field. Let $n$ be an odd integer prime to the characteristic, and assume that $[k(\root_n):k]=\varphi(n)$. Let $a\in k$, and $a\notin k^p$ for all prime $p\mid n$. Let $K$ be the splitting field of $X^n-a$ over $k$.
    
    Then the above homomorphism $\sigma\mapsto M(\sigma)$ is an isomorphism. The commutator group is $\Gal(K/k(\root_n))$, so $k(\root_n)$ is the maximal abelian subextension of $K$. 
  \end{thm}
  \begin{warn}
    When $n$ is even, there are some complications.
  \end{warn}%Galois Theory
%infinite Galois Theory
%
%\part{Dedekind Domain}
%\section{23}


\begin{appendices}
%\renewcommand{\thepart}{}%\Alph{part}.}
\CTEXsetup[name={,},number={}]{part}
\part{Appendix}
%\renewcommand{\thechapter}{}
%\chapter{Category Theory}
\section{Categories}
  \begin{defn}
  A \termin{category} $\Cc$ consists of
  \begin{itemize}
    \item a class $\ob\Cc$ of \termin[objects]{object}.
    \item a class $\hom\Cc$ of \termin[morphisms]{morphism (category theory)}, or \termin[arrows]{arrow (category theory)}, or \termin[maps]{map (category theory)}, between the objects.

             Each morphism $f$ has a unique source object $A$ and target object $B$ where $A$ and $B$ are in $\ob\Cc$.

             We write $f\colon A\To B$, and we say ``$f$ is a morphism from $A$ to $B$''.

             We write $\Hom(A, B)$ (or $\Hom_{\Cc}(A, B)$ when there may be confusion about to which category $\Hom(A, B)$ refers) to denote the hom-class of all morphisms from $A$ to $B$. (Some authors write $\Mor(A, B)$ or simply $\Cc(A,B)$ instead.)
    \item for every three objects $A,B$ and $C$, a binary operation
             \begin{equation*}
               \Hom(A, B) \times \Hom(B, C) \To \Hom(A, C)
             \end{equation*}
             called \termin{composition of morphisms}.

             The composition of $f\colon A \To B$ and $g\colon B \To C$ is written as $g\circ f$ or simply $gf$. (Some authors use ``diagrammatic order'', writing $f;g$ or $fg$.)
  \end{itemize}
  such that the following axioms hold:
  \begin{description}
    \item[associativity] if $f\colon A \To B, g\colon B \To C$ and $h\colon C \To D$ then
                                 \begin{equation*}
                                   h\circ(g\circ f) = (h\circ g)\circ f
                                 \end{equation*}
    \item[identity] for every object $A$, there exists a morphism $1_A\colon A \To A$ (some times write $\id_A$) called the \termin[identity morphism]{identity (morphism)} for $A$, such that for every morphism $f\colon A \To B$, we have $1_B \circ f = f = f \circ 1_A$.
  \end{description}
  From these axioms, one can prove that there is exactly one identity morphism for every object. Some authors use a slight variation of the definition in which each object is identified with the corresponding identity morphism.
  \end{defn}

  \begin{defn}
    A category $\Cc$ is called \termin[small]{small category} if both $\ob\Cc$ and $\hom\Cc$ are actually sets,
    and \termin[large]{large category} otherwise.
    A \termin[locally small category]{locally!small} is a category such that for all objects $A$ and $B$, the hom-class $\Hom(A, B)$ is a set,
    called a hom-set. Many important categories in mathematics (such as the category of sets), although not small, are at least locally small.
  \end{defn}

  \begin{defn}
    Any category $\Cc$ can itself be considered as a new category in a different way: the objects are the same as those in the original category but the arrows are those of the original category reversed. This is called the \termin[dual]{dual!category} or \termin[opposite category]{opposite!category} and is denoted $\Cc^{\op}$.
  \end{defn}

  \begin{defn}
    The \termin{product category} $\Cc\times\Dd$ consists of:
    \begin{itemize}
      \item \emph{objects:}
               pairs of objects $(A, B)$, where $A$ is an object of $\Cc$ and $B$ of $\Dd$;
      \item \emph{arrows from $(A_, B_1)$ to $(A_2, B_2)$:}
               pairs of arrows $(f, g)$, where $f\colon A_1 \To A_2$ is an arrow of $\Cc$ and $g\colon B_1 \To B_2$ is an arrow of $\Dd$;
      \item \emph{compositions:}
               component-wise composition from the contributing categories:
                                    \begin{equation*}
                                      (f_2, g_2) \circ (f_1, g_1) = (f_2 \circ f_1, g_2 \circ g_1);
                                    \end{equation*}
      \item \emph{identities:}
               pairs of identities from the contributing categories:
                               \begin{equation*}
                                 1_{(A, B)} = (1_A, 1_B).
                               \end{equation*}
    \end{itemize}
  \end{defn}

\subsection{Subcategories}
  \begin{defn}
    Let $\Cc$ be a category. A \termin{subcategory} $\Ss$ of $\Cc$ is given by
    \begin{itemize}
      \item a subcollection of objects of $\Cc$, denoted $\ob\Ss$,
      \item a subcollection of morphisms of $\Cc$, denoted $\hom\Ss$.
    \end{itemize}
    such that
    \begin{enumerate}
      \item for every $X$ in $\ob\Ss$, the identity morphism $\id_X$ is in $\hom\Ss$,
      \item for every morphism $f\colon X\To Y$ in $\hom\Ss$, both the source $X$ and the target $Y$ are in $\ob\Ss$,
      \item for every pair of morphisms $f$ and $g$ in $\hom\Ss$ the composite $f\circ g$ is in $\hom\Ss$ whenever it is defined.
    \end{enumerate}
  \end{defn}
  \begin{rem}
    These conditions ensure that $\Ss$ is a category in its own right: the collection of objects is $\ob\Ss$, the collection of morphisms is $\hom\Ss$, and the identities and composition are as in $\Cc$. There is an obvious faithful functor $I\colon \Ss\To\Cc$, called the \termin{inclusion functor} which takes objects and morphisms to themselves.
  \end{rem}

  \begin{defn}
    Let $\Ss$ be a subcategory of a category $\Cc$. We say that $\Ss$ is a \termin[full subcategory]{full!subcategory} of $\Cc$ if the inclusion functor is fully faithful.
  \end{defn}

  \begin{defn}
    A subcategory $\Ss$ of $\Cc$ is said to be \termin{isomorphism-closed} or \termin{replete} if every isomorphism $k\colon X\To Y$ in $\Cc$ such that $Y$ is in $\Ss$ also belongs to $\Ss$. A isomorphism-closed full subcategory is said to be \termin[strictly full]{strictly full!subcategory}.
  \end{defn}

  \begin{defn}
    A subcategory of $\Cc$ is said to be \termin{wide} or \termin{lluf} if it contains all the objects of $\Cc$.
  \end{defn}

\subsection{Reflective Subcategory}

\subsection{Comma Categories}
  \begin{defn}
    Suppose that $\Aa$, $\Bb$, and $\Cc$ are categories, and $S$ and $T$ (for source and target) are functors
          \begin{displaymath}
            \xymatrix{
               \Aa\ar[r]^{S} & \Cc & \Bb\ar[l]_{T}                }
          \end{displaymath}
    We can form the \termin{comma category} $(S\down T)$ as follows:
    \begin{itemize}
      \item The objects are all triples $(\alpha,\beta,f)$ with $\alpha$ an object in $\Aa$, $\beta$ an object in $\Bb$, and $f\colon S(\alpha)\To T(\beta)$ a morphism in $\Cc$.
      \item The morphisms from $(\alpha,\beta,f)$ to $(\alpha',\beta',f')$ are all pairs $(g,h)$
                 where $g\colon \alpha\To\alpha'$ and $h\colon \beta\To\beta'$ are morphisms in $\Aa$ and $\Bb$ respectively, such that the following diagram commutes:
                 \begin{displaymath}
                   \xymatrix{
                       S(\alpha)\ar[r]^{S(g)}\ar[d]_{f} & S(\alpha')\ar[d]^{f'}  \\
                       T(\beta)\ar[r]_{T(h)} & T(\beta')           }
                 \end{displaymath}
      \item Morphisms are composed by taking $(g,h)\circ(g',h')$ to be $(g\circ g',h\circ h')$, whenever the latter expression is defined.
      \item The identity morphism on an object $(\alpha,\beta,f)$ is $(\id_{\alpha},\id_{\beta})$.
    \end{itemize}
  \end{defn}

  \begin{exam}
    \textbf{Slice category}.

    When $\Aa=\Cc$, $S$ is the identity functor, and $\Bb=\one$ (the category with one object $\ast$ and one morphism).
    Then $T(\ast)=A$ for some object $A$ in $\Cc$. In this case, the comma category is written $(\Cc\down A)$, and is often called the \termin{slice category} over $A$ or the category of \emph{\red objects over $A$}.
    The objects $(\alpha,\ast,f)$ can be simplified to pairs $(\alpha,f)$, where $f\colon \alpha\To A$.
    Sometimes, $f$ is denoted $\pi_{\alpha}$.
    A morphism from $(B,\pi_B)$ to $(B',\pi_{B'})$ in the slice category is then an arrow $g\colon B\To B'$ making the following diagram commute:
          \begin{displaymath}
            \xymatrix@R=0.5cm{
               B\ar[rr]^{g}\ar[dr]_{\pi_B} && B'\ar[dl]^{\pi_{B'}} \\
               &A&                }
          \end{displaymath}
  \end{exam}

  \begin{exam}
    \textbf{Coslice category}.

    The dual concept to a slice category is a coslice category. Here, $S$ has domain $\one$ and $T$ is an identity functor. In this case, the comma category is often written $(A\down\Cc)$, where $A$ is the object of $\Cc$ selected by $S$. It is called the \termin{coslice category} with respect to $A$, or the category of \emph{\red objects under $A$}.
    The objects are pairs $(B,i_B)$ with $i_B\colon A\To B$. Given $(B,i_B)$ and $(B',i_{B'})$, a morphism in the coslice category is a map $h\colon B\To B'$ making the following diagram commute:
          \begin{displaymath}
            \xymatrix@R=0.5cm{
               &A\ar[dl]_{i_B}\ar[dr]^{i_{B'}}&  \\
               B\ar[rr]_{h} && B'                }
          \end{displaymath}
  \end{exam}

  \begin{exam}
    \textbf{Arrow category}.

    $S$ and $T$ are identity functors on $\Cc$ (so $\Aa=\Bb=\Cc$).
    In this case, the comma category is the \termin{arrow category} $\Cc^{\to}$. Its objects are the morphisms of $\Cc$, and its morphisms are commuting squares in $\Cc$.
  \end{exam}

  \begin{exam}
    In the case of the slice or coslice category, the identity functor may be replaced with some other functor; this yields a family of categories particularly useful in the study of adjoint functors. Let $s,t$ be given object in $\Cc$.
    An object of $(s\down T)$ is called a \emph{morphism from $s$ to $T$} or a \termin{$T-$structured arrow} with domain $s$ in.
    An object of $(S\down t)$ is called a \emph{morphism from $S$ to $t$} or a \termin{$S-$costructured arrow} with codomain $s$ in.
  \end{exam}

  \begin{prop}
    For each comma category there are forgetful functors from it.
    \begin{itemize}
      \item \termin{domain functor}, $(S\down T)\To\Aa$, which maps:
      \begin{itemize}
        \item objects: $(\alpha,\beta,f)\mapsto\alpha$;
        \item morphisms: $(g,h)\mapsto g$;
      \end{itemize}
      \item \termin{codomain functor}, $(S\down T)\To\Bb$, which maps:
      \begin{itemize}
        \item objects: $(\alpha,\beta,f)\mapsto\beta$;
        \item morphisms: $(g,h)\mapsto h$;
      \end{itemize}
    \end{itemize}
  \end{prop}

  \begin{exam}
    The category of \textbf{pointed sets} is a comma category $(\bullet\down\Set)$, with $\bullet$ being (a functor selecting) any singleton set, and $\Set$ (the identity functor of) the category of sets.
    Each object of this category is a set, together with a function selecting some element of the set: the ``\textbf{basepoint}''. Morphisms are functions on sets which map basepoints to basepoints. In a similar fashion one can form the category of \textbf{pointed spaces} $(\bullet\down\Top)$.
  \end{exam}

  \begin{exam}
    The category of \textbf{graphs} is $(\Set\down D)$, with $D\colon\Set\To\Set$ the functor taking a set $s$ to $s\times s$.
    The objects $(a,b,f)$ then consist of two sets and a function; $a$ is an \textbf{indexing set}, $b$ is a set of \textbf{nodes}, and $f\colon a\To (b\times b)$ chooses pairs of elements of $b$ for each input from $a$. That is, $f$ picks out certain edges from the set $b\times b$ of possible edges.
    A morphism in this category is made up of two functions, one on the indexing set and one on the node set. They must ``agree'' according to the general definition above, meaning that $(g,h)\colon(a,b,f)\To(a',b',f')$ must satisfy $f'\circ g=D(h)\circ f$. In other words, the edge corresponding to a certain element of the indexing set, when translated, must be the same as the edge for the translated index.
  \end{exam}

  \begin{exam}
    Colimits in comma categories may be ``inherited''. If $\Aa$ and $\Bb$ are cocomplete, $S\colon\Aa\To\Cc$ is a cocontinuous functor, and $T\colon\Bb\To\Cc$ another functor (not necessarily cocontinuous), then the comma category $(S\down T)$ produced will also be cocomplete.

    If $\Aa$ and $\Bb$ are complete, and both $S\colon\Aa\To\Cc$ and $T\colon\Bb\To\Cc$ are continuous functors, then the comma category $(S\down T)$ is also complete, and the projection functors $(S\down T)\To\Aa$ and $(S\down T)\To\Bb$ are limit preserving.
  \end{exam}

  \begin{exam}
    \textbf{Adjunctions}.

    Lawvere showed that the functors $F\colon\Cc\To\Dd$ and $G\colon\Dd\To\Cc$ are adjoint if and only if the comma categories $(F\down \id_{\Dd})$ and $(\id_{\Cc}\down G)$, with $\id_{\Dd}$ and $\id_{\Cc}$ the identity functors on $\Dd$ and $\Cc$ respectively, are isomorphic, and equivalent elements in the comma category can be projected onto the same element of $\Cc\times\Dd$. This allows adjunctions to be described without involving sets, and was in fact the original motivation for introducing comma categories.
  \end{exam}

  \begin{exam}
    \textbf{Natural transformations}.

    A natural transformation $\eta\colon S\To T$, with $S,T\colon\Aa\To\Cc$, corresponds to a functor $\Aa\To(S\down T)$ which maps each object $\alpha$ to $(\alpha,\alpha,\eta_{\alpha})$ and maps each morphism $g$ to $(g,g)$.
    This is a bijective correspondence between natural transformations $S\To T$ and functors $\Aa\To(S\down T)$ which are sections of both forgetful functors from $(S\down T)$.
  \end{exam}

\newpage\section{Morphisms}
\subsection{Monomorphisms, Epimorphisms and Zero Morphisms}
  \begin{defn}
    A morphism $f$ is called a \termin{monomorphism}, or \termin{monoic}, if for any morphisms
    $\xymatrix@1{\cdot\ar@<0.5ex>[r]^{\alpha}\ar@<-0.5ex>[r]_{\beta} &\cdot\ar[r]^{f} &\cdot}$, $f\alpha=f\beta$ implies $\alpha=\beta$.
    Dually, $f$ is called an \termin{epimorphism}, or \termin{epi}, if for any morphisms
    $\xymatrix@1{\cdot\ar[r]^{f} &\cdot\ar@<0.5ex>[r]^{\alpha}\ar@<-0.5ex>[r]_{\beta} &\cdot}$, $\alpha f=\beta f$ implies $\alpha=\beta$.
    If $f$ is both a monomorphism and an epimorphism, then we say it is a \termin{bimorphism}.
  \end{defn}
  \begin{defn}
    A morphism $f$ is called a \termin[split monomorphism]{split!monomorphism}, if it has a left inverse, names \termin{retraction}.
    Dually, a morphism $f$  is called a \termin[split epimorphism]{split!epimorphism}, if it has a right inverse, names \termin{section}.
    If $f$ is both a split monomorphism and a split epimorphism, then we say it is an \termin{isomorphism}.
  \end{defn}
  \begin{rem}
    It is clear that any split monomorphism must be monoic and any split epimorphism is epi, hence any isomorphism is a bimorphism. However, the converse is not true in general case.
  \end{rem}

  A bijective morphism may fail to be an isomorphism:
  \begin{exam}
    In $\mathbf{Top}$, the map from the half-open interval $[0,1)$ to the unit circle $S^1$ (thought of as a subspace of the complex plane) which sends $x$ to $e^{2��ix}$ is continuous and bijective but not a \emph{\red  homeomorphism}\index{homeomorphism} since the inverse map is not continuous at $1$.
  \end{exam}
  \begin{rem}
    This counterexample also shows why the concept of \emph{subobject} does not correspond subspace in $\mathbf{Top}$.
  \end{rem}

  An epimorphism may fail to be surjective:

  \begin{exam}
    In the category of rings, $\mathbf{Ring}$, the inclusion map $\ZZ\hookrightarrow\QQ$ is a non-surjective epimorphism; to see this, note that any ring homomorphism on $\QQ$ is determined entirely by its action on $\ZZ$. A similar argument shows that the natural ring homomorphism from any commutative ring $R$ to any one of its localizations is an epimorphism.
  \end{exam}
  \begin{rem}
    This is also a counterexample shows that a quotient object may not be a quotient.
  \end{rem}

  \begin{defn}
    A morphism $f$ is called a \termin[constant morphism]{constant!morphism} (or sometimes \termin[left zero morphism]{left!zero morphism}) if for any morphisms
    $\xymatrix@1{\cdot\ar@<0.5ex>[r]^{\alpha}\ar@<-0.5ex>[r]_{\beta} &\cdot\ar[r]^{f} &\cdot}$, $f\alpha=f\beta$.
    Dually, $f$ is called a \termin{coconstant morphism} (or sometimes \termin[right zero morphism]{right!zero morphism}) if for any morphisms
    $\xymatrix@1{\cdot\ar[r]^{f} &\cdot\ar@<0.5ex>[r]^{\alpha}\ar@<-0.5ex>[r]_{\beta} &\cdot}$, $\alpha f=\beta f$.
     A \termin[zero morphism]{zero!morphism} is one that is both a constant morphism and a coconstant morphism.
  \end{defn}
  \begin{defn}
    A \termin[category with zero morphisms]{category!with zero morphisms} is one where, for every two objects $A$ and $B$ in $\Cc$, there is a fixed morphism $0_{AB} \colon A \To�� B$ such that for all objects $X, Y, Z$ in C and all morphisms $f \colon X \To Y, g \colon Y \To Z$, the following diagram commutes:
      \begin{displaymath}
        \xymatrix{
             X\ar[r]^{0_{XY}}\ar[d]_{f}\ar[dr]|{0_{XZ}} & Y\ar[d]^{g} \\
             Y\ar[r]_{0_{YZ}} & Z                    }
      \end{displaymath}
  \end{defn}
  \begin{rem}
    The morphisms $0_{XY}$ necessarily are zero morphisms and form a \emph{compatible system} of zero morphisms. If $\Cc$ is a category with zero morphisms, then the collection of $0_{XY}$ is unique.
  \end{rem}
  \begin{rem}
    If a category has zero morphisms, then one can define the notions of \emph{kernel} and \emph{cokernel} for any morphism in that category.
  \end{rem}

\subsection{Factorization}
\cite{freyd1972categories}
  \begin{defn}
    If a morphism $f\colon X\To Y$ can be written as a composition $f=g\circ h$ with $g\colon Z\To Y$ and $h\colon X\To Z$,
    then $f$ is said to \termin{factor through} any (and all) of $Z$, $g$, and $h$. We also say $f$ is \emph{\red factorized} as $h$ followed by $g$.
  \end{defn}

  \begin{defn}
    A \termin[factorization system]{factorization!system} $(E, M)$ for a category $\Cc$ consists of two classes of morphisms $E$ and $M$ of $\Cc$ such that:
    \begin{enumerate}
      \item $E$ and $M$ both contain all isomorphisms of $\Cc$ and are closed under composition.
      \item Every morphism $f$ of $\Cc$ can be factored as $f=m\circ e$ for some morphisms $e\in E$ and $m\in M$.
      \item The factorization is \emph{functorial}: if $u$ and $v$ are two morphisms such that $vme=m'e'u$ for some morphisms $e,e'\in E$ and $m,m'\in M$, then there exists a unique morphism $w$ making the following diagram commute:
          \begin{displaymath}
            \xymatrix{
               \ar[r]^{e}\ar[d]_{u} & \ar[r]^{m}\ar@{-->}[d]_{w} & \ar[d]^{v} \\
               \ar[r]_{e'} & \ar[r]_{m'} &                    }
          \end{displaymath}
    \end{enumerate}
  \end{defn}

  \begin{defn}
    Two morphisms $e$ and $m$ are said to be \termin[orthogonal]{orthogonal!morphism}, if for every pair of morphisms $u$ and $v$ such that $ve=mu$ there is a unique morphism $w$ such that the diagram
          \begin{displaymath}
            \xymatrix{
               \ar[r]^{e}\ar[d]_{u} & \ar@{-->}[dl]^-{w}\ar[d]^{v} \\
               \ar[r]_{m} &                     }
          \end{displaymath}
    commutes. If so, denote by $e\down m$,

    This notion can be extended to define the orthogonals of sets of morphisms by
     \begin{equation*}
       H^{\uparrow}\defeq\left\{ e \mid \forall h\in H, e\down h \right\}
     \end{equation*}
     and
     \begin{equation*}
       H^{\down}\defeq\left\{ m \mid \forall h\in H, h\down m \right\}
     \end{equation*}
    Since in a factorization system $E\cap M$ contains all the isomorphisms, the condition 3. of the definition is equivalent to 3':
    \begin{equation*}
      E\subset M^{\uparrow} \qquad M\subset E^{\down}
    \end{equation*}
  \end{defn}

  \begin{prop}
    The pair $(E,M)$ of classes of morphisms of $\Cc$ is a factorization system if and only if it satisfies the following conditions:
    \begin{enumerate}
      \item Every morphism $f$ of $\Cc$ can be factored as $f=m\circ e$ for some morphisms $e\in E$ and $m\in M$.
      \item $E\subset M^{\uparrow}$ and $M\subset E^{\down}$.
    \end{enumerate}
  \end{prop}

\subsection{Endomorphisms}
\cite{balmer2001idempotent}
  \begin{defn}
    An \termin{endomorphism} is a morphism whose domain and co-domain coincide. An \termin{automorphism} is a morphism that is both an endomorphism and an isomorphism.
  \end{defn}
  \begin{defn}
    An \termin{idempotent} $e$ is an endomorphism such that $e\circ e=e$.
    An endomorphism $e$ is said to \termin[split]{split!endomorphism} if it is idempotent, and if there are two morphisms $f,g$ such that $e = g f$ and $\id = f g$.
  \end{defn}
  \begin{defn}
    A category is called \termin[idempotent complete]{idempotent!complete}, if every idempotent splits.
  \end{defn}
  \begin{defn}
    Let $\Cc$ be a category, the \termin{Karoubi envelope} of $\Cc$, sometimes written $\Split(\Cc)$, is the category whose objects are pairs of the form $(A, e)$ where $A$ is an object of $\Cc$ and $e\colon A\To A$ is an idempotent of $\Cc$, and whose morphisms are triples of the form
    \begin{equation*}
      (e,f,e')\colon (A,e) \To (A',e')
    \end{equation*}
    where $f\colon A\To A'$ is a morphism of $\Cc$ satisfying $e'\circ f=f=f\circ e$ (or equivalently $f=e'\circ f\circ e$).

    Composition in $\Split(\Cc)$ is as in $\Cc$, but the identity morphism on $(A,e)$ in $\Split(\Cc)$ is $(e,e,e)$, rather than the identity on $A$.
  \end{defn}

  \begin{prop}
    The Karoubi envelope $\Split(\Cc)$ of $\Cc$ is the \termin[idempotent completion]{idempotent!completion} of $\Cc$, which means that
    $\Cc$ can be fully faithfully embedded into $\Split(\Cc)$, and the embedding $\imath\colon\Cc\To\Split(\Cc)$ satisfying the following universal property:
    \begin{quote}
      For any functor $F\colon\Cc\To\Dd$ with $\Dd$ is idempotent complete, there is a unique functor $F'\colon \Split(\Cc) \To \Dd$ such that the following diagram commutes:
      \begin{displaymath}
        \xymatrix@R=0.5cm{
                &         \Split(\Cc) \ar[dd]^{F'}     \\
              \Cc \ar[ur]^-{\imath} \ar[dr]_{F}                 \\
                &         \Dd                 }
      \end{displaymath}
    \end{quote}
  \end{prop}

\subsection{Initial and Terminal Morphisms}
  \begin{defn}
    Suppose that $U\colon\Dd \To \Cc$ is a functor from a category $\Dd$ to a category $\Cc$, and let $X$ be an object of $\Cc$.
    An \termin[initial morphism]{initial!morphism} from $X$ to $U$
    is an initial object in the category $(X\down U)$ of morphisms from $X$ to $U$.
    A \termin[terminal morphism]{terminal!morphism} from $U$ to $X$
    is a terminal object in the comma category $(U\down X)$ of morphisms from $U$ to $X$.
  \end{defn}
  \begin{rem}
  The term \termin[universal morphism]{universal!morphism} refers either to an initial morphism or a terminal morphism.
  \end{rem}

  \begin{prop}
    Given a functor $U$ and an object $X$ as above, there may or may not exist an initial morphism from $X$ to $U$.
    However, if an initial morphism does exist then it is unique up to a unique isomorphism.
  \end{prop}

  \begin{prop}
    Let $U$ be a functor from $\Dd$ to $\Cc$, and let $X$ be an object of $\Cc$.
    Then the following statements are equivalent:
    \begin{enumerate}[a.]
      \item $(A, \phi)$ is an initial morphism from $X$ to $U$;
      \item $(A, \phi)$ is an initial object of the comma category $(X \down U)$;
      \item $(A, \phi)$ is a representation of $\Hom_{\Cc}(X, U(-))$.
    \end{enumerate}

    The dual statements are also equivalent:
    \begin{enumerate}[a'.]
      \item $(A, \phi)$ is a terminal morphism from $U$ to $X$;
      \item $(A, \phi)$ is a terminal object of the comma category $(U \down X)$;
      \item $(A, \phi)$ is a representation of $\Hom_{\Cc}(U(-), X)$.
    \end{enumerate}
  \end{prop}

  Suppose $(A_1, \phi_1)$ is an initial morphism from $X_1$ to $U$ and $(A_2, \phi_2)$ is an initial morphism from $X_2$ to $U$.
  By the initial property, given any morphism $h\colon X_1 \To X_2$ there exists a unique morphism $g\colon A_1 \To A_2$ such that the following diagram commutes:
      \begin{displaymath}
        \xymatrix{
                X_1\ar[r]^{\phi_1}\ar[d]_{h} & U(A_1)\ar@{-->}[d]^{U(g)} & A_1\ar@{-->}[d]^{g} \\
                X_2\ar[r]_{\phi_2} & U(A_2) & A_2                 }
      \end{displaymath}

  If every object $X_i$ of $\Cc$ admits an initial morphism to $U$, then the assignment $X_i\mapsto A_i$ and $h\mapsto g$ defines a functor $V$ from $\Cc$ to $\Dd$. The maps $\phi_i$ then define a natural transformation from $\id_{\Cc}$ to $UV$. The functors $(V, U)$ are then a pair of adjoint functors, with $V$ left-adjoint to $U$ and $U$ right-adjoint to $V$.

  Similar statements apply to the dual situation of terminal morphisms from $U$. If such morphisms exist for every $X$ in $\Cc$ one obtains a functor $V\colon \Cc \To \Dd$ which is right-adjoint to $U$ (so $U$ is left-adjoint to $V$).

  Indeed, all pairs of adjoint functors arise from \emph{universal constructions} in this manner. Let $F$ and $G$ be a pair of adjoint functors with unit $\eta$ and co-unit $\varepsilon$.
  Then we have a universal morphism for each object in $\Cc$ and $\Dd$:
  \begin{itemize}
    \item For each object $X$ in $\Cc$, $(F(X), \eta_X)$ is an initial morphism from $X$ to $G$.
               That is, for all $f\colon X \To G(Y)$ there exists a unique $g\colon F(X) \To Y$ for which the following diagrams commute.
    \item For each object $Y$ in $\Dd$, $(G(Y), \varepsilon_Y)$ is a terminal morphism from $F$ to $Y$.
               That is, for all $g\colon F(X) \To Y$ there exists a unique $f\colon X \To G(Y)$ for which the following diagrams commute.
  \end{itemize}
      \begin{displaymath}
        \xymatrix{
                X\ar[r]^-{\eta_X}\ar[dr]_{f} & GF(X)\ar@{-->}[d]^{G(g)} & F(X)\ar@{-->}[d]_{F(f)}\ar[dr]^{g} & \\
                & G(Y) & FG(Y)\ar[r]_-{\varepsilon_Y} & Y               }
      \end{displaymath}

  \emph{Universal constructions} are more general than adjoint functor pairs: a universal construction is like an optimization problem; it gives rise to an adjoint pair if and only if this problem has a solution for every object of $\Cc$ (equivalently, every object of $\Dd$).

\newpage\section{Functors}
  \begin{defn}
    Let $\Cc, \Dd$ be two categories, a \termin{functor} $F\colon\Cc\To\Dd$ is a corresponding from $\ob\Cc$ into $\ob\Dd$ and $\hom\Cc$ into $\hom\Dd$ such that
    \begin{enumerate}[(i)]
      \item $F(1_X)=1_{F(X)}$ for every object $X$;
      \item $F(g\circ f)=F(g)\circ F(f)$ for all morphisms $f\colon A\To B, g\colon B\To C$.
    \end{enumerate}
  \end{defn}

  \begin{defn}
    Let $\Cc, \Dd$ be two categories, a \termin{contravariant functor} is a functor from $\Cc^{\op}$ to $\Dd$.
  \end{defn}
  Ordinary functors are also called \termin[covariant functor]{covariant functors} in order to distinguish them from \emph{contravariant} ones.

  \begin{defn}
    Every functor $F\colon\Cc\To\Dd$ induces the \termin[opposite functor]{opposite!functor} $F^{\op}\colon \Cc^{\op} \To \Dd^{\op}$ maps objects and morphisms identically to $F$.
  \end{defn}

  \begin{defn}
    A \termin{bifunctor} is a functor whose domain is a product category.
  \end{defn}

  \begin{exam}
    Consider $S_2\To S_3\To S_2$ in $\Grp$, it is not difficult to show that there is no functor $\Grp\To\Ab$ sending each group to its center.
  \end{exam}

  \begin{exam}
    The functor $\Cc \To \Dd$ which maps every object of $\Cc$ to a fixed object $X$ in $\Dd$ and every morphism in $\Cc$ to the identity morphism on $X$. Such a functor is called a \termin[constant]{constant!functor} or \termin{selection functor}.
  \end{exam}

  \begin{exam}
    The \termin[diagonal functor]{diagonal!functor} is defined as the functor from $\Dd$ to the functor category $[\Cc,\Dd]$ which sends each object in $\Dd$ to the constant functor at that object.
  \end{exam}

\begin{defn}
  A Functor $F\colon \mathcal{C}\to\mathcal{D}$ is said to be
  \begin{enumerate}[a)]
    \setlength{\itemindent}{2ex}
    \item \termin[faithful]{faithful!functor} (resp. \termin[full]{full!functor}, resp. \termin{ fully faithful}) if for any $X,Y\in\ob\mathcal{C}$, the map $\Hom_{\mathcal{C}}(X,Y)\to\Hom_{\mathcal{D}}(F(X),F(Y))$ is injective (resp. surjective, resp. bijective).
    \item \termin{essentially surjective} if for each $Y\in\ob\mathcal{D}$, there exists $X\in\ob\mathcal{C}$ and an isomorphism $F(X)\cong Y$.
    \item \termin{conservative} if any morphism $f\colon X\to Y$ in $\mathcal{C}$ is an isomorphism as soon as $F(f)$ is an isomorphism.
  \end{enumerate}
\end{defn}
\begin{warn}
  A faithful functor need not be injective on objects or morphisms. That is, two objects $X$ and $X'$ may map to the same object in $\Dd$ (which is why the range of a fully faithful functor is not necessarily equivalent to $\Cc$),
  and two morphisms $f \colon X\To Y$ and $f' \colon X'\To Y'$ (with different domains/codomains) may map to the same morphism in $\Dd$.

  Likewise, a full functor need not be surjective on objects or morphisms. There may be objects in $\Dd$ not of the form $F(X)$ for some $X$ in $\Cc$. Morphisms between such objects clearly cannot come from morphisms in $\Cc$.
\end{warn}
\begin{prop}
  \begin{enumerate}[1)]
    \setlength{\itemindent}{2ex}
    \item Let $F\colon \mathcal{C}\to \mathcal{D}$ be a faithful functor and let $f$ be a morphism in $\mathcal{C}$. Then if $F(f)$ is a monomorphism (resp. an epimorphism), then $f$ is a monomorphism (resp. an epimorphism).
    \item Moreover, assume that $F$ is fully faithful. Then if $F(f)$ is an isomorphism, then $f$ is an isomorphism. In other words, fully faithful functors are conservative.
  \end{enumerate}
\end{prop}
\begin{proof}
For any $\xymatrix@1{T\ar@<0.5ex>[r]^{\alpha}\ar@<-0.5ex>[r]_{\beta} &X\ar[r]^{f} &T}$ in $\mathcal{C}$ such that $f\alpha=f\beta$, then $F(f)F(\alpha)=F(f)F(\beta)$. Since $F(f)$ is injective, $F(\alpha)=F(\beta)$. Since $F$ is faithful, $\alpha=\beta$. Hence $f$ is also a monomorphism. The epimorphism case is similar.

Let $\varphi$ be the inverse morphism of $F(f)\colon F(X)\to F(Y)$, since $F$ is fully faithful, there exist a $g\in\Hom_{\mathcal{C}}(Y,X)$ such that $F(g)=\varphi$, moreover, it is the inverse morphism of $f$.
\end{proof}
\begin{cor}
  A fully faithful functor is necessarily injective on objects up to isomorphism.
\end{cor}

\subsection{Natural Transformations and Functor categories}
\begin{defn}
  Let $F,G$ be two functors from $\Cc$ to $\Dd$. A morphism (or \termin[natural transformation]{natural!transformation}) of functors $\pi\colon F\to G$ is the data for all $X\in\ob\Cc$ of a morphism $\pi(X)\colon F(X)\to G(X)$ such that for all $f\colon X\to Y$ in $\Cc$ , the following diagram commutes:
 \begin{displaymath}
      \xymatrix{
         F(X)\ar[r]^{\pi_X}\ar[d]_{F(f)}&G(X)\ar[d]^{G(f)}\\
         F(Y)\ar[r]^{\pi_Y}&G(Y)
      }
\end{displaymath}

Hence, by considering the family of functors from $\Cc$ to $\Dd$ and the morphisms of such functors, we get a new category, denoted by $\Fct(\Cc,\Dd)$ or simply $[\Cc,\Dd]$, or $\Dd^{\Cc}$.
\end{defn}
  \begin{rem}
    If $\Cc$ and $\Dd$ are both \emph{preadditive} categories, then we can consider the category of all additive functors from $\Cc$ to $\Dd$, denoted by $\Add(\Cc,\Dd)$.
  \end{rem}
  \begin{exam}
    Any ring $R$ can be considered as a one-object preadditive category; the category of left modules over $R$ is the same as the additive functor category $\Add(R,\Ab)$, and the category of right $R-$modules is $\Add(R^{\op},\Ab)$. Because of this example, for any preadditive category $\Cc$, the category $\Add(\Cc,\Ab)$ is sometimes called the ``category of \termin[left modules]{module!over a preadditive category} over $\Cc$'' and $\Add(\Cc^{\op},\Ab)$ is the category of \emph{\red right modules} over $\Cc$.
  \end{exam}

  Most constructions that can be carried out in $\Dd$ can also be carried out in $[\Cc,\Dd]$ by performing them ``componentwise'', separately for each object in $\Cc$.

  For instance, if any two objects $X$ and $Y$ in $\Dd$ have a product $X\times Y$, then any two functors $F$ and $G$ in $[\Cc,\Dd]$ have a product $F\times G$, defined by $(F\times G)(c) = F(c)\times G(c)$ for every object $c$ in $\Cc$.

  Similarly, if $\eta_c\colon F(c)\To G(c)$ is a natural transformation and each $\eta_c$ has a kernel $K_c$ in the category $\Dd$, then the kernel of $\eta$ in the functor category $[\Cc,\Dd]$ is the functor $K$ with $K(c) = K_c$ for every object $c$ in $\Cc$.

  As a consequence we have the general rule of thumb that the functor category $[\Cc,\Dd]$ shares most of the ``nice'' properties of $\Dd$:
  \begin{itemize}
    \item if $\Dd$ is complete (or cocomplete), then so is $[\Cc,\Dd]$;
    \item if $\Dd$ is an abelian category, then so is $[\Cc,\Dd]$;
    \item if $\Cc$ is any small category, then the category $[\Cc,\Set]$ of presheaves is a \emph{topos}.
  \end{itemize}
  \begin{prop}
    the categories of directed graphs, $G-$sets and presheaves on a topological space $X$ are all complete and cocomplete topoi, and that the categories of representations of $G$, modules over the ring $R$, and presheaves of abelian groups on a topological space $X$ are all abelian, complete and cocomplete.
  \end{prop}

\begin{prop}
  Every natural transformation $\pi\colon F\To G$ defines a function which sends each arrow $f\colon A\To B$ of $\Cc$ to an arrow $\pi_f\colon F(A)\To G(B)$ of $\Dd$ in such a way that
  \begin{equation*}
    G(g)\circ\pi_f =\pi_{gf} = \pi_g\circ F(f)
  \end{equation*}
  for each composable pair $g,f$.
  Conversely, every such function $\pi$ comes from a unique natural transformation with $\pi_X=\pi_{1_X}$.
\end{prop}
\begin{rem}
  This gives an arrows only description of a natural transformation.
\end{rem}

  \begin{defn}
    An \termin{infranatural transformation} $\eta$ from $F$ to $G$ is simply a family of morphisms $\eta_X\colon F(X) \To G(X)$.
    Thus a natural transformation is an infranatural transformation for which $\eta_Y \circ F(f) = G(f) \circ \eta_X$ for every morphism $f\colon X\To Y$. The \termin{naturalizer} of $\eta$, $\nat(\eta)$, is the largest subcategory of $\Cc$ containing all the objects of $\Cc$ on which $\eta$ restricts to a natural transformation.
  \end{defn}

  \begin{defn}
    If, for every object $X$ in $\Cc$, the morphism $\eta_X$ is an isomorphism in $\Dd$, then $\eta$ is said to be a
    \termin[natural isomorphism]{natural!isomorphism}.
    Two functors $F$ and $G$ are said to be \termin{naturally isomorphic} if there exists a natural isomorphism from $F$ to $G$.
  \end{defn}

  \begin{exam}
    Statements such as
    \begin{quote}
      ``Every group is \emph{naturally isomorphic} to its opposite group''
    \end{quote}
    abound in modern mathematics.

    The content of the above statement is:
    \begin{quote}
      ``The identity functor $\Id \colon \Grp \To \Grp$ is \emph{naturally isomorphic} to the opposite functor $\op \colon \Grp \To \Grp$.''
    \end{quote}
  \end{exam}

  \begin{exam}
    If $K$ is a field, then for every vector space $V$ over $K$ we have a ``natural'' injective linear map $V \To V^{\ast\ast}$ from the vector space into its double dual. These maps are ``natural'' in the following sense: the double dual operation is a functor, and the maps are the components of a natural transformation from the identity functor to the double dual functor.
  \end{exam}

  \begin{defn}
    A particular map between particular objects may be called an \termin{unnatural isomorphism} (or ``this isomorphism is not natural'') if the map cannot be extended to a natural transformation on the entire category.
  \end{defn}
  \begin{rem}
    Some authors distinguish notationally, using $\cong$ for a natural isomorphism and $\approx$ for an unnatural isomorphism, reserving $=$ for equality (usually equality of maps).
  \end{rem}
  \begin{exam}
    In group theory or module theory, a given decomposition of an object into a direct sum is ``not natural'', or rather ``not unique'', as automorphisms exist that do not preserve the direct sum decomposition
  \end{exam}

    \begin{exam}
    \textbf{fundamental group of torus}

    As an example of the distinction between the \emph{\red functorial statement} and \emph{\red individual objects}, consider homotopy groups of a product space, specifically the fundamental group of the torus.
    \begin{quote}
      The homotopy groups of a product space are \emph{naturally} the product of the homotopy groups of the components,
      \begin{equation*}
        \pi_n((X,x_0)\times(Y,y_0))\cong\pi_n(X,x_0)\times\pi_n(Y,y_0)
      \end{equation*}
      with the isomorphism given by projection onto the two factors, fundamentally because maps into a product space are exactly products of maps into the components.

      \emph{This is a \red functorial statement.}
    \end{quote}

    However, given the torus, which is abstractly a product of two circles, and thus has fundamental group isomorphic to $\ZZ^2$,
    but the splitting $\pi_1(T,t_0)\approx\ZZ\times\ZZ$ is not natural. Note the use of $\approx$, $\cong$, and $=$:
    \begin{equation*}
      \pi_1(T,t_0)\approx\pi_1(S^1,x_0)\times\pi_1(S^1,y_0)\cong\ZZ\times\ZZ=\ZZ^2
    \end{equation*}

    This abstract isomorphism with a product \emph{is \red not natural}, as some isomorphisms of $T$ do not preserve the product:
    the self-homeomorphism of $T$ (thought of as the quotient space $\RR^2/\ZZ^2$) given by
    $\begin{pmatrix}
      \begin{smallmatrix}
      1 & 1 \\
      0 & 1 \\
      \end{smallmatrix}
    \end{pmatrix}$
     (geometrically a \emph{Dehn twist} about one of the generating curves) acts as this matrix on $\ZZ^2$ (it's in the general linear group $\GL(\ZZ, 2)$ of invertible integer matrices), which does not preserve the decomposition as a product because it is not diagonal.

     However, if one is given the torus as a product --- equivalently, given a decomposition of the space --- then the splitting of the group follows from the general statement earlier.

     In categorical terms, the relevant category (preserving the structure of a product space) is ``maps of product spaces, namely a pair of maps between the respective components''.

     \emph{\red Naturality is a categorical notion}, and requires being very precise about exactly what data is given --- the torus as a space that happens to be a product (in the category of spaces and continuous maps) is different from the torus presented as a product (in the category of products of two spaces and continuous maps between the respective components).
    \end{exam}

    \begin{exam}
    \textbf{dual of a finite-dimensional vector space}

    Every finite-dimensional vector space is isomorphic to its dual space, but this isomorphism relies on an arbitrary choice of isomorphism (for example, via choosing a basis and then taking the isomorphism sending this basis to the corresponding dual basis).
    There is in general no natural isomorphism between a finite-dimensional vector space and its dual space.

    However, related categories (with additional structure and restrictions on the maps) do have a natural isomorphism.

    In this category (finite-dimensional vector spaces with a \emph{nondegenerate bilinear form}, maps linear transforms that respect the bilinear form), the dual of a map between vector spaces can be identified as a transpose.

    Often for reasons of geometric interest this is specialized to a subcategory, by requiring that the nondegenerate bilinear forms have additional properties, such as being \emph{symmetric} (\emph{orthogonal matrices}), \emph{symmetric} and \emph{positive definite} (\emph{inner product space}), \emph{symmetric sesquilinear} (\emph{Hermitian spaces}), \emph{skew-symmetric} and \emph{totally isotropic} (\emph{symplectic vector space}), etc. --- in all these categories a vector space is naturally identified with its dual, by the nondegenerate bilinear form.
    \end{exam}

\begin{defn}
  A functor $F\colon\mathcal{C}\to\mathcal{D}$ is an \termin[equivalence]{equivalence (functor)} of categories if there exists $G\colon\mathcal{D}\to\mathcal{C}$ such that:
  \begin{align*}
    G\circ F&\cong\id_{\mathcal{C}}\\
    F\circ G&\cong\id_{\mathcal{D}}
  \end{align*}
  If such a functor exist, say $\mathcal{C}$ and $\mathcal{D}$ are \termin[equivalent]{equivalent!category}, denoted by $\mathcal{C}\simeq\mathcal{D}$.
\end{defn}
\begin{rem}
  If two categories are equivalent, all results and concepts in one of them have their counterparts in the other one. This is why this notion of equivalence of categories plays an important role in Mathematics.
\end{rem}

The following properties are easy to check
\begin{thm}
  The functor $F\colon\mathcal{C}\to\mathcal{D}$ is an equivalence of categories if and only if $F$ is fully faithful and essentially surjective.
\end{thm}

\begin{prop}
For any categories $\mathcal{C},\mathcal{D},\mathcal{E}$,
  \begin{enumerate}[1)]
    \setlength{\itemindent}{2ex}
    \item $[\Cc,\Dd]^{\op}\simeq[\Cc^{\op},\Dd^{\op}]$
    \item $[\Cc\times\Dd,\Ee]\simeq[\Dd,[\Cc,\Ee]]\simeq[\Cc,[\Dd,\Ee]]$
  \end{enumerate}
\end{prop}

\subsection{Category of All Categories}

    Given functors and natural transformations:
      \begin{displaymath}
        \xymatrix{
           \Cc \ar@<1.5ex>[r]^{F}="1" \ar@<-1.5ex>[r]_{G}="2" & \Bb \ar@<1.5ex>[r]^{F'}="3" \ar@<-1.5ex>[r]_{G'}="4" & \Aa
           \ar"1";"2"^{\pi} \ar"3";"4"^{\tau}
        }
      \end{displaymath}
    we first have the composite functors $F'F$ and $G'G$ and a commutative square
     \begin{displaymath}
        \xymatrix{
           F'F(X)\ar[r]^{F'(\pi_X)}\ar[d]_{\tau_{F(X)}} & F'G(X)\ar[d]^{\tau_{G(X)}}\\
           G'F(X)\ar[r]^{G'(\pi_X)} & G'G(X)
        }
    \end{displaymath}

    Now define $(\tau\circ\pi)_X$ to be the diagonal of this square. Then $\tau\circ\pi$ is also a natural transformation. (Which will not be confused with the original composition of natural transformations: first, they are compositions in different categories; second, we will always use different nations. Indeed, we will use $\tau\pi$ to denote the original one.)

    It is easy to check that all functors\footnote{Of course, we need a restriction. For instance, the functors between small categories.} form a category $\Cat$ under this composition (the horizontal composition). Moreover, for any functor $F\colon\Cc\To\Dd$, the identity at $F$ in $\Cat$ coincide with in $[\Cc,\Dd]$.

    Apart from this, For any functors and natural transformations:
      \begin{displaymath}
        \xymatrix{
           \Cc \ar@<4ex>[r]^{}="1" \ar[r]^{}="3"_{}="2" \ar@<-4ex>[r]_{}="4"
           & \Bb \ar@<4ex>[r]^{}="5" \ar[r]^{}="7"_{}="6" \ar@<-4ex>[r]_{}="8" & \Aa
           \ar"1";"2"^{\pi} \ar"3";"4"^{\tau} \ar"5";"6"^{\pi'} \ar"7";"8"^{\tau'}
        }
      \end{displaymath}
    there is a identity (\termin{interchange law}):
    \begin{equation*}
      (\tau'\pi')\circ(\tau\pi) = (\tau'\circ\tau)(\pi'\circ\pi)
    \end{equation*}

    Conclusively, we have
    \begin{thm}
      $\Cat$ has two compositions satisfying the interchange law and share the same identities.
    \end{thm}

    \begin{cor}
      The horizontal composition $\circ$ is a functor
      \begin{equation*}
        \circ\colon \Aa^{\Bb}\times\Bb^{\Cc}\To\Aa^{\Cc}
      \end{equation*}
    \end{cor}

    More general,  we define \termin{double category} to be a category with two compositions satisfying interchange law. Further, a \termin{$2-$category} is a double category which the two compositions share the same identities.

    \begin{exam}
      Let $G$  be a topological group with identity element $e$, while $\sigma,\sigma',\tau,\tau'$ are continuous loops in $G$ at $e$.
      Define $\tau\circ\sigma$ to be the path $\sigma$ followed by the path $\tau$.
      Define $\tau\sigma$ to be the pointwise product of $\tau$ and $\sigma$.
      Then they form a $2-$category.
    \end{exam}

    \begin{thm}[Hilton-Eckmann]
      Let $S$ be a set with two binary operations
      \begin{equation*}
        \cdot\colon S\times S\To S \qquad   \circ\colon S\times S\To S
      \end{equation*}
      which both have the same unit element $e$ and satisfying the interchange law. Then $\cdot$ and $\circ$ are equal, and each is commutative.
    \end{thm}
    \begin{cor}
      The fundamental group of a topological group is abelian.
    \end{cor}

    \begin{prop}
      The functor category $\Dd^{\Cc}=[\Cc,\Dd]$ is itself a bijective $\Cat^{\op}\times\Cat\To\Cat$. The arrow function sends a pair of functors $F\colon\Dd\To\Dd'$ and $G\colon\Cc'\To\Cc$ to the functor
      \begin{equation*}
        F^G\colon \Dd^{\Cc} \To \Dd'^{\Cc'}
      \end{equation*}
      which defined on objects $S\in\Dd^{\Cc}$ as $F^G(S)=F\circ S\circ G$ and on arrows $\tau\colon S\To T$ in $\Dd^{\Cc}$ as $F^G(\tau)=F\circ\tau\circ G$.
      \begin{displaymath}
        \xymatrix{
           \Cc' \ar[r]^{G} & \Cc \ar@<1.5ex>[r]^{S}="3" \ar@<-1.5ex>[r]_{T}="4"
           & \Dd \ar[r]^{F} & \Dd'
           \ar"3";"4"^{\tau}
        }
      \end{displaymath}
    \end{prop}

    \begin{prop}\label{product and power}
      For categories $\Aa,\Bb,\Cc$ establish natural isomorphisms:
      \begin{equation*}
        (\Aa\times\Bb)^{\Cc}\simeq\Aa^{\Cc}\times\Bb^{\Cc}  \qquad  \Cc^{\Aa\times\Bb}\simeq(\Cc^{\Bb})^{\Aa}\simeq(\Cc^{\Aa})^{\Bb}
      \end{equation*}
    \end{prop}

\subsection{Yoneda Lemma}
\begin{defn}
  For a category $\mathcal{C}$, one can define two categories:
  \begin{align*}
    \dual{\Cc} & \defeq [\Cc^{\op},\mathbf{Set}] \\
    \codual{\Cc} & \defeq [\Cc,\mathbf{Set}]^{\op}
  \end{align*}
  and two functors:
  \begin{center}
  \parbox{0.40\linewidth}{\longmapdes{\mathcal{M}^{*}}{\mathcal{C}}{\dual{\mathcal{C}}}{X}{\mathcal{M}^{X}}}  \parbox{0.40\linewidth}{\longmapdes{\mathcal{M}_{*}}{\mathcal{C}}{\codual{\mathcal{C}}}{X}{\mathcal{M}_{X}}}
  \end{center}
  where $\mathbf{Set}$ denoted the category of sets, $\mathcal{M}^X$ denoted the functor from $\mathcal{C}$ to $\mathbf{Set}$ which maps $Y\in\ob\mathcal{C}$ to the set $\Hom(Y,X)$, and $\mathcal{M}_X$ is similar.
\end{defn}
\begin{thm}[Yoneda Lemma]\index{Yoneda!Lemma}
  For $F\in\ob\dual{\mathcal{C}}$ and $X\in\ob\mathcal{C}$, there is an isomorphism
  \begin{equation*}
    \Hom_{\dual{\mathcal{C}}}(\mathcal{M}^{X},F)\cong F(X)
  \end{equation*}
  which, moreover, is natural in both $F$ and $X$.
\end{thm}
\begin{proof}
  For any $\alpha\in\Hom_{\dual{\mathcal{C}}}(\mathcal{M}^{X},F)$, let $\Phi(\alpha)$ be $\alpha_X(\id_X)$. Which defined a map to $F(X)$. Conversely, for any $a\in F(X)$, let $\Psi(a)$ be a natural transformation such that for any $Y\in\ob\mathcal{C}$ and morphism $f\colon Y\to X$, $\Psi(a)_Y(f)=F(f)(a)$.

  To show that $\Psi(a)$ is natural, consider the following diagram for each $f\colon Y\to X$
 \begin{displaymath}
      \xymatrix{
         \mathcal{M}^X(X)\ar[r]^-{\Psi(a)_X}\ar[d]_{\mathcal{M}^X(f)}&F(X)\ar[d]^{F(f)}\\
         \mathcal{M}^X(Y)\ar[r]^-{\Psi(a)_Y}&F(Y)
      }
\end{displaymath}
  Let$g\in\mathcal{M}^X(X)$, then
  \begin{align*}
    F(f)\Psi(a)_X(g) &=F(f)F(g)(a) \\
    &=F(gf)(a) \\
    &=\Psi(a)_Y(gf) \\
    &=\Psi(a)_Y(\mathcal{M}^X(f)(g))
  \end{align*}
  Hence the diagram commutes and $\Psi(a)$ is natural.

  For each $\alpha\in\Hom_{\dual{\mathcal{C}}}(\mathcal{M}^{X},F)$, consider the commutative diagram below
 \begin{displaymath}
      \xymatrix{
         \mathcal{M}^X(X)\ar[r]^-{\alpha_X}\ar[d]_{\mathcal{M}^X(f)}&F(X)\ar[d]^{F(f)}\\
         \mathcal{M}^X(Y)\ar[r]^-{\alpha_Y}&F(Y)
      }
\end{displaymath}
  It tells us that
  \begin{align*}
    \Psi(\Phi(\alpha))_Y(f) &=F(f)(\Phi(\alpha)) \\
    &=F(f)(\alpha_X(\id_X)) \\
    &=\alpha_Y(\mathcal{M}^X(f)(\id_X)) \\
    &=\alpha_Y(f)
  \end{align*}
  Thus $\Psi(\Phi(\alpha))=\alpha$. Hence $\Psi\circ\Phi=\id_{\Hom_{\dual{\mathcal{C}}}(\mathcal{M}^{X},F)}$.
  \begin{align*}
    \Phi(\Psi(a))&=\Psi(a)_X(\id_X) \\
    &=F(\id_X)(a) \\
    &=\id_{F(X)}(a)=a
  \end{align*}
  Hence $\Phi\circ\Psi=\id_{F(X)}$. Then, $\Phi$ is isomorphism.

  To show this isomorphism is natural in $F$, consider the diagram below for any natural transformation $\pi\colon F\to G$
   \begin{displaymath}
      \xymatrix{
         \Hom_{\dual{\mathcal{C}}}(\mathcal{M}^{X},F)\ar[r]^-{\Phi}\ar[d]_{\Hom_{\dual{\mathcal{C}}}(\mathcal{M}^{X},\pi)}&F(X)\ar[d]^{\pi_X}\\
         \Hom_{\dual{\mathcal{C}}}(\mathcal{M}^{X},G)\ar[r]^-{\Phi}&G(X)
      }
  \end{displaymath}
  For any $\alpha\in\Hom_{\dual{\mathcal{C}}}(\mathcal{M}^{X},F)$, we have
  \begin{align*}
    \pi_X\Phi(\alpha)&=\pi_X\alpha_X(\id_X) \\
    &=(\pi\alpha)_X(\id_X) \\
    &=\Phi(\pi\alpha) \\
    &=\Phi\Hom_{\dual{\mathcal{C}}}(\mathcal{M}^{X},\pi)(\alpha)
  \end{align*}
  Thus the diagram commutes and $\Phi$ is natural in $F$.

  Similar, for each $f\colon Y\to X$, consider the diagram below
  \begin{displaymath}
      \xymatrix{
         \Hom_{\dual{\mathcal{C}}}(\mathcal{M}^{X},F)\ar[r]^-{\Phi}\ar[d]_{\Hom_{\dual{\mathcal{C}}}(\mathcal{M}^{f},F)}&F(X)\ar[d]^{F(f)}\\
         \Hom_{\dual{\mathcal{C}}}(\mathcal{M}^{Y},F)\ar[r]^-{\Phi}&F(Y)
      }
  \end{displaymath}
  For any $\alpha\in\Hom_{\dual{\mathcal{C}}}(\mathcal{M}^{X},F)$, we have
  \begin{align*}
    F(f)\Phi(\alpha)&=F(f)\alpha_X(\id_X) \\
    &=\alpha_Y\mathcal{M}^{X}(f)(\id_X) \\
    &=\alpha_Y(\id_Xf) \\
    &=\alpha_Y(f)
  \end{align*}
  On the other hand
  \begin{align*}
    \Phi\Hom_{\dual{\mathcal{C}}}(\mathcal{M}^{f},F)(\alpha)&=\Phi(\alpha\mathcal{M}^{f}) \\
    &=(\alpha\mathcal{M}^{f})_Y(\id_Y) \\
    &=\alpha_Y\mathcal{M}^{f}_Y(\id_Y) \\
    &=\alpha_Y(f\id_Y) \\
    &=\alpha_Y(f)
  \end{align*}
  Thus the diagram commutes and $\Phi$ is natural in $X$.
\end{proof}
\begin{cor}
  The functor $\mathcal{M}^{*}$ is fully faithful.
\end{cor}
\begin{proof}
  For any $X,Y\in\ob\mathcal{C}$, we have
 \begin{align*}
   \Hom_{\dual{\mathcal{C}}}(\mathcal{M}^{X},\mathcal{M}^{Y})&\cong \mathcal{M}^{Y}(X)\\
   &= \Hom_{\mathcal{C}}(X,Y)
 \end{align*}
\end{proof}
\begin{rem}
One calls $\mathcal{M}^{*}$ the \termin[Yoneda embedding]{Yoneda!embedding}, sometimes denoted by $Y_{\Cc}$. Hence, one may consider $\mathcal{C}$ as a full subcategory of $\dual{\mathcal{C}}$. In particular, for $X\in\ob\mathcal{C}$, $\mathcal{M}^{X}$ determines $X$ up to unique isomorphism, that is, an isomorphism $\mathcal{M}^{X}\cong\mathcal{M}^{Y}$ determines a unique isomorphism $X\cong Y$.
\end{rem}
  \begin{rem}
     Some authors define an \termin{embedding} to be a fully faithful functor. Such a functor is necessarily injective on objects up-to-isomorphism. For instance, the Yoneda embedding is an embedding in this sense.
  \end{rem}
\begin{exam}
  If $\mathcal{C}$ has products and coproducts, then there is a canonical isomorphism
  \begin{equation*}
    (A\times B)+(A\times C)\cong A\times(B+C).
  \end{equation*}
  To prove this, by the remark above, it is enough to prove
  \begin{equation*}
    \Hom(X,(A\times B)+(A\times C))\cong\Hom(X,A\times(B+C))
  \end{equation*}
  for each $X\in\ob\mathcal{C}$ and this isomorphism is natural in $X$. Which is easy to check.
\end{exam}
\begin{cor}
  Let $\mathcal{C}$ be a category and let $f\colon X\to Y$ be a morphism in $\mathcal{C}$.
  \begin{enumerate}[1)]
    \setlength{\itemindent}{2ex}
    \item Assume that for any $Z\in\mathcal{C}$, the map $\Hom_{\mathcal{C}}(Z,X)\markar{f\circ}\Hom_{\mathcal{C}}(Z,Y)$ is bijective. Then $f$ is an isomorphism.
    \item Assume that for any $Z\in\mathcal{C}$, the map $\Hom_{\mathcal{C}}(X,Z)\markar{\circ f}\Hom_{\mathcal{C}}(Y,Z)$ is bijective. Then $f$ is an isomorphism.
  \end{enumerate}
\end{cor}

  \begin{rem}
    If $\Cc$ is a preadditive category, then the Yoneda's lemma yields a full embedding of $\Cc$ into the functor category $\Add(\Cc^{\op},\Ab)$.
    So $\Cc$ naturally sits inside an abelian category.
  \end{rem}

\subsection{Representable Functors}
\begin{defn}
  One says that a functor $F\colon\op{\mathcal{C}}\to\mathbf{Set}$ is \emph{\red  representable}\index{representable functor} if there exists $X\in\ob\mathcal{C}$ such that $F(Y)\cong\Hom_{\mathcal{C}}(Y,X)$ functorially in $Y\in\mathcal{C}$. In other words, $F\cong\mathcal{M}^X$ in $\dual{\mathcal{C}}$. Such an object $X$ is called a \emph{\red  representative}\index{representative} of $F$. Similarly, a functor $G\colon \mathcal{C}\to\mathbf{Set}$ is \emph{\red  representable} if there exists $X\in\ob\mathcal{C}$ such that $G(Y)\cong\Hom_{\mathcal{C}}(X,Y)$ functorially in $Y\in\mathcal{C}$.
\end{defn}
\begin{rem}
  It is important to notice that the isomorphisms above determine $X$ up to unique isomorphism. More precisely, given two isomorphisms $F\markar{\cong}\mathcal{M}^X$ and $F\markar{\cong}\mathcal{M}^{X'}$ there exists a unique isomorphism $\theta:X\markar{\cong}X'$ making the following diagram commutative:
 \begin{displaymath}
      \xymatrix{
         &F\ar[ld]_{\cong}\ar[rd]^{\cong}&\\
         \mathcal{M}^X\ar[rr]^{\mathcal{M}^{*}(\theta)}_{\cong}&&\mathcal{M}^{X'}
      }
\end{displaymath}
\end{rem}

  \begin{defn}
    Let $\Vv$ be a category. A $\Vv-$valued presheaf $\Fff$ on a category $\Cc$ is a functor $\Fff\colon\Cc^{\op}\To\Vv$.
     Often presheaf is defined to be a $\Set-$valued presheaf.
    A morphism of presheaves is defined to be a natural transformation of functors.
    This makes the collection of all presheaves into a category, often written $\widehat{\Cc}$.
    A functor into $\widehat{\Cc}$ is sometimes called a \termin{profunctor}.
  \end{defn}
  \begin{prop}
    A locally small category $\Cc$ embeds fully and faithfully into the category $\widehat{\Cc}$ of $\Set-$valued presheaves via the Yoneda embedding $Y_{\Cc}$ which to every object $A$ of $\Cc$ associates the hom-set $\Hom_{\Cc}(-,A)$.
  \end{prop}
  \begin{prop}
    The presheaf category $\widehat{\Cc}$ is (up to equivalence of categories) the free colimit completion of the category $\Cc$.
  \end{prop}

\newpage\section{Objects}
\subsection{Initial and Terminal Objects}
\begin{defn}
  Let $\Cc$ be a category. An \termin[initial object]{initial!object} of $\Cc$ is an object $I$ in $\Cc$ such that for every object $X$ in $\Cc$, there exists precisely one morphism $I\to X$. Dually, an object $T$ is \termin[terminal object]{terminal!object} if for every object $X$ in $\Cc$, there exists a single morphism $X\to T$. If an object is both initial and terminal, it is called a \termin[zero object]{zero!object} or \termin{null object}.
\end{defn}
\begin{rem}
  It is easy to see that the initial object and terminal object are unique up to isomorphism. Such universal properties will be detail in the limit theory later.
\end{rem}
  \begin{rem}
    If $\Cc$ has a zero object $0$, then given two objects $X$ and $Y$ in $\Cc$, there are canonical morphisms $f \colon 0 \To X$ and $g \colon Y \To 0$.
    Then, $f\circ g$ is a zero morphism in $\Hom_{\Cc}(Y, X)$. Thus, every category with a zero object is also a category with zero morphisms given by the composition $0_{XY} \colon X \To 0 \To Y$.
  \end{rem}

  Not every category has terminal objects, for example:
  \begin{exam}
    The category of infinite groups do not have a terminal object: given any infinite group $G$ there are infinitely many morphisms $\ZZ\to G$, so $G$ cannot be terminal.
  \end{exam}

\subsection{Subobjects and Quotient Objects}
\begin{defn}
  Let $A,B\in$ be objects in $\mathcal{C}$. If there exist a monomorphism $f\colon A\to B$, then we call $(A,f)$ a \emph{\red  subobject}\index{subobject} of $B$. If there exist an epimorphism $f\colon B\to A$, then we call $(A,f)$ a \emph{\red  quotient object}\index{quotient object} of $B$.
\end{defn}
\begin{warn}
  Notice that the notation of subobject and quotient object may not be suitable abstract of  sub- and quotient in usual sense. For example, Consider $\mathbf{Top}$, the subobjects of an object are not just the subspaces, this concept mixes others. The same story happened in quotient objects.
\end{warn}

\begin{exam}
  In $\mathbf{Top}$, every epimorphism is surjective. However, a quotient object may still not be a quotient space. In fact, for every topological space, the identity map from itself to the trivial topological space on the same underlying set is epimorphism.
\end{exam}

\begin{defn}
  An \emph{\red  extremal monomorphism}\index{extremal monomorphism} is a monomorphism that cannot be nontrivially factored through an epimorphism. In another word, if $m=g\circ e$ with $e$ an epimorphism, then $e$ is an isomorphism. A subobject composed by an object with an extremal monomorphism is called an \emph{\red  extremal subobject}.
\end{defn}
\begin{exam}
  The extremal subobject in $\mathbf{Top}$ is just the subspace with its inclusion map.
\end{exam}
\begin{rem}
  Notice that, in category theory, when we use the word ``is'', is actually under the meaning of ``up to isomorphism''. However, a bijective morphism may not be isomorphism. Which makes lots trouble, especially in epi- case: the concept of quotient object in the category of rings and topological spaces totally lose shape. Even consider special type of quotient objects like extremal ones may not work. The epimorphisms may be very mysterious.
\end{rem}
\begin{rem}
  Given two subobjects $(A,f),(A',f')$ of $B$, the morphism $g$ from $(A,f)$ to $(A',f')$ is the morphism $g$ (in fact, it is unique) in $\mathcal{C}$ such that the following diagram commutes:
  \begin{displaymath}
      \xymatrix{
         A\ar[rd]_{f}\ar[rr]^{g}&&A'\ar[ld]^{f'}\\
         &B&
      }
  \end{displaymath}
  Thus we get a category $\Sub_{\mathcal{C}}(B)$. Similar, we get $\Quot_{\mathcal{C}}(B)$.
\end{rem}
\subsection{Free Objects and Generators}
\begin{defn}
  A \termin{concrete category} $(\mathcal{C},U)$ is a category $\mathcal{C}$ together with a faithful functor $U\colon\mathcal{C}\to\mathbf{Set}$, named
  \termin{forgetful functor}.
  A category $\mathcal{C}$ is called \termin{concretizable} if there exists such a forgetful functor.
\end{defn}
Unlike the literal meaning, a concrete category may be very abstract. In fact, we have
\begin{exam}
 Let $\mathcal{C}$ be any small category, then there exists a faithful functor
 \mapdes{P\colon\dual{\mathcal{C}}}{\mathbf{Set}}{X}{\coprod_{c \in \mathrm{ob}C} X(c)}
 By composing this with the Yoneda embedding $\mathcal{M}^{*}\colon\mathcal{C}\to\dual{\mathcal{C}}$, one obtains a faithful functor $\mathcal{C}\to\mathbf{Set}$.
\end{exam}
Not every category, whose objects are based on sets, are concretizable. For example
\begin{exam}
 The homotopy category of topological spaces $\mathbf{hTop}$, which has same objects as $\mathbf{Top}$ but its morphisms are homotopy classes of continuous functions, is an example of a category that is not concretizable. The fact that there does not exist any faithful functor from $\mathbf{hTop}$ to $\mathbf{Set}$ was first proven by Peter Freyd, see \cite{Freyd70homotopyis}. In the same article, Freyd cites an earlier result that the category of ``small categories and natural equivalence-classes of functors'' also fails to be concretizable.
\end{exam}
A category $\mathcal{C}$ may admit several faithful functors into $\mathbf{Set}$. Hence there may be several concrete categories $(\mathcal{C},U)$ all corresponding to the same category $\mathcal{C}$.
\begin{exam}
  For technical reasons, the category $\mathbf{Ban}_1$ of Banach spaces and linear contractions is often equipped not with the ``obvious'' forgetful functor but the functor $U_1\colon \mathbf{Ban}_1\to\mathbf{Set}$ which maps a Banach space to its (closed) unit ball.
\end{exam}
Notice that, the forgetful functor may map different objects to the same set and, if this occurs, it will also map different morphisms to the same function, which is not contradictory to faithful.
\begin{exam}
  A set $X$ can be equipped different topologies, hence become different objects in $\mathbf{Top}$, and their identity maps are different morphisms. However, the usual forgetful functor maps them to the same set $X$ and their identity maps to one $\id_X$.
\end{exam}
\begin{defn}
  A left adjoint functor $F$ of a forgetful functor $U$ is called the \termin[free functor]{free!functor}. Let $S$ be a set, then $F(S)$ is called the \termin[free object]{free!object} generated by $S$.
\end{defn}
\begin{rem}
  Since $F$ is the adjoint functor of $U$, there must be a natural transformation $\eta\colon\id_{\mathbf{Set}}\to U\circ F$. More explicitly, $F$ is, up to isomorphisms in $\mathcal{C}$, characterized by the following universal property:
\begin{quote}
Whenever $T\in\ob\mathcal{C}$, and $f\colon S\to U(T)$ is a function, then there is a unique $\mathcal{C}-$morphism $g\colon F(S)\to T$ such that $U(g)\circ\eta(S)=f$.
\end{quote}
\end{rem}
Since the last section of Chapter I of Lang's textbook has discussed the free functor of $\mathbf{Grp}$ in detail, we will not repeat them here.

A related but different concept is the generator
\begin{defn}
  A \termin[generator]{generator!of category} (or \termin{separator}) of a category $\mathcal{C}$ is an object $G$, such that for any two different morphisms $f,g\colon X\to Y$, there exist one morphism $h\colon G\to H$ such that $f\circ h\neq g\circ h$.
\end{defn}
\begin{exam}
  $\ZZ$ is a generator in $\mathbf{Ab}$. Similarly, the one-point set is a generator for $\mathbf{Set}$.
\end{exam}
\newpage\section{Limit Theory}
\subsection{Cones and Limits}
\begin{defn}
  Let $\Jj$ and $\Cc$ be categories. A \termin{diagram of type $\Jj$} or a \termin{$\Jj-$diagram} in $\Cc$ is a functor $D\colon \Jj\to \Cc$.
  The category $\Jj$ is called the \termin{index category} or the \termin{scheme} of the diagram $D$.
  For $j$ in the index category, we will write $D(j)$ in the form $D_j$.

  A \termin{cone} to a diagram $D$ consists of an object $C$ in $\Cc$ and a family of arrows in $\Cc$,
  \begin{equation*}
    c_j\colon C\longrightarrow D_j, \forall j\in\ob\mathcal{J}
  \end{equation*}
  such that for each arrow $\alpha\colon i\to j$ in $\Jj$, the following triangle commutes.
  \begin{displaymath}
      \xymatrix{
         &C\ar[ld]_{c_i}\ar[rd]^{c_j}&\\
         D_i\ar[rr]^{D_{\alpha}}&&D_j
      }
  \end{displaymath}

  A morphism of cones
  \begin{equation*}
    \vartheta\colon (C,c_j)\longrightarrow (C',c'_j)
  \end{equation*}
  is an arrow $\vartheta$ in $\Cc$, making each triangle
  \begin{displaymath}
      \xymatrix{
         C\ar[rd]_{c_j}\ar[rr]^{\vartheta}&&C'\ar[ld]^{c'_j}\\
         &D_j&
      }
  \end{displaymath}
  commute.

  Finally, cones to $D$ with morphisms between them form a category $\mathbf{Cone}(D)$ (or denote $\Delta\downarrow D$).
\end{defn}
\begin{defn}
  A \termin{limit} for a diagram $D\colon \mathcal{J}\to \mathcal{C}$ is a terminal object in $\mathbf{Cone}(D)$. In particular, a finite limit is a limit for a diagram on a finite index category $\mathcal{J}$.
\end{defn}
\begin{rem}
One often denote a limit in the form
\begin{equation*}
  p_i\colon\mathop{\underleftarrow{\lim}}\limits_{j}D_j \longrightarrow D_i
\end{equation*}
When $\{p_i\}$ are obvious, one may simply call the object as the limit.
\end{rem}
\begin{exam}
  Let $\mathbf{0}$ be the empty category, then in any category $\mathcal{C}$, there is only one diagram of type $\mathbf{0}$: the empty one. A cone to the empty diagram is essentially just an object of
   $\mathcal{C}$. The limit of the empty diagram is just the \emph{\red terminal} object in $\mathcal{C}$.
\end{exam}
\begin{exam}
  Let $\mathcal{C}$ be a small category, $\id_{\mathcal{C}}\colon\mathcal{C}\to\mathcal{C}$ is the identity functor. If $\mathcal{C}$ has initial object $I$, then $I$ is the limit of $\id_{\mathcal{C}}$. Conversely, if $\{p_B\colon A\to B\mid B\in\ob\mathcal{C}\}$ is the limit of $\id_{\mathcal{C}}$, then its easy to see that it is the \emph{\red initial} object in $\mathcal{C}$.
\end{exam}
\begin{exam}
  Take $\mathcal{J}=\{1,2\}$ the discrete category with two objects and no nonidentity arrows. A diagram $D\colon\mathcal{J}\to\mathcal{C}$ hence is a pair of objects $D_1, D_2\in\mathcal{C}$. A cone to $D$ is an object $C$ equipped with arrows
  \begin{displaymath}
      \xymatrix{
         D_1&C\ar[l]_-{c_1}\ar[r]^-{c_2}&D_2
      }
  \end{displaymath}
  The limit of $D$ is just the \emph{\red  product} of $D_1$ and $D_2$ in $\mathcal{C}$.
\end{exam}
\begin{exam}\label{limit.2}
  Take $\mathcal{J}$ to be the following category:
  \begin{displaymath}
      \xymatrix{
         1\ar@<0.5ex>[r]^{\alpha}\ar@<-0.5ex>[r]_{\beta} &2
      }
  \end{displaymath}
  Hence a diagram $D$ of type $\mathcal{J}$ looks like
  \begin{displaymath}
      \xymatrix{
         D_1\ar@<0.5ex>[r]^{D_{\alpha}}\ar@<-0.5ex>[r]_{D_{\beta}} &D_2
      }
  \end{displaymath}
  The limit of $D$ is the \emph{\red  equalizer} of $D_{\alpha},D_{\beta}$.
\end{exam}
\begin{exam}
  Let $(I,\leqslant)$ be a \termin{filtered partially ordered set} (FPOS), which means that for each two elements $i,j$, there exist an element $k\in I$ such that $k\leqslant i,k\leqslant j$. Treat $I$ as a category, in any category $\mathcal{C}$, a diagram $D\colon I\to\mathcal{C}$ satisfy that for any $i\leqslant j\leqslant k$ in $I$, $D_{j\leqslant k}D_{i\leqslant j}=D_{i\leqslant k}$ is called an \termin[inverse system]{inverse!system} and its limit is called an \termin[inverse limit]{inverse!limit}, or \termin[projective limit]{projective!limit}.
\end{exam}
The limit of a diagram sometimes works like monomorphism, although each arrow in the cone may not be injective.
\begin{prop}\label{collective inj}
  Let $\{p_j\colon A\to D_j\mid j\in\ob\mathcal{J}\}$ be the limit of diagram $D\colon\mathcal{J}\to\mathcal{C}$. For any $f,g\colon B\to A$, if $p_jf=p_jg, \forall j\in\ob\mathcal{J}$, then $f=g$.
\end{prop}
\begin{exam}
  Let $\{(A_i,f_i)\mid i\in I\}$ be a family of subobject of $A$ in $\mathcal{C}$. Treat $\{f_i\colon A_i\to A\mid i\in I\}$ as a subcategory of $\mathcal{C}$. hence its inclusion functor is a diagram in $\mathcal{C}$. If such a diagram has limit $C$, then then arrow from $C$ to $A$ can be determined by each arrow $\alpha_i$ from $C$ to $A_i$. Use the proposition above, it is easy to check that $f_i\alpha_i\colon C\to A$ is injective, hence $(C,f_i\alpha_i)$ is also a subobject of $A$, called the \termin[intersection]{intersection!of subobjects} of $\{(A_i,f_i)\mid i\in I\}$.
\end{exam}
\subsection{Co-cones and Colimits}
Dually, we have corresponding concepts
\begin{defn}
  A \termin{co-cone} to a diagram $D$ consists of an object $C$ in $\mathcal{C}$ and a family of arrows in $\mathcal{C}$,
  \begin{equation*}
    c_j\colon D_j\longrightarrow C, \forall j\in\ob\mathcal{J}
  \end{equation*}
  such that for each arrow $\alpha\colon i\to j$ in $\mathcal{J}$, the following triangle commutes.
  \begin{displaymath}
      \xymatrix{
         D_i\ar[rr]^{D_{\alpha}}\ar[rd]_{c_i}&&D_j\ar[ld]^{c_j} \\
         &C&
      }
  \end{displaymath}

  A morphism of co-cones
  \begin{equation*}
    \vartheta\colon (C,c_j)\longrightarrow (C',c'_j)
  \end{equation*}
  is an arrow $\vartheta$ in $\mathcal{C}$, making each triangle
  \begin{displaymath}
      \xymatrix{
         &D_j\ar[ld]_{c_j}\ar[rd]^{c'_j}& \\
         C\ar[rr]^{\vartheta}&&C'
      }
  \end{displaymath}
  commute.

  Finally, co-cones to $D$ with morphisms between them form a category $\mathbf{Cocone}(D)$ (or denote $D\downarrow\Delta$).
\end{defn}
\begin{defn}
  A \termin{colimit} for a diagram $D\colon \mathcal{J}\to \mathcal{C}$ is a initial object in $\mathbf{Cocone}(D)$. In particular, a finite colimit is a colimit for a diagram on a finite index category $\mathcal{J}$.
\end{defn}
\begin{rem}
One often denote a colimit in the form
\begin{equation*}
  k_i\colon\mathop{\underrightarrow{\lim}}\limits_{j}D_j \longleftarrow D_i
\end{equation*}
When $\{k_i\}$ are obvious, one may simply call the object as the colimit.
\end{rem}
\begin{exam}
  Let $\mathcal{C}$ be a small category, the colimit of the identity functor is just the \emph{\red terminal} object in $\mathcal{C}$.
\end{exam}
\begin{exam}
  Take $\mathcal{J}=\{1,2\}$ the discrete category with two objects and no nonidentity arrows. A diagram $D\colon\mathcal{J}\to\mathcal{C}$ hence is a pair of objects $D_1, D_2\in\mathcal{C}$. A co-cone to $D$ is an object $C$ equipped with arrows
  \begin{displaymath}
      \xymatrix{
         D_1\ar[r]^-{c_1}&C&D_2\ar[l]_-{c_2}
      }
  \end{displaymath}
  The colimit of $D$ is just the \emph{\red  coproduct} of $D_1$ and $D_2$ in $\mathcal{C}$.
\end{exam}
\begin{exam}
  Take noations as in \ref{limit.2}, the colimit of $D$ is the \emph{\red  coequalizer} of $D_{\alpha},D_{\beta}$.
\end{exam}
\begin{exam}
  Let $(I,\leqslant)$ be a \termin{directed partially ordered set} (DPOS), which means that for each two elements $i,j$, there exist an element $k\in I$ such that $i\leqslant k,j\leqslant k$. Treat $I$ as a category, in any category $\mathcal{C}$, a diagram $D\colon I\to\mathcal{C}$ satisfy that for any $i\leqslant j\leqslant k$ in $I$, $D_{j\leqslant k}D_{i\leqslant j}=D_{i\leqslant k}$ is called an \termin[direct system]{direct!system} and its colimit is called an \termin[direct limit]{direct!limit}, or \termin{inductive limit}.
\end{exam}

The colimit of a diagram sometimes works like epimorphism, although each arrow in the cone may not be surjective.
\begin{prop}\label{collective surj}
  Let $\{k_j\colon D_j\to A\mid j\in\ob\mathcal{J}\}$ be the limit of diagram $D\colon\mathcal{J}\to\mathcal{C}$. For any $f,g\colon A\to B$, if $fk_j=gk_j, \forall j\in\ob\mathcal{J}$, then $f=g$.
\end{prop}
\begin{exam}
  Let $\{(A_i,f_i)\mid i\in I\}$ be a family of quotient object of $A$ in $\mathcal{C}$. Treat $\{f_i\colon A\to A_i\mid i\in I\}$ as a subcategory of $\mathcal{C}$. hence its inclusion functor is a diagram in $\mathcal{C}$. If such a diagram has colimit $C$, then then arrow from $A$ to $C$ can be determined by each arrow $\alpha_i$ from $A_i$ to $C$. Use the proposition above, it is easy to check that $\alpha_if_i\colon A\to C$ is surjective, hence $(C,f_i\alpha_i)$ is also a quotient object of $A$, called the \termin{cointersection} of $\{(A_i,f_i)\mid i\in I\}$.
\end{exam}

\begin{prop}\label{Hom-dir.lim}
  Colimits are linked to limits via
  \begin{equation*}
    \Hom(\dirlim_{\Jj} X_i, Y) = \invlim_{\Jj^{\op}} \Hom(X_i,Y)
  \end{equation*}
\end{prop}
  \begin{proof}
    For any connection morphism $\phi^i_j\colon X_i\To X_j$ in $(X_i)$, the corresponding connection map in $(\Hom(X_i,Y))$ is
    \mapdes{\Hom(X_j,Y)}{\Hom(X_i,Y)}{f}{f\circ\phi^i_j}
    Use this corresponding, the statement is easy to verify.
  \end{proof}

  A similar proposition is
  \begin{prop}\label{Hom-inv.lim}
    \begin{equation*}
      \Hom(X,\invlim_{\Jj} Y_i) = \invlim_{\Jj} \Hom(X,Y_i)
    \end{equation*}
  \end{prop}
  \begin{proof}
    For any connection morphism $\phi^i_j\colon Y_i\To Y_j$ in $(Y_i)$, the corresponding connection map in $(\Hom(X,Y_i))$ is
    \mapdes{\Hom(X,Y_i)}{\Hom(X,Y_j)}{f}{\phi^i_j\circ f}
    Use this corresponding, the statement is easy to verify.
  \end{proof}

\subsection{Kernels and Cokernels}
  \begin{defn}
    Let $\{f_i\}$ be a family of parallel morphisms, which can be view as a diagram. the \termin{equalizer} is the limit of the diagram.
    Dually, the \termin{coequalizer} is the colimit.
  \end{defn}
  \begin{rem}
    The correct diagram for the degenerate case with \emph{no morphisms} is slightly subtle:
    one might initially draw the diagram as consisting of the two objects $X,Y$ and no morphisms.
    This is incorrect, however, since the limit of such a diagram is the product of these two objects, rather than the equalizer.
    (And indeed products and equalizers are different concepts: the set-theoretic definition of them are different.)

    Instead, the appropriate insight is that every equalizer diagram is fundamentally concerned with the $X$, including $Y$ only because $Y$ is the codomain of morphisms which appear in the diagram.

    With this view, we see that if there are no morphisms involved, $Y$ does not make an appearance and the equalizer diagram consists of $X$ alone. The limit of this diagram is then any isomorphism to $X$.

    Similarly, the correct coequalizer diagram of \emph{no morphisms} consists of the codomain alone.
  \end{rem}

  \begin{defn}
    An equalizer of exactly two morphisms is sometimes called the \termin[difference kernel]{difference!kernel} of them.
  \end{defn}

  \begin{prop}
    Any equalizer is a monomorphism. Dually, any coequalizer is an epimorphism.
  \end{prop}
  \begin{rem}
    A monomorphism is said to be \termin[regular]{regular!monomorphism} if it is an equalizer of some set of morphisms.
    Dually, an epimorphism is said to be \termin[regular]{regular!epimorphism} if it is a coequalizer of some set of morphisms.
  \end{rem}

  \begin{defn}
    The \termin{kernel} of a morphism $f$ is the equaliser of $f$ and the parallel zero morphism. Dually, The \termin{cokernel} is the coequalizer of $f$ and the parallel zero morphism.
  \end{defn}

  \begin{defn}
    A monomorphism is called \termin[norm]{norm!monomorphism}, if it is a kernel of some morphism. Dually, an epimorphism is called \termin[norm]{norm!epimorphism}, if it is a cokernel of some morphism.
  \end{defn}

  \begin{exam}
    Coequalisers can be large: There are exactly two functors from the category $\mathbf{1}$ having one object and one identity arrow, to the category $\mathbf{2}$ with two objects and exactly one non-identity arrow going between them. The coequaliser of these two functors is the monoid of natural numbers under addition, considered as a one-object category. In particular, this shows that while every coequalising arrow is epic, it is not necessarily surjective.
  \end{exam}

  \begin{prop}
    The kernel of the limit is also the limit of the kernels. Dually, the cokernel of the colimit is also the colimit of the cokernels.
  \end{prop}
  \begin{proof}
     We start with a commutative diagram:
         \begin{displaymath}
            \xymatrix{
               \cdot\ar@{-->}[rd]|{u}\ar@/^/[rrd]^{t}\ar@{-->}@/_/[rdd]_{u_n}&&&\\
                & \cdot\ar[r]^{\ker}\ar[d]_p & \cdot\ar[r]^{f}\ar[d]_p & \cdot\ar[d]_p \\
                & \cdot\ar[r]^{\ker f_n} & \cdot\ar[r]^{f_n} & \cdot             }
          \end{displaymath}
    Here, $f$ is the limit of $f_n$, and $\ker$ is the limit of the kernels, we want to show that it is also the kernel of $f$.

    First, by the collective injectivity of limit (ref. \ref{collective inj}), the composition of $f$ and $\ker$ is equal to $0$.

    Then, for any $t$ such that $f\circ t=0$, we have $f_n\circ p\circ t=0$,
    hence there exist a unique morphism $u_n$ such that $\ker f_n\circ u_n = p\circ t$.
    By the definition of limit, there exist a unique morphism $u$ such that $u_n=p\circ u$.
    Hence $p\circ\ker\circ u=p\circ t$. By the collective injectivity again, $\ker\circ u=t$, which proves $\ker$ is the kernel of $f$.
  \end{proof}

\subsection{Products and Coproducts}
\begin{defn}
  Consider a diagram of type $\mathcal{J}$, where $\mathcal{J}$ is a discrete category. It looks like a family of objects without arrows between them. The limit of such a diagram is called the \termin{product} of these objects. Dually, the colimit of such a diagram is called the \termin{coproduct} of these objects.
\end{defn}
\begin{rem}
  We often denote the product and the coproduct of a family of objects $\{A_i\}_{i\in I}$ as $\prod A_i$ and $\coprod A_i$.
\end{rem}
\begin{exam}
  Since the nullary discrete category is the empty category, the \termin[nullary product]{nullary!product} is just the terminal object. Similar, the \termin[unary product]{unary!product} of any object is itself. Dually, we get \termin[nullary coproduct]{nullary!coproduct} and \termin[unary coproduct]{unary!coproduct}.
\end{exam}
\begin{prop}
  A category has finite products, which means that each family of finite objects has product, if and only if it has binary product and terminal object.
\end{prop}
\begin{proof}
  For $n\geqslant 2$, it is clear that $(\cdots((A_1\times A_2)\times A_3)\cdots\times A_n)$ is the $n-$ary product of $A_1,A_2\cdots,A_n$. Notice that the word ``finite'' include the nullary case, so we still need the existence of terminal object.
\end{proof}
\begin{defn}
  Let $\{f_i\colon A_i\to B_i\}_{i\in I}$ be family of morphisms in a category has products, the product of them is the only morphism $\Pi f_i\colon\Pi A_i\to\Pi B_i$ make the following diagram commutative
  \begin{displaymath}
      \xymatrix{
         \Pi A_i\ar@{-->}[r]^{\Pi f_i}\ar[d]&\Pi B_i\ar[d]\\
         A_j\ar[r]^{f_j}&B_j
      }
  \end{displaymath}
\end{defn}
\begin{prop}
  Product is a functor from $[I,\mathcal{C}]$ to $\mathcal{C}$.
\end{prop}
\begin{proof}
  It's easy to check by definition.
\end{proof}
\begin{rem}
  Similar, we can define coproduct of morphisms and check coproduct is a functor. More general, limit and colimit are functors.
\end{rem}
\begin{prop}
  Monomorphisms are stable under product, which means the product of a family of monomorphisms is also a monomorphism.
\end{prop}
\begin{prop}
  In a category has products, for any $i\in I$, let $(E_i,e_i)$ be the equalizer of $f_i,g_i\colon A_i\to B_i$, then, $(\Pi E_i,\Pi e_i)$ is the equalizer of $\Pi f_i,\Pi g_i\colon \Pi A_i\to\Pi B_i$.
\end{prop}
\begin{prop}
   Regular monomorphisms are stable under product, and so are isomorphisms.
\end{prop}
Dually, we have similar propositions for coproduct.

\subsection{Pullback and Pushout}
\begin{defn}
  Let $\Jj$ be
  \begin{displaymath}
      \xymatrix{
         \cdot\ar[r]&\cdot&\cdot\ar[l]
      }
  \end{displaymath}
  a limit for a $\Jj-$diagram is of the form
  \begin{displaymath}
      \xymatrix{
         A\ar[r]^{f}&C&B\ar[l]_{g}
      }
  \end{displaymath}
  which can be view as a commutative square in $\Cc$:
  \begin{displaymath}
      \xymatrix{
         P\ar[r]^{\overline{f}}\ar[d]_{\overline{g}}&B\ar[d]^{g}\\
         A\ar[r]^{f}&C
      }
  \end{displaymath}
  we call it a \termin[pullback square]{pullback!square} or \termin{cartesian diagram}, and say $\overline{g}$ is the \termin{pullback} of $g$ through $f$, $\overline{f}$ is the \emph{\red  pullback} of $f$ through $g$. We also call this limit the \termin[fibre product]{fibre!product} of $A$ and $B$ over $C$, and denoted by $A\times_CB$.
\end{defn}
Sometime, we consider such kind of category, in which every pullback exist, we call it a \termin[category with pullbacks]{category!with pullbacks} or say it \emph{\red  has} pullbacks.
\begin{prop}\label{pullback}
  Let $\xymatrix@1{A\ar[r]^{f}&C&B\ar[l]_{g}}$ be a pair of morphisms in $\mathcal{C}$, and $\xymatrix@1{A&A\times B\ar[r]^-{p_B}\ar[l]_-{p_A}&B}$ be the product of $A$ and $B$, $e\colon E\to A\times B$ is the equalizer of $fp_A$ and $gp_B$. Then the following diagram is cartesian:
  \begin{displaymath}
      \xymatrix{
         E\ar[r]^{p_Be}\ar[d]_{p_Ae}&B\ar[d]^{g}\\
         A\ar[r]^{f}&C
      }
  \end{displaymath}
\end{prop}
\begin{prop}
  Monomorphisms are \termin{stable} under pullback, which means that the pullback of a monomorphism is also a monomorphism. Moreover, regular monomorphisms are also stable, and so are isomorphisms.
\end{prop}
\begin{warn}
  Epimorphisms may not be stable under pullback in any category with pullbacks.
\end{warn}
The following proportion is a good exercise for diagram chase:
\begin{prop}[Two-pullbacks]\label{Two-pullbacks}
  Consider a commutative diagram in a category with pullbacks as below:
  \begin{displaymath}
      \xymatrix{
         \cdot\ar[r]\ar[d]&\cdot\ar[r]\ar[d]&\cdot\ar[d]\\
         \cdot\ar[r]&\cdot\ar[r]&\cdot
      }
  \end{displaymath}
  \begin{enumerate}[1)]
    \setlength{\itemindent}{2ex}
    \item  If the two small squares are pullbacks, so is the outer rectangle.
    \item  If the right square and the outer rectangle are pullbacks, so is the left square.
  \end{enumerate}
\end{prop}
\begin{cor}
  The pullback of a commutative triangle is a commutative triangle.
  \begin{displaymath}
      \xymatrix{
         \cdot\ar[rr]\ar[dd]\ar[dr]&&\cdot\ar[dd]\ar[dr]&\\
         &\cdot\ar[dl]&&\cdot\ar[dl]\\
         \cdot\ar[rr]&&\cdot
      }
  \end{displaymath}
\end{cor}

The dual concept of pullback is pushout.
\begin{defn}
  Let $\Jj'$ be
  \begin{displaymath}
      \xymatrix{
         \cdot&\cdot\ar[l]\ar[r]&\cdot
      }
  \end{displaymath}
  a limit for a $\Jj'-$diagram is of the form
  \begin{displaymath}
      \xymatrix{
         B&C\ar[l]_{g}\ar[r]^{f}&A
      }
  \end{displaymath}
  which can be view as a commutative square in $\Cc$:
  \begin{displaymath}
      \xymatrix{
         C\ar[r]^{f}\ar[d]_{g}&A\ar[d]^{\overline{g}}\\
         B\ar[r]^{\overline{f}}&P
      }
  \end{displaymath}
  we call it a \termin[pushout square]{pushout!square} or \termin{cocartesian diagram}, and say $\overline{g}$ is the \termin{pushout} of $g$ through $f$, $\overline{f}$ is the \emph{\red  pushout} of $f$ through $g$. We also call this limit the \termin[fibre coproduct]{fibre!coproduct} of $A$ and $B$ over $C$, and denoted by $A\amalg_CB$.
\end{defn}
By the duality principle, the duality of the properties of pullback are also true.

\begin{defn}
  Let $\overline{f}$ be the pullback of $f$ through $g$, it is called \termin[descendable]{descendable pullback} if $f$ is also the pushout of $\overline{f}$ through $g$.
\end{defn}

\begin{prop}\label{Fun.pull-push}
  Let $\Cc$ be a category with pullbacks (resp, pushouts), then taking pullback (resp. pushout) is a functor from $[\Jj,\Cc]$ to $[\Jj',\Cc]$ (resp. from $[\Jj',\Cc]$ to $[\Jj,\Cc]$).
\end{prop}

\subsection{Complete Categories}
\begin{defn}
  A category is said to be \termin[complete]{complete category}, if every diagram in it has a limit. Similar, a \termin[finite complete category]{finite!complete category} is such a category, in which every finite diagram has a limit. Dually, we have concepts of \termin{cocomplete category} and \termin[finite cocomplete category]{finite!cocomplete category}.
\end{defn}
\begin{thm}
  Let $\mathcal{C}$ be a category, the following statements are equivalent:
  \begin{enumerate}[a)]
    \setlength{\itemindent}{2ex}
    \item $C$ is finite complete.
    \item $C$ has finite products and equalizers.
    \item $C$ has pullbacks and terminal object.
  \end{enumerate}
\end{thm}

\newpage\section{Exactness}
\subsection{Exact Categories}
  \begin{defn}
    An \termin[exact category]{exact!category} $\Ee$ is an additive category possessing a class $\EEe$ of ``short exact sequences'': triples of objects connected by arrows
    \begin{equation*}
      M'\To M\To M''
    \end{equation*}
    satisfying the following axioms inspired by the properties of short exact sequences in an abelian category:
    \begin{enumerate}
      \item $\Ee$ is closed under isomorphisms and contains the split exact sequences:
               \begin{equation*}
                 M'\To M'\oplus M''\To M''
               \end{equation*}
      \item Suppose $M\to M''$ occurs as the second arrow of a sequence in $\EEe$ (it is called an \termin[admissible epimorphism]{admissible!epimorphism}) and $N\to M''$ is any arrow in $\Ee$. Then their \emph{pullback} exists and its \emph{projection} to $N$ is also an admissible epimorphism.

          Dually, if $M'\to M$ occurs as the first arrow of a sequence in $\EEe$ (it is called an \termin[admissible monomorphism]{admissible!monomorphism}) and $M'\to N$ is any arrow, then their \emph{pushout} exists and its \emph{coprojection} from $N$ is also an admissible monomorphism.

          \emph{In other words, the admissible epimorphisms are ``stable under pullback'', resp. the admissible monomorphisms are ``stable under pushout''.}
      \item Admissible monomorphisms are \emph{kernels} of their corresponding admissible epimorphisms, and dually.
               The composition of two admissible monomorphisms is admissible (likewise admissible epimorphisms);
      \item Suppose $M\to M''$ is a map in $\Ee$ which admits a kernel in $\Ee$, and suppose $N\to M$ is any map such that the composition
               $N\to M\to M''$ is an admissible epimorphism. Then so is $M\to M''$.

               Dually, if $M'\to M$ admits a cokernel and $M\to N$ is such that $M'\to M\to N$ is an admissible monomorphism, then so is $M'\to M$.
    \end{enumerate}
  \end{defn}
  \begin{rem}
    Admissible monomorphisms are generally denoted $\mono$ and admissible epimorphisms are denoted $\epi$. These axioms are not minimal; in fact, the last one has been shown by Bernhard Keller \cite{keller1990chain} to be redundant.
  \end{rem}

  \begin{defn}
    An \termin[exact functor]{exact!functor} $F$ from an exact category $\Dd$ to another one $\Ee$ is an additive functor such that if
    \begin{equation*}
      M'\mono M\epi M''
    \end{equation*}
    is exact in $\Dd$, then
    \begin{equation*}
      F(M')\mono F(M) \epi F(M'')
    \end{equation*}
    is exact in $\Ee$.
  \end{defn}

  \begin{defn}
    A subcategory $\Dd$ of $\Ee$ is called an \termin[exact subcategory]{exact!subcategory} if the inclusion functor is fully faithful and exact.
  \end{defn}

  \begin{defn}
    A \termin{Serre subcategory} is a non-empty full subcategory $\Ss$ of an abelian category $\Aa$ such that for all short exact sequences
    \begin{equation*}
      0\To M'\To M\To M''\To 0
    \end{equation*}
    in $\Aa$, $M$ belongs to $\Ss$ if and only if both $M'$ and $M''$ do. This notion arises from Serre's C-theory.
  \end{defn}

  \begin{exam}
    Exact categories come from abelian categories in the following way.
    Suppose $\Aa$ is abelian and let $\Ss$ be any Serre subcategory.
    We can take the class $\EEe$ to be simply the sequences in $\Ss$ which are exact in $\Aa$; that is,
    \begin{equation*}
      M'\To M\To M''
    \end{equation*}
    is in $\EEe$ iff
    \begin{equation*}
      0\To M'\To M\To M''\To 0
    \end{equation*}
    is exact in $\Aa$. Then $\Ss$ is an exact category.
  \end{exam}
  \begin{rem}
    The condition Serre subcategory can be weakened to be a strictly full additive subcategory which is closed under taking \emph{\red extensions} in the sense that given an exact sequence
    \begin{equation*}
      0\To M'\To M\To M''\To 0
    \end{equation*}
    in $\Aa$, then if $M',M''$ are in E, so is $M$.
  \end{rem}

  \begin{exam}
    The category $\Abtf$ of torsion-free abelian groups is exact.
  \end{exam}

  \begin{exam}
    The category $\Ab_{\tor}$ of abelian groups with torsion (and also the zero group) is exact.
  \end{exam}

\subsection{Exact Functors}
  \begin{defn}
    Let $\Aa,\Bb$ be two abelian categories, $F\colon\Aa\To\Bb$ is an additive functor. Let
    \begin{equation*}
      0\To A\To B\To C\To 0
    \end{equation*}
    be a short exact sequence in $\Aa$. We say that $F$ is
    \begin{itemize}
      \item \termin{half exact} if $F(A)\To F(B)\To F(C)$ is exact.
      \item \termin[left exact]{left!exact} if $0\To F(A)\To F(B)\To F(C)$ is exact.
      \item \termin[right exact]{right!exact} if $F(A)\To F(B)\To F(C)\To 0$ is exact.
      \item \termin[exact]{exact!functor} if $0\To F(A)\To F(B)\To F(C)\To 0$ is exact.
    \end{itemize}
  \end{defn}

  For contravariant functor, the definition is similar.

  \begin{prop}
    A covariant (not necessarily additive) functor is left exact if and only if it turns finite limits into limits; a covariant functor is right exact if and only if it turns finite colimits into colimits; a contravariant functor is left exact if and only if it turns finite colimits into limits; a contravariant functor is right exact if and only if it turns finite limits into colimits. A functor is exact if and only if it is both left exact and right exact.
  \end{prop}
  \begin{prop}
    If the functor $F$ is left adjoint to $G$, then $F$ is right exact and $G$ is left exact.
  \end{prop}

  The degree to which a left exact functor fails to be exact can be measured with its right derived functors; the degree to which a right exact functor fails to be exact can be measured with its left derived functors.

\newpage\section{Diagram Lammas in Abelian Categories}
\emph{Throughout of this section, the category is assumed to be abelian unless otherwise specified.}
\subsection{Abelian Category}

The concept of abelian category has been introduced in Chapter 1. We discuss some properties of abelian categories.

\begin{prop}\label{monoic epi 1}
  A morphism $f\colon A\To B$ is monoic (resp. epi) if and only if $\ker f=0$ (resp. $\coker f=0$) if and only if $\coim f=1_A$ (resp. $\im f=1_B$) if and only if $f=\im f$ (resp. $f=\coim f$).
\end{prop}
\begin{proof}
  We prove the monoic case only, the epi case is dual to it.

  (monoic $\Longrightarrow$ $\ker f=0$.) We have $f\circ\ker f=0=f\circ 0$, since$f$ is monoic, $\ker f=0$.

  ($\ker f=0$ $\Longrightarrow$ $\coim f=1_A$.) It is easy to check that $1_A$ satisfying the universal property of $\coim f$.

   ($\coim f=1_A$ $\Longrightarrow$ $f=\im f$.) By the natural isomorphism $\Psi$.

  ($f=\im f$ $\Longrightarrow$ monoic.) A kernel is always monoic.
\end{proof}

\begin{prop}
  Any bimorphism must be a isomorphism.
\end{prop}
\begin{proof}
  Suppose $f\colon A\To B$ is a bimorphism, which means that $f$ is both monoic and epi, hence $\coim f=1_A$ and $\im f=1_B$ by Proposition \ref{monoic epi 1}. Whence the standard factorization become $f=\Psi$ which is a isomorphism in an abelian category.
\end{proof}

\begin{cor}\label{epi-mono}
  Every morphism $f$ can be uniquely factorized as an epimorphism $e$ followed by a monomorphism $m$. Moreover, $e=\coim f$, $m=\im f$.
\end{cor}

\begin{lem}\label{factor through}
  Let $f$ be factorized as $h$ followed by $g$, then
  \begin{enumerate}[(i)]
    \item If $g$ is monoic, then $\ker f=\ker h$;
    \item If $h$ is epi, then $\coker f=\coker g$.
  \end{enumerate}
\end{lem}
\begin{proof}
  By check the definition, the statements are clearly true and hold in any category where related concepts make sense.
\end{proof}

\begin{defn}
  A monomorphism is called \termin[norm]{norm!monomorphism}, if it is a kernel of some morphism. Dually, an epimorphism is called \termin[norm]{norm!epimorphism}, if it is a cokernel of some morphism.
\end{defn}

We now introduce some equivalent conditions of abelian category.

\begin{prop}
  A pre-abelian category becomes abelian if and only if all monomorphisms and epimorphisms are normal.
\end{prop}
\begin{proof}
  The ``only if'' comes from Proposition \ref{monoic epi 1}.

  Conversely, if all monomorphisms and epimorphisms are normal, then for every morphism $f$,
\end{proof}

\subsection{Cartesian Diagrams}
\begin{lem}
  Consider the following diagram:
  \begin{displaymath}
      \xymatrix{
         E\ar[r]^{f'}\ar[d]_{g'}&B\ar[d]^{g}\\
         A\ar[r]^{f}&C
      }
  \end{displaymath}
  \begin{enumerate}[(i)]

    \item The diagram is commutative if and only if the composition
             \begin{equation*}
               \longexseq{E}{\<f',g'\>}{A\oplus B}{\<f,-g\>}{C}
             \end{equation*}
             is equal to $0$. Where $\<f',g'\>$ is the unique morphism such that $f'=p_2\<f',g'\>$ and $g'=p_1\<f',g'\>$, $\<f,-g\>$ is the unique morphism such that $f=\<f,-g\>i_1$ and $-g=\<f,-g\>i_2$.
    \item This diagram is cartesian if and only if $\<f',g'\>=\ker\<f,-g\>$.
    \item This diagram is cocartesian if and only if $\<f,-g\>=\coker\<f',g'\>$.
  \end{enumerate}
\end{lem}
\begin{proof}
  \begin{enumerate}[(i)]
    \item By the definition, we have
             \begin{equation*}
               fg'-gf'=\<f,-g\>i_1p_1\<f',g'\>+\<f,-g\>i_2p_2\<f',g'\>=\<f,-g\>\<f',g'\>
             \end{equation*}
             Hence $fg'=gf'$ if and only if $\<f,-g\>\<f',g'\>=0$.
    \item It follows from Proposition \ref{pullback}.
    \item It is the dual of \emph{(ii)}.
  \end{enumerate}
\end{proof}

\begin{prop}\label{bicartesian}
  In a cartesian diagram, if $f$ is an epimorphism, then so is $f'$, and the diagram is also cocartesian.
\end{prop}
\begin{proof}
  If $f$ is epimorphism, then so is $\<f,-g\>$: let $u\colon C\To T$ be an arbitrary morphism such that $u\<f,-g\>=0$, then $uf=u\<f,-g\>i_1=0$, which implies $u=0$. Thus we get an exact sequence:
  \begin{equation*}
    0\To \longexseq{E}{\<f',g'\>}{A\oplus B}{\<f,-g\>}{C} \To 0
  \end{equation*}
  Hence $\<f,-g\>=\coker\<f',g'\>$, and the diagram is cocartesian.

  Let $v\colon B\To T$ be an arbitrary morphism such that $vf'=0$, then $vp_2\<f',g'\>=0$, hence there exists a morphism $w\colon C\To T$ such that $vp_2=w\<f,-g\>$.
  \begin{displaymath}
      \xymatrix{
          E\ar[r]^-{\<f',g'\>} & A\oplus B\ar[r]^-{\<f,-g\>}\ar[dr]_{vp_2} & C\ar@{-->}[d]^{w}  \\
          & & T
      }
  \end{displaymath}

  We then have (notice that $p_2i_1=0$)
  \begin{equation*}
    0=vp_2i_1=w\<f,-g\>i_1=wf
  \end{equation*}
  hence $w=0$ and therefore $v=0$.
\end{proof}

\begin{prop}
  In a cartesian diagram, let $k\colon K\To A$ be the kernel of $f$. Then $k$ can be factor as $k=g'k'$ where $k'$ is a kernel of $f'$.
\end{prop}
\begin{proof}
  First, we show that $k$ can be factor as $k=g'k'$:   Since $\<f',g'\>=\ker\<f,-g\>$ and
  \begin{equation*}
    \<f,-g\>i_1k=fk=0
  \end{equation*}
  there exist a unique morphism $k'\colon K\To E$ such that
  \begin{equation*}
     \<f',g'\>k'=i_1k
  \end{equation*}
  \begin{displaymath}
      \xymatrix{
          E\ar[r]^-{\<f',g'\>} & A\oplus B\ar[r]^-{\<f,-g\>} & C  \\
          K\ar[ur]_{i_1k}\ar@{-->}[u]^{k'} & &
      }
  \end{displaymath}
  Hence
  \begin{equation*}
    g'k'=p_1\<f',g'\>k'=p_1i_1k=k
  \end{equation*}

  We now prove that $k'$ is a kernel of $f'$:
  Let $t\colon T\To E$ be an arbitrary morphism such that $f't=0$. Then $fg't=gf't=0$, hence there exists a unique morphism $u$ such that $ku=g't$.
  \begin{displaymath}
      \xymatrix{
          T\ar[r]^{t}\ar@{-->}[d]_{u} & E\ar[r]^{f'}\ar[d]^{g'} & B\ar[d]^{g}  \\
          K\ar[ur]_{k'}\ar[r]_{k} & A\ar[r]_{f} & C
      }
  \end{displaymath}
  Notice that
  \begin{equation*}
    f'k'u=p_2\<f',g'\>k'u=p_2i_1ku=0=f't
  \end{equation*}
  and $g'k'u=ku=g't$. Hence $k'u=t$. Which proved $k'$ is a kernel of $f'$.
\end{proof}

\subsection{Snake Lemma}
\emph{When we drawing diagrams contain many kernels and cokernels, to simplify the notation, we denoted the domain of the kernel of $f$ by $\ker f$ while the codomain of the cokernel by $\coker f$. $\im f$ and $\coim f$ are similar. This will not make ambiguity since they are just be used as notations of objects in this situation.}

\begin{thm}[Weak Snake Lemma]\label{w.snake}
  The short exact sequences of morphisms $\alpha,\beta,\gamma$, which means a commutative diagram like below
\begin{displaymath}
      \xymatrix{
         0\ar[r] & A\ar[r]\ar[d]_{\alpha} & B\ar[r]\ar[d]_{\beta} & C\ar[r]\ar[d]_{\gamma} & 0 \\
         0\ar[r] & A'\ar[r] & B'\ar[r] & C'\ar[r] & 0
      }
\end{displaymath}
  induces an exact sequence relating kernels and cokernels
    \begin{equation*}
      0\To\ker\alpha\To\ker\beta\To\ker\gamma\markar{\delta}\coker\alpha\To\coker\beta\To\coker\gamma\To0
    \end{equation*}
\end{thm}

\begin{proof}
  We proceed in steps:
  \begin{enumerate}
    \item
    The morphisms between kers are clearly given by the efinition and the commutative of diagram. By chasing the diagram and use the definition, one can verify that the sequence of kers is exact. For the cokers, the argument is similar. Let's proof that $m_0\colon\ker\alpha\To\ker\beta$ is the kernel of $e_0\colon\ker\beta\To\ker\gamma$ as an example:
    \begin{displaymath}
      \xymatrix{
         T\ar@{-->}[rd]|{t}\ar@/^/[rrd]^{\tau}\ar@{-->}@/_/[rdd]_{\tau'}&&&\\
         &\ker\alpha\ar[r]^{m_0}\ar[d]_{i} & \ker\beta\ar[r]^-{e_0}\ar[d]_{j} & \ker\gamma\ar[d]_{k} \\
         &A\ar[r]^{m}\ar[d]_{\alpha} & B\ar[r]^{e}\ar[d]_{\beta} & C\ar[d]_{\gamma}\\
         &A'\ar[r]^{m'} & B'\ar[r]^{e'} & C'
      }
    \end{displaymath}

    By choosing any object $T$ with morphism $\tau$ such that $e_0\tau=0$, we have $ke_0\tau=0$, hence $ej\tau=0$. Since $m$ is the kernel of $e$, there exist a unique $\tau'\colon T\To A$ such that $j\tau=m\tau'$.

    We have
    \begin{equation*}
      m'\alpha\tau'=\beta m\tau'=\beta j\tau=0
    \end{equation*}

    Which implies $\alpha\tau'=0$ since $m'$ is monoic. Hence there exist a unique $t\colon T\To\ker\alpha$ such that $it=\tau'$, hence
     \begin{equation*}
       jm_0t=mit=m\tau'=j\tau
     \end{equation*}

    Which implies $m_0t=\tau$ since $j$ is monoic.
    Which shows that $m_0$ is a kernel of $e_0$ and the sequence is exact by lemma \ref{exactsq}.
    \item
      The morphism $\delta$ can be obtained by this way:

      Consider the following diagram
    \begin{displaymath}
      \xymatrix{
         A\ar@{-->}[r]^{s}\ar@{=}[d] & D\ar[r]^-{u}\ar[d]_{k'}      & \ker\gamma\ar[d]_{k} \\
         A\ar[r]^{m}\ar[d]_{\alpha}       & B\ar[r]^{e}\ar[d]_{\beta}   & C\ar[d]_{\gamma}        \\
         A'\ar[r]^{m'}\ar[d]_{c}              & B'\ar[r]^{e'}\ar[d]_{c'}     & C'\ar@{=}[d]                \\
         \coker\alpha\ar[r]_-{v}           & D'\ar@{-->}[r]_{t}            & C'
      }
    \end{displaymath}
    where the upper-right square is cartesian and the lower-left square is cocartesian. Since $e$ is epi, then $u$ is epi and $u=\coker s$. Similarly,  $v$ is a monoic and $v=\ker t$.

    By the commutativity, $tc'\beta k'=\gamma ku=0$, hence there exists a unique morphism $d\colon D\To \coker\alpha$ such that
    \begin{equation*}
      vd=c'\beta k'
    \end{equation*}

    Similarly, $vds=c'\beta k's=vc\alpha=0$. Then $ds=0$ because $v$ is monoic. Hence there exists a unique morphism
    \begin{equation*}
      \delta\colon\ker\gamma\To\coker\alpha
    \end{equation*}
    such that $\delta u=d$.
    Moreover,
    \begin{equation*}
      v\delta u=c'\beta k'
    \end{equation*}
    \item
      We now prove that the sequence
    \begin{equation*}
      \ker\beta\markar{e_0}\ker\gamma\markar{\delta}\coker\alpha\markar{m_1}\coker\beta
    \end{equation*}
     is exact, we prove the exactness at $\ker\gamma$ only, the case at $\coker\alpha$ is dual to this.

    Let $x\colon\ker\delta\To\ker\gamma$ be the kernel of $\delta$, and $t\colon\im e_0\To \ker\gamma$ be the image of $e_0$. We need to show the are equivalent. Since both $x$ and $t$ are monoic, it suffices to show that they can factor through each other.

    (\emph{1})
    First, we prove that $e_0\delta=0$.
    \begin{displaymath}
      \xymatrix{
         \ker\beta\ar@{-->}[rd]|{\tau}\ar@/^/[rrd]^{e_0}\ar@/_/[rdd]_{j}&&\\
         &D\ar[r]^{u}\ar[d]_{k'} & \ker\gamma\ar[d]^{k} \\
         & B\ar[r]_{e} & C
      }
    \end{displaymath}
    Since $ej=ke_0$ there exist a unique $\tau\colon\ker\beta\To D$ such that $u\tau=e_0$ and $k'\tau=j$. Therefore
    \begin{equation*}
      v\delta e_0=v\delta u\tau=c'\beta k'\tau=c'\beta j=0
    \end{equation*}
    Which implies $\delta e_0=0$ since $v$ is monoic.

    (\emph{2})
    Therefore $t$ can factor through $x$ by a unique morphism $\mu$. To get a factorization of $x$ through $t$, we pullback $t$ through $x$ and obtain a monomorphism $t'$:
    \begin{displaymath}
      \xymatrix{
         \cdot\ar[r]^{t'}\ar[d]_{x'}   & \ker\delta\ar[d]^{x}        \\
         \im e_0 \ar[r]_-{t}\ar@{-->}[ur]|{\mu}     & \coker\gamma
      }
    \end{displaymath}

    If $t'$ is an isomorphism, then it is clearly that $x=tx't'^{-1}$. To prove this, it suffices to show $t'$ is epi.

    (\emph{3})
    Consider the diagram in 2. Since $e/\beta k'=\gamma ku=0$, there exist a unique $f\colon D\To A'$ such that $m'f=\beta k'$. We have
    \begin{equation*}
      vcf=c'm'f=c'\beta k'=v\delta u
    \end{equation*}
    which implies $cf=\delta u$ because $v$ is monoic.

    Pullback $x$ through $u$:
    \begin{displaymath}
      \xymatrix{
         Y\ar[r]^-{y_1}\ar[d]_{y}        & \ker\delta\ar[d]^{x}       \\
         D\ar[r]_-{u}                          & \ker\gamma
      }
    \end{displaymath}

    We have
    \begin{equation*}
      cfy=\delta uy=\delta xy_1=0
    \end{equation*}
    Hence there exist a unique $f'\colon Y\To\im\alpha=\ker c$ such that
    \begin{equation*}
      \alpha_0f'=fy
    \end{equation*}

    We pullback through the $f'$ the epimorphism $\alpha_1$:
    \begin{displaymath}
      \xymatrix{
         Z\ar[r]^{z_1}\ar[d]_{z}        & Y\ar[d]^{f'}       \\
         A\ar[r]_-{\alpha_1}                          & \im\alpha
      }
    \end{displaymath}

    In order to see more clearly,  we put these morphisms in the following diagram which may not commutative at the upper-left square:
    \begin{displaymath}
      \xymatrix{
         Z\ar@{-->}[r]^{z_1}\ar@{-->}[d]_{z}
         & Y\ar@{-->}[r]^{y_1}\ar@{-->}[d]^{y}\ar@{-->}[ddl]_{f'}
         & \ker\delta\ar[d]^{x} \\
         A\ar[dr]^{m}\ar[d]_{\alpha_1}   & D\ar[r]^-{u}\ar[d]^{k'}\ar[ddl]^{f}      & \ker\gamma\ar[d]^{k} \\
         \im\alpha\ar[d]_{\alpha_0}        & B\ar[r]^{e}\ar[d]_{\beta}   & C\ar[d]_{\gamma}        \\
         A'\ar[r]^{m'}\ar[d]_{c}               & B'\ar[r]^{e'}                     & C'                \\
         \coker\alpha           &              &
      }
    \end{displaymath}

    (\emph{4})
    To measure the non-commutativity, define
    \begin{equation*}
      \Delta=k'yz_1-mz
    \end{equation*}
    Since
    \begin{equation*}
      \beta k' y z_1 = m' f y z_1 = m' \alpha_0 f' z_1 = m' \alpha_0 \alpha_1 z = \beta m z
    \end{equation*}
    We have $\beta\Delta=0$ and hence there exist a unique $\theta\colon Z\To\ker\beta$ such that $j\theta=\Delta$.

    On the other hand,
    \begin{equation*}
      e\Delta = ek'yz_1-emz = ek'yz_1-0 = kxy_1z_1
    \end{equation*}
    Hence $ej\theta=kxy_1z_1$.

    Consider the diagram below:
    \begin{displaymath}
      \xymatrix{
         Z\ar@{-->}[dr]_{\tau}\ar@/^/[drr]^{y_1z_1}\ar[dd]_{\theta} & & \\
                & \cdot\ar[r]_-{t'}\ar[d]_{x'}   & \ker\delta\ar[d]^{x}        \\
         \ker\beta\ar[r]^-{e_1}\ar[d]_{j}     & \im e_0\ar[r]^{t}        & \ker\gamma\ar[d]^{k}  \\
         B\ar[rr]_{e}           & & C
      }
    \end{displaymath}
    We have
    \begin{equation*}
      kte_1\theta = ke_0\theta=ej\theta=kxy_1z_1
    \end{equation*}
    which implies $te_1\theta=xy_1z_1$ because $k$ is monoic.

    By the universality of pullback, there exist a unique morphism $\tau$, such that $t'\tau=y_1z_1$ which is an epimorphism, hence so is $t'$.
  \end{enumerate}
\end{proof}

\begin{thm}[Snake Lemma]
  The following commutative diagram of exact sequences
\begin{displaymath}
      \xymatrix{
         & A\ar[r]^{f}\ar[d]_{\alpha} & B\ar[r]^{g}\ar[d]_{\beta} & C\ar[d]_{\gamma}\ar[r] & 0 \\
         0\ar[r] & A'\ar[r]_{f'} & B'\ar[r]_{g'} & C' &
      }
\end{displaymath}
  induces an exact sequence relating kernels and cokernels
    \begin{equation*}
      \ker\alpha\longrightarrow\ker\beta\longrightarrow\ker\gamma\markar{\delta}\coker\alpha\longrightarrow\coker\beta\longrightarrow\coker\gamma
    \end{equation*}
\end{thm}
\begin{proof}
First, we prove that the kernel of $\alpha$ is a pullback of the kernel of $\beta$ through $f$:

    \begin{displaymath}
      \xymatrix{
         T\ar@{-->}[rd]|{\tau}\ar@/^/[rrd]^{t'}\ar@/_/[rdd]_{t}&&\\
         & \ker\alpha\ar[r]^{f_0}\ar[d]_{i} & \ker\beta\ar[d]^{j} \\
         & A\ar[r]^{f}\ar[d]_{\alpha} & B\ar[d]^{\beta} \\
         & A'\ar[r]_{f'} & B'
      }
    \end{displaymath}

  Let $T$ be an arbitrary object with morphisms $t,t'$ such that $ft=jt'$. Then
  \begin{equation*}
    f'\alpha t=\beta ft=\beta jt'=0
  \end{equation*}
  which implies $\alpha t=0$ because $f'$ is monoic. Hence there exist a unique $\tau\colon T\To\ker\alpha$ such that $i\tau=t$.
  Then
  \begin{equation*}
    jf_0\tau=fi\tau=ft=jt'
  \end{equation*}
  which implies $f_0\tau=t'$ because $j$ is monoic. Hence $i$ is the pullback of $j$ through $f$.

  Let $f_m\colon K\To B$ be the kernel of $g$ and $g_e'\colon B'\To K'$ cokernel of $f'$. Then we get a commutative diagram
  \begin{displaymath}
      \xymatrix{
         A\ar[r]^{f_e}\ar[dr]_-{\alpha} & K\ar[r]^{f_m}\ar@{-->}[d]^{a} & B\ar[r]^{g}\ar[d]^{\beta} & C\ar[dr]^{\gamma}\ar[r]\ar@{-->}[d]_{c} & 0 \\
         0\ar[r] & A'\ar[r]_{f'} & B'\ar[r]_{g'_e} & K'\ar[r]_{g'_m} & C'
      }
  \end{displaymath}
  where $(f_e,f_m), (g'_e,g'_m)$ are the \emph{epi-mono factorizations}\footnote{Notice that the sequence is exact in $B$ means that $\im f=\ker g$ and, dually, $\coker f=\coim g$. Hence we get the epi-mono factorization.} of $f$ and $g$ respectively, and $a,c$ determined uniquely by the fact that $A'=\ker g'$ and $C=\coker f$.

  Then By the Weak Snake Lemma (\ref{w.snake}), the diagram induces an exact sequence:
    \begin{equation*}
      0\To\ker a\To\ker\beta\To\ker c\markar{\delta}\coker a\To\coker\beta\To\coker c\To0
    \end{equation*}

    Since $f_e$ is epi, by Lemma \ref{factor through}, the cokernel of $a$ and $\alpha$ coincide. Dually, the kernel of $c$ and $\gamma$ coincide.

    Hence it suffices to show that $\ker a\To\ker\beta$ is the image of $\ker\alpha\To\ker\beta$, and the cokernel case is dual to this.

    Consider the following diagram, where $e$ exists and make the diagram commutative since $af_ei=fi=0$.
  \begin{displaymath}
      \xymatrix{
         \ker\alpha\ar[r]^{e}\ar[d]_{i} & \ker a\ar[r]^{m}\ar[d]_{j} & \ker\beta\ar[d]_{k} \\
         A\ar[r]_{f_e} & K\ar[r]_{f_m} & B
      }
  \end{displaymath}
  By the universal property of kernel, $me$ is the morphism $\ker\alpha\To\ker\beta$ we discussed above.

  By the statements we proved at first, the right square and the outer rectangle are pullbacks, so is the left square by Proposition \ref{Two-pullbacks}.
  Therefore $e$ is a pullback of $f_e$ and hence epi. By the uniqueness of epi-mono factorization (\ref{epi-mono}), $m$ is the image of $\ker\alpha\To\ker\beta$.
\end{proof}

  \begin{rem}
    The morphism $\delta$ is called \termin{connection morphism}, while the long exact sequence is called \termin{snake sequence}.
  \end{rem}
  \begin{prop}
    The the snake sequence is \emph{natural} in the sense that if
    \begin{displaymath}
        \xymatrix@!0{
          &&& \cdot\ar[rr]\ar'[d]^{\alpha'}[dd] && \cdot\ar[rr]\ar'[d]^{\beta'}[dd] && \cdot\ar[rr]\ar'[d]^{\gamma'}[dd] && 0 \\
          && \cdot\ar[rr]\ar[dd]^(.3){\alpha}\ar[ru] && \cdot\ar[rr]\ar[dd]^(.3){\beta}\ar[ru] && \cdot\ar[rr]\ar[dd]^(.3){\gamma}\ar[ru] && 0 & \\
          & 0\ar'[r][rr] && \cdot\ar'[r][rr] && \cdot\ar'[r][rr] && \cdot && \\
          0\ar[rr] && \cdot\ar[rr]\ar[ru] && \cdot\ar[rr]\ar[ru] && \cdot\ar[ru] &&&
        }
    \end{displaymath}
    is a commutative diagram with exact rows, then the snake lemma can be applied twice, to the ``front'' and to the ``back'',
    yielding two snake sequences; these are related by a commutative diagram of the form:
    \begin{displaymath}
        \xymatrix@!0{
          &\ker\alpha'\ar[rr] && \ker\beta'\ar[rr] && \ker\gamma'\ar[rr]^-{\delta'} && \coker\alpha'\ar[rr] && \coker\beta'\ar[rr] && \coker\gamma' \\
          \ker\alpha\ar[rr]\ar[ur] && \ker\beta\ar[rr]\ar[ur] && \ker\gamma\ar[rr]^-{\delta}\ar[ur] && \coker\alpha\ar[rr]\ar[ur] && \coker\beta\ar[rr]\ar[ur] && \coker\gamma\ar[ur] &
        }
    \end{displaymath}
  \end{prop}
  \begin{proof}
    The commutativity of between kers (resp. cokers) are clear. It suffices to check commutativity at the connection morphisms.

    For $\alpha',\beta',\gamma'$, we have the same diagrams as $\alpha,\beta,\gamma$ in the construction of connection morphism.
    Without causing ambiguity, we can use the same labels and denote the morphisms from the ``front'' to the ``back'' by $\sigma$.

    Since taking pullback and pushout are functors (\ref{Fun.pull-push}), we have
    \begin{equation*}
      v\delta'\sigma u = v\delta'u\sigma = c'\beta'k' \sigma = \cdots = \sigma c'\beta k' = \sigma v\delta u = v\sigma \delta u
    \end{equation*}
    Hence $\delta'\sigma=\sigma\delta$.
  \end{proof}

  \begin{thm}[Short Five Lemma]
    Consider the following commutative diagram of exact sequences:
    \begin{displaymath}
        \xymatrix{
          & \cdot\ar[r]^{f}\ar[d]_{\alpha} & \cdot\ar[r]^{g}\ar[d]_{\beta} & \cdot\ar[d]_{\gamma}\ar[r] & 0 \\
          0\ar[r] & \cdot\ar[r]_{f'} & \cdot\ar[r]_{g'} & \cdot &
        }
    \end{displaymath}
    Then,  if $\alpha,\gamma$ are monoic (resp. epi), then so is $\beta$. Moreover, assume $f$ is monoic and $g'$ is epi, then any two of $\alpha,\beta,\gamma$ are isomorphisms implies so is the third.
  \end{thm}
  \begin{proof}
    Check the snake sequence, then the statements are obvious.
  \end{proof}

  \begin{thm}[Five Lemma]
    Consider the following exact sequences of five morphisms:
    \begin{displaymath}
        \xymatrix{
          \cdot\ar[r]^{a}\ar[d]_{f_1} & \cdot\ar[r]^{b}\ar[d]_{f_2} & \cdot\ar[r]^{c}\ar[d]_{f_3} & \cdot\ar[r]^{d}\ar[d]_{f_4} & \cdot\ar[d]_{f_5} \\
          \cdot\ar[r]_{a'} & \cdot\ar[r]_{b'} & \cdot\ar[r]_{c'} & \cdot\ar[r]_{d'} & \cdot
        }
    \end{displaymath}
    Then
    \begin{enumerate}[a)]
      \item If $f_1$ is epi and $f_2,f_4$ are monoic, then $f_3$ is monoic;
      \item If $f_5$ is monoic and $f_2,f_4$ are epi, then $f_3$ is epi;
      \item If $f_1,f_2,f_4,f_5$ are isomorphisms, then so is $f_3$.
    \end{enumerate}
  \end{thm}
  \begin{proof}
    First of all, notice that any exact sequence of morphisms can be factored into short exact sequences. For example, our exact sequence above
    can be factored as
    \begin{displaymath}
        \xymatrix@1{
          \cdot\ar[r]^-{a_e}\ar[d]_{f_1} & \im a \ar[r]^-{a_m}\ar[d]_{f_a}
          & \cdot\ar[r]^-{b_e}\ar[d]_{f_2} & \im b \ar[r]^-{b_m}\ar[d]_{f_b}
          & \cdot\ar[r]^-{c_e}\ar[d]_{f_3} & \im c \ar[r]^-{c_m}\ar[d]_{f_c}
          & \cdot\ar[r]^-{d_e}\ar[d]_{f_4} & \im d \ar[r]^-{d_m}\ar[d]_{f_d}
          & \cdot\ar[d]_{f_5} \\
          \cdot\ar[r]_-{a'_e} & \im a' \ar[r]_-{a'_m}
          & \cdot\ar[r]_-{b'_e} & \im b' \ar[r]_-{b'_m}
          & \cdot\ar[r]_-{c'_e} & \im c' \ar[r]_-{c'_m}
          & \cdot\ar[r]_-{d'_e} & \im d' \ar[r]_-{d'_m}
          & \cdot
        }
    \end{displaymath}

    We prove only \emph{a)} as a example:

    Since $a'_ef_1$ is epi, so is $f_a$. Consider the short exact sequence of $f_a,f_2,f_b$, by snake lemma, we have the snake sequence:
    \begin{equation*}
      \ker f_a\To\ker f_2\To\ker f_b\markar{\delta}\coker f_a\To\coker f_2\To\coker f_b
    \end{equation*}
    Since $f_2$ is monoic and $f_a$ is epi, the exact sequence at $\ker f_b$ become
    \begin{equation*}
      0\To\ker f_b\markar{\delta}0
    \end{equation*}
    Hence $f_b$ is monoic.

    Since $f_4c_m$ is monoic, so is $f_c$. By the \emph{short five lemma}, $f_3$ is monoic as desired.
  \end{proof}


\newpage\section{Appendix: Some Counterexamples}



\subsection*{About Category Theory}
  In this chapter, I try to introduce some concepts in ``category theory''.

  Nowadays, there are numerous books introducing category theory, like \cite{lawvere1997conceptual} and \cite{awodey2010category} which are easily readable books. Of course, The standard textbook is \cite{lane1998categories}.
  Unless otherwise specified, most of the contents in this chapter comes from them.
  One can also find them in a modern homological algebra textbook.

%\chapter{Category}

  \chapter{Set Theory}
  In this chapter, we supplement some important facts in set theory.
\minitoc
\newpage
\section{The fundamental axiom system}
\subsection{What should a set be?}
It is common to use foundations of mathematics in which ``set'' is an undefined term; this is set theory as a foundation. In a pure material set theory like ZFC, every object is a set. Even in a structural approach such as ETCS, it is common for every object to be a structured set in some way or another.

Material set theory conflates two notions of sets, which were elegantly (but not first) described by Mathieu Dupont in a blog post as ``set$^1$'' and ``set$^2$'', which we will here call ``abstract set'' and ``concrete set''. In the latter case (set$^2$), we have some fixed universe of discourse (for example, collection of all real numbers), and a \textbf{concrete set} is a set of elements of this universe (in our example, a set of real numbers). In the former case (set$^1$), an \textbf{abstract set} is a purely abstract concept, an unstructured (except perhaps for the equality relation) collection of unlabelled elements. Arguably (this argument goes back at least to Lawvere), Cantor's original concept of cardinal number (Kardinalzahl) was the abstraction from a concrete set to its underlying abstract set.

\subsection{ZFC}
  The commonly accepted standard foundation of mathematics today is a material set theory, \termin{ZFC} or \emph{Zermelo-Fraenkel set theory with the axiom of choice}.
  \begin{axiom}[Extensionality]
    If two sets have the same members, then they are equal and themselves members of the same sets.
  \end{axiom}
  \begin{axiom}[Specification]
    Given any set $X$ and any property $P$ of the elements of $X$, there is a set $\{x\in X\mid P(x)\}$ consisting precisely of those elements of $X$ for which $P$ holds.
  \end{axiom}
  \begin{axiom}[Empty set]
    There is an \textbf{empty set}: a set $\varnothing$ with no elements.
  \end{axiom}
  \begin{axiom}[Union]
    If $S$ is a set, then there is a set $\bigcup S$, \textbf{the union of $S$}, whose elements are precisely the elements of the elements of $S$.
  \end{axiom}
  \begin{axiom}[Paring]
    If $A$ and $B$ are sets, then there is a set $\{A,B\}$, \textbf{the unordered pairing of $A$ and $B$}, whose elements are precisely $A$ and $B$.
  \end{axiom}
  \begin{axiom}[Power set]
    If $X$ is a set, then there is a set $\Pp(X)$, \textbf{the power set of $X$}, whose elements are precisely \textbf{the subsets of $X$}, that is the sets whose elements are all elements of $X$.
  \end{axiom}
  \begin{defn}
    The union of a set $X$ and its singleton $\{X\}$ is called the \termin{successor} of $X$. An inductive set $N$ is a set such that whenever any $x$ is a member of $N$, its successor is also a member of $N$.
  \end{defn}
  \begin{axiom}[Infinity]
    Nonempty inductive set exists.
  \end{axiom}
  \begin{axiom}[Replacement]
    Given a property $\psi[x,Y]$ with the chosen free variables shown, if $X$ is a set and if for every $x$ in $X$ there is a unique $Y$ (denoted by $\Psi(x)$) such that $\psi[x,Y]$ holds, then there is set $\{\Psi(x)\mid x\in X\}$, \textbf{the image of $X$ under $\psi$}.
  \end{axiom}
  \begin{axiom}[Choice]
    If $X$ is a set, each of whose elements has an element, then there is a set with exactly one element from each element of $X$.
  \end{axiom}
  If we abandon the axiom of choice, then the rest axioms form a weaker system called \termin{ZF}, \emph{the Zermelo-Fraenkel set theory}.

  For more detail, see \href{http://en.wikipedia.org/wiki/Zermelo-Fraenkel_set_theory}{\emph{Wikipedia}} or any textbook on set theory.

\subsection{Universe}
  \begin{defn}
    A \termin{Grothendieck universe} $\Uu$ is a set $\Uu$ such that:
    \begin{enumerate}
      \item If $x$ is an element of $\Uu$ and if $y$ is an element of $x$, then $y$ is also an element of $\Uu$.
%      \item If $x$ and $y$ are both elements of $\Uu$, then $\{x,y\}$ is an element of $\Uu$.
      \item If $x$ is an element of $\Uu$, then $\Pp(x)$, the power set of $x$, is also an element of $\Uu$.
      \item If $\{x_i\}_{i\in I}$ is a family of elements of $\Uu$, and if $I$ is an element of $\Uu$, then the union $\bigcup_{i\in I} x_i$ is an element of $\Uu$.
      \item $\N\in\Uu$.
    \end{enumerate}

    Elements of a Grothendieck universe $\Uu$ are called \termin[$\Uu-$small sets]{$\Uu-$small set}, while a subset of $\Uu$ is called \termin[$\Uu-$moderate]{$\Uu-$moderate set}.  If the universe $\Uu$ is understood, we may simply say \termin[small]{small set} and \termin[moderate]{moderate set}.
  \end{defn}
  We now prove a useful lemma for a given Grothendieck universe $\Uu$.
  \begin{lem}
    If $x \in \Uu$ and $y \subseteq x$, then $y \in \Uu$.
  \end{lem}
  \begin{proof}
    $y \in \Pp(x)$ because $y \subseteq x$. $\Pp(x) \in \Uu$ because $x \in \Uu$, so $y \in \Uu$.
  \end{proof}
  If $\Uu$ is a Grothendieck universe, then it is easy to show that $\Uu$ is itself a model of ZFC. Therefore, one cannot prove in ZFC the existence of a Grothendieck universe, and so we need extra set-theoretic axioms to ensure that uncountable universes exist. Grothendieck's original proposal was to add the following axiom of universes to the usual axioms of set theory:
  \begin{axiom}[Universes]
    Every set belongs to a universe.
  \end{axiom}
  Whenever any operation leads one outside of a given Grothendieck universe, there is guaranteed to be a bigger Grothendieck universe in which one lands. In other words, every set can be small if your universe is large enough!

  Later, Mac Lane pointed out that often, it suffices to assume the existence of one uncountable universe. In particular, any discussion of ``small'' and ``large'' that can be stated in terms of sets and proper classes can also be stated in terms of a single universe $\Uu$.

\subsection{Other axiom system}
  There are other material set theories, such as NBG, \emph{the Von Neumann-Bernays-G\"{o}del set theory}, MK, \emph{the Morse-Kelley set theory} and so on.

  Unlike \textbf{material} set theories, A \textbf{structural} set theory is a set theory which describes structural mathematics, and only structural mathematics. As conceived by the structuralist, mathematics is the study of structures whose form is independent of the particular attributes of the things that make them up.

  For example, the structuralist says, essentially, that the number ``3'' should denote ``the third place in a natural numbers object'' rather than some particular set such as $\{\varnothing,\{\varnothing\},\{\varnothing,\{\varnothing\}\}\}$ as it does in any definition of ``the set of natural numbers'' in ZF.

  Among category theorists, it's popular to state the axioms of a structural set theory by specifying elementary properties of the category of sets. For this reason structural set theory is often called \textbf{categorial} set theory.

  The original, and most commonly cited, categorially presented structural set theory is Bill Lawvere's \href{http://ncatlab.org/nlab/show/ETCS}{\textbf{ETCS}}. It is weaker than ZFC and must be supplemented with an axiom of collection to handle some esoteric parts of modern mathematics, although it suffices for most everyday uses.

  Another structural set theory, which is stronger than ETCS (since it includes the axiom of collection by default) and also less closely tied to category theory, is \href{http://ncatlab.org/nlab/show/SEAR}{\textbf{SEAR}}.
\subsection{Exercises}
\begin{ex}
  Show that any Grothendieck universe $\Uu$ contains:
  \begin{itemize}
    \item All singletons of each of its elements,
    \item All products of all families of elements of $\Uu$ indexed by an element of $\Uu$,
    \item All disjoint unions of all families of elements of $\Uu$ indexed by an element of $\Uu$,
    \item All intersections of all families of elements of $\Uu$ indexed by an element of $\Uu$,
    \item All functions between any two elements of $\Uu$,
    \item All subsets of $\Uu$ whose cardinal is an element of $\Uu$.
  \end{itemize}
\end{ex}


\newpage\section{Axiom of choice}
  In the viewpoint of category theory, the axiom of choice (\termin{AC}) is the following statement:
  \begin{axiom}[AC]
    Every surjection in the category $\Set$ splits.
  \end{axiom}

  In this section, we introduce some equivalence statements of AC under the assumption ZF. The most important among them are Zorn's lemma and the well-ordering theorem.

  Here is a very short list from $n$Lab; much longer lists can be found elsewhere, such as at \emph{\href{http://en.wikipedia.org/wiki/Axiom_of_choice\#Equivalents}{Wikipedia}}. Some of the statements on this list, though, may be of interest to $n$Labbers but are not commonly mentioned as equivalents of choice.
  \begin{itemize}
    \item The well-ordering theorem.
    \item Zorn's lemma,
    \item That (L= monomorphisms, R= epimorphisms) is a weak factorization system on $\Set$.
    \item That $\Set$ is equivalent to its own free exact completion.
    \item That there exists a group structure on every inhabited set (see \emph{\href{http://mathoverflow.net/questions/12973/does-every-non-empty-set-admit-a-group-structure-in-zf/12988\#12988}{this MO answer}}).
    \item That every fully faithful and essentially surjective functor between strict categories is a strong equivalence of categories.
    \item That the nonabelian cohomology $H^1(X;G)$ is trivial for every discrete set $X$ and every group $G$ (see \emph{\href{http://golem.ph.utexas.edu/category/2013/07/cohomology_detects_failures_of.html}{this post}}).
  \end{itemize}
\subsection{Zorn's lemma}
  \begin{thm}[Zorn's lemma]
    Each proset $S$ has a maximal element if every chain in $S$ has an upper bound.
  \end{thm}
  A sketch of proof can be found in \emph{\href{http://ncatlab.org/nlab/show/Zorn's+lemma}{$n$Lab}}.
\subsection{The well-ordering theorem}
  \begin{defn}
    A \termin{well-order} $\leqslant$ on a set $S$ is a total order that is \termin{well-founded}, that means, every nonempty subset of $S$ has a minimal element.
    A set equipped with a well-order is called a \termin{well-ordered set}, or a \termin{woset}.
  \end{defn}
  This notion is sometimes very useful since it allows us to extent mathematical induction to any well-ordered sets.
  \begin{prop}
    Let $P(x)$ be a statement for elements of a woset $S$ whose strict order is denoted by $<$. Then, to show $P(x)$ holds for all elements $x$ of $S$, it suffices to show that $P(x_0)$ holds for the minimal element $x_0$ in $S$, and that
    \begin{quote}
      If $x$ is an element of $S$ and $P(y)$ is true for all $y$ such that $y<x$, then $P(x)$ must also be true.
    \end{quote}
  \end{prop}

  \begin{thm}[The well-ordering theorem]
    Given any set $S$, there exists a well-order $\leqslant$ on $S$.
  \end{thm}
  A sketch of proof can be found in \emph{\href{http://en.wikipedia.org/wiki/Well-ordering_theorem\#Statement_and_sketch_of_proof}{Wikipedia}} and \emph{\href{http://ncatlab.org/nlab/show/well-ordering+theorem\#statement_and_proof}{$n$Lab}}




\newpage\section{Ordinals and cardinals}
  Naively, a \termin{cardinal number} should be an isomorphism class of sets, and the \termin{cardinality} of a set $S$ would be its isomorphism class. That is:
  \begin{enumerate}
    \item every set has a unique cardinal number as its cardinality;
    \item every cardinal number is the cardinality of some set;
    \item two sets have the same cardinality if and only if they are isomorphic as sets.
  \end{enumerate}
  Then a \termin{finite cardinal} is the cardinality of a finite set, while an \termin{infinite cardinal} or \termin{transfinite cardinal} is the cardinality of an infinite set.

  Formally, instead of define what is a cardinal number, we define the category of cardinal numbers $\Card$ as the skeleton of $\Set$.

  Naively, a \termin{ordinal number} should be an isomorphism class of wosets, and the \termin{ordinal rank} of a woset $(S,\leqslant)$ would be its isomorphism class. That is:
  \begin{enumerate}
    \item every woset has a unique ordinal number as its ordinal rank;
    \item every ordinal number is the ordinal rank of some woset;
    \item two wosets have the same ordinal rank if and only if they are isomorphic as wosets.
  \end{enumerate}
  Then a \termin{finite ordinal} is the ordinal rank of a finite woset, while an \termin{infinite ordinal} or \termin{transfinite ordinal} is the ordinal rank of an infinite woset.

  Formally, instead of define what is a ordinal number, we define the category of ordinal numbers $\Ord$ as the skeleton of $\Wos$, the category of wosets.

\subsection{Cardinal arithmetic}
  As a skeleton of $\Set$, the operations on $\Set$ define the natural arithmetic operations on $\Card$.

  More explicitly, denote the cardinality of a set $S$ by $|S|$, we have
  \begin{itemize}
    \item $\sum_i|S_i|=|\bigsqcup_i S_i|$;
    \item $\prod_i|S_i|=|\prod_i S_i|$;
    \item $|Y|^{|X|}=|\Hom(X,Y)|$;
  \end{itemize}

\section{The category of sets}


%  \chapter{Famous Books}
Here is a list of some famous books in present. The articles discuss them are copied from $n$Lab.
\minitoc
\newpage
\section{Handbook of Analysis and its Foundations} 
Erich Schechter's \emph{\href{http://www.math.vanderbilt.edu/~schectex/ccc/}{Handbook of Analysis and its Foundations}} is a large book, intended for self study by beginning graduate students or senior-level undergraduates, on all of the basic topics of abstract analysis and then some. Its logical flow is very much like that of Bourbaki, but focussed on that which applies to analysis and written in a modern style. Except for the slightly broken index (check the \href{http://www.math.vanderbilt.edu/~schectex/ccc/addenda/}{\textbf{errata}}!), it is very user-friendly, with sketched proofs phrased as exercises with hints, many examples (eventually), and abundant cross references. It is also extremely self contained; the only prerequisite is mathematical maturity, and the first section even helps with that!

It begins, as the name implies, with foundations: not only the usual na\"{i}ve set theory, but also a discussion of ZFC, constructive mathematics, and enough model theory to do nonstandard analysis. There is special emphasis on the axiom of choice; throughout the book, it is explicitly pointed out whenever anything beyond dependent choice and excluded middle is required. (The axioms of replacement and foundation, on the few occasions when they appear, are also pointed out, so logically the book takes place in ${Z^-} + {DC}$.)

The book then moves on to algebra, moving from monoids to fields, on the grounds that such algebra also serves as a foundation for analysis. This culminates in a treatment of category theory; this is somewhat unsatisfactory (although very good for an analysis book!) and is not much more than Bourbaki's theory of structures reinterpreted as a theory of concrete categories. After algebra comes topology, and then analysis proper.

Those aspects of analysis that do not depend on the theory of the real numbers are also covered when appropriate in the set-theory and algebra sections of the book; thus topological spaces and convex sets (for example) are defined early, although they are not thoroughly studied until later (after the foundational parts are finished). The analysis in the book is soft and abstract, on the grounds that this material serves as the proper foundation for hard results in concrete cases. However, many concrete examples are given to illustrate the abstract ideas. The breadth of topics covered, even within analysis itself, is quite wide; from convergence spaces to ultrabarrels, from the Henstock integral to the Brouwer fixed-point theorem, it has it all.

But everything must stop somewhere; it does not cover \textbf{complex analysis}.

\subsection*{Contents}

\begin{itemize}%
\item Part A: Sets and Orderings (Chapters 1--7)
\item Part B: Algebra (Chapters 8--14)
\item Part C: Topology and Uniformity (Chapters 15--21)
\item Part D: Topological Vector Spaces (Chapters 22--30)
\end{itemize}
\begin{enumerate}%
\item Sets
\item Functions
\item Relations and Orderings
\item More About Sups and Infs
\item Filters, Topologies, and Other Sets of Sets
\item Constructivism and Choice
\item Nets and Convergences
\item Elementary Algebraic Systems
\item Concrete Categories
\item The Real Numbers
\item Linearity
\item Convexity
\item Boolean Algebras
\item Logic and Intangibles
\item Topological Spaces
\item Separation and Regularity Axioms
\item Compactness
\item Uniform Spaces
\item Metric and Uniform Completeness
\item Baire Theory
\item Positive Measure and Integration
\item Norms
\item Normed Operators
\item Generalized Riemann Integrals
\item Fr\'{e}chet Derivatives
\item Metrization of Groups and Vector Spaces
\item Barrels and Other Features of TVS��s
\item Duality and Weak Compactness
\item Vector Measures
\item Initial Value Problems
\end{enumerate}





\newpage
\section{Elephant}
The Elephant is a book on topos theory by Peter Johnstone.

The full title is \emph{Sketches of an Elephant: A Topos Theory Compendium. Like Gravitation}, the title can be taken to refer not only to the subject matter but also to the immense size and scope of the book itself. Like The \emph{Lord of the Rings}, it consists of 6 parts arranged evenly into 3 volumes (but without appendices). Actually, Volume 3 has not yet been published (so who knows? it may have appendices after all!).

The Elephant is a good reference for anything related to topos theory, and we may often cite it here. However, it introduced many terminological changes, some of which may not be widely accepted or even known. (Fortunately, it will tell you about these in the text.) 

\subsection*{Contents}
\begin{enumerate}[A]
  \item Toposes as Categories
  \begin{enumerate}[{A}1]
    \item Regular and cartesian closed categories
    \begin{enumerate}[{A1}.1]
      \item Preliminary assumptions
      \item Cartesian categories
      \item Regular categories
      \item Coherent categories
      \item Cartesian closed categories
      \item Subobject classifiers
    \end{enumerate}
    \item Toposes - basic theory
    \begin{enumerate}[{A2}.1]
      \item Definition and examples
      \item The monadicity theorem
      \item The Fundamental Theorem
      \item Effectiveness, positivity and partial maps
      \item Natural number objects
      \item Quasitoposes
    \end{enumerate}
    \item Allegories
    \begin{enumerate}[{A3}.1]
      \item Relations in regular categories
      \item Allegories and tabulations
      \item Splitting symmetric idempotents
      \item Division allegories and power allegories
    \end{enumerate}
    \item Geometric morphisms - basic theory
    \begin{enumerate}[{A4}.1]
      \item Definition and examples
    \end{enumerate}
  \end{enumerate}
  \item $2$-Categorical Aspects of Topos Theory
  \begin{enumerate}[{B}1]
    \item Indexed categories and fibrations
    \begin{enumerate}[{B1}.1]
      \item Review of $2-$categories
      \item Indexed categories
      \item Fibrations
      \item Limits and colimits
      \item Descent conditions and stacks
    \end{enumerate}    
    \item Internal and localy internal categories
    \begin{enumerate}[{B2}.1]
      \item Review of enriched categories
      \item Locally internal categories
      \item Internal categories and diagram categories
      \item The indexed adjoint functor theorem
      \item Discrete opfibrations
      \item Filtered colimits
      \item Internal profunctors
    \end{enumerate}
    \item Toposes over a base
    \begin{enumerate}[{B3}.1]
      \item $S-$Toposes as $S-$indexed categories
      \item Diaconescu's theorem
      \item Giraud's theorem
      \item Colimits in Top
    \end{enumerate}
  \end{enumerate}
  \item Toposes as Spaces
  \begin{enumerate}[{C}1]
    \item Sheaves on a locale
    \begin{enumerate}[{C1}.1]
      \item Frames and nuclei
      \item Locales and spaces
      \item Sheaves, local homeomorphisms and frame-valued sets
      \item Continuous maps
      \item Some topological properties of toposes
    \end{enumerate}
    \item Sheaves on a site
    \begin{enumerate}[{C2}.1]
      \item Sites and coverages
      \item The topos of sheaves
      \item Morphisms of sites
      \item Internal sites and pullbacks
      \item Fibrations of sites
    \end{enumerate}
    \item Classes of geometric morphisms
    \begin{enumerate}[{C3}.1]
      \item Proper maps
      \item Locally connected morphisms
      \item Atomic morphisms
      \item Local maps
    \end{enumerate}
  \end{enumerate}
  \item Toposes as theories
  \begin{enumerate}[{D}1]
    \item First-order categorical logic
    \begin{enumerate}[{D1}.1]
      \item First-order languages
      \item Categorical semantics
      \item First-order logic
      \item Syntactic categories
      \item Classical completeness
    \end{enumerate}
    \item Sketches
    \begin{enumerate}[{D2}.1]
      \item The concept of sketch
      \item Sketches and theories
      \item Sketchable and accessible categories
      \item Properties of model categories
    \end{enumerate}
    \item Classifying toposes
    \begin{enumerate}[{D3}.1]
      \item Classifying toposes via syntactic sites
      \item The object classifier
      \item Coherent toposes
    \end{enumerate}
    \item Higher-order logic
    \begin{enumerate}[{D4}.1]
      \item Predicative type theories
      \item Axioms of choice and Booleanness
      \item Real numbers in a topos
    \end{enumerate}
    \item Aspects of finiteness
    \begin{enumerate}[{D5}.1]
      \item Finite cardinals
      \item Finitary algebraic theories
      \item Kuratowski-finiteness
    \end{enumerate}
  \end{enumerate}
  \item Homotopy and Cohomology
  \begin{enumerate}[{E}1]
    \item Homotopy theory for toposes
    \begin{enumerate}[{E1}.1]
      \item Homotopy theory of toposes
      \item The fundamental groupoid via paths
      \item The fundamental groupoid via coverings
      \item Natural homotopy
    \end{enumerate}
    \item Algebraic homotopy theory
    \begin{enumerate}[{E2}.1]
      \item Quillen model structures
      \item Model structure for simplicial sets
      \item Model structures for sheaves
    \end{enumerate}
    \item Cohomology theory
    \begin{enumerate}[{E3}.1]
      \item Abelian groups and modules in a topos
      \item Cech cohomology
      \item Torsors and non-abelian cohomology
      \item Cohomological applications of descent theory
    \end{enumerate}
  \end{enumerate}
  \item Toposes a Mathematical Universes
  \begin{enumerate}[{F}1]
    \item Synthetic differential geometry
    \begin{enumerate}[{F1}.1]
      \item Properties of the generic ring
      \item Rings of line type
      \item Well-adapted models
      \item Tiny objects
      \item Synthetic integration theory
      \item Intrinsic infinitesimal
    \end{enumerate}
    \item Topos theory and set theory
    \begin{enumerate}[{F2}.1]
      \item Internal sets in a topos
      \item Algebraic set theory
      \item Independence proofs via classifying toposes
      \item Independence of the axiom of choice
    \end{enumerate}
  \end{enumerate}
\end{enumerate}


\end{appendices}

\newpage
\backmatter

%Chapter=Table of Categories
\printglossaries

\newpage
%Chapter=Index
\phantomsection
\addcontentsline{toc}{chapter}{Index}
\printindex

\newpage
%Chapter=Bibliography
\phantomsection
\addcontentsline{toc}{chapter}{Bibliography}
\bibliographystyle{apalike}
\bibliography{bib}

\end{document}
