%%%%%%%%%%%%%%%%%%%%%%%%%%%%%%%%%%
%
%               A Common Setting File Made by Gau Syu
%                              GauSyu@Gmail.com
%
%###########################################
% @ Basic Document Packages
%     @@ Typeface
%     @@ Latex Graphics
%     @@ Index and Nomenclatures
%     @@ Define Colors
%     @@ Terminology Format
% @ Theorems and References
%     @@ Change the Indentation
%     @@ Define Theorem Environments
%     @@ Define Cross Reference Names
% @ Chapters and Sections
%%%%%%%%%%%%%%%%%%%%%%%%%%%%%%%%%%
%
%                          Basic Document Packages
%
%%%%%%%%%%%%%%%%%%%%%%%%%%%%%%%%%%
\documentclass[11pt,a4paper,oneside,nocap,fancyhdr,hyperref]{ctexbook}
%nocap means English
\usepackage{appendix} % Allowed Special Appendix Chapters
\usepackage{enumerate} % This package gives the enumerate environment an optional argument which determines the style in which the counter is printed.
%
%                                         Typefaces
%
    \newfontinstance{\Edward}{Edwardian Script ITC}
    \newfontinstance{\Frak}{Euclid Fraktur}
    \newcommand{\Giant}{\fontsize{72pt}{\baselineskip}\selectfont}
    \newcommand{\giant}{\fontsize{27pt}{\baselineskip}\selectfont}
%
%                                     Latex Graphics
%
\usepackage{graphicx}% Able to Insert Pictures
\usepackage{epic} % Extending Latex Graphics
\usepackage[all]{xy} % Able to Draw Diagrams
\xyoption{2cell}
\UseAllTwocells
\usepackage{tikz} % Able to Draw Pictures
%
%                           Index and Nomenclatures
%
\usepackage{makeidx}
  \makeindex
\usepackage[
  symbols,                %list of symbols
%  nonumberlist,       %do not show page numbers
  seeautonumberlist,
  hyperfirst=false,
  toc,                         %show listings as entries in table of contents
  section=chapter, %use section level for toc entries
  counter=section] %countered by section
{glossaries}

  \altnewglossary{categories}{cat}{Categories}
  \glsenablehyper
  \makeglossaries

%%  Usage of glossaryentry
%\newglossaryentry{<\label>}  {name=<\what occurs in the glossary>, description=<\>,  text=<\what occurs in the context>, sort=<\How this term by sorted>, type=<\>}
%%
%\usepackage[refpage,intoc]{nomencl}
%  \def\nomname{Notations}
%  \setlength{\nomlabelwidth}{.20\hsize}
%  \setlength{\nomitemsep}{-\parsep}
%    \makenomenclature

\usepackage{xifthen}% provides \isempty test
\newcommand{\termin}[2][]{%
  \ifthenelse{\isempty{#1}}%
    {\textbf{{#2}}\index{#2}}% if #1 is empty
    {\textbf{{#1}}\index{#2}}% if #1 is not empty
}
%
%                                    Define Colors
%
\usepackage{color}
  \newcommand{\red}{\color{red}}
  \newcommand{\blue}{\color{blue}}

%%%%%%%%%%%%%%%%%%%%%%%%%%%%%%%%%%
%
%                        Define Theorem Environments
%
%%%%%%%%%%%%%%%%%%%%%%%%%%%%%%%%%%
\usepackage{amsmath,amsthm} % the Standard AMS Package
\usepackage{mathtools} %Many tools

%\renewcommand{\proofname}{\textbf{Solution}}

\newtheoremstyle{question}{1.5ex plus 1ex minus .2ex}{1.5ex plus 1ex minus .2ex}{\large\itshape}{}{\bfseries}{}{1em}{}
\theoremstyle{question}
\newtheorem{qst}{Question}[section]
\renewcommand{\theqst}{\arabic{qst}}
\newtheorem{subqst}{Question}[qst]
\newtheorem*{qst*}{Question}

\theoremstyle{theorem}
\newtheorem{thm}{定理}[section]
\newtheorem{lem}[thm]{引理}
\newtheorem{prop}[thm]{命题}
\newtheorem{cor}[thm]{推论}

\theoremstyle{definition}
\newtheorem{defn}[thm]{定义}
\newtheorem{exam}[thm]{例子}
\newtheorem{ex}{}[chapter]
\theoremstyle{remark}
\newtheorem*{rem}{注}
\newtheorem*{rec}{回顾}

\renewcommand{\proofname}{证明}
%%%%%%%%%%%%%%%%%%%%%%%%%%%%%%%%%%
%
%                              Chapters and Sections
%
%%%%%%%%%%%%%%%%%%%%%%%%%%%%%%%%%%
\CTEXsetup[name={{$\S$},}]{section}
\CTEXsetup[name={{\Frak Cap}.,},aftername={\quad},format={\centering}]{chapter}
\CTEXoptions[contentsname={\textbf{Contents}}]
%\CTEXsetup[name={$\mathscr{P}$,}]{part}
\setcounter{secnumdepth}{2}% Depth of Sections
\setcounter{tocdepth}{1}% Depth of Contents
\usepackage{minitoc}
